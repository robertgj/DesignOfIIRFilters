\pdfminorversion=7 % Suppress warning messages
\documentclass[a4paper,twoside,10pt,english]{article}
%\usepackage{showframe} % Helps debug vbox and hbox warnings
\usepackage{geometry}
\geometry{verbose,nomarginpar,tmargin=1.5cm,bmargin=1.5cm,lmargin=1.5cm,
rmargin=1.5cm,headheight=1cm,headsep=1cm,footskip=1cm}

\usepackage[section]{algorithm}
\usepackage{algpseudocode}
\usepackage{amsfonts}
\usepackage{amsmath}
\usepackage{amssymb}
\usepackage{appendix}
\usepackage{array}
\usepackage{babel}
\usepackage{booktabs}
\usepackage{color} % For monochrome printing
\usepackage{enumitem}
\usepackage{esint}
\usepackage{float}
\usepackage[T1]{fontenc}
\usepackage{framed}
\usepackage[utf8]{inputenc}
\usepackage{lmodern}
\usepackage{mathtools}
\usepackage{mflogo}
\usepackage{multirow}
\usepackage{nag}
\usepackage{needspace}
\usepackage{subcaption}
\usepackage{threeparttable}
\usepackage{times}
\usepackage[nottoc,notbib]{tocbibind}
\usepackage{upquote}
\usepackage{verbatim}

\ifdefined\DesignOfSchurLatticeFiltersMono{}
  \begin{comment}
  To build a monochrome version:
  1. Set 'export OCTAVE_ENABLE_MONOCHROME=1' in the shell to enable the
     following line in src/test_common.m:
   if getenv("OCTAVE_ENABLE_MONOCHROME")
     set(0,"defaultaxescolororder",zeros(size(get(0,"defaultaxescolororder"))));
   endif
  2. Use the command 'make DesignOfSchurLatticeFilters_monochrome' to call:
   pdflatex '\newcommand\DesignOfSchurLatticeFiltersMono{} \
             \pdfminorversion=7 % Suppress warning messages
\documentclass[a4paper,twoside,10pt,english]{article}
%\usepackage{showframe} % Helps debug vbox and hbox warnings
\usepackage{geometry}
\geometry{verbose,nomarginpar,tmargin=1.5cm,bmargin=1.5cm,lmargin=1.5cm,
rmargin=1.5cm,headheight=1cm,headsep=1cm,footskip=1cm}

\usepackage[section]{algorithm}
\usepackage{algpseudocode}
\usepackage{amsfonts}
\usepackage{amsmath}
\usepackage{amssymb}
\usepackage{appendix}
\usepackage{array}
\usepackage{babel}
\usepackage{booktabs}
\usepackage{color} % For monochrome printing
\usepackage{enumitem}
\usepackage{esint}
\usepackage{float}
\usepackage[T1]{fontenc}
\usepackage{framed}
\usepackage[utf8]{inputenc}
\usepackage{lmodern}
\usepackage{mathtools}
\usepackage{mflogo}
\usepackage{multirow}
\usepackage{nag}
\usepackage{needspace}
\usepackage{subcaption}
\usepackage{threeparttable}
\usepackage{times}
\usepackage[nottoc,notbib]{tocbibind}
\usepackage{upquote}
\usepackage{verbatim}

\ifdefined\DesignOfSchurLatticeFiltersMono{}
  \begin{comment}
  To build a monochrome version:
  1. Set 'export OCTAVE_ENABLE_MONOCHROME=1' in the shell to enable the
     following line in src/test_common.m:
   if getenv("OCTAVE_ENABLE_MONOCHROME")
     set(0,"defaultaxescolororder",zeros(size(get(0,"defaultaxescolororder"))));
   endif
  2. Use the command 'make DesignOfSchurLatticeFilters_monochrome' to call:
   pdflatex '\newcommand\DesignOfSchurLatticeFiltersMono{} \
             \pdfminorversion=7 % Suppress warning messages
\documentclass[a4paper,twoside,10pt,english]{article}
%\usepackage{showframe} % Helps debug vbox and hbox warnings
\usepackage{geometry}
\geometry{verbose,nomarginpar,tmargin=1.5cm,bmargin=1.5cm,lmargin=1.5cm,
rmargin=1.5cm,headheight=1cm,headsep=1cm,footskip=1cm}

\usepackage[section]{algorithm}
\usepackage{algpseudocode}
\usepackage{amsfonts}
\usepackage{amsmath}
\usepackage{amssymb}
\usepackage{appendix}
\usepackage{array}
\usepackage{babel}
\usepackage{booktabs}
\usepackage{color} % For monochrome printing
\usepackage{enumitem}
\usepackage{esint}
\usepackage{float}
\usepackage[T1]{fontenc}
\usepackage{framed}
\usepackage[utf8]{inputenc}
\usepackage{lmodern}
\usepackage{mathtools}
\usepackage{mflogo}
\usepackage{multirow}
\usepackage{nag}
\usepackage{needspace}
\usepackage{subcaption}
\usepackage{threeparttable}
\usepackage{times}
\usepackage[nottoc,notbib]{tocbibind}
\usepackage{upquote}
\usepackage{verbatim}

\ifdefined\DesignOfSchurLatticeFiltersMono{}
  \begin{comment}
  To build a monochrome version:
  1. Set 'export OCTAVE_ENABLE_MONOCHROME=1' in the shell to enable the
     following line in src/test_common.m:
   if getenv("OCTAVE_ENABLE_MONOCHROME")
     set(0,"defaultaxescolororder",zeros(size(get(0,"defaultaxescolororder"))));
   endif
  2. Use the command 'make DesignOfSchurLatticeFilters_monochrome' to call:
   pdflatex '\newcommand\DesignOfSchurLatticeFiltersMono{} \
             \pdfminorversion=7 % Suppress warning messages
\documentclass[a4paper,twoside,10pt,english]{article}
%\usepackage{showframe} % Helps debug vbox and hbox warnings
\usepackage{geometry}
\geometry{verbose,nomarginpar,tmargin=1.5cm,bmargin=1.5cm,lmargin=1.5cm,
rmargin=1.5cm,headheight=1cm,headsep=1cm,footskip=1cm}

\usepackage[section]{algorithm}
\usepackage{algpseudocode}
\usepackage{amsfonts}
\usepackage{amsmath}
\usepackage{amssymb}
\usepackage{appendix}
\usepackage{array}
\usepackage{babel}
\usepackage{booktabs}
\usepackage{color} % For monochrome printing
\usepackage{enumitem}
\usepackage{esint}
\usepackage{float}
\usepackage[T1]{fontenc}
\usepackage{framed}
\usepackage[utf8]{inputenc}
\usepackage{lmodern}
\usepackage{mathtools}
\usepackage{mflogo}
\usepackage{multirow}
\usepackage{nag}
\usepackage{needspace}
\usepackage{subcaption}
\usepackage{threeparttable}
\usepackage{times}
\usepackage[nottoc,notbib]{tocbibind}
\usepackage{upquote}
\usepackage{verbatim}

\ifdefined\DesignOfSchurLatticeFiltersMono{}
  \begin{comment}
  To build a monochrome version:
  1. Set 'export OCTAVE_ENABLE_MONOCHROME=1' in the shell to enable the
     following line in src/test_common.m:
   if getenv("OCTAVE_ENABLE_MONOCHROME")
     set(0,"defaultaxescolororder",zeros(size(get(0,"defaultaxescolororder"))));
   endif
  2. Use the command 'make DesignOfSchurLatticeFilters_monochrome' to call:
   pdflatex '\newcommand\DesignOfSchurLatticeFiltersMono{} \
             \input{DesignOfSchurLatticeFilters}'
  \end{comment}
  \usepackage[hidelinks,colorlinks=false]{hyperref}
\else
  \usepackage[unicode=true,pdfusetitle,bookmarks=true,backref=page,
              bookmarksnumbered=false,bookmarksopen=false,breaklinks=false,
              pdfborder={1 0 0},colorlinks=true]{hyperref}
  \usepackage[all]{hypcap} % Needed to help hyperlinks direct correctly
\fi

\setlength{\parindent}{0em}
\setlength{\parskip}{\bigskipamount}
\floatstyle{ruled}
\newfloat{algorithm}{tbp}{loa}
\floatname{algorithm}{Algorithm}

\newcommand{\DesignOfSchurLatticeFiltersPdfScale}{1}
\newcommand{\DesignOfSchurLatticeFiltersIncludeScale}{0.85}

\newcommand{\coloneq}{\mathrel{\resizebox{\widthof{$\mathord{=}$}}
    {\height}{ $\!\!\resizebox{1.2\width}{0.8\height}
      {\raisebox{0.23ex}{$\mathop{:}$}}\!\!=\!\!$ }}}

\DeclareMathOperator{\sinc}{sinc}
\DeclareMathOperator{\jsn}{sn}
\DeclareMathOperator{\jns}{ns}
\DeclareMathOperator{\jcn}{cn}
\DeclareMathOperator{\jnc}{nc}
\DeclareMathOperator{\jdn}{dn}
\DeclareMathOperator{\jnd}{nd}
\DeclareMathOperator{\jsd}{sd}
\DeclareMathOperator{\jds}{ds}
\DeclareMathOperator{\jcd}{cd}
\DeclareMathOperator{\jdc}{dc}
\DeclareMathOperator{\jsc}{sc}
\DeclareMathOperator{\jcs}{cs}
\DeclareMathOperator{\jam}{am}
\DeclareMathOperator{\jarcsn}{arcsn}
\DeclareMathOperator{\jarcsc}{arcsc}
\DeclareMathOperator{\jarccn}{arccn}
\DeclareMathOperator{\jarcdn}{arcdn}
\DeclareMathOperator{\jarccd}{arccd}
\DeclareMathOperator{\jarccs}{arccs}
\DeclareMathOperator{\cosec}{cosec}
\DeclareMathOperator{\cotan}{cotan}
\DeclareMathOperator{\sech}{sech}
\DeclareMathOperator{\arctanh}{arctanh}
\DeclareMathOperator{\arccosh}{arccosh}
\DeclareMathOperator{\sign}{sign}
\DeclarePairedDelimiter{\ceil}{\lceil}{\rceil}
\DeclarePairedDelimiter{\floor}{\lfloor}{\rfloor}
\DeclarePairedDelimiter{\nint}{\lfloor}{\rceil}
\DeclarePairedDelimiter{\mathabs}{\lvert}{\rvert}
\DeclarePairedDelimiter{\mathnorm}{\lVert}{\rVert}
\DeclareMathOperator{\mathnull}{null}
\DeclareMathOperator{\mathspan}{span}
\DeclareMathOperator{\mathrank}{rank}
\DeclareMathOperator{\mathrange}{range}
\DeclareMathOperator{\mathtrace}{trace}
\DeclareMathOperator{\mathadj}{adj}
\DeclareMathOperator{\mathdiag}{diag}
\DeclareMathOperator{\mathtpltz}{Toeplitz}
\DeclareMathOperator{\mathcohull}{cohull}
\DeclareMathOperator{\mathcone}{cone}
\DeclareMathOperator{\mathri}{ri}
\DeclareMathOperator{\mathaff}{aff}
\DeclareMathOperator{\mathconv}{conv}
\DeclareMathOperator{\mathhe}{He}
\DeclareMathOperator{\mathsq}{Sq}
\DeclareMathOperator{\mathvec}{vec}
\DeclareMathOperator{\mathconj}{\ast} % {\textsf{\tiny{H}}}
\DeclareMathOperator{\mathreverse}{\textsf{\tiny{F}}}

\title{On the design of Schur lattice digital filters}
\author{Robert G. Jenssen}
\date{\today}

\begin{document}

\maketitle

\begin{abstract}
  This article describes a computer-aided procedure for approximating an
  arbitrary transfer function by a tapped one-multiplier all pass lattice
  filter with integer coefficients. The all pass lattice has a simple
  stability criterion, that the lattice coefficients have magnitude less
  than unity, and, in addition, has good coefficient sensitivity and round-off
  noise properties. The procedure consists of finding an initial filter
  transfer function, performing constrained minimum-mean-squared-error
  optimisation of the filter response and searching for the fixed point
  filter coefficients.
\end{abstract}

\section{Introduction}

The Infinite Impulse Response (IIR) digital filter transfer function is
a rational polynomial:
\begin{align}
  F\left(z\right) = \frac{B_{N}\left(z\right)}{A_{N}\left(z\right)}
  &= \frac{b_{0}+b_{1}z^{-1}+\cdots + b_{N}z^{-N}}
    {1+a_{1}z^{-1} + \cdots + a_{N}z^{-N}}
\label{eqn:Transfer-function-F}
\end{align}
Here $z$ is a complex number and the response of the filter with respect to
angular frequency, $\omega$, is $F\left(e^{\imath\omega}\right)$, where
$\omega=2\pi$ corresponds to the digital sampling frequency. 
This article describes a method for computer-aided design of the frequency
response of $F\left(e^{\imath\omega}\right)$. If the filter has
$A_{N}\left(z\right)=1$ then it is a Finite Impulse Response (FIR) filter and
the optimisation problem is convex, has no constraints on the coefficients and
is solved by well-known algorithms that minimise the peak error or the
least-squared-error~\cite{Medlin_LagrangeMultiplierFIRMultirate,SelesnickBurrus_ExchangeAlgorithmsLinearPhaseFIRFilters,SelesnickLangBurrus_ConstrainedLeastSquareFIRFilters,SelesnickLangBurrus_ConstrainedLeastSquareMultiBandFIRFilters,McClellanParks_ComputerProgramFIRLinearPhaseFilters,Iwasaki_GeneralizedKYPFrequencyDomainInequalities,Davidson_LMISpectralConstraints,ParksMcClellan_ChebyshevApproxNonRecursiveDigitalFilters}.
Otherwise, the filter is an Infinite Impulse (IIR) filter. An IIR filter is
stable if the roots, $P_{k}$, of the denominator polynomial,
$A_{N}\left(z\right)$, lie within the unit circle, $\left|P_{k}\right|<1$.
\emph{Lang}~\cite{Lang_LSDesignIIRDigitalFiltersPoleRadiusConstraint} applies
Rouch\'{e}'s theorem to constrain the zeros of $A_{N}\left(z\right)$ during
optimisation. \emph{Lu} and
\emph{Hinamoto}~\cite{LuHinamoto_IIRRobustStabilityQuadraticProgramming}
decompose $F\left(z\right)$ into a series connection of second-order sections
and apply the ``stabiliy triangle'' to each section to ensure filter stability
after optimisation. If the transfer function is represented in gain-pole-zero
form:
\begin{align*}
  F\left(z\right)&= G\frac{\prod_{k=1}^{N}\left(z-Z_{k}\right)}
                   {\prod_{k=1}^{N}\left(z-P_{k}\right)}
\end{align*}
where $G$ is the gain factor, the $Z_{k}$ are the roots of $B_{N}\left(z\right)$
and the $P_{k}$ are the roots of $A_{N}\left(z\right)$, then the constraints on
the pole locations of $F\left(z\right)$ are $\left|P_{k}\right|<1$. 
\emph{Deczky}~\cite{Deczky_MinPSynthesisIIRDigitalFilters} and
\emph{Richards}~\cite{Richards_DeczkyRecursiveDecimator} demonstrate the
optimisation of stable IIR filter transfer functions expressed in gain-pole-zero
form.
Lattice filters are an alternative implementation of the IIR filter
transfer function~\cite{VaidyanathanMitra_RobustFilterStructures}.
\emph{Vaidyanathan et
  al.}~\cite{VaidyanathanMitraNuevo_LowSensitivityIIRDigitalFilters,
  RegaliaMitraVaidyanathan_DigitalAllPassFilterVersatileSignalProcessing} show
that a subset of the $F\left(z\right)$ transfer functions can be factored into
the parallel sum of two all pass filters and that the resulting implementation
has good coefficient sensitivity properties. They show lattice filter
implementations of these all pass sections that are ``structurally bounded''
meaning that the section is all pass regardless of the values of the
coefficients. \emph{Gray} and
\emph{Markel}~\cite{GrayMarkel_DigitalLatticeAndLadderFilterSynthesis,
  MarkelGray_AutocorrelationSpeechAnalysis} describe the implementation of
$F\left(z\right)$ as a tapped all pass lattice filter based on the Schur
polynomial decomposition algorithm~\cite{SchurMethodsInOperatorTheory_Kailath,
  SchurMethodsInOperatorTheory_Schur_I}.
The Schur algorithm enables the decomposition of the denominator polynomial of
the transfer function, $F\left(z\right)$, into a set of orthogonal
polynomials. A result of this decompostion is a set of coefficients, often
called ``reflection coefficients'', that have magnitude less than unity
if-and-only-if the poles of the transfer function lie within the unit circle in
the complex plane. Further, the numerator polynomial of $F\left(z\right)$ can be
expressed as a weighted sum of the orthogonal polynomials. These coefficients and
orthogonal polynomials correspond to a tapped all pass lattice filter
implementation of the transfer function that has good round-off noise and
coefficient sensitivity when the coefficients are truncated to integer values.
This article reviews computer-aided techniques for designing tapped Schur
lattice filters with integer coefficients : finding an initial IIR filter
transfer function, constrained minimum-mean-squared-error optimisation of the
filter frequency response in terms of the lattice coefficients and searching for
integer filter coefficients. As an example, I demonstrate the design of a tapped
Schur lattice implementation of a low pass differentiator filter. The
coefficients and frequency response plots for this example are created with
Octave~\cite{Eaton_Octave,Octave_OptimPackage,Octave_SignalPackage}.
The source code for this article is available at
\url{https://github.com/robertgj/DesignOfIIRFilters}.

\section{Schur lattice filters}
\subsection{The Schur polynomial decomposition}
In 1918 \emph{Issai Schur}~\cite{SchurMethodsInOperatorTheory_Schur_I}
published a paper entitled ``\emph{On power series which are bounded in the
  interior of the unit circle}''. From the abstract:
\begin{quotation}
  The continued fraction algorithm introduced here very easily supplies an
  intrinsically important parametric representation for the coefficients of the
  power series to be considered.
\end{quotation}
\emph{Kailath}~\cite{SchurMethodsInOperatorTheory_Kailath} reviews the impact of
this algorithm on modern signal processing. \emph{Gray} and \emph{Markel}
describe the application of a similar algorithm to the autocorrelation analysis
of speech~\cite{MarkelGray_AutocorrelationSpeechAnalysis} and to the synthesis
of lattice digital
filters~\cite{GrayMarkel_DigitalLatticeAndLadderFilterSynthesis}. 
In the following I use the inner product method of \emph{Markel} and
\emph{Gray}~\cite{MarkelGray_AutocorrelationSpeechAnalysis} and
\emph{Parhi}~\cite[Chapter 12 and Appendix
D]{Parhi_VLSIDigitalSignalProcessingSystems}.

Initialise an $N$'th order polynomial as:
\begin{align*}
  \Phi_{N}\left(z\right) &= A_{N}\left(z\right) = \sum_{n=0}^{N}\phi_{n}z^{n}
\end{align*}
and define the reverse polynomial:
\begin{align*}
\hat{\Phi}_{N}\left(z\right)=z^{N}\Phi_{N}\left(z^{-1}\right)
\end{align*}
The transfer function $\hat{\Phi}_{N}\left(z\right)/\Phi_{N}\left(z\right)$ 
represents an all pass filter.
The Schur decomposition forms the polynomial $\Phi_{N-1}\left(z\right)$ as:
\begin{align*}
  \Phi_{N-1}\left(z\right) &= \frac{z^{-1}\left[\Phi_{N}\left(z\right)-
                             k_{N}\hat{\Phi}_{N}\left(z\right)\right]}
                             {s_{N}} \\
\end{align*}
where $s_{N}$ is a scaling constant and
$k_{N}=\phi_{0}/\phi_{N}=\Phi_{N}\left(0\right)/\hat{\Phi}_{N}\left(0\right)$.
The degree of $\Phi_{N-1}\left(z\right)$ is $1$ less than that of
$\Phi_{N}\left(z\right)$ since, by a change of variables, the numerator is:
\begin{align*}
 z^{-1}\left\{ \phi_{N}\Phi_{N}\left(z\right)-\phi_{0}\hat{\Phi}_{N}\left(z\right)\right\}
& = z^{-1}\sum^{N}_{n=0}\left\{\phi_{N}\phi_{n}z^{n}-\phi_{0}\phi_{N-i}z^{n}\right\}\\
& = \sum^{N}_{n=1}\left\{\phi_{N}\phi_{n}-\phi_{0}\phi_{N-i}\right\}z^{n-1}\\
\end{align*}
This degree reduction procedure is continued, resulting in the set of
polynomials $\left\{\Phi_{N}\left(z\right),\Phi_{N-1}\left(z\right),\ldots,
 \Phi_{0}\left(z\right)\right\}$:
\begin{align}
  \Phi_{n-1}\left(z\right) &= \frac{z^{-1}\left\{\Phi_{n}\left(z\right)-
                             k_{n}\hat{\Phi}_{n}\left(z\right)\right\}}
                             {s_{n}}
\label{eq:SchurDegreeReduction}
\end{align}
and
\begin{align*}
  \hat{\Phi}_{n-1}\left(z\right) &= \frac{\left\{\hat{\Phi}_{n}\left(z\right)-
                                         k_{n}\Phi_{n}\left(z\right)\right\}}
                                        {s_{n}}
\end{align*}

\emph{Markel} and \emph{Gray}~\cite[Equations 13a and 13b]
{MarkelGray_AutocorrelationSpeechAnalysis} define an inner product of two
polynomials, $P\left(z\right)$ and $Q\left(z\right)$:
\begin{align*}
  \langle P\left(z\right),Q\left(z\right)\rangle &= \frac{1}{2\pi \imath}
    \ointctrclockwise_{C}\frac{P\left(z\right)Q\left(z^{-1}\right)}
    {A_{N}\left(z\right)A_{N}\left(z^{-1}\right)}\frac{dz}{z}
\end{align*}
where all the zeros of $A_{N}\left(z\right)$ lie within the contour $C$.
Two properties of $\Phi_{n}\left(z\right)$ under this inner product are:
\begin{enumerate}
\item The $\Phi_{n}\left(z\right)$ polynomials are
  orthogonal~\cite[Appendix D]{Parhi_VLSIDigitalSignalProcessingSystems}:
  \begin{align*}
    \langle \Phi_{m}\left(z\right),\Phi_{n}\left(z\right)\rangle = 0
    \;\text{if}\; m \ne n
  \end{align*}
\item For $1\le n \le N$, $\mathabs{k_{n}} < 1$ if-and-only-if the zeros of
  $A_N\left(z\right)$ lie within the unit
  circle~\cite[pages 72 and 74]{MarkelGray_AutocorrelationSpeechAnalysis}
\end{enumerate}
In addition, if $s_{n}=\sqrt{1-k_{n}^{2}}$\;, then the $\Phi_{n}$ polynomials are
orthonormal~\cite[Appendix D]{Parhi_VLSIDigitalSignalProcessingSystems}:
\begin{align*}
  \langle \Phi_{n}\left(z\right),\Phi_{n}\left(z\right)\rangle = 1
\end{align*}

\subsection{The Schur one-multiplier lattice filter}

In Equation~\ref{eq:SchurDegreeReduction}, choose $s_{n}=1-\epsilon_{n}k_{n}$,
where $\epsilon_{n}=\pm 1$, and initialise $\Lambda_{N}\left(z\right)=A_{N}\left(z\right)$. Then:
\begin{subequations}
\begin{align}
  \Lambda_{n-1}\left(z\right) &= \frac{z^{-1}\left\{\Lambda_{n}\left(z\right)-
                             k_{n}\hat{\Lambda}_{n}\left(z\right)\right\}}
                                {1-\epsilon_{n}k_{n}}
  \label{eqn:SchurOneMLambda} \\
  \hat{\Lambda}_{n-1}\left(z\right)
                              &= \frac{\left\{\hat{\Lambda}_{n}\left(z\right)-
                                       k_{n}\Lambda_{n}\left(z\right)\right\}}
                                      {1-\epsilon_{n}k_{n}}
  \label{eqn:SchurOneMLambdaHat}
\end{align}
\end{subequations}
where:
\begin{align*}
  k_{n}=\frac{\Lambda_{n}\left(0\right)}{\hat{\Lambda}_{n}\left(0\right)}
  =\frac{\Phi_{n}\left(0\right)}{\hat{\Phi}_{n}\left(0\right)}
\end{align*}
Rearranging Equation~\ref{eqn:SchurOneMLambdaHat} and substituting it into
Equation~\ref{eqn:SchurOneMLambda} gives:
\begin{align*}
  z\Lambda_{n-1}\left(z\right)
  &=\left(1+\epsilon_{n}k_{n}\right)\Lambda_{n}\left(z\right)-
    k_{n}\hat{\Lambda}_{n-1}\left(z\right)\\
  \hat{\Lambda}_{n}\left(z\right) & = k_{n}\Lambda_{n}\left(z\right)+
              \left(1-\epsilon_{n}k_{n}\right)\hat{\Lambda}_{n-1}\left(z\right)
\end{align*}
Figure~\ref{fig:Schur-one-multiplier-lattice-section} shows the corresponding
\emph{all pass Schur one multiplier lattice} filter section.

\begin{figure}
\centering
\includegraphics{schur_OneMultiplier_lattice_section}
\caption{All pass Schur one multiplier lattice filter section}
\label{fig:Schur-one-multiplier-lattice-section}
\end{figure}

The expansion of the numerator polynomial of the transfer function,
$F\left(z\right)$, in the orthogonal basis, $\Lambda_{n}\left(z\right)$,
is~\cite[Section 12.2.3]{Parhi_VLSIDigitalSignalProcessingSystems}:
\begin{align*}
B_{N}\left(z\right) & = \sum_{n=0}^{N}c_{n}\Lambda_{n}\left(z\right)
\end{align*}

Figure~\ref{fig:Second-order-tapped-Schur-one-multiplier-lattice-filter}
shows the tapped Schur one multiplier lattice filter implementation of a
second order transfer function.

\begin{figure}
\centering
\includegraphics{schur_OneMultiplier_2nd_Order}
\caption{Second-order tapped Schur one-multiplier lattice filter}
\label{fig:Second-order-tapped-Schur-one-multiplier-lattice-filter}
\end{figure}

The relation between $\Lambda_{n}\left(z\right)$ and $\Phi_{n}\left(z\right)$ is:
\begin{align*}
\Lambda_{N}\left(z\right) & = \Phi_{N}\left(z\right)\\
  \Lambda_{n}\left(z\right)
  &= \Phi_{n}\left(z\right)\sqrt{\frac{\left(1+\epsilon_{N}k_{N}\right)
    \left(1+\epsilon_{N-1}k_{N-1}\right)\cdots
    \left(1+\epsilon_{n+1}k_{n+1}\right)}
    {\left(1-\epsilon_{N}k_{N}\right)
    \left(1-\epsilon_{N-1}k_{N-1}\right)\cdots
    \left(1-\epsilon_{n+1}k_{n+1}\right)}}
\end{align*}
The $\Lambda_{n}\left(z\right)$ polynomials are orthogonal but not orthonormal
since:
\begin{align*}
\langle \Lambda_{n}\left(z\right),\Lambda_{n}\left(z\right)\rangle
&= \langle\hat{\Lambda}_{n}\left(z\right),\hat{\Lambda}_{n}\left(z\right)\rangle\\
&=\frac{\left(1+\epsilon_{N}k_{N}\right)\left(1+\epsilon_{N-1}k_{N-1}\right)
  \cdots\left(1+\epsilon_{n+1}k_{n+1}\right)}
  {\left(1-\epsilon_{N}k_{N}\right)\left(1-\epsilon_{N-1}k_{N-1}\right)
  \cdots\left(1-\epsilon_{n+1}k_{n+1}\right)}
\end{align*}

If the input signal is random and white with unit power then the average power
at an internal node, $x_{n}$, is
$\langle\Lambda_{n}\left(z\right),\Lambda_{n}\left(z\right)\rangle$. The
magnitude of $\langle\Lambda_{n}\left(z\right),\Lambda_{n}\left(z\right)\rangle$
can be adjusted by choosing the sign parameters,
$\epsilon_{N},\epsilon_{N-1}\hdots,\epsilon_{n+1}$.
\emph{Gray} and
\emph{Markel}~\cite[p. 496]{GrayMarkel_DigitalLatticeAndLadderFilterSynthesis}
suggest that one criterion for choosing the sign parameters is to require that
the node associated with the largest $k_{n}$ parameter in magnitude have the
largest amplitude. The sign parameters are found recursively by requiring that
the amplitudes at other nodes be as large as possible without exceeding the
maximum value. If the maximum occurs for $k_{l}$ then the recursion proceeds
for $n=l-1,l-2,\ldots,0$ and again for $n=l+1,l+2,\ldots,N$. The
recursion is simple because:
\begin{align*}
  \frac{\langle \Lambda_{n}\left(z\right),\Lambda_{n}\left(z\right)\rangle}
       {\langle \Lambda_{n+1}\left(z\right),\Lambda_{n+1}\left(z\right)\rangle}
  &= \frac{1+\epsilon_{n+1}k_{n+1}}{1-\epsilon_{n+1}k_{n+1}}
\end{align*}
By changing the sign parameter, this ratio can always be made smaller or larger 
than one. Algorithm~\ref{alg:One-multiplier-sign-assignment} shows the method
used to assign the sign parameters described by \emph{Gray} and
\emph{Markel}~\cite[p. 496]{GrayMarkel_DigitalLatticeAndLadderFilterSynthesis}.

\begin{algorithm}
Assume that $k_{l}$ has the largest magnitude of the $k_{n}$ for
$n=1,2,\ldots,N$. Define the quantities
\begin{align*}
  Q_{n}&=\frac{\langle\Lambda_{n}\left(z\right),\Lambda_{n}\left(z\right)\rangle}
             {\langle\Lambda_{l}\left(z\right),\Lambda_{l}\left(z\right)\rangle}\\
  q_{n}&=\frac{1+\left|k_{n}\right|}{1-\left|k_{n}\right|}
\end{align*}

so that $Q_{l}=1$. Each $Q_{n}$ should be as large as possible without
exceeding $Q_{l}$. Successive ratios are:
\begin{align}
\frac{Q_{n}}{Q_{n+1}} = \begin{cases}
q_{n} & if\,\epsilon_{n}=\phantom{-}\sign\left(k_{n}\right)\\
1/q_{n} & if\,\epsilon_{n}=-\sign\left(k_{n}\right)
\end{cases}\label{eq:OneMultRecursiveKCalc}
\end{align}

Now assign the $\epsilon_{n}$:
\begin{algorithmic}
\For {$n=l-1,l-2,\hdots,1$}
  \If {$Q_{n+1}<1/q_{n}$}
    \State $\epsilon_{n}=-\sign\left(k_{n}\right)$
  \Else
    \State $\epsilon_{n}=\sign\left(k_{n}\right)$
  \EndIf
\EndFor
\For {$n=l+1,l+2,\hdots,N$}
  \If {$Q_{n}<1/q_{n}$}
    \State $\epsilon_{n}=\sign\left(k_{n}\right)$
  \Else
    \State $\epsilon_{n}=-\sign\left(k_{n}\right)$
  \EndIf
\EndFor
\end{algorithmic}
\caption{One multiplier lattice sign
  assignment~\cite[p. 496]{GrayMarkel_DigitalLatticeAndLadderFilterSynthesis}.} 
\label{alg:One-multiplier-sign-assignment}
\end{algorithm}

\subsection{State variable descriptions of the Schur one multiplier lattice filter}
The state variable representation of a filter is:
\begin{align}
\left[\begin{array}{c}
\mathbf{x}^{\prime}\\
\mathbf{y}\end{array}\right] &= \left[\begin{array}{cc}
    \mathbf{A} & \mathbf{B} \\
    \mathbf{C} & \mathbf{D} \\
\end{array}\right]\left[\begin{array}{c}
\mathbf{x}\\
\mathbf{u}\end{array}\right]
\label{eqn:State-variable-representation}
\end{align}
where $\mathbf{u}$ is the input, $\mathbf{x}$ is the filter state, for
convenience $\mathbf{x}^{\prime}=z\,\mathbf{x}$, and $\mathbf{y}$ is the filter
output. $\mathbf{u}$ and $\mathbf{y}$ may be vectors. $\mathbf{A}$ is called
the state transition matrix.
Examination of Figure~\ref{fig:Schur-one-multiplier-lattice-section} and
Figure~\ref{fig:Second-order-tapped-Schur-one-multiplier-lattice-filter}
suggests that the calculation of the all pass output, $\hat{y}$, of the
tapped Schur lattice filter can be represented as a matrix product:

\begin{align*}
\left[\begin{array}{c}
\hat{y}_{0}\\
x_{1}\\
x_{2}\\
\vdots\\
x_{N-1}\\
u \end{array}\right] & = \left[\begin{array}{cccccc}
1 & 0 & \cdots &  & \cdots & 0\\
0 & 1 &  &  &  & \vdots\\
\vdots &  & \ddots\\
 &  &  & \ddots &  & \vdots\\
0 &  &  &  & 1 & 0\\
0 & \cdots &  & \cdots & 0 & 1
\end{array}\right]\left[\begin{array}{c}
x_{0}\\
x_{1}\\
x_{2}\\
\vdots\\
x_{N-1}\\
u
\end{array}\right]
\end{align*}
\begin{align*}
\left[\begin{array}{c}
x_{0}^{\prime}\\
\hat{y}_{1}\\
x_{2}\\
\vdots\\
x_{N-1}\\
u
\end{array}\right] & = \left[\begin{array}{cccccc}
-k_{1} & \left(1+k_{1}\epsilon_{1}\right) & 0 & \cdots & \cdots & 0\\
\left(1-k_{1}\epsilon_{1}\right) & k_{1} & 0 &  &  & \vdots\\
0 & 0 & 1\\
\vdots &  &  & \ddots &  & \vdots\\
0 &  &  &  & 1 & 0\\
0 & \cdots &  & \cdots & 0 & 1
\end{array}\right]\left[\begin{array}{c}
\hat{y}_{0}\\
x_{1}\\
x_{2}\\
\vdots\\
x_{N-1}\\
u
\end{array}\right]
\end{align*}
\begin{align*}
\left[\begin{array}{c}
x_{0}^{\prime}\\
x_{1}^{\prime}\\
\hat{y}_{2}\\
\vdots\\
x_{N-1}\\     
u
\end{array}\right] & = \left[\begin{array}{cccccc}
1 & 0 & 0 & 0 & \cdots & 0\\
0 & -k_{2} & \left(1+k_{2}\epsilon_{2}\right) & 0 &  & \vdots\\
0 & \left(1-k_{2}\epsilon_{2}\right) & k_{2} & 0\\
\vdots &  &  & \ddots &  & \vdots\\
0 &  &  &  & 1 & 0\\
0 & \cdots &  & \cdots & 0 & 1
\end{array}\right]\left[\begin{array}{c}
x_{0}^{\prime}\\
\hat{y}_{1}\\
x_{2}\\
\vdots\\
x_{N-1}\\
u
\end{array}\right]
\end{align*}
\begin{align*}
  \vdots\\
\end{align*}
\begin{align*}
\left[\begin{array}{c}
x_{0}^{\prime}\\
x_{1}^{\prime}\\
\vdots\\
x_{N-2}^{\prime}\\
\hat{y}_{N-1}\\
u
\end{array}\right] & = \left[\begin{array}{cccccc}
1 & 0 & \cdots &  & \cdots & 0\\
0 & 1 &  &  &  & \vdots\\
\vdots &  & \ddots &  &  & \vdots\\
0 &  &  & -k_{N-1} & \left(1+k_{N-1}\epsilon_{N-1}\right) & 0\\
0 &  &  & \left(1-k_{N-1}\epsilon_{N-1}\right) & k_{N-1} & 0\\
0 & \cdots & \cdots & 0 & 0 & 1
\end{array}\right]\left[\begin{array}{c}
x_{0}^{\prime}\\
\vdots\\
x_{N-3}^{\prime}\\
\hat{y}_{N-2}\\
x_{N-1}\\
u
\end{array}\right]
\end{align*}
\begin{align*}
\left[\begin{array}{c}
x_{0}^{\prime}\\
x_{1}^{\prime}\\
x_{2}^{\prime}\\
\vdots\\
x_{N-1}^{\prime}\\
\hat{y}
\end{array}\right] & = \left[\begin{array}{cccccc}
1 & 0 & \cdots &  & \cdots & 0\\
0 & 1 &  &  &  & \vdots\\
\vdots &  & \ddots & & & \vdots\\
0 &  &  & 1 & 0 & 0\\
0 &  &  &  0 & -k_{N} & \left(1+k_{N}\epsilon_{N}\right)\\
0 & \cdots &  & 0 & \left(1-k_{N}\epsilon_{N}\right) & k_{N}
\end{array}\right]\left[\begin{array}{c}
x_{0}^{\prime}\\
x_{1}^{\prime}\\
\vdots\\
x_{N-2}^{\prime}\\
\hat{y}_{N-1}\\
u
\end{array}\right]
\end{align*}
The construction of the state variable description of the Schur one multiplier
lattice filter is summarised in
Algorithm~\ref{alg:Construction-state-variable-One-Multiplier-lattice}.

\begin{algorithm}
Given $\left\{k_{1},k_{2},\hdots,k_{N}\right\}$,
$\left\{\epsilon_{1},\epsilon_{2},\hdots,\epsilon_{N}\right\}$ and
$\left\{c_{0},c_{1},\hdots,c_{N}\right\}$:
\begin{algorithmic}
\State $\hat{y}_{0} = x_{0}$
\For{$n=1,\hdots,N-1$}
  \State $x_{n-1}^{\prime}=-k_{n}\hat{y}_{n-1}+\left(1+k_{n}\epsilon_{n}\right)x_{n}$
  \State $\hat{y}_{n}= \left(1-k_{n}\epsilon_{n}\right)\hat{y}_{n-1}+k_{n}x_{n}$
\EndFor
\State $x_{N-1}^{\prime} = -k_{N}\hat{y}_{N-1}+\left(1+k_{N}\epsilon_{N}\right)u$
\State $\hat{y} = \left(1-k_{N}\epsilon_{N}\right)\hat{y}_{N-1}+k_{N}u$
\State $y  = c_{0}x_{0}+c_{1}x_{1}+\cdots+c_{N-1}x_{N-1}+c_{N}u$
\end{algorithmic}
\caption{Construction of a state variable description of the Schur
  one multiplier lattice filter.}
\label{alg:Construction-state-variable-One-Multiplier-lattice}
\end{algorithm}

The state variable matrix calculations for a second order tapped Schur one
muliplier lattice filter are:
\begin{align*}
  \left[\begin{array}{c}
    x_{0}^{\prime} \\
    x_{1}^{\prime} \\
    \hat{y} \\
    y\end{array}\right] &= 
  \left[\begin{array}{ccc}- k_{1} & 1+\epsilon_{1} k_{1} & 0\\
-k_{2}\left(1-\epsilon_{1}k_{1}\right) & -k_{1}k_{2}& 1+\epsilon_{2}k_{2} \\
 \left(1-\epsilon_{1}k_{1}\right)\left(1-\epsilon_{2}k_{2}\right) &
   k_{1}\left(1-\epsilon_{2}k_{2}\right) & k_{2} \\
 c_{0} & c_{1} & c_{2} \end{array}\right] \left[\begin{array}{c}
    x_{0} \\
    x_{1} \\
    u \end{array} \right]
\end{align*}
The latency in the arithmetic calculation of $y$ and $\hat{y}$
increases with the filter order. \emph{Parhi}~\cite[Chapter 4 and Section 12.8]
{Parhi_VLSIDigitalSignalProcessingSystems} describes the retiming and pipelining
of lattice filters for reduced latency. One method is to design the lattice
filter with coefficients of the denominator polynomial, $A_{N}\left(z\right)$,
only in $z^{-2n}$, where $N$ is even and $n=1,\hdots,\frac{N}{2}$.
\emph{Deczky}~\cite{Deczky_MinPSynthesisIIRDigitalFilters} and
\emph{Richards}~\cite{Richards_DeczkyRecursiveDecimator} describe such filters
as $R=2$.
Figure~\ref{fig:Fourth-order-tapped-Schur-one-multiplier-lattice-filter-R2}
shows such a fourth order tapped Schur one multiplier lattice filter with and
without pipelining.

\begin{figure}
\begin{subfigure}[b]{\textwidth}
\centering
\includegraphics[scale=\DesignOfSchurLatticeFiltersIncludeScale]{schur_OneMultiplier_4th_Order_R2}
\caption{Without pipelining}
\label{subfig:Fourth-order-tapped-Schur-one-multiplier-lattice-R2-without}
\end{subfigure}
\begin{subfigure}[b]{\textwidth}
\centering
\includegraphics[scale=\DesignOfSchurLatticeFiltersIncludeScale]{schur_OneMultiplier_4th_Order_R2_Pipelined}
\caption{With pipelining}
\label{subfig:Fourth-order-tapped-Schur-one-multiplier-lattice-R2-with}
\end{subfigure}
\caption{Fourth-order tapped Schur one-multiplier lattice filter with denominator
  coefficients only for $z^{-2n}$}
\label{fig:Fourth-order-tapped-Schur-one-multiplier-lattice-filter-R2}
\end{figure}

The state variable matrix calculations for a pipelined fourth order tapped Schur
one-muliplier lattice filter with denominator polynomial coefficients only for
$z^{-2n}$ are:

\begin{align*}
  \left[\begin{array}{c}
    x_{0}^{\prime} \\
    x_{1}^{\prime} \\
    x_{2}^{\prime} \\
    x_{3}^{\prime} \\
    x_{4}^{\prime} \\
    \hat{y} \\
    y\end{array}\right] &= 
  \left[\begin{array}{cccccc} 0 & 1 & 0 & 0 & 0 & 0\\
    -k_{2} & 0 & 0 & 0 & 1+\epsilon_{2}k_{2} & 0\\
    1-\epsilon_{2}k_{2} & 0 & 0 & 0 & k_{2} & 0\\
    c_{0} & c_{1} & 0 & 0 & c_{2} & 0 \\
    0 & 0 & -k_{4} & 0 & 0 & 1+\epsilon_{4}k_{4}\\
    0 & 0 & 1-\epsilon_{4}k_{4}  & 0 & 0 & k_{4} \\
    0 & 0 & 0 & 1 & c_{3} & c_{4}\end{array}\right] \left[\begin{array}{c}
    x_{0} \\
    x_{1} \\
    x_{2} \\
    x_{3} \\
    x_{4} \\
    u \end{array} \right]
\end{align*}

The construction of the state variable description of the pipelined tapped Schur
one multiplier lattice filter with denominator polynomial coefficients only for
$z^{-2n}$ is summarised in
Algorithm~\ref{alg:State-variable-pipelined-One-Multiplier-R2-lattice}.

\begin{algorithm}
Given $N$ even, $\left\{k_{2},k_{4}\hdots,k_{N}\right\}$,
$\left\{\epsilon_{2},\epsilon_{4},\hdots,\epsilon_{N}\right\}$ and
$\left\{c_{0},c_{1},\hdots,c_{N}\right\}$:
\begin{algorithmic}
\State $x^{\prime}_{0}=x_{1}$
\State $x^{\prime}_{1}=-k_{2}x_{0}+\left(1+k_{2}\epsilon_{2}\right)x_{4}$
\State $x^{\prime}_{2}=\left(1-k_{2}\epsilon_{2}\right)x_{0}+k_{2}x_{4}$
\State $x^{\prime}_{3}=c_{0}x_{0}+c_{1}x_{1}+c_{2}x_{4}$

\For{$n=2,\hdots,\frac{N}{2}-1$}
\State $x^{\prime}_{3n-2}=-k_{2n}x_{3n-4}+\left(1+k_{2n}\epsilon_{2n}\right)x_{3n+1}$
\State $x^{\prime}_{3n-1}=\left(1-k_{2n}\epsilon_{2n}\right)x_{3n-4}+ k_{2n}x_{3n+1}$
\State $x^{\prime}_{3n}=x_{3n-3}+c_{2n-1}x_{3n-2}+c_{2n}x_{3n+1}$
\EndFor

\State $x_{3\frac{N}{2}-2}^{\prime} = -k_{N}x_{3\frac{N}{2}-4}+
\left(1+k_{N}\epsilon_{N}\right)u$
\State $\hat{y} = \left(1-k_{N}\epsilon_{N}\right)x_{3\frac{N}{2}-4}+k_{N}u$
\State $y = x_{3\frac{N}{2}-3}+c_{N-1}x_{3\frac{N}{2}-2}+c_{N}u$
\end{algorithmic}
\caption{Construction of a state variable description of the pipelined tapped
  Schur one multiplier lattice filter with denominator polynomial coefficients
  only for $z^{-2n}$.}
\label{alg:State-variable-pipelined-One-Multiplier-R2-lattice}
\end{algorithm}

\subsection{Frequency response of the tapped Schur one multiplier lattice filter\label{sec:Frequency-response-of-tapped-Schur-one-multiplier}}
Eliminating the filter state, $\mathbf{x}$, from
Equation~\ref{eqn:State-variable-representation} gives:
\begin{align}
\mathbf{F}\left(z\right) &= 
    \mathbf{C}\left(z\mathbf{I}-\mathbf{A}\right)^{-1}\mathbf{B}+\mathbf{D}
\label{eqn:State-variable-response}
\end{align}
\emph{Thiele}~\cite{Theile_SensitivityLinearStateSpaceSystems} uses the identity:
\begin{align*}
  \mathbf{R}\mathbf{R}^{-1} &= \mathbf{I} \\
\frac{\partial\mathbf{R}}{\partial\chi}\mathbf{R}^{-1} 
  +\mathbf{R}\frac{\partial\mathbf{R}^{-1}}{\partial\chi} &= 0 \\
  \frac{\partial\mathbf{R}}{\partial\chi} 
    &=-\mathbf{R}\frac{\partial\mathbf{R}^{-1}}{\partial\chi}\mathbf{R}
\end{align*}
\emph{Theile} shows the sensitivity functions of $\mathbf{F}\left(z\right)$ with
respect to the components $\alpha$, $\beta$, $\gamma$ and $\delta$ of
$\mathbf{A}$, $\mathbf{B}$, $\mathbf{C}$ and $\mathbf{D}$ respectively:
\begin{align*}
\frac{\partial\mathbf{F}}{\partial{}z} &= 
-\mathbf{C}\left(z\mathbf{I}-\mathbf{A}\right)^{-2}\mathbf{B} \\
  \frac{\partial\mathbf{F}}{\partial\alpha} &= 
\mathbf{C}\left(z\mathbf{I}-\mathbf{A}\right)^{-1}\frac{\partial\mathbf{A}}{\partial\alpha}
\left(z\mathbf{I}-\mathbf{A}\right)^{-1}\mathbf{B} \\
\frac{\partial\mathbf{F}}{\partial\beta} &= 
\mathbf{C}\left(z\mathbf{I}-\mathbf{A}\right)^{-1}\\
\frac{\partial\mathbf{F}}{\partial\gamma} &= 
\left(z\mathbf{I}-\mathbf{A}\right)^{-1}\mathbf{B}\\
\frac{\partial\mathbf{F}}{\partial\delta} &= \mathbf{I}
\end{align*}
Substituting $z=e^{\imath\omega}$ into Equation~\ref{eqn:State-variable-response}
gives the complex frequency response of the state variable filter on the unit
circle. In the following I will use
$\mathbf{R}=\left(e^{\imath\omega}\mathbf{I}-\mathbf{A}\right)^{-1}$. The
components $\alpha$ of $\mathbf{A}$, etc. may themselves be functions of other
variables (for example, the $c$ and $k$ coefficients of a tapped Schur one
multiplier lattice filter) that I will represent by $\chi$. The sensitivity
functions of the complex frequency response
$\mathbf{F}\left(e^{\imath\omega}\right)$ with respect to $\omega$ and $\chi$ are:
\begin{align*}
  \frac{\partial\mathbf{R}}{\partial{}\omega}
  =& -\imath{}e^{\imath\omega}\mathbf{R}\mathbf{R}\\
\frac{\partial\mathbf{F}}{\partial\omega}=&
-\imath e^{\imath\omega}\mathbf{C}\mathbf{R}\mathbf{R}\mathbf{B} \\
  \frac{\partial\mathbf{F}}{\partial{}\chi}=
   &\frac{\partial\mathbf{C}}{\partial\chi}\mathbf{R}\mathbf{B}+
\mathbf{C}\mathbf{R}\frac{\partial\mathbf{A}}{\partial\chi}\mathbf{R}\mathbf{B}+
\mathbf{C}\mathbf{R}\frac{\partial\mathbf{B}}{\partial\chi}+
     \frac{\partial\mathbf{D}}{\partial\chi}\\
\frac{\partial{}^{2}\mathbf{F}}{\partial\chi\partial\omega}=&
-\imath{}e^{\imath\omega}\left[
\frac{\partial\mathbf{C}}{\partial\chi}\mathbf{R}\mathbf{R}\mathbf{B}+
\mathbf{C}\mathbf{R}\mathbf{R}\frac{\partial\mathbf{A}}{\partial\chi}\mathbf{R}\mathbf{B}+
\mathbf{C}\mathbf{R}\frac{\partial\mathbf{A}}{\partial\chi}\mathbf{R}\mathbf{R}\mathbf{B}+
\mathbf{C}\mathbf{R}\mathbf{R}\frac{\partial\mathbf{B}}{\partial\chi}\right]
\end{align*}

The complex frequency response of the filter is:
\begin{align*}
\mathbf{F} &=\Re\mathbf{F} + \imath\Im\mathbf{F}
\end{align*}

The squared-magnitude response of the filter is:
\begin{align*}
\left|\mathbf{F}\right|^{2} &=\Im \mathbf{F}^{2} + \Re \mathbf{F}^{2}
\end{align*}

The gradients of the squared-magnitude response of the filter with respect to 
the Schur lattice filter coefficients are:
\begin{align*}
\frac{\partial\left|\mathbf{F}\right|^{2}}{\partial\chi}
&= 2\left[\Im \mathbf{F}\Im\frac{\partial\mathbf{F}}{\partial\chi}+
          \Re \mathbf{F}\Re\frac{\partial\mathbf{F}}{\partial\chi}\right]
\end{align*}

The gradient of the squared-magnitude response of the filter with respect to 
the angular frequency is:
\begin{align*}
\frac{\partial\left|\mathbf{F}\right|^{2}}{\partial\omega}
&= 2\left[\Im \mathbf{F}\Im\frac{\partial{}\mathbf{F}}{\partial\omega}+
          \Re \mathbf{F}\Re\frac{\partial{}\mathbf{F}}{\partial\omega}\right]
\end{align*}
so that:
\begin{align*}
  \frac{\partial^{2}  \left|\mathbf{F}\right|^{2}}
  {\partial\omega\partial\chi}
  &= 2\left[ \Im\frac{\partial{}\mathbf{F}}{\partial\chi}
             \Im\frac{\partial{}\mathbf{F}}{\partial\omega}+
    \Im \mathbf{F}\Im\frac{\partial^{2}\mathbf{F}}{\partial\omega\partial\chi}+
    \Re\frac{\partial{}\mathbf{F}}{\partial\chi}
    \Re\frac{\partial{}\mathbf{F}}{\partial\omega}+
    \Re \mathbf{F}\Re\frac{\partial^{2}\mathbf{F}}{\partial\omega\partial\chi}
  \right]
\end{align*}

The phase response, $P$, of the filter is:
\begin{align*}%
P&=\arctan\frac{\Im \mathbf{F}}{\Re \mathbf{F}}
\end{align*}

The gradients of the phase response of the filter with respect to 
the Schur lattice filter coefficients are given by:
\begin{align*}
\left|\mathbf{F}\right|^{2}\frac{\partial{}P}{\partial\chi}&=
\Re \mathbf{F}\Im\frac{\partial{}\mathbf{F}}{\partial\chi}-
\Im \mathbf{F}\Re\frac{\partial{}\mathbf{F}}{\partial\chi}
\end{align*}

The group delay response, $T$, of the filter is:
\begin{align*}
T &= -\frac{\partial P}{\partial\omega}
\end{align*}
so that:
\begin{align*}
\left|\mathbf{F}\right|^{2}T &= -\left[\Re \mathbf{F}\Im\frac{\partial \mathbf{F}}{\partial\omega} -
                              \Im \mathbf{F}\Re\frac{\partial \mathbf{F}}{\partial\omega}\right]
\end{align*}

The gradients of the group delay response with respect to the Schur lattice
filter coefficients are given by:
\begin{align*}
  \frac{\partial\left|\mathbf{F}\right|^{2}}{\partial\chi}T+
  \left|\mathbf{F}\right|^{2}\frac{\partial{}T}{\partial\chi} = 
&-\left[\Re\frac{\partial{}\mathbf{F}}{\partial\chi}
\Im\frac{\partial{}\mathbf{F}}{\partial\omega} +
\Re \mathbf{F} \Im\frac{\partial^{2}\mathbf{F}}{\partial\chi\partial\omega} -
  \Im\frac{\partial{}\mathbf{F}}{\partial\chi}
  \Re\frac{\partial{}\mathbf{F}}{\partial\omega} -
  \Im \mathbf{F}\Re\frac{\partial^{2}\mathbf{F}}
                        {\partial\chi\partial\omega}\right]
\end{align*}

\section{Computer-Aided-Design of Schur lattice filters}
The filter design procedure commences with the specification of the required
filter amplitude, phase and group delay responses. The optimisation of an IIR
frequency response with respect to the coefficients is not a convex problem. I
hope to find a ``good enough'' design rather than a global optimum.

One formulation of the filter optimisation problem is to minimise the
\emph{weighted squared error} of the frequency response of $F\left(z\right)$:
\begin{align}
\begin{split}
  \textbf{minimise}\quad &\mathcal{E}_{F}\left(\chi\right) =
  \int W\left(\omega\right)\left|F\left(\omega\right)
    -F_{d}\left(\omega\right)\right|^{2}d\omega \\
\textbf{subject to}\quad &F\left(\omega\right)\text{ is stable} 
\end{split}
\label{eqn:Definition-of-squared-response-error}
\end{align}
where $\chi$ is the coefficient vector of the filter, $F\left(z\right)$,
$\mathcal{E}_{F}\left(\chi\right)$ is the weighted sum of the squared error,
$F\left(\omega\right)$ is the filter frequency response,
$W\left(\omega\right)$ is the frequency weighting and
$F_{d}\left(\omega\right)$ is the desired filter complex frequency response.
The integrand of Equation~\ref{eqn:Definition-of-squared-response-error} could
be the weighted sum of one or more of the squared-magnitude, phase or group
delay responses.

The filter response optimisation problem can also be expressed as a
\emph{mini-max} optimisation problem:
\begin{align}
\label{eqn:Definition-of-minimax-response-error}
\begin{split}
\textbf{minimise}\quad &\max\;\left|F\left(\omega\right)
  -F_{d}\left(\omega\right)\right|\\
\textbf{subject to}\quad &F\left(\omega\right)\text{ is stable} 
\end{split}
\end{align}

I begin by finding an initial filter design that approximates the desired
response, then optimising the minimum mean-squared-error of the
calculated response and, if desired, finding a mini-max approximation to
the desired response. Finally, I describe two methods of
searching for integer valued filter coefficients: a depth-first branch-and-bound
search and a successive relaxation method.

\subsection{Finding an initial filter\label{sec:Finding-initial-filter}}
The initial filter could be an FIR filter or a ``classical'' Butterworth,
Chebyshev, etc. IIR filter design.
\emph{Deczky}~\cite{Deczky_MinPSynthesisIIRDigitalFilters} and
\emph{Richards}~\cite{Richards_DeczkyRecursiveDecimator} begin with an
initial filter consisting of a ``by-eye'' arrangement of the filter poles and
zeros. \emph{Tarczynski et
  al.}~\cite{TarczynskiCainHermanowiczRojewski_WISEMethodDesignIIRFilters}
propose minimising the filter response mean-squared-error, $\mathcal{E}_{F}$,
weighted with a barrier function:
\begin{align}
\label{eqn:WISE-barrier-function}
\left(1-\lambda\right)\mathcal{E}_{F}+\lambda\sum^{T+M}_{t=T+1}h^{2}\left(t\right)
\end{align}
where $\lambda$, $T$ and $M$ are suitable constants and $h\left(t\right)$ is the
impulse response of the filter $H\left(z\right)=\frac{1}{A_{N}\left(z\right)}$.
The barrier function (the second part of
Equation ~\ref{eqn:WISE-barrier-function}) is intended to be small when the
filter, $F\left(z\right)$, is stable and increase rapidly when the poles
approach the unit circle. \emph{Tarczynski et al.} provide heuristics for
selecting $\lambda$, $T$ and $M$. Typically,
$\lambda \in \left[10^{-10},10^{-3}\right]$, $T\in \left[ 100, 500 \right]$ and
$M=NR$. \emph{Roberts and Mullis}~\cite[Section 8.3]
{RobertsMullis_DigitalSignalProcessing} show that the state space description of
the direct-form implementation of $H\left(z\right)$ is:
\begin{align*}
\left[\begin{array}{c}
x\left(t+1\right)\\
y\left(t\right)
\end{array}\right] &= \left[\begin{array}{cc}
A & B\\
C & D
\end{array}\right]\left[\begin{array}{c}
x\left(t\right)\\
u\left(t\right)
\end{array}\right]
\end{align*}
where:
\begin{align*}
A &= \left[\begin{array}{cccc}
0 & 1 & \cdots & 0\\
\vdots & \vdots & \ddots & \vdots\\
0 & 0 & \cdots & 1\\
-a_{N} & -a_{N-1} & \cdots & -a_{1}
\end{array}\right]\\
B &= \left[\begin{array}{ccccc}0 & 0 & \cdots & 0 & 1\end{array}\right]^{\top}\\
C &= \left[\begin{array}{ccc}-a_{N} & \cdots & -a_{1}\end{array}\right]\\
D &= 1
\end{align*}

The corresponding impulse response
is~\cite[Equation 8.3.21]{RobertsMullis_DigitalSignalProcessing}:
\begin{align*}
h\left(t\right) &= \begin{cases}
0 & t<0\\
D & t=0\\
CA^{t-1}B & t>0
\end{cases}
\end{align*}

\subsection{Minimum-Mean-Squared-Error optimisation of IIR filters}
Minimisation of the mean squared error of the filter response proceeds by
successive solution of the following linear approximation problem
for $\Delta_{\chi}$:
\begin{align}
\begin{split}
  \textbf{minimise}\quad & \epsilon+\beta \\
  \textbf{subject to}\quad & \mathnorm{\mathcal{E}_{F}\left(\chi\right) +
          \Delta_{\chi}\nabla\mathcal{E}_{F}\left(\chi\right)}\le\epsilon \\
  & \mathnorm{\Delta_{\chi}}\le\beta \\
  & F\left(\omega\right)\text{ is stable}
\end{split}
\label{eqn:Optimisation-of-minimum-mean-squared-response-error}
\end{align}
After each step the new coefficient vector is $\chi+\Delta_{\chi}$. 

In fact I replace $\mathcal{E}_{F}$ with an approximation to the sum at discrete
frequencies of the weighted squared error of the real valued squared magnitude,
phase, group delay and the gradient of the squared magnitude with respect to
angular frequency. For example, the first order approximation to the squared
error of the group delay, $T$, is:
\begin{align*}
  \bar{\mathcal{E}}_{T}\left(\chi+\Delta_{\chi}\right)
  &\approx \sum_{\omega} W_{T}\left(\omega\right)
     \left[T\left(\chi,\omega\right) + \right.
    \left. \Delta_{\chi}\nabla_{\chi}T\left(\chi,\omega\right) - \right.
    \left. T_{d}\left(\omega\right)\right]^{2}
\end{align*}
with gradient with respect to $\chi$:
\begin{align*}
  \nabla_{\chi} \bar{\mathcal{E}}_{T}\left(\chi\right)
  &\approx 2 \sum_{\omega} W_{T}\left(\omega\right)
    \left[T\left(\chi,\omega\right)-T_{d}\left(\omega\right)\right]
    \nabla_{\chi}T\left(\chi,\omega\right)
\end{align*}
The required gradients of the state variable filter frequency response were
derived in Section~\ref{sec:Frequency-response-of-tapped-Schur-one-multiplier}.

\subsection{Peak-Constrained-Least-Squares optimisation of IIR filters}
Linear phase FIR filter design
algorithms~\cite{Hofstetter_DesignNonRecursiveDigitalFilters,
  ParksMcClellan_ChebyshevApproxNonRecursiveDigitalFilters,
  SelesnickBurrus_ExchangeAlgorithmsLinearPhaseFIRFilters,
  SelesnickLangBurrus_ConstrainedLeastSquareMultiBandFIRFilters} commonly
employ an ``exchange algorithm'' that, given an initial approximation to the
desired amplitude response, finds the extremal frequencies of that response and
interpolates the coefficients to obtain a new set of filter coefficients that
achieves the desired ripple values at all of those frequencies. The process
repeats until a satisfactory mini-max amplitude response is found. I have
successfully solved the IIR mini-max problem of
Equation~\ref{eqn:Definition-of-minimax-response-error} with a modified version
of the ``Peak-Constrained-Least-Squares'' (PCLS) algorithm of \emph{Selesnick},
\emph{Lang} and \emph{Burrus}~\cite[p.498]
{SelesnickLangBurrus_ConstrainedLeastSquareMultiBandFIRFilters}. They describe
the exchange of amplitude response, $A\left(\omega\right)$, constraints as
follows:
\begin{quotation}
  After each iteration, the algorithm checks the values of
  $A\left(\omega\right)$ over the previous constraint set frequencies. $\hdots$
  However, if it is found that $A\left(\omega\right)$ violates the constraints
  at some frequency belonging to the previous constraint set, $R$, then i) that
  frequency where the violation is greatest is appended to the current
  constraint set, $S$, and ii) the same frequency is removed from the
  record of previous constraint set frequencies, $R$.
\end{quotation}

Figure~\ref{fig:Modified-Selesnick-Lang-Burrus-algorithm} shows a flow-chart of
the modified \emph{Selesnick}, \emph{Lang} and \emph{Burrus} exchange algorithm.

\begin{figure}
\centering
\includegraphics[scale=\DesignOfSchurLatticeFiltersIncludeScale]{slb_exchange_algorithm}
\caption{Modified Selesnick, Lang and Burrus exchange algorithm~\cite[p.498]
  {SelesnickLangBurrus_ConstrainedLeastSquareMultiBandFIRFilters}.}
\label{fig:Modified-Selesnick-Lang-Burrus-algorithm}
\end{figure}

Similarly to $\bar{\mathcal{E}}_{T}$, the PCLS constraints for the group delay
response, $T\left(\chi,\omega\right)$, are approximated to first order by:
\begin{align*}
  T_{l}\left(\omega\right) \le
  T\left(\chi,\omega\right) + \Delta_{\chi}\nabla_{\chi}T\left(\chi,\omega\right)
  \le T_{u}\left(\omega\right)
\end{align*}
where $T_{l}\left(\omega\right)$ and $T_{u}\left(\omega\right)$ are the
lower and upper constraints on the group delay.

The modified PCLS mini-max optimisation is:
\begin{align}
\begin{split}
  \textbf{minimise}\quad & \epsilon+\beta \\
  \textbf{subject to}\quad & \mathnorm{\bar{\mathcal{E}}_{F}\left(\chi\right) +
          \Delta_{\chi}\nabla\bar{\mathcal{E}}_{F}\left(\chi\right)}\le\epsilon \\
  & \mathnorm{\Delta_{\chi}}\le\beta \\
  & F\left(\omega\right)\text{ is stable} \\
  & \text{PCLS response constraints are satisfied}
\end{split}
\label{eqn:Optimisation-of-PCLS-response-error}
\end{align}

\subsection{Second Order Cone Programming}
 \emph{Alizadeh} and \emph{Goldfarb}~\cite
{AlizadehGoldfarb_SecondOrderCone} describe Second Order Cone
Programming (SOCP) as the solution of a class of convex optimisation problems in
which a linear function is minimised subject to a set of conic constraints.
In the following example I use the \emph{SeDuMi}
(\emph{Se}lf-\emph{Du}al-\emph{Mi}nimisation) SOCP solver originally written
by \emph{Jos Sturm}~\cite{Sturm_SeDuMi_GitHub}. 
\emph{Lu}~\cite[Section III]{Lu_SedumiRemarks} provides an example of expressing
an optimisation problem of
Equation~\ref{eqn:Optimisation-of-PCLS-response-error} in the form accepted by
SeDuMi. In \emph{Lu}'s notation the problem is:
\begin{subequations}
\begin{align}
\textbf{minimise}  \quad&\boldsymbol{b}^{\top}\boldsymbol{x}
\label{eqn:Lu-SeDuMi-Remarks-problem-format}\\
\textbf{subject to}\quad&\|\boldsymbol{A}_{i}^{\top}\boldsymbol{x}+
\boldsymbol{c}_{i}\| \le \boldsymbol{b}_{i}^{\top}\boldsymbol{x}+d_{i}
\quad \text{for}\;i=1,\hdots,q
\label{eqn:Lu-SeDuMi-Remarks-problem-format-conic-constraints}\\
&\boldsymbol{D}^{\top}\boldsymbol{x}+\boldsymbol{f}\ge\boldsymbol{0}
\label{eqn:Lu-SeDuMi-Remarks-problem-format-linear-constraints}
\end{align}
\end{subequations}
where $\boldsymbol{x}\in\mathfrak{\mathbb{R}}^{m\times 1}$, 
$\boldsymbol{b}\in\mathfrak{\mathbb{R}}^{m\times 1}$,
$\boldsymbol{A}_{i}\in\mathfrak{\mathbb{R}}^{m\times\left(n_{i}-1\right)}$
\footnote{$n_{i}-1$ to allow for the column $b_{i}$ in the matrix 
$\boldsymbol{A}_{t}^{\left(i\right)}$ and $d_{i}$ in the column vector 
$\boldsymbol{c}_{t}^{\left(i\right)}$}, 
$\boldsymbol{c}_{i}\in\mathfrak{\mathbb{R}}^{\left(n_{i}-1\right)\times 1}$,
$\boldsymbol{b}_{i}\in\mathfrak{\mathbb{R}}^{m\times 1}$, 
$d_{i}\in\mathfrak{\mathbb{R}}$ for $1 \le i \le q$,
$\boldsymbol{D}\in\mathfrak{\mathbb{R}}^{m\times p}$ and
$\boldsymbol{f}\in\mathfrak{\mathbb{R}}^{p\times 1}$. The problem is cast into
SeDuMi format by defining:
\begin{align*}
\boldsymbol{A}_{t}&=\left[-\boldsymbol{D} \quad \boldsymbol{A}_{t}^{\left(1\right)}
\hdots \quad \boldsymbol{A}_{t}^{\left(q\right)}\right]\\
\boldsymbol{A}_{t}^{\left(i\right)} &= -\left[\boldsymbol{b}_{i} \quad
\boldsymbol{A}_{i}\right]\nonumber\\
\boldsymbol{b}_{t}&=-\boldsymbol{b}\nonumber\\
\boldsymbol{c}_{t}&=\left[\boldsymbol{f};\quad \boldsymbol{c}_{t}^{\left(1\right)}
;\quad \hdots\quad \boldsymbol{c}_{t}^{\left(q\right)}\right]\nonumber\\
\boldsymbol{c}_{t}^{\left(i\right)}&=\left[d_{i} ;\quad \boldsymbol{c}_{i}\right]
\nonumber
\end{align*}

For Equation~\ref{eqn:Optimisation-of-PCLS-response-error},
$\boldsymbol{x}=\left[\epsilon,\beta,\chi\right]$,
Equation~\ref{eqn:Lu-SeDuMi-Remarks-problem-format-conic-constraints} applies
the conic constraints on the linear approximation to the minimum mean squared
error of the frequency response and 
Equation~\ref{eqn:Lu-SeDuMi-Remarks-problem-format-linear-constraints} applies
the linear constraints on both the reflection coefficients,
$\mathabs{k_{n}}<1$, and the PCLS frequency response constraints.

\section{\label{sec:Design-IIR-lattice-filters-with-signed-digit-coefficients}Design of digital filters with signed-digit coefficients}
\subsection{The canonical signed-digit representation of filter coefficients}

\emph{Parhi}~\cite[Section 13.6]{Parhi_VLSIDigitalSignalProcessingSystems} lists
the following properties of the \emph{canonical} binary signed-digit (CSD)
representation of a binary signed-digit number $A=a_{W-1}a_{W-2}\cdots{}a_{1}a_{0}$
where each $a_{i}\in \left\{-1,0,1\right\}$:
\begin{itemize}
\item no 2 consecutive bits in a CSD number are non-zero 
\item the CSD representation of a number contains the miniumum possible number
of non-zero bits, thus the name \emph{canonic}
\item the CSD representation of a number is unique
\end{itemize}
The first property breaks the carry chain when adding two CSD numbers.
\emph{Parhi}~\cite[Section 13.6.1]{Parhi_VLSIDigitalSignalProcessingSystems}
shows an algorithm that calculates the \emph{canonical} binary signed-digit 
representation from the two's complement representation, reproduced here as
Algorithm~\ref{alg:twos-complement-to-signed-digit-conversion}.

\begin{algorithm}[htbp]
Denote the two's complement representation of the number $A$ as 
$A=\hat{a}_{W-1}\hat{a}_{W-2}\cdots\hat{a}_{1}\hat{a}_{0}$. \\
Denote the CSD representation of $A$ as $A=a_{W-1}a_{W-2}\cdots{}a_{1}a_{0}$.

\begin{algorithmic}
\State $\hat{a}_{-1}=0$
\State $\gamma_{-1}=0$
\State $\hat{a}_{W}=\hat{a}_{W-1}$
\For{$k=0,\hdots,W-1$}
\State $\theta_{i}=\hat{a}_{i}\oplus\hat{a}_{i-1}$
\State $\gamma_{i}=\bar{\gamma}_{i-1}\theta_{i}$
\State $a_{i}=\left(1-2\hat{a}_{i+1}\right)\gamma_{i}$
\EndFor
\end{algorithmic}
\caption{Conversion of 2's complement numbers to the canonical signed-digit
  representation (\emph{Parhi}~\cite[Section
  13.6.1]{Parhi_VLSIDigitalSignalProcessingSystems}).}
\label{alg:twos-complement-to-signed-digit-conversion}
\end{algorithm}

\emph{Lim et al.}~\cite{Lim_SignedPowerOfTwoAllocationDigitalFilters} describe
a method of allocating  a limited number of signed power-of-two digit terms 
to the fixed point coefficients of a digital filter. The method is based 
on the belief that ``\emph{allocating the SPT terms in such a way that all the 
coefficient values have the same quantisation step-size to coefficient
sensitivity ratio will lead to a good design}''. 

\emph{Lim et al.} first prove properties of the canonical signed-digit
representation~\cite[Section II]{Lim_SignedPowerOfTwoAllocationDigitalFilters}.
In particular:

\emph{Property 1}: Define $S_{Q}$ as the set of contiguous integers that can be 
represented by up to $Q$ signed digits. The largest integer in $S_{Q}$ is
$J_{Q}=\sum_{l=0}^{Q-1}2^{2l+1}$.

\emph{Property 5}: On average, $0.72Q$ signed-digits are required to represent
the integers in $S_{Q}$.

\emph{Lim et al.} show that an estimate of the number of signed digits, $Q$, 
required to represent $J_{Q}$ is:
\begin{align*}
  Q \approx \frac{1}{2}\log_{2}J_{Q}+0.31
\end{align*}
Replacing $J_{Q}$ by an integer $n\in S_{Q}$, an estimate of the average number
of terms, $Q_{A}$, required to represent $n$ is:
\begin{align*}
  Q_{A}&\approx 0.72Q\\
      &\approx 0.36 \log_{2}n+0.22
\end{align*}

Now suppose that $D$ signed digits are available to represent two positive
integers $n_{1}$ and $n_{2}$. If $n_{1}\approx n_{2}$ then each integer is
allocated $\frac{D}{2}$ bits. If $n_{1}>n_{2}$ then \emph{Lim et al.} argue 
that the number of additional signed-digits, $Q_{E}$, required to represent 
$n_{1}$ is:
\begin{align*}
  Q_{E}&\approx 0.36\log_{2}\lfloor \frac{n_{1}}{n_{2}}\rfloor
\end{align*}
where $\lfloor\;\rfloor$ represents the integer part. In general, 
$Q_{E}$ is not an integer.

\emph{Lim et al.} go on to consider the allocation of signed digits to the
coefficients of a symmetric FIR filter. The change in the frequency response,
$F\left(\omega\right)$, of a filter due to a change $\Delta\chi_{n}$ in
coefficient $\chi_{n}$ is:
\begin{align*}
\Delta{} F\left(\omega,n\right)&  \approx
\frac{\partial{}F\left(\omega\right)}{\partial\chi_{n}}\Delta{}\chi_{n}
\end{align*}
\emph{Lim et al.} use the average of the coefficient sensitivity to define a
cost for the $n$'th coefficient:
\begin{align*}
cost_{n}&=0.36\log_{2}\left|\chi_{n}\right| + 0.36\log_{2}\int^{\pi}_{0}
\left|\frac{\partial{}F\left(\omega\right)}{\partial{}\chi_{}}\right|d\omega
\end{align*}

Given a total of $D$ signed-digits, \emph{Lim et al.} assign a single 
signed-digit at a time to the coefficient with the largest cost. After a 
coefficient is given a signed-digit, its cost is decreased by one. The
process is repeated until all $D$ digits have been allocated.

\subsection{Branch-and-bound search for the filter coefficients}
Given the $K$ floating point coefficients, $\chi$, of a filter, an exhaustive
search of the upper and lower bounds on the integer or signed-digit
approximations to these coefficients would require
$\mathcal{O}\left(2^{K}\right)$ comparisons of the corresponding filter
approximation error.
\emph{Branch-and-bound}~\cite{LandDoig_AutomaticMethodSolvingDiscreteProgrammingProblems},~\cite[p.627]{Sedgewick_AlgorithmsInCPlusPlus}
is a heuristic for reducing the 
number of branches searched in a binary decision tree. At each branch of the
binary tree the solution is compared to the estimated lower bounds on the cost
of the full path proceeding from that branch. If the cost of that full path is
greater than that of the best full path found so far then further search on
that path is abandoned.  Figure~\ref{fig:branch-bound-tree-search-algorithm}
shows a flow diagram of an implementation of the algorithm using a stack. The
floating point filter coefficients, $\chi$, are approximated by the
signed-digit coefficients, $\bar{\chi}$. Each coefficient, $\chi_{n}$
and $\bar{\chi}_{n}$, is bounded by the corresponding signed-digit numbers $u_{n}$
and $l_{n}$ so that $l_{n}\le\chi_{n}\le{}u_{n}$ and
$l_{n}\le\bar{\chi}_{n}\le u_{n}$. The search for the set of coefficients with
minimum cost is ``depth-first'', starting at the root of the search tree and
fixing successive coefficients.  \emph{Ito et
  al.}~\cite{Ito_PowersOfTwoAllocationFIR} recommend choosing, at each branch,
the $\chi_{n}$ with the greatest difference $u_{n}-l_{n}$. The two sub-problems
at that branch fix $\bar{\chi}_{n}$ to $l_{n}$ and $u_{n}$. One of the two
sub-problems is pushed onto a stack and the other is solved and the cost
calculated. I assume that this cost is the least possible for the remaining
coefficients on the current branch. If the cost of the current sub-problem is
greater than the current minimum cost then the current branch is abandoned and a
new sub-problem is popped off the problem stack. Otherwise, if the search has
reached the maximum depth of the tree then the current solution is a new
signed-digit minimum cost solution. If the current cost is less than the
minimum cost and the search has not reached the maximum depth then the search
continues with a new branch.

\begin{figure}
\centering
\includegraphics[scale=\DesignOfSchurLatticeFiltersIncludeScale]{branch_bound_algorithm}
\caption{Branch-and-bound depth-first search algorithm}
\label{fig:branch-bound-tree-search-algorithm}
\end{figure}

\subsection{Successive relaxation search for the filter coefficients}
A successive relaxation search for the filter coefficients fixes one
coefficient and optimises the filter frequency response with respect to the
remaining free coefficients. This is called a \emph{relaxation} of the
optimisation. The relaxation is repeated until all the coefficients are fixed.
At each coefficient relaxation step the script finds the upper and lower
signed-digit approximations to the current set of active coefficients and
selects the coefficient with the largest difference in those approximations.

\section{An example : design of a low pass differentiator filter}
This section describes the design of a low pass differentiator filter
implemented as $F_{0}\left(z\right)=1-z^{-1}$ in series with a tapped Schur
one-multiplier lattice correction filter, $C\left(z\right)$, having denominator
polynomial coefficients only for terms in $z^{-2n}$. In the pass band, the
desired complex frequency response of a differentiator filter is:
\begin{align*}
  F_{d}\left(\omega\right)=-\imath\frac{\omega}{2}e ^{-\imath\omega t_{p}}
\end{align*}
where $t_{p}$ is the nominal filter pass band group delay in samples.
The squared-magnitude response of the correction filter is:
\begin{align*}
  \mathabs{C\left(\omega\right)}^{2}
  &= \frac{\mathabs{F_{d}\left(\omega\right)}^{2}}
    {\mathabs{F_{0}\left(\omega\right)}^{2}}
\end{align*}
where:
\begin{align*}
  \mathabs{F_{d}\left(\omega\right)} &= \frac{\omega}{2} \\
  \mathabs{F_{d}\left(\omega\right)}^{2}&= \frac{\omega^{2}}{4} \\  
  \frac{d\mathabs{F_{d}\left(\omega\right)}^{2}}{d\omega}
     &= \frac{\omega}{2} = \mathabs{F_{d}\left(\omega\right)}
\end{align*}
and the response of the zero at $z=1$ is:
\begin{align*}
  F_{0}\left(\omega\right) &= 1-e^{\imath\omega} \\
        &=2\imath{} e^{\imath\frac{\omega}{2}}\sin\frac{\omega}{2} \\
  \mathabs{F_{0}\left(\omega\right)} &= 2\sin\frac{\omega}{2} \\
  \frac{d\mathabs{F_{0}\left(\omega\right)}^{2}}{d\omega}
        &=2\mathabs{F_{0}\left(\omega\right)}
          \frac{d\mathabs{F_{0}\left(\omega\right)}}{d\omega} \\
        &= 2\sin\omega
\end{align*}
By the chain rule for differentiation, the gradient of
$\mathabs{F\left(\omega\right)}^{2}$ is:
\begin{align*}
  \frac{d\mathabs{F_{d}\left(\omega\right)}^{2}}{d\omega}
  &= \mathabs{F_{0}\left(\omega\right)}^{2}
     \frac{d\mathabs{C\left(\omega\right)}^{2}}{d\omega} +
    \frac{\mathabs{F_{d}\left(\omega\right)}^{2}}
    {\mathabs{F_{0}\left(\omega\right)}^{2}}
    \frac{d\mathabs{F_{0}\left(\omega\right)}^{2}}{d\omega}
\end{align*}
Substituting and rearranging to obtain the desired gradient of
$\mathabs{C\left(\omega\right)}^{2}$ in terms of the desired overall squared
amplitude response, $\mathabs{F_{d}\left(\omega\right)}^{2}$, and the
squared magnitude response of the zero at $z=1$,
$\mathabs{F_{0}\left(\omega\right)}^{2}$:
\begin{align*}
  \frac{d\mathabs{C\left(\omega\right)}^{2}}{d\omega}
  &= \mathabs{F_{d}\left(\omega\right)}
\left[\frac{1-\mathabs{F_{d}\left(\omega\right)}\cot\frac{\omega}{2}}{\mathabs{F_{0}\left(\omega\right)}^{2} }\right]
\end{align*}

\subsection{Initial R=2 low pass differentiator filter}
Figure~\ref{fig:Schur-OneM-lattice-correction-lowpass-differentiator-R2-initial}
shows the frequency response of the initial R=2 low pass differentiator filter
found by the method of \emph{Tarczynski et al.}, described in
Section~\ref{sec:Finding-initial-filter}. The correction filter order is $N=10$,
filter pass band and stop band edge frequencies are $0.2$ and $0.4$\;(normalised
to the digital sample rate), the nominal pass band phase is $1.5\pi$
radians (adjusted for delay) and the nominal pass band group delay is $9$
samples.

\begin{figure}
\centering
\scalebox{\DesignOfSchurLatticeFiltersPdfScale}{\input{schurOneMlattice_socp_slb_lowpass_differentiator_R2_test_initial_response}}
\caption{Amplitude, phase and group delay responses of an initial 
  low pass differentiator filter composed of a Schur one-multiplier lattice
  correction filter with a denominator polynomial having terms only in $z^{-2}$
  in series with $1-z^{-1}$.}
\label{fig:Schur-OneM-lattice-correction-lowpass-differentiator-R2-initial}
\end{figure}

The coefficients of the numerator and denominator polynomials of the initial
correction filter transfer function are:
\begin{small}
\verbatiminput{schurOneMlattice_socp_slb_lowpass_differentiator_R2_test_N0_coef.m}
\verbatiminput{schurOneMlattice_socp_slb_lowpass_differentiator_R2_test_D0R_coef.m}
\end{small}

The corresponding Schur one-multiplier lattice coefficients of the initial
correction filter are:
\begin{small}
\verbatiminput{schurOneMlattice_socp_slb_lowpass_differentiator_R2_test_k0_coef.m}
\verbatiminput{schurOneMlattice_socp_slb_lowpass_differentiator_R2_test_epsilon0_coef.m}
\verbatiminput{schurOneMlattice_socp_slb_lowpass_differentiator_R2_test_c0_coef.m}
\end{small}

\subsection{PCLS design of the R=2 low pass differentiator filter}
The low pass differentiator correction filter was PCLS optimised with 
amplitude pass band error peak-to-peak ripple of $0.0009$,
amplitude stop band peak ripple of $0.007$,
phase pass band peak-to-peak ripple of $0.0002\pi$\; radians,
group delay pass band peak-to-peak ripple of $0.006$\; samples and
correction filter gradient of amplitude-squared pass band error peak-to-peak
ripple of $0.02$.

The Schur one-multiplier lattice coefficients of the PCLS SOCP optimised
correction filter are:
\begin{small}
\verbatiminput{schurOneMlattice_socp_slb_lowpass_differentiator_R2_test_k2_coef.m}
\verbatiminput{schurOneMlattice_socp_slb_lowpass_differentiator_R2_test_epsilon2_coef.m}
\verbatiminput{schurOneMlattice_socp_slb_lowpass_differentiator_R2_test_c2_coef.m}
\end{small}

The corresponding correction filter numerator and denominator polynomial
coefficients are:
\begin{small}
\verbatiminput{schurOneMlattice_socp_slb_lowpass_differentiator_R2_test_N2_coef.m}
\verbatiminput{schurOneMlattice_socp_slb_lowpass_differentiator_R2_test_D2_coef.m}
\end{small}
Figure~\ref{fig:Schur-OneM-lattice-lowpass-differentiator-R2}
shows the  amplitude error, phase and group delay response errors of the $R=2$
low pass differentiator filter after PCLS SOCP optimisation.
Figure~\ref{fig:Schur-OneM-lattice-lowpass-differentiator-R2-dCsqdw-error}
shows the error in the gradient of the squared-magnitude response of the $R=2$
correction filter after PCLS SOCP optimisation.
Figure~\ref{fig:Schur-OneM-lattice-lowpass-differentiator-R2-pz} shows
the pole-zero plot of the $R=2$ low pass differentiator filter.
Figure~\ref{fig:Schur-OneM-lattice-correction-lowpass-differentiator-direct-sens}
shows the gradients of the correction filter squared-magnitude, phase and delay
with respect to the coefficients of the direct form implementation of the filter.
Figure~\ref{fig:Schur-OneM-lattice-correction-lowpass-differentiator-Schur-sens}
shows the gradients of the correction filter squared-magnitude, phase and delay
with respect to the coefficients of the Schur lattice implementation of the
filter.

\begin{figure}
\centering
\scalebox{\DesignOfSchurLatticeFiltersPdfScale}{\input{schurOneMlattice_socp_slb_lowpass_differentiator_R2_test_pcls_error}}
\caption{PCLS optimised amplitude error, phase and group delay responses of a
  low pass differentiator filter composed of a Schur one-multiplier lattice
  correction filter with a denominator polynomial having terms only in $z^{-2}$
  in series with $1-z^{-1}$.}
\label{fig:Schur-OneM-lattice-lowpass-differentiator-R2}
\end{figure}

\begin{figure}
\centering
\scalebox{\DesignOfSchurLatticeFiltersPdfScale}{\input{schurOneMlattice_socp_slb_lowpass_differentiator_R2_test_pcls_dCsqdw_error}}
\caption{PCLS optimised pass band error of the gradient of the squared magnitude
  response of the correction filter after PCLS SOCP optimisation.}
\label{fig:Schur-OneM-lattice-lowpass-differentiator-R2-dCsqdw-error}
\end{figure}

\begin{figure}
\centering
\scalebox{\DesignOfSchurLatticeFiltersPdfScale}{\input{schurOneMlattice_socp_slb_lowpass_differentiator_R2_test_pcls_pz}}
\caption{Pole-zero plot of the PCLS optimised low pass differentiator
  filter composed of a Schur one-multiplier lattice correction filter with a
  denominator polynomial having terms only in $z^{-2}$ in series with $1-z^{-1}$.}
\label{fig:Schur-OneM-lattice-lowpass-differentiator-R2-pz}
\end{figure}

\begin{figure}
\centering
\scalebox{\DesignOfSchurLatticeFiltersPdfScale}{\input{schurOneMlattice_socp_slb_lowpass_differentiator_R2_test_direct_sensitivity}}
\caption{Gradients of the pass band response with respect to the coefficients of
  the direct form implementation of the PCLS optimised low pass differentiator
  correction filter.}
\label{fig:Schur-OneM-lattice-correction-lowpass-differentiator-direct-sens}
\end{figure}

\begin{figure}
\centering
\scalebox{\DesignOfSchurLatticeFiltersPdfScale}{\input{schurOneMlattice_socp_slb_lowpass_differentiator_R2_test_schur_sensitivity}}
\caption{Gradients of the pass band response with respect to the coefficients of
  the tapped Schur lattice implementation of the PCLS optimised low pass
  differentiator correction filter.}
\label{fig:Schur-OneM-lattice-correction-lowpass-differentiator-Schur-sens}
\end{figure}

\subsection{Search for the signed-digit coefficients of the R=2 low pass differentiator filter}

When truncated to $12$ bits and $3$ signed-digits the Schur one-multiplier
lattice coefficients of the PCLS optimised correction filter are:
\begin{small}
\verbatiminput{socp_relaxation_schurOneMlattice_lowpass_differentiator_R2_12_nbits_test_k0_sd_no_alloc_coef.m}
\verbatiminput{socp_relaxation_schurOneMlattice_lowpass_differentiator_R2_12_nbits_test_c0_sd_no_alloc_coef.m}
\end{small}

The numbers of signed-digits that the heuristic of \emph{Lim et al.} allocates to
each cofficient of the PCLS optimised correction filter are:
\begin{small}
\verbatiminput{socp_relaxation_schurOneMlattice_lowpass_differentiator_R2_12_nbits_test_k_allocsd_digits.m}
\verbatiminput{socp_relaxation_schurOneMlattice_lowpass_differentiator_R2_12_nbits_test_c_allocsd_digits.m}
\end{small}

The signed-digit coefficients of the PCLS optimised filter with the
numbers of signed-digits allocated by the heuristic of \emph{Lim et al.} are:
\begin{small}
\verbatiminput{socp_relaxation_schurOneMlattice_lowpass_differentiator_R2_12_nbits_test_k0_sd_coef.m}
\verbatiminput{socp_relaxation_schurOneMlattice_lowpass_differentiator_R2_12_nbits_test_c0_sd_coef.m}
\end{small}

The signed-digit coefficients of the PCLS optimised filter with the
numbers of signed-digits allocated by the heuristic of \emph{Lim et al.} and
found with branch-and-bound search are:
\begin{small}
\verbatiminput{branch_bound_schurOneMlattice_lowpass_differentiator_R2_12_nbits_test_k_min_coef.m}
\verbatiminput{branch_bound_schurOneMlattice_lowpass_differentiator_R2_12_nbits_test_c_min_coef.m}
\end{small}
The corresponding correction filter transfer function polynomials found with
branch-and-bound search are:
\begin{small}
\verbatiminput{branch_bound_schurOneMlattice_lowpass_differentiator_R2_12_nbits_test_N_min_coef.m}
\verbatiminput{branch_bound_schurOneMlattice_lowpass_differentiator_R2_12_nbits_test_D_min_coef.m}
\end{small}

The signed-digit coefficients of the PCLS optimised filter with the
numbers of signed-digits allocated by the heuristic of \emph{Lim et al.} and
found with SOCP relaxation search are:
\begin{small}
\verbatiminput{socp_relaxation_schurOneMlattice_lowpass_differentiator_R2_12_nbits_test_k_min_coef.m}
\verbatiminput{socp_relaxation_schurOneMlattice_lowpass_differentiator_R2_12_nbits_test_c_min_coef.m}
\end{small}
The corresponding correction filter transfer function polynomials found with
SOCP relaxation search are:
\begin{small}
\verbatiminput{socp_relaxation_schurOneMlattice_lowpass_differentiator_R2_12_nbits_test_N_min_coef.m}
\verbatiminput{socp_relaxation_schurOneMlattice_lowpass_differentiator_R2_12_nbits_test_D_min_coef.m}
\end{small}

Table~\ref{tab:Comp-schurOneMlattice-lowpass-diff-R2-12-nbits-cost}
compares, for each filter, a cost function, the maximum pass band and stop band
response errors, the total number of signed-digits required by the $12$-bit 
coefficients and the number of shift-and-add operations required to implement
the correction filter coefficient multiplications of the correction filter.

\begin{table}
\centering
\begin{threeparttable}
\begin{tabular}{lrrrrrrrr} \\ \toprule
 &Correction&Max. pass&Max. pass &Max. stop &Max. pass   &Max. pass  &$12$-bit&Shift-\\
 &filter    &amplitude&amplitude &amplitude &phase error &delay error&signed  &and- \\
 &cost      &error    &rel. error&error     &(rad./$\pi$)&(samples)  &digits  &adds \\
\midrule
\input{comparison_no_POP_schurOneMlattice_lowpass_differentiator_R2_12_nbits_test_cost.tab}
\\ \bottomrule
\end{tabular}
\end{threeparttable}
\caption[Comparison of low pass differentiator filter fixed point implementations]
{Comparison of the response errors and the total number of signed digits in the
  12-bit coefficients, each having an average of 3-signed digits, allocated by
  the method of \emph{Lim et al.}, required for a tapped Schur one-multiplier
  lattice, R=2, low pass differentiator correction filter found by
  branch-and-bound and SOCP relaxation search.}
\label{tab:Comp-schurOneMlattice-lowpass-diff-R2-12-nbits-cost}
\end{table}

Figure~\ref{fig:Comp-schurOneMlattice-lowpass-diff-R2-12-nbits-pass}
compares the pass band amplitude response of each filter for
$12$-bit $3$-signed-digit coefficients and $12$-bit coefficients with an average
of $3$-signed-digits allocated by the method of \emph{Lim et al.} found with
branch-and-bound search and SOCP relaxation search.
Figure~\ref{fig:Comp-schurOneMlattice-lowpass-diff-R2-12-nbits-relative-pass}
compares the pass band relative amplitude response of each filter.
Figure~\ref{fig:Comp-schurOneMlattice-lowpass-diff-R2-12-nbits-stop}
compares the stop band amplitude response of each filter.
Figure~\ref{fig:Comp-schurOneMlattice-lowpass-diff-R2-12-nbits-phase} shows
the pass band phase reslonse of each filter.
~\ref{fig:Comp-schurOneMlattice-lowpass-diff-R2-12-nbits-delay} shows the
pass band group-delay response of each filter.
Figure~\ref{fig:Comp-schurOneMlattice-lowpass-diff-R2-12-nbits-dCsqdw} shows
the gradient of the squared magnitude response of the correction filter.

\begin{figure}
\centering
\scalebox{\DesignOfSchurLatticeFiltersPdfScale}{\input{comparison_no_POP_schurOneMlattice_lowpass_differentiator_R2_12_nbits_test_pass}}
\caption{Comparison of the pass band response error of
  a tapped Schur one-multiplier, R=2, lattice lowpass differentiator correction
  filter with floating point coefficients, 12-bit 3-signed-digit coefficients,
  and 12-bit signed-digit coefficients each having an average of 3-signed-digits
  allocated by the method of \emph{Lim et al.}, found by branch-and-bound and
  SOCP relaxation search.}
\label{fig:Comp-schurOneMlattice-lowpass-diff-R2-12-nbits-pass}
\end{figure}

\begin{figure}
\centering
\scalebox{\DesignOfSchurLatticeFiltersPdfScale}{\input{comparison_no_POP_schurOneMlattice_lowpass_differentiator_R2_12_nbits_test_relative_pass}}
\caption{Comparison of the pass band relative response error of
  a tapped Schur one-multiplier, R=2, lattice lowpass differentiator correction
  filter with floating point coefficients, 12-bit 3-signed-digit coefficients,
  and 12-bit signed-digit coefficients each having an average of 3-signed-digits
  allocated by the method of \emph{Lim et al.}, found by branch-and-bound and
  SOCP relaxation search.}
\label{fig:Comp-schurOneMlattice-lowpass-diff-R2-12-nbits-relative-pass}
\end{figure}

\begin{figure}
\centering
\scalebox{\DesignOfSchurLatticeFiltersPdfScale}{\input{comparison_no_POP_schurOneMlattice_lowpass_differentiator_R2_12_nbits_test_stop}}
\caption{Comparison of the stop band response error of
  a tapped Schur one-multiplier, R=2, lattice lowpass differentiator correction
  filter with floating point coefficients, 12-bit 3-signed-digit coefficients,
  and 12-bit signed-digit coefficients each having an average of 3-signed-digits
  allocated by the method of \emph{Lim et al.}, found by branch-and-bound and
  SOCP relaxation search.}
\label{fig:Comp-schurOneMlattice-lowpass-diff-R2-12-nbits-stop}
\end{figure}

\begin{figure}
\centering
\scalebox{\DesignOfSchurLatticeFiltersPdfScale}{\input{comparison_no_POP_schurOneMlattice_lowpass_differentiator_R2_12_nbits_test_phase}}
\caption{Comparison of the pass band phase response errors of a tapped Schur
  one-multiplier, R=2, lattice low pass differentiator correction filter with
  floating point coefficients, 12-bit 3-signed-digit coefficients, and 12-bit
  signed-digit coefficients each having an average of 3-signed-digits allocated
  by the method of \emph{Lim et al.}, found by branch-and-bound and
  SOCP relaxation search.}
\label{fig:Comp-schurOneMlattice-lowpass-diff-R2-12-nbits-phase}
\end{figure}

\begin{figure}
\centering
\scalebox{\DesignOfSchurLatticeFiltersPdfScale}{\input{comparison_no_POP_schurOneMlattice_lowpass_differentiator_R2_12_nbits_test_delay}}
\caption{Comparison of the pass band group delay response errors of a tapped
  Schur one-multiplier, R=2, lattice low pass differentiator correction filter
  with floating point coefficients, 12-bit 3-signed-digit coefficients, and
  12-bit signed-digit coefficients each having an average of 3-signed-digits
  allocated by the method of \emph{Lim et al.}, found by branch-and-bound and
  SOCP relaxation search.}
\label{fig:Comp-schurOneMlattice-lowpass-diff-R2-12-nbits-delay}
\end{figure}

\begin{figure}
\centering
\scalebox{\DesignOfSchurLatticeFiltersPdfScale}{\input{comparison_no_POP_schurOneMlattice_lowpass_differentiator_R2_12_nbits_test_dCsqdw}}
\caption{Comparison of the error in the pass band gradient of the squared
  magnitude response of a tapped Schur one-multiplier, R=2, lattice low pass
  differentiator correction filter with floating point coefficients, 12-bit
  3-signed-digit coefficients, and 12-bit signed-digit coefficients each having
  an average of 3-signed-digits allocated by the method of \emph{Lim et al.},
  found by branch-and-bound and SOCP relaxation search.}
\label{fig:Comp-schurOneMlattice-lowpass-diff-R2-12-nbits-dCsqdw}
\end{figure}

\subsection{Comparison with a maximally-linear  FIR low pass differentiator filter}

\emph{Selesnick}~\cite{Selesnick_MaximallyFlatLowPassDigitalDifferentiators}
describes the design of maximally linear low pass FIR differentiator filters:
\begin{align*}
  F\left(z\right)&=\left(\frac{1-z^{-1}}{2}\right)
                   \left(\frac{1+z^{-1}}{2}\right)^{K}
                   z^{-L}\sum_{n=0}^{L}c_{n}\left[\frac{-z+2-z^{-1}}{4}\right]^{n}
\end{align*}
where the $c_{n}$ are defined recursively:
\begin{align*}
c_{0}&=2\\
c_{1}&=K+\frac{1}{3} \\
    & \hdots \\
c_{n}&=\frac{\left(8n^{2}+4Kn-10n-K+3\right)c_{n-1}-\left(2n+K-3\right)^{2}c_{n-2}}
       {2n\left(2n+1\right)}
\end{align*}
The length of the impulse response is $N=K+2L+2$. The coefficients of the maximally linear anti-symmetric FIR
low pass differentiator with $L=20$ and $K=21$ are\footnote{For odd length,
  $N=63$, $h0\left(32\right)=0$.}:
\begin{small}
\verbatiminput{socp_relaxation_directFIRantisymmetric_lowpass_differentiator_12_nbits_test_h0_coef.m}
\end{small}
By construction, the phase response of the anti-symmetric FIR low pass
differentiator filter is $\frac{\pi}{2}$ radians. This FIR filter was truncated
to $16$ distinct coefficients, the same number as the Schur lattice, and a group
delay of $16$ samples. After SOCP-relaxation optimisation to $12$-bits with an
average of $3$ signed-digits allocated by the heuristic of \emph{Lim et al.}
the distinct coefficients of the filter are:
\begin{small}
\verbatiminput{socp_relaxation_directFIRantisymmetric_lowpass_differentiator_12_nbits_test_hM_min_coef.m}
\end{small}

Table~\ref{tab:Comp-schurOneMlattice-lowpass-diff-R2-12-nbits-hM-min-cost}
compares a cost function, the maximum pass band and stop band response errors,
the total number of signed-digits required by the $12$-bit coefficients and the
number of shift-and-add operations required to implement the coefficient
multiplications of the floating point FIR, floating point truncated FIR and the
SOCP-relaxation optimised truncated FIR and the the SOCP-relaxation optimised
truncated FIR and the branch-and-bound optimised truncated FIR filter.

\begin{table}
\centering
\begin{threeparttable}
\begin{tabular}{lrrrrrr} \\ \toprule
 &Low pass        &Max. pass&Max. pass &Max. stop &$12$-bit&Shift-\\
 &differentiator  &amplitude&amplitude &amplitude &signed  &and- \\
 &cost            &error    &rel. error&error     &digits  &adds \\
\midrule
\input{comparison_no_POP_schurOneMlattice_lowpass_differentiator_R2_12_nbits_test_hM_min_cost.tab}
\\ \bottomrule
\end{tabular}
\end{threeparttable}
\caption[Comparison of low pass differentiator filter FIR implementations]
{Comparison of the response errors and the total number of signed digits in the
  12-bit coefficients, each having an average of 3-signed digits, allocated by
  the method of \emph{Lim et al.}, required for a low pass differentiator filter
  implemented as an anti-symmetric FIR filter found by SOCP relaxation search
  and branch-and-bound search.}
\label{tab:Comp-schurOneMlattice-lowpass-diff-R2-12-nbits-hM-min-cost}
\end{table}

Figure~\ref{fig:Comp-schurOneMlattice-lowpass-diff-R2-12-nbits-hM-min-pass}
compares the pass band error amplitude responses of a low pass
differentiator filter implemented as a maximally linear FIR filter with floating
point coefficients, that FIR filter truncated to $16$ distinct coefficients and
the truncated filter coefficients SOCP-relaxation optimised for $12$-bit
$3$-signed-digit coefficients and $12$-bit coefficients with an average of
$3$-signed-digits allocated by the method of \emph{Lim et al.}.
Similarly,
Figure~\ref{fig:Comp-schurOneMlattice-lowpass-diff-R2-12-nbits-hM-min-pass-relative}
compares the pass band relative error amplitude responses and
Figure~\ref{fig:Comp-schurOneMlattice-lowpass-diff-R2-12-nbits-hM-min-stop}
compares the stop band error amplitude responses.

\begin{figure}
\centering
\scalebox{\DesignOfSchurLatticeFiltersPdfScale}{\input{comparison_no_POP_schurOneMlattice_lowpass_differentiator_R2_12_nbits_test_hM_min_pass_error}}
\caption{Comparison of the pass band error amplitude responses of a low pass
  differentiator filter implemented as a maximally linear FIR filter with
  floating point coefficients, that FIR filter truncated to $16$ distinct
  coefficients and the truncated filter coefficients SOCP-relaxation optimised
  for $12$-bit $3$-signed-digit coefficients and $12$-bit coefficients with an
  average of $3$-signed-digits allocated by the method of \emph{Lim et al.}.}
\label{fig:Comp-schurOneMlattice-lowpass-diff-R2-12-nbits-hM-min-pass}
\end{figure}

\begin{figure}
\centering
\scalebox{\DesignOfSchurLatticeFiltersPdfScale}{\input{comparison_no_POP_schurOneMlattice_lowpass_differentiator_R2_12_nbits_test_hM_min_pass_relative_error}}
\caption{Comparison of the pass band relative error responses of a low pass
  differentiator filter implemented as a maximally linear FIR filter with
  floating point coefficients, that FIR filter truncated to $16$ distinct
  coefficients and the truncated filter coefficients SOCP-relaxation optimised
  for $12$-bit $3$-signed-digit coefficients and $12$-bit coefficients with an
  average of $3$-signed-digits allocated by the method of \emph{Lim et al.}.}
\label{fig:Comp-schurOneMlattice-lowpass-diff-R2-12-nbits-hM-min-pass-relative}
\end{figure}

\begin{figure}
\centering
\scalebox{\DesignOfSchurLatticeFiltersPdfScale}{\input{comparison_no_POP_schurOneMlattice_lowpass_differentiator_R2_12_nbits_test_hM_min_stop}}
\caption{Comparison of the stop band error amplitude responses of a low pass
  differentiator filter implemented as a maximally linear FIR filter with
  floating point coefficients, that FIR filter truncated to $16$ distinct
  coefficients and the truncated filter coefficients SOCP-relaxation optimised
  for $12$-bit $3$-signed-digit coefficients and $12$-bit coefficients with an
  average of $3$-signed-digits allocated by the method of \emph{Lim et al.}.}
\label{fig:Comp-schurOneMlattice-lowpass-diff-R2-12-nbits-hM-min-stop}
\end{figure}

Table~\ref{tab:Comp-schurOneMlattice-lowpass-diff-R2-12-nbits-Schur-FIR-cost}
compares a cost function, the maximum pass band and stop band response errors,
the total number of
signed-digits required by the $12$-bit coefficients and the number of
shift-and-add operations required to
implement the coefficient multiplications of the SOCP-relaxation optimised
truncated FIR and Schur lattice filters and the branch-and-bound optimised
truncated FIR and Schur lattice filters.

\begin{table}
\centering
\begin{threeparttable}
\begin{tabular}{lrrrrrr} \\ \toprule
 &Low pass        &Max. pass&Max. pass &Max. stop &$12$-bit&Shift-\\
 &differentiator  &amplitude&amplitude &amplitude &signed  &and- \\
 &cost            &error    &rel. error&error     &digits  &adds \\
\midrule
\input{comparison_no_POP_schurOneMlattice_lowpass_differentiator_R2_12_nbits_test_Schur_FIR_cost.tab}
\\ \bottomrule
\end{tabular}
\end{threeparttable}
\caption[Comparison of low pass differentiator filter FIR and Schur lattice implementations]
{Comparison of the response errors and the total number of signed digits in the
  12-bit coefficients, each having an average of 3-signed digits, allocated by
  the method of \emph{Lim et al.}, required for a low pass differentiator filter
  implemented as an anti-symmetric FIR filter and a tapped Schur one-multiplier
  lattice, R=2, correction filter found by SOCP relaxation search and
  branch-and-bound search.}
\label{tab:Comp-schurOneMlattice-lowpass-diff-R2-12-nbits-Schur-FIR-cost}
\end{table}

Figure~\ref{fig:Comp-schurOneMlattice-lowpass-diff-R2-12-nbits-Schur-FIR-pass}
compares the pass band amplitude error responses of a low pass
differentiator filter implemented as a maximally linear FIR filter truncated to
$16$ distinct coefficients and a tapped Schur one-multiplier lattice, R=2,
correction filter. The coefficients are SOCP-relaxation and branch-and-bound
optimised for $12$-bit coefficients with an average of
$3$-signed-digits each, allocated by the method of \emph{Lim et al.}\;.
Similarly,
Figure~\ref{fig:Comp-schurOneMlattice-lowpass-diff-R2-12-nbits-Schur-FIR-pass-relative}
compares the pass band relative error amplitude responses and
Figure~\ref{fig:Comp-schurOneMlattice-lowpass-diff-R2-12-nbits-Schur-FIR-stop}
compares the stop band error amplitude responses.

\begin{figure}
\centering
\scalebox{\DesignOfSchurLatticeFiltersPdfScale}{\input{comparison_no_POP_schurOneMlattice_lowpass_differentiator_R2_12_nbits_test_Schur_FIR_pass_error}}
\caption{Comparison of the pass band amplitude error responses of a low pass
  differentiator filter implemented as a maximally linear FIR filter truncated
  to $16$ distinct coefficients and a tapped Schur one-multiplier lattice, R=2,
  correction filter. The coefficients are SOCP-relaxation and branch-and-bound
  optimised for $12$-bit coefficients with an average of
  $3$-signed-digits each, allocated by the method of \emph{Lim et al.}\;.}
\label{fig:Comp-schurOneMlattice-lowpass-diff-R2-12-nbits-Schur-FIR-pass}
\end{figure}

\begin{figure}
\centering
\scalebox{\DesignOfSchurLatticeFiltersPdfScale}{\input{comparison_no_POP_schurOneMlattice_lowpass_differentiator_R2_12_nbits_test_Schur_FIR_pass_relative_error}}
\caption{Comparison of the pass band amplitude relative error responses of a low
  pass differentiator filter implemented as a maximally linear FIR filter
  truncated to $16$ distinct coefficients and a tapped Schur one-multiplier
  lattice, R=2, correction filter. The coefficients are SOCP-relaxation and
  branch-and-bound optimised for $12$-bit coefficients with an average of
  $3$-signed-digits each, allocated by the method of \emph{Lim et al.}\;.}
\label{fig:Comp-schurOneMlattice-lowpass-diff-R2-12-nbits-Schur-FIR-pass-relative}
\end{figure}

\begin{figure}
\centering
\scalebox{\DesignOfSchurLatticeFiltersPdfScale}{\input{comparison_no_POP_schurOneMlattice_lowpass_differentiator_R2_12_nbits_test_Schur_FIR_stop}}
\caption{Comparison of the stop band amplitude responses of a low pass
  differentiator filter implemented as a maximally linear FIR filter truncated
  to $16$ distinct coefficients and a tapped Schur one-multiplier lattice, R=2,
  correction filter. The coefficients are SOCP-relaxation and branch-and-bound
  optimised for $12$-bit coefficients with an average of
  $3$-signed-digits each, allocated by the method of \emph{Lim et al.}\;.}
\label{fig:Comp-schurOneMlattice-lowpass-diff-R2-12-nbits-Schur-FIR-stop}
\end{figure}

\clearpage
\section{Summary}
The Schur lattice digital filter has a simple stability constraint. This article
demonstrates the realisation of an IIR digital filter transfer function with
stability and frequency response constraints by optimising the floating point
and fixed point coefficients of a tapped one-multiplier Schur lattice digital
filter. The performance of the tapped, R=2, Schur lattice filter is compared
to that of a maximally-flat anti-symmetric FIR low pass differentiator filter.
The tapped, R=2, Schur lattice filter has lower group delay and better pass band
relative amplitude error. The truncated anti-symmetric FIR filter has better
stop band amplitude suppression and, by construction, constant phase and group
delay.

\bibliographystyle{plain}
\bibliography{DesignOfSchurLatticeFilters}

\end{document}
'
  \end{comment}
  \usepackage[hidelinks,colorlinks=false]{hyperref}
\else
  \usepackage[unicode=true,pdfusetitle,bookmarks=true,backref=page,
              bookmarksnumbered=false,bookmarksopen=false,breaklinks=false,
              pdfborder={1 0 0},colorlinks=true]{hyperref}
  \usepackage[all]{hypcap} % Needed to help hyperlinks direct correctly
\fi

\setlength{\parindent}{0em}
\setlength{\parskip}{\bigskipamount}
\floatstyle{ruled}
\newfloat{algorithm}{tbp}{loa}
\floatname{algorithm}{Algorithm}

\newcommand{\DesignOfSchurLatticeFiltersPdfScale}{1}
\newcommand{\DesignOfSchurLatticeFiltersIncludeScale}{0.85}

\newcommand{\coloneq}{\mathrel{\resizebox{\widthof{$\mathord{=}$}}
    {\height}{ $\!\!\resizebox{1.2\width}{0.8\height}
      {\raisebox{0.23ex}{$\mathop{:}$}}\!\!=\!\!$ }}}

\DeclareMathOperator{\sinc}{sinc}
\DeclareMathOperator{\jsn}{sn}
\DeclareMathOperator{\jns}{ns}
\DeclareMathOperator{\jcn}{cn}
\DeclareMathOperator{\jnc}{nc}
\DeclareMathOperator{\jdn}{dn}
\DeclareMathOperator{\jnd}{nd}
\DeclareMathOperator{\jsd}{sd}
\DeclareMathOperator{\jds}{ds}
\DeclareMathOperator{\jcd}{cd}
\DeclareMathOperator{\jdc}{dc}
\DeclareMathOperator{\jsc}{sc}
\DeclareMathOperator{\jcs}{cs}
\DeclareMathOperator{\jam}{am}
\DeclareMathOperator{\jarcsn}{arcsn}
\DeclareMathOperator{\jarcsc}{arcsc}
\DeclareMathOperator{\jarccn}{arccn}
\DeclareMathOperator{\jarcdn}{arcdn}
\DeclareMathOperator{\jarccd}{arccd}
\DeclareMathOperator{\jarccs}{arccs}
\DeclareMathOperator{\cosec}{cosec}
\DeclareMathOperator{\cotan}{cotan}
\DeclareMathOperator{\sech}{sech}
\DeclareMathOperator{\arctanh}{arctanh}
\DeclareMathOperator{\arccosh}{arccosh}
\DeclareMathOperator{\sign}{sign}
\DeclarePairedDelimiter{\ceil}{\lceil}{\rceil}
\DeclarePairedDelimiter{\floor}{\lfloor}{\rfloor}
\DeclarePairedDelimiter{\nint}{\lfloor}{\rceil}
\DeclarePairedDelimiter{\mathabs}{\lvert}{\rvert}
\DeclarePairedDelimiter{\mathnorm}{\lVert}{\rVert}
\DeclareMathOperator{\mathnull}{null}
\DeclareMathOperator{\mathspan}{span}
\DeclareMathOperator{\mathrank}{rank}
\DeclareMathOperator{\mathrange}{range}
\DeclareMathOperator{\mathtrace}{trace}
\DeclareMathOperator{\mathadj}{adj}
\DeclareMathOperator{\mathdiag}{diag}
\DeclareMathOperator{\mathtpltz}{Toeplitz}
\DeclareMathOperator{\mathcohull}{cohull}
\DeclareMathOperator{\mathcone}{cone}
\DeclareMathOperator{\mathri}{ri}
\DeclareMathOperator{\mathaff}{aff}
\DeclareMathOperator{\mathconv}{conv}
\DeclareMathOperator{\mathhe}{He}
\DeclareMathOperator{\mathsq}{Sq}
\DeclareMathOperator{\mathvec}{vec}
\DeclareMathOperator{\mathconj}{\ast} % {\textsf{\tiny{H}}}
\DeclareMathOperator{\mathreverse}{\textsf{\tiny{F}}}

\title{On the design of Schur lattice digital filters}
\author{Robert G. Jenssen}
\date{\today}

\begin{document}

\maketitle

\begin{abstract}
  This article describes a computer-aided procedure for approximating an
  arbitrary transfer function by a tapped one-multiplier all pass lattice
  filter with integer coefficients. The all pass lattice has a simple
  stability criterion, that the lattice coefficients have magnitude less
  than unity, and, in addition, has good coefficient sensitivity and round-off
  noise properties. The procedure consists of finding an initial filter
  transfer function, performing constrained minimum-mean-squared-error
  optimisation of the filter response and searching for the fixed point
  filter coefficients.
\end{abstract}

\section{Introduction}

The Infinite Impulse Response (IIR) digital filter transfer function is
a rational polynomial:
\begin{align}
  F\left(z\right) = \frac{B_{N}\left(z\right)}{A_{N}\left(z\right)}
  &= \frac{b_{0}+b_{1}z^{-1}+\cdots + b_{N}z^{-N}}
    {1+a_{1}z^{-1} + \cdots + a_{N}z^{-N}}
\label{eqn:Transfer-function-F}
\end{align}
Here $z$ is a complex number and the response of the filter with respect to
angular frequency, $\omega$, is $F\left(e^{\imath\omega}\right)$, where
$\omega=2\pi$ corresponds to the digital sampling frequency. 
This article describes a method for computer-aided design of the frequency
response of $F\left(e^{\imath\omega}\right)$. If the filter has
$A_{N}\left(z\right)=1$ then it is a Finite Impulse Response (FIR) filter and
the optimisation problem is convex, has no constraints on the coefficients and
is solved by well-known algorithms that minimise the peak error or the
least-squared-error~\cite{Medlin_LagrangeMultiplierFIRMultirate,SelesnickBurrus_ExchangeAlgorithmsLinearPhaseFIRFilters,SelesnickLangBurrus_ConstrainedLeastSquareFIRFilters,SelesnickLangBurrus_ConstrainedLeastSquareMultiBandFIRFilters,McClellanParks_ComputerProgramFIRLinearPhaseFilters,Iwasaki_GeneralizedKYPFrequencyDomainInequalities,Davidson_LMISpectralConstraints,ParksMcClellan_ChebyshevApproxNonRecursiveDigitalFilters}.
Otherwise, the filter is an Infinite Impulse (IIR) filter. An IIR filter is
stable if the roots, $P_{k}$, of the denominator polynomial,
$A_{N}\left(z\right)$, lie within the unit circle, $\left|P_{k}\right|<1$.
\emph{Lang}~\cite{Lang_LSDesignIIRDigitalFiltersPoleRadiusConstraint} applies
Rouch\'{e}'s theorem to constrain the zeros of $A_{N}\left(z\right)$ during
optimisation. \emph{Lu} and
\emph{Hinamoto}~\cite{LuHinamoto_IIRRobustStabilityQuadraticProgramming}
decompose $F\left(z\right)$ into a series connection of second-order sections
and apply the ``stabiliy triangle'' to each section to ensure filter stability
after optimisation. If the transfer function is represented in gain-pole-zero
form:
\begin{align*}
  F\left(z\right)&= G\frac{\prod_{k=1}^{N}\left(z-Z_{k}\right)}
                   {\prod_{k=1}^{N}\left(z-P_{k}\right)}
\end{align*}
where $G$ is the gain factor, the $Z_{k}$ are the roots of $B_{N}\left(z\right)$
and the $P_{k}$ are the roots of $A_{N}\left(z\right)$, then the constraints on
the pole locations of $F\left(z\right)$ are $\left|P_{k}\right|<1$. 
\emph{Deczky}~\cite{Deczky_MinPSynthesisIIRDigitalFilters} and
\emph{Richards}~\cite{Richards_DeczkyRecursiveDecimator} demonstrate the
optimisation of stable IIR filter transfer functions expressed in gain-pole-zero
form.
Lattice filters are an alternative implementation of the IIR filter
transfer function~\cite{VaidyanathanMitra_RobustFilterStructures}.
\emph{Vaidyanathan et
  al.}~\cite{VaidyanathanMitraNuevo_LowSensitivityIIRDigitalFilters,
  RegaliaMitraVaidyanathan_DigitalAllPassFilterVersatileSignalProcessing} show
that a subset of the $F\left(z\right)$ transfer functions can be factored into
the parallel sum of two all pass filters and that the resulting implementation
has good coefficient sensitivity properties. They show lattice filter
implementations of these all pass sections that are ``structurally bounded''
meaning that the section is all pass regardless of the values of the
coefficients. \emph{Gray} and
\emph{Markel}~\cite{GrayMarkel_DigitalLatticeAndLadderFilterSynthesis,
  MarkelGray_AutocorrelationSpeechAnalysis} describe the implementation of
$F\left(z\right)$ as a tapped all pass lattice filter based on the Schur
polynomial decomposition algorithm~\cite{SchurMethodsInOperatorTheory_Kailath,
  SchurMethodsInOperatorTheory_Schur_I}.
The Schur algorithm enables the decomposition of the denominator polynomial of
the transfer function, $F\left(z\right)$, into a set of orthogonal
polynomials. A result of this decompostion is a set of coefficients, often
called ``reflection coefficients'', that have magnitude less than unity
if-and-only-if the poles of the transfer function lie within the unit circle in
the complex plane. Further, the numerator polynomial of $F\left(z\right)$ can be
expressed as a weighted sum of the orthogonal polynomials. These coefficients and
orthogonal polynomials correspond to a tapped all pass lattice filter
implementation of the transfer function that has good round-off noise and
coefficient sensitivity when the coefficients are truncated to integer values.
This article reviews computer-aided techniques for designing tapped Schur
lattice filters with integer coefficients : finding an initial IIR filter
transfer function, constrained minimum-mean-squared-error optimisation of the
filter frequency response in terms of the lattice coefficients and searching for
integer filter coefficients. As an example, I demonstrate the design of a tapped
Schur lattice implementation of a low pass differentiator filter. The
coefficients and frequency response plots for this example are created with
Octave~\cite{Eaton_Octave,Octave_OptimPackage,Octave_SignalPackage}.
The source code for this article is available at
\url{https://github.com/robertgj/DesignOfIIRFilters}.

\section{Schur lattice filters}
\subsection{The Schur polynomial decomposition}
In 1918 \emph{Issai Schur}~\cite{SchurMethodsInOperatorTheory_Schur_I}
published a paper entitled ``\emph{On power series which are bounded in the
  interior of the unit circle}''. From the abstract:
\begin{quotation}
  The continued fraction algorithm introduced here very easily supplies an
  intrinsically important parametric representation for the coefficients of the
  power series to be considered.
\end{quotation}
\emph{Kailath}~\cite{SchurMethodsInOperatorTheory_Kailath} reviews the impact of
this algorithm on modern signal processing. \emph{Gray} and \emph{Markel}
describe the application of a similar algorithm to the autocorrelation analysis
of speech~\cite{MarkelGray_AutocorrelationSpeechAnalysis} and to the synthesis
of lattice digital
filters~\cite{GrayMarkel_DigitalLatticeAndLadderFilterSynthesis}. 
In the following I use the inner product method of \emph{Markel} and
\emph{Gray}~\cite{MarkelGray_AutocorrelationSpeechAnalysis} and
\emph{Parhi}~\cite[Chapter 12 and Appendix
D]{Parhi_VLSIDigitalSignalProcessingSystems}.

Initialise an $N$'th order polynomial as:
\begin{align*}
  \Phi_{N}\left(z\right) &= A_{N}\left(z\right) = \sum_{n=0}^{N}\phi_{n}z^{n}
\end{align*}
and define the reverse polynomial:
\begin{align*}
\hat{\Phi}_{N}\left(z\right)=z^{N}\Phi_{N}\left(z^{-1}\right)
\end{align*}
The transfer function $\hat{\Phi}_{N}\left(z\right)/\Phi_{N}\left(z\right)$ 
represents an all pass filter.
The Schur decomposition forms the polynomial $\Phi_{N-1}\left(z\right)$ as:
\begin{align*}
  \Phi_{N-1}\left(z\right) &= \frac{z^{-1}\left[\Phi_{N}\left(z\right)-
                             k_{N}\hat{\Phi}_{N}\left(z\right)\right]}
                             {s_{N}} \\
\end{align*}
where $s_{N}$ is a scaling constant and
$k_{N}=\phi_{0}/\phi_{N}=\Phi_{N}\left(0\right)/\hat{\Phi}_{N}\left(0\right)$.
The degree of $\Phi_{N-1}\left(z\right)$ is $1$ less than that of
$\Phi_{N}\left(z\right)$ since, by a change of variables, the numerator is:
\begin{align*}
 z^{-1}\left\{ \phi_{N}\Phi_{N}\left(z\right)-\phi_{0}\hat{\Phi}_{N}\left(z\right)\right\}
& = z^{-1}\sum^{N}_{n=0}\left\{\phi_{N}\phi_{n}z^{n}-\phi_{0}\phi_{N-i}z^{n}\right\}\\
& = \sum^{N}_{n=1}\left\{\phi_{N}\phi_{n}-\phi_{0}\phi_{N-i}\right\}z^{n-1}\\
\end{align*}
This degree reduction procedure is continued, resulting in the set of
polynomials $\left\{\Phi_{N}\left(z\right),\Phi_{N-1}\left(z\right),\ldots,
 \Phi_{0}\left(z\right)\right\}$:
\begin{align}
  \Phi_{n-1}\left(z\right) &= \frac{z^{-1}\left\{\Phi_{n}\left(z\right)-
                             k_{n}\hat{\Phi}_{n}\left(z\right)\right\}}
                             {s_{n}}
\label{eq:SchurDegreeReduction}
\end{align}
and
\begin{align*}
  \hat{\Phi}_{n-1}\left(z\right) &= \frac{\left\{\hat{\Phi}_{n}\left(z\right)-
                                         k_{n}\Phi_{n}\left(z\right)\right\}}
                                        {s_{n}}
\end{align*}

\emph{Markel} and \emph{Gray}~\cite[Equations 13a and 13b]
{MarkelGray_AutocorrelationSpeechAnalysis} define an inner product of two
polynomials, $P\left(z\right)$ and $Q\left(z\right)$:
\begin{align*}
  \langle P\left(z\right),Q\left(z\right)\rangle &= \frac{1}{2\pi \imath}
    \ointctrclockwise_{C}\frac{P\left(z\right)Q\left(z^{-1}\right)}
    {A_{N}\left(z\right)A_{N}\left(z^{-1}\right)}\frac{dz}{z}
\end{align*}
where all the zeros of $A_{N}\left(z\right)$ lie within the contour $C$.
Two properties of $\Phi_{n}\left(z\right)$ under this inner product are:
\begin{enumerate}
\item The $\Phi_{n}\left(z\right)$ polynomials are
  orthogonal~\cite[Appendix D]{Parhi_VLSIDigitalSignalProcessingSystems}:
  \begin{align*}
    \langle \Phi_{m}\left(z\right),\Phi_{n}\left(z\right)\rangle = 0
    \;\text{if}\; m \ne n
  \end{align*}
\item For $1\le n \le N$, $\mathabs{k_{n}} < 1$ if-and-only-if the zeros of
  $A_N\left(z\right)$ lie within the unit
  circle~\cite[pages 72 and 74]{MarkelGray_AutocorrelationSpeechAnalysis}
\end{enumerate}
In addition, if $s_{n}=\sqrt{1-k_{n}^{2}}$\;, then the $\Phi_{n}$ polynomials are
orthonormal~\cite[Appendix D]{Parhi_VLSIDigitalSignalProcessingSystems}:
\begin{align*}
  \langle \Phi_{n}\left(z\right),\Phi_{n}\left(z\right)\rangle = 1
\end{align*}

\subsection{The Schur one-multiplier lattice filter}

In Equation~\ref{eq:SchurDegreeReduction}, choose $s_{n}=1-\epsilon_{n}k_{n}$,
where $\epsilon_{n}=\pm 1$, and initialise $\Lambda_{N}\left(z\right)=A_{N}\left(z\right)$. Then:
\begin{subequations}
\begin{align}
  \Lambda_{n-1}\left(z\right) &= \frac{z^{-1}\left\{\Lambda_{n}\left(z\right)-
                             k_{n}\hat{\Lambda}_{n}\left(z\right)\right\}}
                                {1-\epsilon_{n}k_{n}}
  \label{eqn:SchurOneMLambda} \\
  \hat{\Lambda}_{n-1}\left(z\right)
                              &= \frac{\left\{\hat{\Lambda}_{n}\left(z\right)-
                                       k_{n}\Lambda_{n}\left(z\right)\right\}}
                                      {1-\epsilon_{n}k_{n}}
  \label{eqn:SchurOneMLambdaHat}
\end{align}
\end{subequations}
where:
\begin{align*}
  k_{n}=\frac{\Lambda_{n}\left(0\right)}{\hat{\Lambda}_{n}\left(0\right)}
  =\frac{\Phi_{n}\left(0\right)}{\hat{\Phi}_{n}\left(0\right)}
\end{align*}
Rearranging Equation~\ref{eqn:SchurOneMLambdaHat} and substituting it into
Equation~\ref{eqn:SchurOneMLambda} gives:
\begin{align*}
  z\Lambda_{n-1}\left(z\right)
  &=\left(1+\epsilon_{n}k_{n}\right)\Lambda_{n}\left(z\right)-
    k_{n}\hat{\Lambda}_{n-1}\left(z\right)\\
  \hat{\Lambda}_{n}\left(z\right) & = k_{n}\Lambda_{n}\left(z\right)+
              \left(1-\epsilon_{n}k_{n}\right)\hat{\Lambda}_{n-1}\left(z\right)
\end{align*}
Figure~\ref{fig:Schur-one-multiplier-lattice-section} shows the corresponding
\emph{all pass Schur one multiplier lattice} filter section.

\begin{figure}
\centering
\includegraphics{schur_OneMultiplier_lattice_section}
\caption{All pass Schur one multiplier lattice filter section}
\label{fig:Schur-one-multiplier-lattice-section}
\end{figure}

The expansion of the numerator polynomial of the transfer function,
$F\left(z\right)$, in the orthogonal basis, $\Lambda_{n}\left(z\right)$,
is~\cite[Section 12.2.3]{Parhi_VLSIDigitalSignalProcessingSystems}:
\begin{align*}
B_{N}\left(z\right) & = \sum_{n=0}^{N}c_{n}\Lambda_{n}\left(z\right)
\end{align*}

Figure~\ref{fig:Second-order-tapped-Schur-one-multiplier-lattice-filter}
shows the tapped Schur one multiplier lattice filter implementation of a
second order transfer function.

\begin{figure}
\centering
\includegraphics{schur_OneMultiplier_2nd_Order}
\caption{Second-order tapped Schur one-multiplier lattice filter}
\label{fig:Second-order-tapped-Schur-one-multiplier-lattice-filter}
\end{figure}

The relation between $\Lambda_{n}\left(z\right)$ and $\Phi_{n}\left(z\right)$ is:
\begin{align*}
\Lambda_{N}\left(z\right) & = \Phi_{N}\left(z\right)\\
  \Lambda_{n}\left(z\right)
  &= \Phi_{n}\left(z\right)\sqrt{\frac{\left(1+\epsilon_{N}k_{N}\right)
    \left(1+\epsilon_{N-1}k_{N-1}\right)\cdots
    \left(1+\epsilon_{n+1}k_{n+1}\right)}
    {\left(1-\epsilon_{N}k_{N}\right)
    \left(1-\epsilon_{N-1}k_{N-1}\right)\cdots
    \left(1-\epsilon_{n+1}k_{n+1}\right)}}
\end{align*}
The $\Lambda_{n}\left(z\right)$ polynomials are orthogonal but not orthonormal
since:
\begin{align*}
\langle \Lambda_{n}\left(z\right),\Lambda_{n}\left(z\right)\rangle
&= \langle\hat{\Lambda}_{n}\left(z\right),\hat{\Lambda}_{n}\left(z\right)\rangle\\
&=\frac{\left(1+\epsilon_{N}k_{N}\right)\left(1+\epsilon_{N-1}k_{N-1}\right)
  \cdots\left(1+\epsilon_{n+1}k_{n+1}\right)}
  {\left(1-\epsilon_{N}k_{N}\right)\left(1-\epsilon_{N-1}k_{N-1}\right)
  \cdots\left(1-\epsilon_{n+1}k_{n+1}\right)}
\end{align*}

If the input signal is random and white with unit power then the average power
at an internal node, $x_{n}$, is
$\langle\Lambda_{n}\left(z\right),\Lambda_{n}\left(z\right)\rangle$. The
magnitude of $\langle\Lambda_{n}\left(z\right),\Lambda_{n}\left(z\right)\rangle$
can be adjusted by choosing the sign parameters,
$\epsilon_{N},\epsilon_{N-1}\hdots,\epsilon_{n+1}$.
\emph{Gray} and
\emph{Markel}~\cite[p. 496]{GrayMarkel_DigitalLatticeAndLadderFilterSynthesis}
suggest that one criterion for choosing the sign parameters is to require that
the node associated with the largest $k_{n}$ parameter in magnitude have the
largest amplitude. The sign parameters are found recursively by requiring that
the amplitudes at other nodes be as large as possible without exceeding the
maximum value. If the maximum occurs for $k_{l}$ then the recursion proceeds
for $n=l-1,l-2,\ldots,0$ and again for $n=l+1,l+2,\ldots,N$. The
recursion is simple because:
\begin{align*}
  \frac{\langle \Lambda_{n}\left(z\right),\Lambda_{n}\left(z\right)\rangle}
       {\langle \Lambda_{n+1}\left(z\right),\Lambda_{n+1}\left(z\right)\rangle}
  &= \frac{1+\epsilon_{n+1}k_{n+1}}{1-\epsilon_{n+1}k_{n+1}}
\end{align*}
By changing the sign parameter, this ratio can always be made smaller or larger 
than one. Algorithm~\ref{alg:One-multiplier-sign-assignment} shows the method
used to assign the sign parameters described by \emph{Gray} and
\emph{Markel}~\cite[p. 496]{GrayMarkel_DigitalLatticeAndLadderFilterSynthesis}.

\begin{algorithm}
Assume that $k_{l}$ has the largest magnitude of the $k_{n}$ for
$n=1,2,\ldots,N$. Define the quantities
\begin{align*}
  Q_{n}&=\frac{\langle\Lambda_{n}\left(z\right),\Lambda_{n}\left(z\right)\rangle}
             {\langle\Lambda_{l}\left(z\right),\Lambda_{l}\left(z\right)\rangle}\\
  q_{n}&=\frac{1+\left|k_{n}\right|}{1-\left|k_{n}\right|}
\end{align*}

so that $Q_{l}=1$. Each $Q_{n}$ should be as large as possible without
exceeding $Q_{l}$. Successive ratios are:
\begin{align}
\frac{Q_{n}}{Q_{n+1}} = \begin{cases}
q_{n} & if\,\epsilon_{n}=\phantom{-}\sign\left(k_{n}\right)\\
1/q_{n} & if\,\epsilon_{n}=-\sign\left(k_{n}\right)
\end{cases}\label{eq:OneMultRecursiveKCalc}
\end{align}

Now assign the $\epsilon_{n}$:
\begin{algorithmic}
\For {$n=l-1,l-2,\hdots,1$}
  \If {$Q_{n+1}<1/q_{n}$}
    \State $\epsilon_{n}=-\sign\left(k_{n}\right)$
  \Else
    \State $\epsilon_{n}=\sign\left(k_{n}\right)$
  \EndIf
\EndFor
\For {$n=l+1,l+2,\hdots,N$}
  \If {$Q_{n}<1/q_{n}$}
    \State $\epsilon_{n}=\sign\left(k_{n}\right)$
  \Else
    \State $\epsilon_{n}=-\sign\left(k_{n}\right)$
  \EndIf
\EndFor
\end{algorithmic}
\caption{One multiplier lattice sign
  assignment~\cite[p. 496]{GrayMarkel_DigitalLatticeAndLadderFilterSynthesis}.} 
\label{alg:One-multiplier-sign-assignment}
\end{algorithm}

\subsection{State variable descriptions of the Schur one multiplier lattice filter}
The state variable representation of a filter is:
\begin{align}
\left[\begin{array}{c}
\mathbf{x}^{\prime}\\
\mathbf{y}\end{array}\right] &= \left[\begin{array}{cc}
    \mathbf{A} & \mathbf{B} \\
    \mathbf{C} & \mathbf{D} \\
\end{array}\right]\left[\begin{array}{c}
\mathbf{x}\\
\mathbf{u}\end{array}\right]
\label{eqn:State-variable-representation}
\end{align}
where $\mathbf{u}$ is the input, $\mathbf{x}$ is the filter state, for
convenience $\mathbf{x}^{\prime}=z\,\mathbf{x}$, and $\mathbf{y}$ is the filter
output. $\mathbf{u}$ and $\mathbf{y}$ may be vectors. $\mathbf{A}$ is called
the state transition matrix.
Examination of Figure~\ref{fig:Schur-one-multiplier-lattice-section} and
Figure~\ref{fig:Second-order-tapped-Schur-one-multiplier-lattice-filter}
suggests that the calculation of the all pass output, $\hat{y}$, of the
tapped Schur lattice filter can be represented as a matrix product:

\begin{align*}
\left[\begin{array}{c}
\hat{y}_{0}\\
x_{1}\\
x_{2}\\
\vdots\\
x_{N-1}\\
u \end{array}\right] & = \left[\begin{array}{cccccc}
1 & 0 & \cdots &  & \cdots & 0\\
0 & 1 &  &  &  & \vdots\\
\vdots &  & \ddots\\
 &  &  & \ddots &  & \vdots\\
0 &  &  &  & 1 & 0\\
0 & \cdots &  & \cdots & 0 & 1
\end{array}\right]\left[\begin{array}{c}
x_{0}\\
x_{1}\\
x_{2}\\
\vdots\\
x_{N-1}\\
u
\end{array}\right]
\end{align*}
\begin{align*}
\left[\begin{array}{c}
x_{0}^{\prime}\\
\hat{y}_{1}\\
x_{2}\\
\vdots\\
x_{N-1}\\
u
\end{array}\right] & = \left[\begin{array}{cccccc}
-k_{1} & \left(1+k_{1}\epsilon_{1}\right) & 0 & \cdots & \cdots & 0\\
\left(1-k_{1}\epsilon_{1}\right) & k_{1} & 0 &  &  & \vdots\\
0 & 0 & 1\\
\vdots &  &  & \ddots &  & \vdots\\
0 &  &  &  & 1 & 0\\
0 & \cdots &  & \cdots & 0 & 1
\end{array}\right]\left[\begin{array}{c}
\hat{y}_{0}\\
x_{1}\\
x_{2}\\
\vdots\\
x_{N-1}\\
u
\end{array}\right]
\end{align*}
\begin{align*}
\left[\begin{array}{c}
x_{0}^{\prime}\\
x_{1}^{\prime}\\
\hat{y}_{2}\\
\vdots\\
x_{N-1}\\     
u
\end{array}\right] & = \left[\begin{array}{cccccc}
1 & 0 & 0 & 0 & \cdots & 0\\
0 & -k_{2} & \left(1+k_{2}\epsilon_{2}\right) & 0 &  & \vdots\\
0 & \left(1-k_{2}\epsilon_{2}\right) & k_{2} & 0\\
\vdots &  &  & \ddots &  & \vdots\\
0 &  &  &  & 1 & 0\\
0 & \cdots &  & \cdots & 0 & 1
\end{array}\right]\left[\begin{array}{c}
x_{0}^{\prime}\\
\hat{y}_{1}\\
x_{2}\\
\vdots\\
x_{N-1}\\
u
\end{array}\right]
\end{align*}
\begin{align*}
  \vdots\\
\end{align*}
\begin{align*}
\left[\begin{array}{c}
x_{0}^{\prime}\\
x_{1}^{\prime}\\
\vdots\\
x_{N-2}^{\prime}\\
\hat{y}_{N-1}\\
u
\end{array}\right] & = \left[\begin{array}{cccccc}
1 & 0 & \cdots &  & \cdots & 0\\
0 & 1 &  &  &  & \vdots\\
\vdots &  & \ddots &  &  & \vdots\\
0 &  &  & -k_{N-1} & \left(1+k_{N-1}\epsilon_{N-1}\right) & 0\\
0 &  &  & \left(1-k_{N-1}\epsilon_{N-1}\right) & k_{N-1} & 0\\
0 & \cdots & \cdots & 0 & 0 & 1
\end{array}\right]\left[\begin{array}{c}
x_{0}^{\prime}\\
\vdots\\
x_{N-3}^{\prime}\\
\hat{y}_{N-2}\\
x_{N-1}\\
u
\end{array}\right]
\end{align*}
\begin{align*}
\left[\begin{array}{c}
x_{0}^{\prime}\\
x_{1}^{\prime}\\
x_{2}^{\prime}\\
\vdots\\
x_{N-1}^{\prime}\\
\hat{y}
\end{array}\right] & = \left[\begin{array}{cccccc}
1 & 0 & \cdots &  & \cdots & 0\\
0 & 1 &  &  &  & \vdots\\
\vdots &  & \ddots & & & \vdots\\
0 &  &  & 1 & 0 & 0\\
0 &  &  &  0 & -k_{N} & \left(1+k_{N}\epsilon_{N}\right)\\
0 & \cdots &  & 0 & \left(1-k_{N}\epsilon_{N}\right) & k_{N}
\end{array}\right]\left[\begin{array}{c}
x_{0}^{\prime}\\
x_{1}^{\prime}\\
\vdots\\
x_{N-2}^{\prime}\\
\hat{y}_{N-1}\\
u
\end{array}\right]
\end{align*}
The construction of the state variable description of the Schur one multiplier
lattice filter is summarised in
Algorithm~\ref{alg:Construction-state-variable-One-Multiplier-lattice}.

\begin{algorithm}
Given $\left\{k_{1},k_{2},\hdots,k_{N}\right\}$,
$\left\{\epsilon_{1},\epsilon_{2},\hdots,\epsilon_{N}\right\}$ and
$\left\{c_{0},c_{1},\hdots,c_{N}\right\}$:
\begin{algorithmic}
\State $\hat{y}_{0} = x_{0}$
\For{$n=1,\hdots,N-1$}
  \State $x_{n-1}^{\prime}=-k_{n}\hat{y}_{n-1}+\left(1+k_{n}\epsilon_{n}\right)x_{n}$
  \State $\hat{y}_{n}= \left(1-k_{n}\epsilon_{n}\right)\hat{y}_{n-1}+k_{n}x_{n}$
\EndFor
\State $x_{N-1}^{\prime} = -k_{N}\hat{y}_{N-1}+\left(1+k_{N}\epsilon_{N}\right)u$
\State $\hat{y} = \left(1-k_{N}\epsilon_{N}\right)\hat{y}_{N-1}+k_{N}u$
\State $y  = c_{0}x_{0}+c_{1}x_{1}+\cdots+c_{N-1}x_{N-1}+c_{N}u$
\end{algorithmic}
\caption{Construction of a state variable description of the Schur
  one multiplier lattice filter.}
\label{alg:Construction-state-variable-One-Multiplier-lattice}
\end{algorithm}

The state variable matrix calculations for a second order tapped Schur one
muliplier lattice filter are:
\begin{align*}
  \left[\begin{array}{c}
    x_{0}^{\prime} \\
    x_{1}^{\prime} \\
    \hat{y} \\
    y\end{array}\right] &= 
  \left[\begin{array}{ccc}- k_{1} & 1+\epsilon_{1} k_{1} & 0\\
-k_{2}\left(1-\epsilon_{1}k_{1}\right) & -k_{1}k_{2}& 1+\epsilon_{2}k_{2} \\
 \left(1-\epsilon_{1}k_{1}\right)\left(1-\epsilon_{2}k_{2}\right) &
   k_{1}\left(1-\epsilon_{2}k_{2}\right) & k_{2} \\
 c_{0} & c_{1} & c_{2} \end{array}\right] \left[\begin{array}{c}
    x_{0} \\
    x_{1} \\
    u \end{array} \right]
\end{align*}
The latency in the arithmetic calculation of $y$ and $\hat{y}$
increases with the filter order. \emph{Parhi}~\cite[Chapter 4 and Section 12.8]
{Parhi_VLSIDigitalSignalProcessingSystems} describes the retiming and pipelining
of lattice filters for reduced latency. One method is to design the lattice
filter with coefficients of the denominator polynomial, $A_{N}\left(z\right)$,
only in $z^{-2n}$, where $N$ is even and $n=1,\hdots,\frac{N}{2}$.
\emph{Deczky}~\cite{Deczky_MinPSynthesisIIRDigitalFilters} and
\emph{Richards}~\cite{Richards_DeczkyRecursiveDecimator} describe such filters
as $R=2$.
Figure~\ref{fig:Fourth-order-tapped-Schur-one-multiplier-lattice-filter-R2}
shows such a fourth order tapped Schur one multiplier lattice filter with and
without pipelining.

\begin{figure}
\begin{subfigure}[b]{\textwidth}
\centering
\includegraphics[scale=\DesignOfSchurLatticeFiltersIncludeScale]{schur_OneMultiplier_4th_Order_R2}
\caption{Without pipelining}
\label{subfig:Fourth-order-tapped-Schur-one-multiplier-lattice-R2-without}
\end{subfigure}
\begin{subfigure}[b]{\textwidth}
\centering
\includegraphics[scale=\DesignOfSchurLatticeFiltersIncludeScale]{schur_OneMultiplier_4th_Order_R2_Pipelined}
\caption{With pipelining}
\label{subfig:Fourth-order-tapped-Schur-one-multiplier-lattice-R2-with}
\end{subfigure}
\caption{Fourth-order tapped Schur one-multiplier lattice filter with denominator
  coefficients only for $z^{-2n}$}
\label{fig:Fourth-order-tapped-Schur-one-multiplier-lattice-filter-R2}
\end{figure}

The state variable matrix calculations for a pipelined fourth order tapped Schur
one-muliplier lattice filter with denominator polynomial coefficients only for
$z^{-2n}$ are:

\begin{align*}
  \left[\begin{array}{c}
    x_{0}^{\prime} \\
    x_{1}^{\prime} \\
    x_{2}^{\prime} \\
    x_{3}^{\prime} \\
    x_{4}^{\prime} \\
    \hat{y} \\
    y\end{array}\right] &= 
  \left[\begin{array}{cccccc} 0 & 1 & 0 & 0 & 0 & 0\\
    -k_{2} & 0 & 0 & 0 & 1+\epsilon_{2}k_{2} & 0\\
    1-\epsilon_{2}k_{2} & 0 & 0 & 0 & k_{2} & 0\\
    c_{0} & c_{1} & 0 & 0 & c_{2} & 0 \\
    0 & 0 & -k_{4} & 0 & 0 & 1+\epsilon_{4}k_{4}\\
    0 & 0 & 1-\epsilon_{4}k_{4}  & 0 & 0 & k_{4} \\
    0 & 0 & 0 & 1 & c_{3} & c_{4}\end{array}\right] \left[\begin{array}{c}
    x_{0} \\
    x_{1} \\
    x_{2} \\
    x_{3} \\
    x_{4} \\
    u \end{array} \right]
\end{align*}

The construction of the state variable description of the pipelined tapped Schur
one multiplier lattice filter with denominator polynomial coefficients only for
$z^{-2n}$ is summarised in
Algorithm~\ref{alg:State-variable-pipelined-One-Multiplier-R2-lattice}.

\begin{algorithm}
Given $N$ even, $\left\{k_{2},k_{4}\hdots,k_{N}\right\}$,
$\left\{\epsilon_{2},\epsilon_{4},\hdots,\epsilon_{N}\right\}$ and
$\left\{c_{0},c_{1},\hdots,c_{N}\right\}$:
\begin{algorithmic}
\State $x^{\prime}_{0}=x_{1}$
\State $x^{\prime}_{1}=-k_{2}x_{0}+\left(1+k_{2}\epsilon_{2}\right)x_{4}$
\State $x^{\prime}_{2}=\left(1-k_{2}\epsilon_{2}\right)x_{0}+k_{2}x_{4}$
\State $x^{\prime}_{3}=c_{0}x_{0}+c_{1}x_{1}+c_{2}x_{4}$

\For{$n=2,\hdots,\frac{N}{2}-1$}
\State $x^{\prime}_{3n-2}=-k_{2n}x_{3n-4}+\left(1+k_{2n}\epsilon_{2n}\right)x_{3n+1}$
\State $x^{\prime}_{3n-1}=\left(1-k_{2n}\epsilon_{2n}\right)x_{3n-4}+ k_{2n}x_{3n+1}$
\State $x^{\prime}_{3n}=x_{3n-3}+c_{2n-1}x_{3n-2}+c_{2n}x_{3n+1}$
\EndFor

\State $x_{3\frac{N}{2}-2}^{\prime} = -k_{N}x_{3\frac{N}{2}-4}+
\left(1+k_{N}\epsilon_{N}\right)u$
\State $\hat{y} = \left(1-k_{N}\epsilon_{N}\right)x_{3\frac{N}{2}-4}+k_{N}u$
\State $y = x_{3\frac{N}{2}-3}+c_{N-1}x_{3\frac{N}{2}-2}+c_{N}u$
\end{algorithmic}
\caption{Construction of a state variable description of the pipelined tapped
  Schur one multiplier lattice filter with denominator polynomial coefficients
  only for $z^{-2n}$.}
\label{alg:State-variable-pipelined-One-Multiplier-R2-lattice}
\end{algorithm}

\subsection{Frequency response of the tapped Schur one multiplier lattice filter\label{sec:Frequency-response-of-tapped-Schur-one-multiplier}}
Eliminating the filter state, $\mathbf{x}$, from
Equation~\ref{eqn:State-variable-representation} gives:
\begin{align}
\mathbf{F}\left(z\right) &= 
    \mathbf{C}\left(z\mathbf{I}-\mathbf{A}\right)^{-1}\mathbf{B}+\mathbf{D}
\label{eqn:State-variable-response}
\end{align}
\emph{Thiele}~\cite{Theile_SensitivityLinearStateSpaceSystems} uses the identity:
\begin{align*}
  \mathbf{R}\mathbf{R}^{-1} &= \mathbf{I} \\
\frac{\partial\mathbf{R}}{\partial\chi}\mathbf{R}^{-1} 
  +\mathbf{R}\frac{\partial\mathbf{R}^{-1}}{\partial\chi} &= 0 \\
  \frac{\partial\mathbf{R}}{\partial\chi} 
    &=-\mathbf{R}\frac{\partial\mathbf{R}^{-1}}{\partial\chi}\mathbf{R}
\end{align*}
\emph{Theile} shows the sensitivity functions of $\mathbf{F}\left(z\right)$ with
respect to the components $\alpha$, $\beta$, $\gamma$ and $\delta$ of
$\mathbf{A}$, $\mathbf{B}$, $\mathbf{C}$ and $\mathbf{D}$ respectively:
\begin{align*}
\frac{\partial\mathbf{F}}{\partial{}z} &= 
-\mathbf{C}\left(z\mathbf{I}-\mathbf{A}\right)^{-2}\mathbf{B} \\
  \frac{\partial\mathbf{F}}{\partial\alpha} &= 
\mathbf{C}\left(z\mathbf{I}-\mathbf{A}\right)^{-1}\frac{\partial\mathbf{A}}{\partial\alpha}
\left(z\mathbf{I}-\mathbf{A}\right)^{-1}\mathbf{B} \\
\frac{\partial\mathbf{F}}{\partial\beta} &= 
\mathbf{C}\left(z\mathbf{I}-\mathbf{A}\right)^{-1}\\
\frac{\partial\mathbf{F}}{\partial\gamma} &= 
\left(z\mathbf{I}-\mathbf{A}\right)^{-1}\mathbf{B}\\
\frac{\partial\mathbf{F}}{\partial\delta} &= \mathbf{I}
\end{align*}
Substituting $z=e^{\imath\omega}$ into Equation~\ref{eqn:State-variable-response}
gives the complex frequency response of the state variable filter on the unit
circle. In the following I will use
$\mathbf{R}=\left(e^{\imath\omega}\mathbf{I}-\mathbf{A}\right)^{-1}$. The
components $\alpha$ of $\mathbf{A}$, etc. may themselves be functions of other
variables (for example, the $c$ and $k$ coefficients of a tapped Schur one
multiplier lattice filter) that I will represent by $\chi$. The sensitivity
functions of the complex frequency response
$\mathbf{F}\left(e^{\imath\omega}\right)$ with respect to $\omega$ and $\chi$ are:
\begin{align*}
  \frac{\partial\mathbf{R}}{\partial{}\omega}
  =& -\imath{}e^{\imath\omega}\mathbf{R}\mathbf{R}\\
\frac{\partial\mathbf{F}}{\partial\omega}=&
-\imath e^{\imath\omega}\mathbf{C}\mathbf{R}\mathbf{R}\mathbf{B} \\
  \frac{\partial\mathbf{F}}{\partial{}\chi}=
   &\frac{\partial\mathbf{C}}{\partial\chi}\mathbf{R}\mathbf{B}+
\mathbf{C}\mathbf{R}\frac{\partial\mathbf{A}}{\partial\chi}\mathbf{R}\mathbf{B}+
\mathbf{C}\mathbf{R}\frac{\partial\mathbf{B}}{\partial\chi}+
     \frac{\partial\mathbf{D}}{\partial\chi}\\
\frac{\partial{}^{2}\mathbf{F}}{\partial\chi\partial\omega}=&
-\imath{}e^{\imath\omega}\left[
\frac{\partial\mathbf{C}}{\partial\chi}\mathbf{R}\mathbf{R}\mathbf{B}+
\mathbf{C}\mathbf{R}\mathbf{R}\frac{\partial\mathbf{A}}{\partial\chi}\mathbf{R}\mathbf{B}+
\mathbf{C}\mathbf{R}\frac{\partial\mathbf{A}}{\partial\chi}\mathbf{R}\mathbf{R}\mathbf{B}+
\mathbf{C}\mathbf{R}\mathbf{R}\frac{\partial\mathbf{B}}{\partial\chi}\right]
\end{align*}

The complex frequency response of the filter is:
\begin{align*}
\mathbf{F} &=\Re\mathbf{F} + \imath\Im\mathbf{F}
\end{align*}

The squared-magnitude response of the filter is:
\begin{align*}
\left|\mathbf{F}\right|^{2} &=\Im \mathbf{F}^{2} + \Re \mathbf{F}^{2}
\end{align*}

The gradients of the squared-magnitude response of the filter with respect to 
the Schur lattice filter coefficients are:
\begin{align*}
\frac{\partial\left|\mathbf{F}\right|^{2}}{\partial\chi}
&= 2\left[\Im \mathbf{F}\Im\frac{\partial\mathbf{F}}{\partial\chi}+
          \Re \mathbf{F}\Re\frac{\partial\mathbf{F}}{\partial\chi}\right]
\end{align*}

The gradient of the squared-magnitude response of the filter with respect to 
the angular frequency is:
\begin{align*}
\frac{\partial\left|\mathbf{F}\right|^{2}}{\partial\omega}
&= 2\left[\Im \mathbf{F}\Im\frac{\partial{}\mathbf{F}}{\partial\omega}+
          \Re \mathbf{F}\Re\frac{\partial{}\mathbf{F}}{\partial\omega}\right]
\end{align*}
so that:
\begin{align*}
  \frac{\partial^{2}  \left|\mathbf{F}\right|^{2}}
  {\partial\omega\partial\chi}
  &= 2\left[ \Im\frac{\partial{}\mathbf{F}}{\partial\chi}
             \Im\frac{\partial{}\mathbf{F}}{\partial\omega}+
    \Im \mathbf{F}\Im\frac{\partial^{2}\mathbf{F}}{\partial\omega\partial\chi}+
    \Re\frac{\partial{}\mathbf{F}}{\partial\chi}
    \Re\frac{\partial{}\mathbf{F}}{\partial\omega}+
    \Re \mathbf{F}\Re\frac{\partial^{2}\mathbf{F}}{\partial\omega\partial\chi}
  \right]
\end{align*}

The phase response, $P$, of the filter is:
\begin{align*}%
P&=\arctan\frac{\Im \mathbf{F}}{\Re \mathbf{F}}
\end{align*}

The gradients of the phase response of the filter with respect to 
the Schur lattice filter coefficients are given by:
\begin{align*}
\left|\mathbf{F}\right|^{2}\frac{\partial{}P}{\partial\chi}&=
\Re \mathbf{F}\Im\frac{\partial{}\mathbf{F}}{\partial\chi}-
\Im \mathbf{F}\Re\frac{\partial{}\mathbf{F}}{\partial\chi}
\end{align*}

The group delay response, $T$, of the filter is:
\begin{align*}
T &= -\frac{\partial P}{\partial\omega}
\end{align*}
so that:
\begin{align*}
\left|\mathbf{F}\right|^{2}T &= -\left[\Re \mathbf{F}\Im\frac{\partial \mathbf{F}}{\partial\omega} -
                              \Im \mathbf{F}\Re\frac{\partial \mathbf{F}}{\partial\omega}\right]
\end{align*}

The gradients of the group delay response with respect to the Schur lattice
filter coefficients are given by:
\begin{align*}
  \frac{\partial\left|\mathbf{F}\right|^{2}}{\partial\chi}T+
  \left|\mathbf{F}\right|^{2}\frac{\partial{}T}{\partial\chi} = 
&-\left[\Re\frac{\partial{}\mathbf{F}}{\partial\chi}
\Im\frac{\partial{}\mathbf{F}}{\partial\omega} +
\Re \mathbf{F} \Im\frac{\partial^{2}\mathbf{F}}{\partial\chi\partial\omega} -
  \Im\frac{\partial{}\mathbf{F}}{\partial\chi}
  \Re\frac{\partial{}\mathbf{F}}{\partial\omega} -
  \Im \mathbf{F}\Re\frac{\partial^{2}\mathbf{F}}
                        {\partial\chi\partial\omega}\right]
\end{align*}

\section{Computer-Aided-Design of Schur lattice filters}
The filter design procedure commences with the specification of the required
filter amplitude, phase and group delay responses. The optimisation of an IIR
frequency response with respect to the coefficients is not a convex problem. I
hope to find a ``good enough'' design rather than a global optimum.

One formulation of the filter optimisation problem is to minimise the
\emph{weighted squared error} of the frequency response of $F\left(z\right)$:
\begin{align}
\begin{split}
  \textbf{minimise}\quad &\mathcal{E}_{F}\left(\chi\right) =
  \int W\left(\omega\right)\left|F\left(\omega\right)
    -F_{d}\left(\omega\right)\right|^{2}d\omega \\
\textbf{subject to}\quad &F\left(\omega\right)\text{ is stable} 
\end{split}
\label{eqn:Definition-of-squared-response-error}
\end{align}
where $\chi$ is the coefficient vector of the filter, $F\left(z\right)$,
$\mathcal{E}_{F}\left(\chi\right)$ is the weighted sum of the squared error,
$F\left(\omega\right)$ is the filter frequency response,
$W\left(\omega\right)$ is the frequency weighting and
$F_{d}\left(\omega\right)$ is the desired filter complex frequency response.
The integrand of Equation~\ref{eqn:Definition-of-squared-response-error} could
be the weighted sum of one or more of the squared-magnitude, phase or group
delay responses.

The filter response optimisation problem can also be expressed as a
\emph{mini-max} optimisation problem:
\begin{align}
\label{eqn:Definition-of-minimax-response-error}
\begin{split}
\textbf{minimise}\quad &\max\;\left|F\left(\omega\right)
  -F_{d}\left(\omega\right)\right|\\
\textbf{subject to}\quad &F\left(\omega\right)\text{ is stable} 
\end{split}
\end{align}

I begin by finding an initial filter design that approximates the desired
response, then optimising the minimum mean-squared-error of the
calculated response and, if desired, finding a mini-max approximation to
the desired response. Finally, I describe two methods of
searching for integer valued filter coefficients: a depth-first branch-and-bound
search and a successive relaxation method.

\subsection{Finding an initial filter\label{sec:Finding-initial-filter}}
The initial filter could be an FIR filter or a ``classical'' Butterworth,
Chebyshev, etc. IIR filter design.
\emph{Deczky}~\cite{Deczky_MinPSynthesisIIRDigitalFilters} and
\emph{Richards}~\cite{Richards_DeczkyRecursiveDecimator} begin with an
initial filter consisting of a ``by-eye'' arrangement of the filter poles and
zeros. \emph{Tarczynski et
  al.}~\cite{TarczynskiCainHermanowiczRojewski_WISEMethodDesignIIRFilters}
propose minimising the filter response mean-squared-error, $\mathcal{E}_{F}$,
weighted with a barrier function:
\begin{align}
\label{eqn:WISE-barrier-function}
\left(1-\lambda\right)\mathcal{E}_{F}+\lambda\sum^{T+M}_{t=T+1}h^{2}\left(t\right)
\end{align}
where $\lambda$, $T$ and $M$ are suitable constants and $h\left(t\right)$ is the
impulse response of the filter $H\left(z\right)=\frac{1}{A_{N}\left(z\right)}$.
The barrier function (the second part of
Equation ~\ref{eqn:WISE-barrier-function}) is intended to be small when the
filter, $F\left(z\right)$, is stable and increase rapidly when the poles
approach the unit circle. \emph{Tarczynski et al.} provide heuristics for
selecting $\lambda$, $T$ and $M$. Typically,
$\lambda \in \left[10^{-10},10^{-3}\right]$, $T\in \left[ 100, 500 \right]$ and
$M=NR$. \emph{Roberts and Mullis}~\cite[Section 8.3]
{RobertsMullis_DigitalSignalProcessing} show that the state space description of
the direct-form implementation of $H\left(z\right)$ is:
\begin{align*}
\left[\begin{array}{c}
x\left(t+1\right)\\
y\left(t\right)
\end{array}\right] &= \left[\begin{array}{cc}
A & B\\
C & D
\end{array}\right]\left[\begin{array}{c}
x\left(t\right)\\
u\left(t\right)
\end{array}\right]
\end{align*}
where:
\begin{align*}
A &= \left[\begin{array}{cccc}
0 & 1 & \cdots & 0\\
\vdots & \vdots & \ddots & \vdots\\
0 & 0 & \cdots & 1\\
-a_{N} & -a_{N-1} & \cdots & -a_{1}
\end{array}\right]\\
B &= \left[\begin{array}{ccccc}0 & 0 & \cdots & 0 & 1\end{array}\right]^{\top}\\
C &= \left[\begin{array}{ccc}-a_{N} & \cdots & -a_{1}\end{array}\right]\\
D &= 1
\end{align*}

The corresponding impulse response
is~\cite[Equation 8.3.21]{RobertsMullis_DigitalSignalProcessing}:
\begin{align*}
h\left(t\right) &= \begin{cases}
0 & t<0\\
D & t=0\\
CA^{t-1}B & t>0
\end{cases}
\end{align*}

\subsection{Minimum-Mean-Squared-Error optimisation of IIR filters}
Minimisation of the mean squared error of the filter response proceeds by
successive solution of the following linear approximation problem
for $\Delta_{\chi}$:
\begin{align}
\begin{split}
  \textbf{minimise}\quad & \epsilon+\beta \\
  \textbf{subject to}\quad & \mathnorm{\mathcal{E}_{F}\left(\chi\right) +
          \Delta_{\chi}\nabla\mathcal{E}_{F}\left(\chi\right)}\le\epsilon \\
  & \mathnorm{\Delta_{\chi}}\le\beta \\
  & F\left(\omega\right)\text{ is stable}
\end{split}
\label{eqn:Optimisation-of-minimum-mean-squared-response-error}
\end{align}
After each step the new coefficient vector is $\chi+\Delta_{\chi}$. 

In fact I replace $\mathcal{E}_{F}$ with an approximation to the sum at discrete
frequencies of the weighted squared error of the real valued squared magnitude,
phase, group delay and the gradient of the squared magnitude with respect to
angular frequency. For example, the first order approximation to the squared
error of the group delay, $T$, is:
\begin{align*}
  \bar{\mathcal{E}}_{T}\left(\chi+\Delta_{\chi}\right)
  &\approx \sum_{\omega} W_{T}\left(\omega\right)
     \left[T\left(\chi,\omega\right) + \right.
    \left. \Delta_{\chi}\nabla_{\chi}T\left(\chi,\omega\right) - \right.
    \left. T_{d}\left(\omega\right)\right]^{2}
\end{align*}
with gradient with respect to $\chi$:
\begin{align*}
  \nabla_{\chi} \bar{\mathcal{E}}_{T}\left(\chi\right)
  &\approx 2 \sum_{\omega} W_{T}\left(\omega\right)
    \left[T\left(\chi,\omega\right)-T_{d}\left(\omega\right)\right]
    \nabla_{\chi}T\left(\chi,\omega\right)
\end{align*}
The required gradients of the state variable filter frequency response were
derived in Section~\ref{sec:Frequency-response-of-tapped-Schur-one-multiplier}.

\subsection{Peak-Constrained-Least-Squares optimisation of IIR filters}
Linear phase FIR filter design
algorithms~\cite{Hofstetter_DesignNonRecursiveDigitalFilters,
  ParksMcClellan_ChebyshevApproxNonRecursiveDigitalFilters,
  SelesnickBurrus_ExchangeAlgorithmsLinearPhaseFIRFilters,
  SelesnickLangBurrus_ConstrainedLeastSquareMultiBandFIRFilters} commonly
employ an ``exchange algorithm'' that, given an initial approximation to the
desired amplitude response, finds the extremal frequencies of that response and
interpolates the coefficients to obtain a new set of filter coefficients that
achieves the desired ripple values at all of those frequencies. The process
repeats until a satisfactory mini-max amplitude response is found. I have
successfully solved the IIR mini-max problem of
Equation~\ref{eqn:Definition-of-minimax-response-error} with a modified version
of the ``Peak-Constrained-Least-Squares'' (PCLS) algorithm of \emph{Selesnick},
\emph{Lang} and \emph{Burrus}~\cite[p.498]
{SelesnickLangBurrus_ConstrainedLeastSquareMultiBandFIRFilters}. They describe
the exchange of amplitude response, $A\left(\omega\right)$, constraints as
follows:
\begin{quotation}
  After each iteration, the algorithm checks the values of
  $A\left(\omega\right)$ over the previous constraint set frequencies. $\hdots$
  However, if it is found that $A\left(\omega\right)$ violates the constraints
  at some frequency belonging to the previous constraint set, $R$, then i) that
  frequency where the violation is greatest is appended to the current
  constraint set, $S$, and ii) the same frequency is removed from the
  record of previous constraint set frequencies, $R$.
\end{quotation}

Figure~\ref{fig:Modified-Selesnick-Lang-Burrus-algorithm} shows a flow-chart of
the modified \emph{Selesnick}, \emph{Lang} and \emph{Burrus} exchange algorithm.

\begin{figure}
\centering
\includegraphics[scale=\DesignOfSchurLatticeFiltersIncludeScale]{slb_exchange_algorithm}
\caption{Modified Selesnick, Lang and Burrus exchange algorithm~\cite[p.498]
  {SelesnickLangBurrus_ConstrainedLeastSquareMultiBandFIRFilters}.}
\label{fig:Modified-Selesnick-Lang-Burrus-algorithm}
\end{figure}

Similarly to $\bar{\mathcal{E}}_{T}$, the PCLS constraints for the group delay
response, $T\left(\chi,\omega\right)$, are approximated to first order by:
\begin{align*}
  T_{l}\left(\omega\right) \le
  T\left(\chi,\omega\right) + \Delta_{\chi}\nabla_{\chi}T\left(\chi,\omega\right)
  \le T_{u}\left(\omega\right)
\end{align*}
where $T_{l}\left(\omega\right)$ and $T_{u}\left(\omega\right)$ are the
lower and upper constraints on the group delay.

The modified PCLS mini-max optimisation is:
\begin{align}
\begin{split}
  \textbf{minimise}\quad & \epsilon+\beta \\
  \textbf{subject to}\quad & \mathnorm{\bar{\mathcal{E}}_{F}\left(\chi\right) +
          \Delta_{\chi}\nabla\bar{\mathcal{E}}_{F}\left(\chi\right)}\le\epsilon \\
  & \mathnorm{\Delta_{\chi}}\le\beta \\
  & F\left(\omega\right)\text{ is stable} \\
  & \text{PCLS response constraints are satisfied}
\end{split}
\label{eqn:Optimisation-of-PCLS-response-error}
\end{align}

\subsection{Second Order Cone Programming}
 \emph{Alizadeh} and \emph{Goldfarb}~\cite
{AlizadehGoldfarb_SecondOrderCone} describe Second Order Cone
Programming (SOCP) as the solution of a class of convex optimisation problems in
which a linear function is minimised subject to a set of conic constraints.
In the following example I use the \emph{SeDuMi}
(\emph{Se}lf-\emph{Du}al-\emph{Mi}nimisation) SOCP solver originally written
by \emph{Jos Sturm}~\cite{Sturm_SeDuMi_GitHub}. 
\emph{Lu}~\cite[Section III]{Lu_SedumiRemarks} provides an example of expressing
an optimisation problem of
Equation~\ref{eqn:Optimisation-of-PCLS-response-error} in the form accepted by
SeDuMi. In \emph{Lu}'s notation the problem is:
\begin{subequations}
\begin{align}
\textbf{minimise}  \quad&\boldsymbol{b}^{\top}\boldsymbol{x}
\label{eqn:Lu-SeDuMi-Remarks-problem-format}\\
\textbf{subject to}\quad&\|\boldsymbol{A}_{i}^{\top}\boldsymbol{x}+
\boldsymbol{c}_{i}\| \le \boldsymbol{b}_{i}^{\top}\boldsymbol{x}+d_{i}
\quad \text{for}\;i=1,\hdots,q
\label{eqn:Lu-SeDuMi-Remarks-problem-format-conic-constraints}\\
&\boldsymbol{D}^{\top}\boldsymbol{x}+\boldsymbol{f}\ge\boldsymbol{0}
\label{eqn:Lu-SeDuMi-Remarks-problem-format-linear-constraints}
\end{align}
\end{subequations}
where $\boldsymbol{x}\in\mathfrak{\mathbb{R}}^{m\times 1}$, 
$\boldsymbol{b}\in\mathfrak{\mathbb{R}}^{m\times 1}$,
$\boldsymbol{A}_{i}\in\mathfrak{\mathbb{R}}^{m\times\left(n_{i}-1\right)}$
\footnote{$n_{i}-1$ to allow for the column $b_{i}$ in the matrix 
$\boldsymbol{A}_{t}^{\left(i\right)}$ and $d_{i}$ in the column vector 
$\boldsymbol{c}_{t}^{\left(i\right)}$}, 
$\boldsymbol{c}_{i}\in\mathfrak{\mathbb{R}}^{\left(n_{i}-1\right)\times 1}$,
$\boldsymbol{b}_{i}\in\mathfrak{\mathbb{R}}^{m\times 1}$, 
$d_{i}\in\mathfrak{\mathbb{R}}$ for $1 \le i \le q$,
$\boldsymbol{D}\in\mathfrak{\mathbb{R}}^{m\times p}$ and
$\boldsymbol{f}\in\mathfrak{\mathbb{R}}^{p\times 1}$. The problem is cast into
SeDuMi format by defining:
\begin{align*}
\boldsymbol{A}_{t}&=\left[-\boldsymbol{D} \quad \boldsymbol{A}_{t}^{\left(1\right)}
\hdots \quad \boldsymbol{A}_{t}^{\left(q\right)}\right]\\
\boldsymbol{A}_{t}^{\left(i\right)} &= -\left[\boldsymbol{b}_{i} \quad
\boldsymbol{A}_{i}\right]\nonumber\\
\boldsymbol{b}_{t}&=-\boldsymbol{b}\nonumber\\
\boldsymbol{c}_{t}&=\left[\boldsymbol{f};\quad \boldsymbol{c}_{t}^{\left(1\right)}
;\quad \hdots\quad \boldsymbol{c}_{t}^{\left(q\right)}\right]\nonumber\\
\boldsymbol{c}_{t}^{\left(i\right)}&=\left[d_{i} ;\quad \boldsymbol{c}_{i}\right]
\nonumber
\end{align*}

For Equation~\ref{eqn:Optimisation-of-PCLS-response-error},
$\boldsymbol{x}=\left[\epsilon,\beta,\chi\right]$,
Equation~\ref{eqn:Lu-SeDuMi-Remarks-problem-format-conic-constraints} applies
the conic constraints on the linear approximation to the minimum mean squared
error of the frequency response and 
Equation~\ref{eqn:Lu-SeDuMi-Remarks-problem-format-linear-constraints} applies
the linear constraints on both the reflection coefficients,
$\mathabs{k_{n}}<1$, and the PCLS frequency response constraints.

\section{\label{sec:Design-IIR-lattice-filters-with-signed-digit-coefficients}Design of digital filters with signed-digit coefficients}
\subsection{The canonical signed-digit representation of filter coefficients}

\emph{Parhi}~\cite[Section 13.6]{Parhi_VLSIDigitalSignalProcessingSystems} lists
the following properties of the \emph{canonical} binary signed-digit (CSD)
representation of a binary signed-digit number $A=a_{W-1}a_{W-2}\cdots{}a_{1}a_{0}$
where each $a_{i}\in \left\{-1,0,1\right\}$:
\begin{itemize}
\item no 2 consecutive bits in a CSD number are non-zero 
\item the CSD representation of a number contains the miniumum possible number
of non-zero bits, thus the name \emph{canonic}
\item the CSD representation of a number is unique
\end{itemize}
The first property breaks the carry chain when adding two CSD numbers.
\emph{Parhi}~\cite[Section 13.6.1]{Parhi_VLSIDigitalSignalProcessingSystems}
shows an algorithm that calculates the \emph{canonical} binary signed-digit 
representation from the two's complement representation, reproduced here as
Algorithm~\ref{alg:twos-complement-to-signed-digit-conversion}.

\begin{algorithm}[htbp]
Denote the two's complement representation of the number $A$ as 
$A=\hat{a}_{W-1}\hat{a}_{W-2}\cdots\hat{a}_{1}\hat{a}_{0}$. \\
Denote the CSD representation of $A$ as $A=a_{W-1}a_{W-2}\cdots{}a_{1}a_{0}$.

\begin{algorithmic}
\State $\hat{a}_{-1}=0$
\State $\gamma_{-1}=0$
\State $\hat{a}_{W}=\hat{a}_{W-1}$
\For{$k=0,\hdots,W-1$}
\State $\theta_{i}=\hat{a}_{i}\oplus\hat{a}_{i-1}$
\State $\gamma_{i}=\bar{\gamma}_{i-1}\theta_{i}$
\State $a_{i}=\left(1-2\hat{a}_{i+1}\right)\gamma_{i}$
\EndFor
\end{algorithmic}
\caption{Conversion of 2's complement numbers to the canonical signed-digit
  representation (\emph{Parhi}~\cite[Section
  13.6.1]{Parhi_VLSIDigitalSignalProcessingSystems}).}
\label{alg:twos-complement-to-signed-digit-conversion}
\end{algorithm}

\emph{Lim et al.}~\cite{Lim_SignedPowerOfTwoAllocationDigitalFilters} describe
a method of allocating  a limited number of signed power-of-two digit terms 
to the fixed point coefficients of a digital filter. The method is based 
on the belief that ``\emph{allocating the SPT terms in such a way that all the 
coefficient values have the same quantisation step-size to coefficient
sensitivity ratio will lead to a good design}''. 

\emph{Lim et al.} first prove properties of the canonical signed-digit
representation~\cite[Section II]{Lim_SignedPowerOfTwoAllocationDigitalFilters}.
In particular:

\emph{Property 1}: Define $S_{Q}$ as the set of contiguous integers that can be 
represented by up to $Q$ signed digits. The largest integer in $S_{Q}$ is
$J_{Q}=\sum_{l=0}^{Q-1}2^{2l+1}$.

\emph{Property 5}: On average, $0.72Q$ signed-digits are required to represent
the integers in $S_{Q}$.

\emph{Lim et al.} show that an estimate of the number of signed digits, $Q$, 
required to represent $J_{Q}$ is:
\begin{align*}
  Q \approx \frac{1}{2}\log_{2}J_{Q}+0.31
\end{align*}
Replacing $J_{Q}$ by an integer $n\in S_{Q}$, an estimate of the average number
of terms, $Q_{A}$, required to represent $n$ is:
\begin{align*}
  Q_{A}&\approx 0.72Q\\
      &\approx 0.36 \log_{2}n+0.22
\end{align*}

Now suppose that $D$ signed digits are available to represent two positive
integers $n_{1}$ and $n_{2}$. If $n_{1}\approx n_{2}$ then each integer is
allocated $\frac{D}{2}$ bits. If $n_{1}>n_{2}$ then \emph{Lim et al.} argue 
that the number of additional signed-digits, $Q_{E}$, required to represent 
$n_{1}$ is:
\begin{align*}
  Q_{E}&\approx 0.36\log_{2}\lfloor \frac{n_{1}}{n_{2}}\rfloor
\end{align*}
where $\lfloor\;\rfloor$ represents the integer part. In general, 
$Q_{E}$ is not an integer.

\emph{Lim et al.} go on to consider the allocation of signed digits to the
coefficients of a symmetric FIR filter. The change in the frequency response,
$F\left(\omega\right)$, of a filter due to a change $\Delta\chi_{n}$ in
coefficient $\chi_{n}$ is:
\begin{align*}
\Delta{} F\left(\omega,n\right)&  \approx
\frac{\partial{}F\left(\omega\right)}{\partial\chi_{n}}\Delta{}\chi_{n}
\end{align*}
\emph{Lim et al.} use the average of the coefficient sensitivity to define a
cost for the $n$'th coefficient:
\begin{align*}
cost_{n}&=0.36\log_{2}\left|\chi_{n}\right| + 0.36\log_{2}\int^{\pi}_{0}
\left|\frac{\partial{}F\left(\omega\right)}{\partial{}\chi_{}}\right|d\omega
\end{align*}

Given a total of $D$ signed-digits, \emph{Lim et al.} assign a single 
signed-digit at a time to the coefficient with the largest cost. After a 
coefficient is given a signed-digit, its cost is decreased by one. The
process is repeated until all $D$ digits have been allocated.

\subsection{Branch-and-bound search for the filter coefficients}
Given the $K$ floating point coefficients, $\chi$, of a filter, an exhaustive
search of the upper and lower bounds on the integer or signed-digit
approximations to these coefficients would require
$\mathcal{O}\left(2^{K}\right)$ comparisons of the corresponding filter
approximation error.
\emph{Branch-and-bound}~\cite{LandDoig_AutomaticMethodSolvingDiscreteProgrammingProblems},~\cite[p.627]{Sedgewick_AlgorithmsInCPlusPlus}
is a heuristic for reducing the 
number of branches searched in a binary decision tree. At each branch of the
binary tree the solution is compared to the estimated lower bounds on the cost
of the full path proceeding from that branch. If the cost of that full path is
greater than that of the best full path found so far then further search on
that path is abandoned.  Figure~\ref{fig:branch-bound-tree-search-algorithm}
shows a flow diagram of an implementation of the algorithm using a stack. The
floating point filter coefficients, $\chi$, are approximated by the
signed-digit coefficients, $\bar{\chi}$. Each coefficient, $\chi_{n}$
and $\bar{\chi}_{n}$, is bounded by the corresponding signed-digit numbers $u_{n}$
and $l_{n}$ so that $l_{n}\le\chi_{n}\le{}u_{n}$ and
$l_{n}\le\bar{\chi}_{n}\le u_{n}$. The search for the set of coefficients with
minimum cost is ``depth-first'', starting at the root of the search tree and
fixing successive coefficients.  \emph{Ito et
  al.}~\cite{Ito_PowersOfTwoAllocationFIR} recommend choosing, at each branch,
the $\chi_{n}$ with the greatest difference $u_{n}-l_{n}$. The two sub-problems
at that branch fix $\bar{\chi}_{n}$ to $l_{n}$ and $u_{n}$. One of the two
sub-problems is pushed onto a stack and the other is solved and the cost
calculated. I assume that this cost is the least possible for the remaining
coefficients on the current branch. If the cost of the current sub-problem is
greater than the current minimum cost then the current branch is abandoned and a
new sub-problem is popped off the problem stack. Otherwise, if the search has
reached the maximum depth of the tree then the current solution is a new
signed-digit minimum cost solution. If the current cost is less than the
minimum cost and the search has not reached the maximum depth then the search
continues with a new branch.

\begin{figure}
\centering
\includegraphics[scale=\DesignOfSchurLatticeFiltersIncludeScale]{branch_bound_algorithm}
\caption{Branch-and-bound depth-first search algorithm}
\label{fig:branch-bound-tree-search-algorithm}
\end{figure}

\subsection{Successive relaxation search for the filter coefficients}
A successive relaxation search for the filter coefficients fixes one
coefficient and optimises the filter frequency response with respect to the
remaining free coefficients. This is called a \emph{relaxation} of the
optimisation. The relaxation is repeated until all the coefficients are fixed.
At each coefficient relaxation step the script finds the upper and lower
signed-digit approximations to the current set of active coefficients and
selects the coefficient with the largest difference in those approximations.

\section{An example : design of a low pass differentiator filter}
This section describes the design of a low pass differentiator filter
implemented as $F_{0}\left(z\right)=1-z^{-1}$ in series with a tapped Schur
one-multiplier lattice correction filter, $C\left(z\right)$, having denominator
polynomial coefficients only for terms in $z^{-2n}$. In the pass band, the
desired complex frequency response of a differentiator filter is:
\begin{align*}
  F_{d}\left(\omega\right)=-\imath\frac{\omega}{2}e ^{-\imath\omega t_{p}}
\end{align*}
where $t_{p}$ is the nominal filter pass band group delay in samples.
The squared-magnitude response of the correction filter is:
\begin{align*}
  \mathabs{C\left(\omega\right)}^{2}
  &= \frac{\mathabs{F_{d}\left(\omega\right)}^{2}}
    {\mathabs{F_{0}\left(\omega\right)}^{2}}
\end{align*}
where:
\begin{align*}
  \mathabs{F_{d}\left(\omega\right)} &= \frac{\omega}{2} \\
  \mathabs{F_{d}\left(\omega\right)}^{2}&= \frac{\omega^{2}}{4} \\  
  \frac{d\mathabs{F_{d}\left(\omega\right)}^{2}}{d\omega}
     &= \frac{\omega}{2} = \mathabs{F_{d}\left(\omega\right)}
\end{align*}
and the response of the zero at $z=1$ is:
\begin{align*}
  F_{0}\left(\omega\right) &= 1-e^{\imath\omega} \\
        &=2\imath{} e^{\imath\frac{\omega}{2}}\sin\frac{\omega}{2} \\
  \mathabs{F_{0}\left(\omega\right)} &= 2\sin\frac{\omega}{2} \\
  \frac{d\mathabs{F_{0}\left(\omega\right)}^{2}}{d\omega}
        &=2\mathabs{F_{0}\left(\omega\right)}
          \frac{d\mathabs{F_{0}\left(\omega\right)}}{d\omega} \\
        &= 2\sin\omega
\end{align*}
By the chain rule for differentiation, the gradient of
$\mathabs{F\left(\omega\right)}^{2}$ is:
\begin{align*}
  \frac{d\mathabs{F_{d}\left(\omega\right)}^{2}}{d\omega}
  &= \mathabs{F_{0}\left(\omega\right)}^{2}
     \frac{d\mathabs{C\left(\omega\right)}^{2}}{d\omega} +
    \frac{\mathabs{F_{d}\left(\omega\right)}^{2}}
    {\mathabs{F_{0}\left(\omega\right)}^{2}}
    \frac{d\mathabs{F_{0}\left(\omega\right)}^{2}}{d\omega}
\end{align*}
Substituting and rearranging to obtain the desired gradient of
$\mathabs{C\left(\omega\right)}^{2}$ in terms of the desired overall squared
amplitude response, $\mathabs{F_{d}\left(\omega\right)}^{2}$, and the
squared magnitude response of the zero at $z=1$,
$\mathabs{F_{0}\left(\omega\right)}^{2}$:
\begin{align*}
  \frac{d\mathabs{C\left(\omega\right)}^{2}}{d\omega}
  &= \mathabs{F_{d}\left(\omega\right)}
\left[\frac{1-\mathabs{F_{d}\left(\omega\right)}\cot\frac{\omega}{2}}{\mathabs{F_{0}\left(\omega\right)}^{2} }\right]
\end{align*}

\subsection{Initial R=2 low pass differentiator filter}
Figure~\ref{fig:Schur-OneM-lattice-correction-lowpass-differentiator-R2-initial}
shows the frequency response of the initial R=2 low pass differentiator filter
found by the method of \emph{Tarczynski et al.}, described in
Section~\ref{sec:Finding-initial-filter}. The correction filter order is $N=10$,
filter pass band and stop band edge frequencies are $0.2$ and $0.4$\;(normalised
to the digital sample rate), the nominal pass band phase is $1.5\pi$
radians (adjusted for delay) and the nominal pass band group delay is $9$
samples.

\begin{figure}
\centering
\scalebox{\DesignOfSchurLatticeFiltersPdfScale}{\input{schurOneMlattice_socp_slb_lowpass_differentiator_R2_test_initial_response}}
\caption{Amplitude, phase and group delay responses of an initial 
  low pass differentiator filter composed of a Schur one-multiplier lattice
  correction filter with a denominator polynomial having terms only in $z^{-2}$
  in series with $1-z^{-1}$.}
\label{fig:Schur-OneM-lattice-correction-lowpass-differentiator-R2-initial}
\end{figure}

The coefficients of the numerator and denominator polynomials of the initial
correction filter transfer function are:
\begin{small}
\verbatiminput{schurOneMlattice_socp_slb_lowpass_differentiator_R2_test_N0_coef.m}
\verbatiminput{schurOneMlattice_socp_slb_lowpass_differentiator_R2_test_D0R_coef.m}
\end{small}

The corresponding Schur one-multiplier lattice coefficients of the initial
correction filter are:
\begin{small}
\verbatiminput{schurOneMlattice_socp_slb_lowpass_differentiator_R2_test_k0_coef.m}
\verbatiminput{schurOneMlattice_socp_slb_lowpass_differentiator_R2_test_epsilon0_coef.m}
\verbatiminput{schurOneMlattice_socp_slb_lowpass_differentiator_R2_test_c0_coef.m}
\end{small}

\subsection{PCLS design of the R=2 low pass differentiator filter}
The low pass differentiator correction filter was PCLS optimised with 
amplitude pass band error peak-to-peak ripple of $0.0009$,
amplitude stop band peak ripple of $0.007$,
phase pass band peak-to-peak ripple of $0.0002\pi$\; radians,
group delay pass band peak-to-peak ripple of $0.006$\; samples and
correction filter gradient of amplitude-squared pass band error peak-to-peak
ripple of $0.02$.

The Schur one-multiplier lattice coefficients of the PCLS SOCP optimised
correction filter are:
\begin{small}
\verbatiminput{schurOneMlattice_socp_slb_lowpass_differentiator_R2_test_k2_coef.m}
\verbatiminput{schurOneMlattice_socp_slb_lowpass_differentiator_R2_test_epsilon2_coef.m}
\verbatiminput{schurOneMlattice_socp_slb_lowpass_differentiator_R2_test_c2_coef.m}
\end{small}

The corresponding correction filter numerator and denominator polynomial
coefficients are:
\begin{small}
\verbatiminput{schurOneMlattice_socp_slb_lowpass_differentiator_R2_test_N2_coef.m}
\verbatiminput{schurOneMlattice_socp_slb_lowpass_differentiator_R2_test_D2_coef.m}
\end{small}
Figure~\ref{fig:Schur-OneM-lattice-lowpass-differentiator-R2}
shows the  amplitude error, phase and group delay response errors of the $R=2$
low pass differentiator filter after PCLS SOCP optimisation.
Figure~\ref{fig:Schur-OneM-lattice-lowpass-differentiator-R2-dCsqdw-error}
shows the error in the gradient of the squared-magnitude response of the $R=2$
correction filter after PCLS SOCP optimisation.
Figure~\ref{fig:Schur-OneM-lattice-lowpass-differentiator-R2-pz} shows
the pole-zero plot of the $R=2$ low pass differentiator filter.
Figure~\ref{fig:Schur-OneM-lattice-correction-lowpass-differentiator-direct-sens}
shows the gradients of the correction filter squared-magnitude, phase and delay
with respect to the coefficients of the direct form implementation of the filter.
Figure~\ref{fig:Schur-OneM-lattice-correction-lowpass-differentiator-Schur-sens}
shows the gradients of the correction filter squared-magnitude, phase and delay
with respect to the coefficients of the Schur lattice implementation of the
filter.

\begin{figure}
\centering
\scalebox{\DesignOfSchurLatticeFiltersPdfScale}{\input{schurOneMlattice_socp_slb_lowpass_differentiator_R2_test_pcls_error}}
\caption{PCLS optimised amplitude error, phase and group delay responses of a
  low pass differentiator filter composed of a Schur one-multiplier lattice
  correction filter with a denominator polynomial having terms only in $z^{-2}$
  in series with $1-z^{-1}$.}
\label{fig:Schur-OneM-lattice-lowpass-differentiator-R2}
\end{figure}

\begin{figure}
\centering
\scalebox{\DesignOfSchurLatticeFiltersPdfScale}{\input{schurOneMlattice_socp_slb_lowpass_differentiator_R2_test_pcls_dCsqdw_error}}
\caption{PCLS optimised pass band error of the gradient of the squared magnitude
  response of the correction filter after PCLS SOCP optimisation.}
\label{fig:Schur-OneM-lattice-lowpass-differentiator-R2-dCsqdw-error}
\end{figure}

\begin{figure}
\centering
\scalebox{\DesignOfSchurLatticeFiltersPdfScale}{\input{schurOneMlattice_socp_slb_lowpass_differentiator_R2_test_pcls_pz}}
\caption{Pole-zero plot of the PCLS optimised low pass differentiator
  filter composed of a Schur one-multiplier lattice correction filter with a
  denominator polynomial having terms only in $z^{-2}$ in series with $1-z^{-1}$.}
\label{fig:Schur-OneM-lattice-lowpass-differentiator-R2-pz}
\end{figure}

\begin{figure}
\centering
\scalebox{\DesignOfSchurLatticeFiltersPdfScale}{\input{schurOneMlattice_socp_slb_lowpass_differentiator_R2_test_direct_sensitivity}}
\caption{Gradients of the pass band response with respect to the coefficients of
  the direct form implementation of the PCLS optimised low pass differentiator
  correction filter.}
\label{fig:Schur-OneM-lattice-correction-lowpass-differentiator-direct-sens}
\end{figure}

\begin{figure}
\centering
\scalebox{\DesignOfSchurLatticeFiltersPdfScale}{\input{schurOneMlattice_socp_slb_lowpass_differentiator_R2_test_schur_sensitivity}}
\caption{Gradients of the pass band response with respect to the coefficients of
  the tapped Schur lattice implementation of the PCLS optimised low pass
  differentiator correction filter.}
\label{fig:Schur-OneM-lattice-correction-lowpass-differentiator-Schur-sens}
\end{figure}

\subsection{Search for the signed-digit coefficients of the R=2 low pass differentiator filter}

When truncated to $12$ bits and $3$ signed-digits the Schur one-multiplier
lattice coefficients of the PCLS optimised correction filter are:
\begin{small}
\verbatiminput{socp_relaxation_schurOneMlattice_lowpass_differentiator_R2_12_nbits_test_k0_sd_no_alloc_coef.m}
\verbatiminput{socp_relaxation_schurOneMlattice_lowpass_differentiator_R2_12_nbits_test_c0_sd_no_alloc_coef.m}
\end{small}

The numbers of signed-digits that the heuristic of \emph{Lim et al.} allocates to
each cofficient of the PCLS optimised correction filter are:
\begin{small}
\verbatiminput{socp_relaxation_schurOneMlattice_lowpass_differentiator_R2_12_nbits_test_k_allocsd_digits.m}
\verbatiminput{socp_relaxation_schurOneMlattice_lowpass_differentiator_R2_12_nbits_test_c_allocsd_digits.m}
\end{small}

The signed-digit coefficients of the PCLS optimised filter with the
numbers of signed-digits allocated by the heuristic of \emph{Lim et al.} are:
\begin{small}
\verbatiminput{socp_relaxation_schurOneMlattice_lowpass_differentiator_R2_12_nbits_test_k0_sd_coef.m}
\verbatiminput{socp_relaxation_schurOneMlattice_lowpass_differentiator_R2_12_nbits_test_c0_sd_coef.m}
\end{small}

The signed-digit coefficients of the PCLS optimised filter with the
numbers of signed-digits allocated by the heuristic of \emph{Lim et al.} and
found with branch-and-bound search are:
\begin{small}
\verbatiminput{branch_bound_schurOneMlattice_lowpass_differentiator_R2_12_nbits_test_k_min_coef.m}
\verbatiminput{branch_bound_schurOneMlattice_lowpass_differentiator_R2_12_nbits_test_c_min_coef.m}
\end{small}
The corresponding correction filter transfer function polynomials found with
branch-and-bound search are:
\begin{small}
\verbatiminput{branch_bound_schurOneMlattice_lowpass_differentiator_R2_12_nbits_test_N_min_coef.m}
\verbatiminput{branch_bound_schurOneMlattice_lowpass_differentiator_R2_12_nbits_test_D_min_coef.m}
\end{small}

The signed-digit coefficients of the PCLS optimised filter with the
numbers of signed-digits allocated by the heuristic of \emph{Lim et al.} and
found with SOCP relaxation search are:
\begin{small}
\verbatiminput{socp_relaxation_schurOneMlattice_lowpass_differentiator_R2_12_nbits_test_k_min_coef.m}
\verbatiminput{socp_relaxation_schurOneMlattice_lowpass_differentiator_R2_12_nbits_test_c_min_coef.m}
\end{small}
The corresponding correction filter transfer function polynomials found with
SOCP relaxation search are:
\begin{small}
\verbatiminput{socp_relaxation_schurOneMlattice_lowpass_differentiator_R2_12_nbits_test_N_min_coef.m}
\verbatiminput{socp_relaxation_schurOneMlattice_lowpass_differentiator_R2_12_nbits_test_D_min_coef.m}
\end{small}

Table~\ref{tab:Comp-schurOneMlattice-lowpass-diff-R2-12-nbits-cost}
compares, for each filter, a cost function, the maximum pass band and stop band
response errors, the total number of signed-digits required by the $12$-bit 
coefficients and the number of shift-and-add operations required to implement
the correction filter coefficient multiplications of the correction filter.

\begin{table}
\centering
\begin{threeparttable}
\begin{tabular}{lrrrrrrrr} \\ \toprule
 &Correction&Max. pass&Max. pass &Max. stop &Max. pass   &Max. pass  &$12$-bit&Shift-\\
 &filter    &amplitude&amplitude &amplitude &phase error &delay error&signed  &and- \\
 &cost      &error    &rel. error&error     &(rad./$\pi$)&(samples)  &digits  &adds \\
\midrule
\input{comparison_no_POP_schurOneMlattice_lowpass_differentiator_R2_12_nbits_test_cost.tab}
\\ \bottomrule
\end{tabular}
\end{threeparttable}
\caption[Comparison of low pass differentiator filter fixed point implementations]
{Comparison of the response errors and the total number of signed digits in the
  12-bit coefficients, each having an average of 3-signed digits, allocated by
  the method of \emph{Lim et al.}, required for a tapped Schur one-multiplier
  lattice, R=2, low pass differentiator correction filter found by
  branch-and-bound and SOCP relaxation search.}
\label{tab:Comp-schurOneMlattice-lowpass-diff-R2-12-nbits-cost}
\end{table}

Figure~\ref{fig:Comp-schurOneMlattice-lowpass-diff-R2-12-nbits-pass}
compares the pass band amplitude response of each filter for
$12$-bit $3$-signed-digit coefficients and $12$-bit coefficients with an average
of $3$-signed-digits allocated by the method of \emph{Lim et al.} found with
branch-and-bound search and SOCP relaxation search.
Figure~\ref{fig:Comp-schurOneMlattice-lowpass-diff-R2-12-nbits-relative-pass}
compares the pass band relative amplitude response of each filter.
Figure~\ref{fig:Comp-schurOneMlattice-lowpass-diff-R2-12-nbits-stop}
compares the stop band amplitude response of each filter.
Figure~\ref{fig:Comp-schurOneMlattice-lowpass-diff-R2-12-nbits-phase} shows
the pass band phase reslonse of each filter.
~\ref{fig:Comp-schurOneMlattice-lowpass-diff-R2-12-nbits-delay} shows the
pass band group-delay response of each filter.
Figure~\ref{fig:Comp-schurOneMlattice-lowpass-diff-R2-12-nbits-dCsqdw} shows
the gradient of the squared magnitude response of the correction filter.

\begin{figure}
\centering
\scalebox{\DesignOfSchurLatticeFiltersPdfScale}{\input{comparison_no_POP_schurOneMlattice_lowpass_differentiator_R2_12_nbits_test_pass}}
\caption{Comparison of the pass band response error of
  a tapped Schur one-multiplier, R=2, lattice lowpass differentiator correction
  filter with floating point coefficients, 12-bit 3-signed-digit coefficients,
  and 12-bit signed-digit coefficients each having an average of 3-signed-digits
  allocated by the method of \emph{Lim et al.}, found by branch-and-bound and
  SOCP relaxation search.}
\label{fig:Comp-schurOneMlattice-lowpass-diff-R2-12-nbits-pass}
\end{figure}

\begin{figure}
\centering
\scalebox{\DesignOfSchurLatticeFiltersPdfScale}{\input{comparison_no_POP_schurOneMlattice_lowpass_differentiator_R2_12_nbits_test_relative_pass}}
\caption{Comparison of the pass band relative response error of
  a tapped Schur one-multiplier, R=2, lattice lowpass differentiator correction
  filter with floating point coefficients, 12-bit 3-signed-digit coefficients,
  and 12-bit signed-digit coefficients each having an average of 3-signed-digits
  allocated by the method of \emph{Lim et al.}, found by branch-and-bound and
  SOCP relaxation search.}
\label{fig:Comp-schurOneMlattice-lowpass-diff-R2-12-nbits-relative-pass}
\end{figure}

\begin{figure}
\centering
\scalebox{\DesignOfSchurLatticeFiltersPdfScale}{\input{comparison_no_POP_schurOneMlattice_lowpass_differentiator_R2_12_nbits_test_stop}}
\caption{Comparison of the stop band response error of
  a tapped Schur one-multiplier, R=2, lattice lowpass differentiator correction
  filter with floating point coefficients, 12-bit 3-signed-digit coefficients,
  and 12-bit signed-digit coefficients each having an average of 3-signed-digits
  allocated by the method of \emph{Lim et al.}, found by branch-and-bound and
  SOCP relaxation search.}
\label{fig:Comp-schurOneMlattice-lowpass-diff-R2-12-nbits-stop}
\end{figure}

\begin{figure}
\centering
\scalebox{\DesignOfSchurLatticeFiltersPdfScale}{\input{comparison_no_POP_schurOneMlattice_lowpass_differentiator_R2_12_nbits_test_phase}}
\caption{Comparison of the pass band phase response errors of a tapped Schur
  one-multiplier, R=2, lattice low pass differentiator correction filter with
  floating point coefficients, 12-bit 3-signed-digit coefficients, and 12-bit
  signed-digit coefficients each having an average of 3-signed-digits allocated
  by the method of \emph{Lim et al.}, found by branch-and-bound and
  SOCP relaxation search.}
\label{fig:Comp-schurOneMlattice-lowpass-diff-R2-12-nbits-phase}
\end{figure}

\begin{figure}
\centering
\scalebox{\DesignOfSchurLatticeFiltersPdfScale}{\input{comparison_no_POP_schurOneMlattice_lowpass_differentiator_R2_12_nbits_test_delay}}
\caption{Comparison of the pass band group delay response errors of a tapped
  Schur one-multiplier, R=2, lattice low pass differentiator correction filter
  with floating point coefficients, 12-bit 3-signed-digit coefficients, and
  12-bit signed-digit coefficients each having an average of 3-signed-digits
  allocated by the method of \emph{Lim et al.}, found by branch-and-bound and
  SOCP relaxation search.}
\label{fig:Comp-schurOneMlattice-lowpass-diff-R2-12-nbits-delay}
\end{figure}

\begin{figure}
\centering
\scalebox{\DesignOfSchurLatticeFiltersPdfScale}{\input{comparison_no_POP_schurOneMlattice_lowpass_differentiator_R2_12_nbits_test_dCsqdw}}
\caption{Comparison of the error in the pass band gradient of the squared
  magnitude response of a tapped Schur one-multiplier, R=2, lattice low pass
  differentiator correction filter with floating point coefficients, 12-bit
  3-signed-digit coefficients, and 12-bit signed-digit coefficients each having
  an average of 3-signed-digits allocated by the method of \emph{Lim et al.},
  found by branch-and-bound and SOCP relaxation search.}
\label{fig:Comp-schurOneMlattice-lowpass-diff-R2-12-nbits-dCsqdw}
\end{figure}

\subsection{Comparison with a maximally-linear  FIR low pass differentiator filter}

\emph{Selesnick}~\cite{Selesnick_MaximallyFlatLowPassDigitalDifferentiators}
describes the design of maximally linear low pass FIR differentiator filters:
\begin{align*}
  F\left(z\right)&=\left(\frac{1-z^{-1}}{2}\right)
                   \left(\frac{1+z^{-1}}{2}\right)^{K}
                   z^{-L}\sum_{n=0}^{L}c_{n}\left[\frac{-z+2-z^{-1}}{4}\right]^{n}
\end{align*}
where the $c_{n}$ are defined recursively:
\begin{align*}
c_{0}&=2\\
c_{1}&=K+\frac{1}{3} \\
    & \hdots \\
c_{n}&=\frac{\left(8n^{2}+4Kn-10n-K+3\right)c_{n-1}-\left(2n+K-3\right)^{2}c_{n-2}}
       {2n\left(2n+1\right)}
\end{align*}
The length of the impulse response is $N=K+2L+2$. The coefficients of the maximally linear anti-symmetric FIR
low pass differentiator with $L=20$ and $K=21$ are\footnote{For odd length,
  $N=63$, $h0\left(32\right)=0$.}:
\begin{small}
\verbatiminput{socp_relaxation_directFIRantisymmetric_lowpass_differentiator_12_nbits_test_h0_coef.m}
\end{small}
By construction, the phase response of the anti-symmetric FIR low pass
differentiator filter is $\frac{\pi}{2}$ radians. This FIR filter was truncated
to $16$ distinct coefficients, the same number as the Schur lattice, and a group
delay of $16$ samples. After SOCP-relaxation optimisation to $12$-bits with an
average of $3$ signed-digits allocated by the heuristic of \emph{Lim et al.}
the distinct coefficients of the filter are:
\begin{small}
\verbatiminput{socp_relaxation_directFIRantisymmetric_lowpass_differentiator_12_nbits_test_hM_min_coef.m}
\end{small}

Table~\ref{tab:Comp-schurOneMlattice-lowpass-diff-R2-12-nbits-hM-min-cost}
compares a cost function, the maximum pass band and stop band response errors,
the total number of signed-digits required by the $12$-bit coefficients and the
number of shift-and-add operations required to implement the coefficient
multiplications of the floating point FIR, floating point truncated FIR and the
SOCP-relaxation optimised truncated FIR and the the SOCP-relaxation optimised
truncated FIR and the branch-and-bound optimised truncated FIR filter.

\begin{table}
\centering
\begin{threeparttable}
\begin{tabular}{lrrrrrr} \\ \toprule
 &Low pass        &Max. pass&Max. pass &Max. stop &$12$-bit&Shift-\\
 &differentiator  &amplitude&amplitude &amplitude &signed  &and- \\
 &cost            &error    &rel. error&error     &digits  &adds \\
\midrule
\input{comparison_no_POP_schurOneMlattice_lowpass_differentiator_R2_12_nbits_test_hM_min_cost.tab}
\\ \bottomrule
\end{tabular}
\end{threeparttable}
\caption[Comparison of low pass differentiator filter FIR implementations]
{Comparison of the response errors and the total number of signed digits in the
  12-bit coefficients, each having an average of 3-signed digits, allocated by
  the method of \emph{Lim et al.}, required for a low pass differentiator filter
  implemented as an anti-symmetric FIR filter found by SOCP relaxation search
  and branch-and-bound search.}
\label{tab:Comp-schurOneMlattice-lowpass-diff-R2-12-nbits-hM-min-cost}
\end{table}

Figure~\ref{fig:Comp-schurOneMlattice-lowpass-diff-R2-12-nbits-hM-min-pass}
compares the pass band error amplitude responses of a low pass
differentiator filter implemented as a maximally linear FIR filter with floating
point coefficients, that FIR filter truncated to $16$ distinct coefficients and
the truncated filter coefficients SOCP-relaxation optimised for $12$-bit
$3$-signed-digit coefficients and $12$-bit coefficients with an average of
$3$-signed-digits allocated by the method of \emph{Lim et al.}.
Similarly,
Figure~\ref{fig:Comp-schurOneMlattice-lowpass-diff-R2-12-nbits-hM-min-pass-relative}
compares the pass band relative error amplitude responses and
Figure~\ref{fig:Comp-schurOneMlattice-lowpass-diff-R2-12-nbits-hM-min-stop}
compares the stop band error amplitude responses.

\begin{figure}
\centering
\scalebox{\DesignOfSchurLatticeFiltersPdfScale}{\input{comparison_no_POP_schurOneMlattice_lowpass_differentiator_R2_12_nbits_test_hM_min_pass_error}}
\caption{Comparison of the pass band error amplitude responses of a low pass
  differentiator filter implemented as a maximally linear FIR filter with
  floating point coefficients, that FIR filter truncated to $16$ distinct
  coefficients and the truncated filter coefficients SOCP-relaxation optimised
  for $12$-bit $3$-signed-digit coefficients and $12$-bit coefficients with an
  average of $3$-signed-digits allocated by the method of \emph{Lim et al.}.}
\label{fig:Comp-schurOneMlattice-lowpass-diff-R2-12-nbits-hM-min-pass}
\end{figure}

\begin{figure}
\centering
\scalebox{\DesignOfSchurLatticeFiltersPdfScale}{\input{comparison_no_POP_schurOneMlattice_lowpass_differentiator_R2_12_nbits_test_hM_min_pass_relative_error}}
\caption{Comparison of the pass band relative error responses of a low pass
  differentiator filter implemented as a maximally linear FIR filter with
  floating point coefficients, that FIR filter truncated to $16$ distinct
  coefficients and the truncated filter coefficients SOCP-relaxation optimised
  for $12$-bit $3$-signed-digit coefficients and $12$-bit coefficients with an
  average of $3$-signed-digits allocated by the method of \emph{Lim et al.}.}
\label{fig:Comp-schurOneMlattice-lowpass-diff-R2-12-nbits-hM-min-pass-relative}
\end{figure}

\begin{figure}
\centering
\scalebox{\DesignOfSchurLatticeFiltersPdfScale}{\input{comparison_no_POP_schurOneMlattice_lowpass_differentiator_R2_12_nbits_test_hM_min_stop}}
\caption{Comparison of the stop band error amplitude responses of a low pass
  differentiator filter implemented as a maximally linear FIR filter with
  floating point coefficients, that FIR filter truncated to $16$ distinct
  coefficients and the truncated filter coefficients SOCP-relaxation optimised
  for $12$-bit $3$-signed-digit coefficients and $12$-bit coefficients with an
  average of $3$-signed-digits allocated by the method of \emph{Lim et al.}.}
\label{fig:Comp-schurOneMlattice-lowpass-diff-R2-12-nbits-hM-min-stop}
\end{figure}

Table~\ref{tab:Comp-schurOneMlattice-lowpass-diff-R2-12-nbits-Schur-FIR-cost}
compares a cost function, the maximum pass band and stop band response errors,
the total number of
signed-digits required by the $12$-bit coefficients and the number of
shift-and-add operations required to
implement the coefficient multiplications of the SOCP-relaxation optimised
truncated FIR and Schur lattice filters and the branch-and-bound optimised
truncated FIR and Schur lattice filters.

\begin{table}
\centering
\begin{threeparttable}
\begin{tabular}{lrrrrrr} \\ \toprule
 &Low pass        &Max. pass&Max. pass &Max. stop &$12$-bit&Shift-\\
 &differentiator  &amplitude&amplitude &amplitude &signed  &and- \\
 &cost            &error    &rel. error&error     &digits  &adds \\
\midrule
\input{comparison_no_POP_schurOneMlattice_lowpass_differentiator_R2_12_nbits_test_Schur_FIR_cost.tab}
\\ \bottomrule
\end{tabular}
\end{threeparttable}
\caption[Comparison of low pass differentiator filter FIR and Schur lattice implementations]
{Comparison of the response errors and the total number of signed digits in the
  12-bit coefficients, each having an average of 3-signed digits, allocated by
  the method of \emph{Lim et al.}, required for a low pass differentiator filter
  implemented as an anti-symmetric FIR filter and a tapped Schur one-multiplier
  lattice, R=2, correction filter found by SOCP relaxation search and
  branch-and-bound search.}
\label{tab:Comp-schurOneMlattice-lowpass-diff-R2-12-nbits-Schur-FIR-cost}
\end{table}

Figure~\ref{fig:Comp-schurOneMlattice-lowpass-diff-R2-12-nbits-Schur-FIR-pass}
compares the pass band amplitude error responses of a low pass
differentiator filter implemented as a maximally linear FIR filter truncated to
$16$ distinct coefficients and a tapped Schur one-multiplier lattice, R=2,
correction filter. The coefficients are SOCP-relaxation and branch-and-bound
optimised for $12$-bit coefficients with an average of
$3$-signed-digits each, allocated by the method of \emph{Lim et al.}\;.
Similarly,
Figure~\ref{fig:Comp-schurOneMlattice-lowpass-diff-R2-12-nbits-Schur-FIR-pass-relative}
compares the pass band relative error amplitude responses and
Figure~\ref{fig:Comp-schurOneMlattice-lowpass-diff-R2-12-nbits-Schur-FIR-stop}
compares the stop band error amplitude responses.

\begin{figure}
\centering
\scalebox{\DesignOfSchurLatticeFiltersPdfScale}{\input{comparison_no_POP_schurOneMlattice_lowpass_differentiator_R2_12_nbits_test_Schur_FIR_pass_error}}
\caption{Comparison of the pass band amplitude error responses of a low pass
  differentiator filter implemented as a maximally linear FIR filter truncated
  to $16$ distinct coefficients and a tapped Schur one-multiplier lattice, R=2,
  correction filter. The coefficients are SOCP-relaxation and branch-and-bound
  optimised for $12$-bit coefficients with an average of
  $3$-signed-digits each, allocated by the method of \emph{Lim et al.}\;.}
\label{fig:Comp-schurOneMlattice-lowpass-diff-R2-12-nbits-Schur-FIR-pass}
\end{figure}

\begin{figure}
\centering
\scalebox{\DesignOfSchurLatticeFiltersPdfScale}{\input{comparison_no_POP_schurOneMlattice_lowpass_differentiator_R2_12_nbits_test_Schur_FIR_pass_relative_error}}
\caption{Comparison of the pass band amplitude relative error responses of a low
  pass differentiator filter implemented as a maximally linear FIR filter
  truncated to $16$ distinct coefficients and a tapped Schur one-multiplier
  lattice, R=2, correction filter. The coefficients are SOCP-relaxation and
  branch-and-bound optimised for $12$-bit coefficients with an average of
  $3$-signed-digits each, allocated by the method of \emph{Lim et al.}\;.}
\label{fig:Comp-schurOneMlattice-lowpass-diff-R2-12-nbits-Schur-FIR-pass-relative}
\end{figure}

\begin{figure}
\centering
\scalebox{\DesignOfSchurLatticeFiltersPdfScale}{\input{comparison_no_POP_schurOneMlattice_lowpass_differentiator_R2_12_nbits_test_Schur_FIR_stop}}
\caption{Comparison of the stop band amplitude responses of a low pass
  differentiator filter implemented as a maximally linear FIR filter truncated
  to $16$ distinct coefficients and a tapped Schur one-multiplier lattice, R=2,
  correction filter. The coefficients are SOCP-relaxation and branch-and-bound
  optimised for $12$-bit coefficients with an average of
  $3$-signed-digits each, allocated by the method of \emph{Lim et al.}\;.}
\label{fig:Comp-schurOneMlattice-lowpass-diff-R2-12-nbits-Schur-FIR-stop}
\end{figure}

\clearpage
\section{Summary}
The Schur lattice digital filter has a simple stability constraint. This article
demonstrates the realisation of an IIR digital filter transfer function with
stability and frequency response constraints by optimising the floating point
and fixed point coefficients of a tapped one-multiplier Schur lattice digital
filter. The performance of the tapped, R=2, Schur lattice filter is compared
to that of a maximally-flat anti-symmetric FIR low pass differentiator filter.
The tapped, R=2, Schur lattice filter has lower group delay and better pass band
relative amplitude error. The truncated anti-symmetric FIR filter has better
stop band amplitude suppression and, by construction, constant phase and group
delay.

\bibliographystyle{plain}
\bibliography{DesignOfSchurLatticeFilters}

\end{document}
'
  \end{comment}
  \usepackage[hidelinks,colorlinks=false]{hyperref}
\else
  \usepackage[unicode=true,pdfusetitle,bookmarks=true,backref=page,
              bookmarksnumbered=false,bookmarksopen=false,breaklinks=false,
              pdfborder={1 0 0},colorlinks=true]{hyperref}
  \usepackage[all]{hypcap} % Needed to help hyperlinks direct correctly
\fi

\setlength{\parindent}{0em}
\setlength{\parskip}{\bigskipamount}
\floatstyle{ruled}
\newfloat{algorithm}{tbp}{loa}
\floatname{algorithm}{Algorithm}

\newcommand{\DesignOfSchurLatticeFiltersPdfScale}{1}
\newcommand{\DesignOfSchurLatticeFiltersIncludeScale}{0.85}

\newcommand{\coloneq}{\mathrel{\resizebox{\widthof{$\mathord{=}$}}
    {\height}{ $\!\!\resizebox{1.2\width}{0.8\height}
      {\raisebox{0.23ex}{$\mathop{:}$}}\!\!=\!\!$ }}}

\DeclareMathOperator{\sinc}{sinc}
\DeclareMathOperator{\jsn}{sn}
\DeclareMathOperator{\jns}{ns}
\DeclareMathOperator{\jcn}{cn}
\DeclareMathOperator{\jnc}{nc}
\DeclareMathOperator{\jdn}{dn}
\DeclareMathOperator{\jnd}{nd}
\DeclareMathOperator{\jsd}{sd}
\DeclareMathOperator{\jds}{ds}
\DeclareMathOperator{\jcd}{cd}
\DeclareMathOperator{\jdc}{dc}
\DeclareMathOperator{\jsc}{sc}
\DeclareMathOperator{\jcs}{cs}
\DeclareMathOperator{\jam}{am}
\DeclareMathOperator{\jarcsn}{arcsn}
\DeclareMathOperator{\jarcsc}{arcsc}
\DeclareMathOperator{\jarccn}{arccn}
\DeclareMathOperator{\jarcdn}{arcdn}
\DeclareMathOperator{\jarccd}{arccd}
\DeclareMathOperator{\jarccs}{arccs}
\DeclareMathOperator{\cosec}{cosec}
\DeclareMathOperator{\cotan}{cotan}
\DeclareMathOperator{\sech}{sech}
\DeclareMathOperator{\arctanh}{arctanh}
\DeclareMathOperator{\arccosh}{arccosh}
\DeclareMathOperator{\sign}{sign}
\DeclarePairedDelimiter{\ceil}{\lceil}{\rceil}
\DeclarePairedDelimiter{\floor}{\lfloor}{\rfloor}
\DeclarePairedDelimiter{\nint}{\lfloor}{\rceil}
\DeclarePairedDelimiter{\mathabs}{\lvert}{\rvert}
\DeclarePairedDelimiter{\mathnorm}{\lVert}{\rVert}
\DeclareMathOperator{\mathnull}{null}
\DeclareMathOperator{\mathspan}{span}
\DeclareMathOperator{\mathrank}{rank}
\DeclareMathOperator{\mathrange}{range}
\DeclareMathOperator{\mathtrace}{trace}
\DeclareMathOperator{\mathadj}{adj}
\DeclareMathOperator{\mathdiag}{diag}
\DeclareMathOperator{\mathtpltz}{Toeplitz}
\DeclareMathOperator{\mathcohull}{cohull}
\DeclareMathOperator{\mathcone}{cone}
\DeclareMathOperator{\mathri}{ri}
\DeclareMathOperator{\mathaff}{aff}
\DeclareMathOperator{\mathconv}{conv}
\DeclareMathOperator{\mathhe}{He}
\DeclareMathOperator{\mathsq}{Sq}
\DeclareMathOperator{\mathvec}{vec}
\DeclareMathOperator{\mathconj}{\ast} % {\textsf{\tiny{H}}}
\DeclareMathOperator{\mathreverse}{\textsf{\tiny{F}}}

\title{On the design of Schur lattice digital filters}
\author{Robert G. Jenssen}
\date{\today}

\begin{document}

\maketitle

\begin{abstract}
  This article describes a computer-aided procedure for approximating an
  arbitrary transfer function by a tapped one-multiplier all pass lattice
  filter with integer coefficients. The all pass lattice has a simple
  stability criterion, that the lattice coefficients have magnitude less
  than unity, and, in addition, has good coefficient sensitivity and round-off
  noise properties. The procedure consists of finding an initial filter
  transfer function, performing constrained minimum-mean-squared-error
  optimisation of the filter response and searching for the fixed point
  filter coefficients.
\end{abstract}

\section{Introduction}

The Infinite Impulse Response (IIR) digital filter transfer function is
a rational polynomial:
\begin{align}
  F\left(z\right) = \frac{B_{N}\left(z\right)}{A_{N}\left(z\right)}
  &= \frac{b_{0}+b_{1}z^{-1}+\cdots + b_{N}z^{-N}}
    {1+a_{1}z^{-1} + \cdots + a_{N}z^{-N}}
\label{eqn:Transfer-function-F}
\end{align}
Here $z$ is a complex number and the response of the filter with respect to
angular frequency, $\omega$, is $F\left(e^{\imath\omega}\right)$, where
$\omega=2\pi$ corresponds to the digital sampling frequency. 
This article describes a method for computer-aided design of the frequency
response of $F\left(e^{\imath\omega}\right)$. If the filter has
$A_{N}\left(z\right)=1$ then it is a Finite Impulse Response (FIR) filter and
the optimisation problem is convex, has no constraints on the coefficients and
is solved by well-known algorithms that minimise the peak error or the
least-squared-error~\cite{Medlin_LagrangeMultiplierFIRMultirate,SelesnickBurrus_ExchangeAlgorithmsLinearPhaseFIRFilters,SelesnickLangBurrus_ConstrainedLeastSquareFIRFilters,SelesnickLangBurrus_ConstrainedLeastSquareMultiBandFIRFilters,McClellanParks_ComputerProgramFIRLinearPhaseFilters,Iwasaki_GeneralizedKYPFrequencyDomainInequalities,Davidson_LMISpectralConstraints,ParksMcClellan_ChebyshevApproxNonRecursiveDigitalFilters}.
Otherwise, the filter is an Infinite Impulse (IIR) filter. An IIR filter is
stable if the roots, $P_{k}$, of the denominator polynomial,
$A_{N}\left(z\right)$, lie within the unit circle, $\left|P_{k}\right|<1$.
\emph{Lang}~\cite{Lang_LSDesignIIRDigitalFiltersPoleRadiusConstraint} applies
Rouch\'{e}'s theorem to constrain the zeros of $A_{N}\left(z\right)$ during
optimisation. \emph{Lu} and
\emph{Hinamoto}~\cite{LuHinamoto_IIRRobustStabilityQuadraticProgramming}
decompose $F\left(z\right)$ into a series connection of second-order sections
and apply the ``stabiliy triangle'' to each section to ensure filter stability
after optimisation. If the transfer function is represented in gain-pole-zero
form:
\begin{align*}
  F\left(z\right)&= G\frac{\prod_{k=1}^{N}\left(z-Z_{k}\right)}
                   {\prod_{k=1}^{N}\left(z-P_{k}\right)}
\end{align*}
where $G$ is the gain factor, the $Z_{k}$ are the roots of $B_{N}\left(z\right)$
and the $P_{k}$ are the roots of $A_{N}\left(z\right)$, then the constraints on
the pole locations of $F\left(z\right)$ are $\left|P_{k}\right|<1$. 
\emph{Deczky}~\cite{Deczky_MinPSynthesisIIRDigitalFilters} and
\emph{Richards}~\cite{Richards_DeczkyRecursiveDecimator} demonstrate the
optimisation of stable IIR filter transfer functions expressed in gain-pole-zero
form.
Lattice filters are an alternative implementation of the IIR filter
transfer function~\cite{VaidyanathanMitra_RobustFilterStructures}.
\emph{Vaidyanathan et
  al.}~\cite{VaidyanathanMitraNuevo_LowSensitivityIIRDigitalFilters,
  RegaliaMitraVaidyanathan_DigitalAllPassFilterVersatileSignalProcessing} show
that a subset of the $F\left(z\right)$ transfer functions can be factored into
the parallel sum of two all pass filters and that the resulting implementation
has good coefficient sensitivity properties. They show lattice filter
implementations of these all pass sections that are ``structurally bounded''
meaning that the section is all pass regardless of the values of the
coefficients. \emph{Gray} and
\emph{Markel}~\cite{GrayMarkel_DigitalLatticeAndLadderFilterSynthesis,
  MarkelGray_AutocorrelationSpeechAnalysis} describe the implementation of
$F\left(z\right)$ as a tapped all pass lattice filter based on the Schur
polynomial decomposition algorithm~\cite{SchurMethodsInOperatorTheory_Kailath,
  SchurMethodsInOperatorTheory_Schur_I}.
The Schur algorithm enables the decomposition of the denominator polynomial of
the transfer function, $F\left(z\right)$, into a set of orthogonal
polynomials. A result of this decompostion is a set of coefficients, often
called ``reflection coefficients'', that have magnitude less than unity
if-and-only-if the poles of the transfer function lie within the unit circle in
the complex plane. Further, the numerator polynomial of $F\left(z\right)$ can be
expressed as a weighted sum of the orthogonal polynomials. These coefficients and
orthogonal polynomials correspond to a tapped all pass lattice filter
implementation of the transfer function that has good round-off noise and
coefficient sensitivity when the coefficients are truncated to integer values.
This article reviews computer-aided techniques for designing tapped Schur
lattice filters with integer coefficients : finding an initial IIR filter
transfer function, constrained minimum-mean-squared-error optimisation of the
filter frequency response in terms of the lattice coefficients and searching for
integer filter coefficients. As an example, I demonstrate the design of a tapped
Schur lattice implementation of a low pass differentiator filter. The
coefficients and frequency response plots for this example are created with
Octave~\cite{Eaton_Octave,Octave_OptimPackage,Octave_SignalPackage}.
The source code for this article is available at
\url{https://github.com/robertgj/DesignOfIIRFilters}.

\section{Schur lattice filters}
\subsection{The Schur polynomial decomposition}
In 1918 \emph{Issai Schur}~\cite{SchurMethodsInOperatorTheory_Schur_I}
published a paper entitled ``\emph{On power series which are bounded in the
  interior of the unit circle}''. From the abstract:
\begin{quotation}
  The continued fraction algorithm introduced here very easily supplies an
  intrinsically important parametric representation for the coefficients of the
  power series to be considered.
\end{quotation}
\emph{Kailath}~\cite{SchurMethodsInOperatorTheory_Kailath} reviews the impact of
this algorithm on modern signal processing. \emph{Gray} and \emph{Markel}
describe the application of a similar algorithm to the autocorrelation analysis
of speech~\cite{MarkelGray_AutocorrelationSpeechAnalysis} and to the synthesis
of lattice digital
filters~\cite{GrayMarkel_DigitalLatticeAndLadderFilterSynthesis}. 
In the following I use the inner product method of \emph{Markel} and
\emph{Gray}~\cite{MarkelGray_AutocorrelationSpeechAnalysis} and
\emph{Parhi}~\cite[Chapter 12 and Appendix
D]{Parhi_VLSIDigitalSignalProcessingSystems}.

Initialise an $N$'th order polynomial as:
\begin{align*}
  \Phi_{N}\left(z\right) &= A_{N}\left(z\right) = \sum_{n=0}^{N}\phi_{n}z^{n}
\end{align*}
and define the reverse polynomial:
\begin{align*}
\hat{\Phi}_{N}\left(z\right)=z^{N}\Phi_{N}\left(z^{-1}\right)
\end{align*}
The transfer function $\hat{\Phi}_{N}\left(z\right)/\Phi_{N}\left(z\right)$ 
represents an all pass filter.
The Schur decomposition forms the polynomial $\Phi_{N-1}\left(z\right)$ as:
\begin{align*}
  \Phi_{N-1}\left(z\right) &= \frac{z^{-1}\left[\Phi_{N}\left(z\right)-
                             k_{N}\hat{\Phi}_{N}\left(z\right)\right]}
                             {s_{N}} \\
\end{align*}
where $s_{N}$ is a scaling constant and
$k_{N}=\phi_{0}/\phi_{N}=\Phi_{N}\left(0\right)/\hat{\Phi}_{N}\left(0\right)$.
The degree of $\Phi_{N-1}\left(z\right)$ is $1$ less than that of
$\Phi_{N}\left(z\right)$ since, by a change of variables, the numerator is:
\begin{align*}
 z^{-1}\left\{ \phi_{N}\Phi_{N}\left(z\right)-\phi_{0}\hat{\Phi}_{N}\left(z\right)\right\}
& = z^{-1}\sum^{N}_{n=0}\left\{\phi_{N}\phi_{n}z^{n}-\phi_{0}\phi_{N-i}z^{n}\right\}\\
& = \sum^{N}_{n=1}\left\{\phi_{N}\phi_{n}-\phi_{0}\phi_{N-i}\right\}z^{n-1}\\
\end{align*}
This degree reduction procedure is continued, resulting in the set of
polynomials $\left\{\Phi_{N}\left(z\right),\Phi_{N-1}\left(z\right),\ldots,
 \Phi_{0}\left(z\right)\right\}$:
\begin{align}
  \Phi_{n-1}\left(z\right) &= \frac{z^{-1}\left\{\Phi_{n}\left(z\right)-
                             k_{n}\hat{\Phi}_{n}\left(z\right)\right\}}
                             {s_{n}}
\label{eq:SchurDegreeReduction}
\end{align}
and
\begin{align*}
  \hat{\Phi}_{n-1}\left(z\right) &= \frac{\left\{\hat{\Phi}_{n}\left(z\right)-
                                         k_{n}\Phi_{n}\left(z\right)\right\}}
                                        {s_{n}}
\end{align*}

\emph{Markel} and \emph{Gray}~\cite[Equations 13a and 13b]
{MarkelGray_AutocorrelationSpeechAnalysis} define an inner product of two
polynomials, $P\left(z\right)$ and $Q\left(z\right)$:
\begin{align*}
  \langle P\left(z\right),Q\left(z\right)\rangle &= \frac{1}{2\pi \imath}
    \ointctrclockwise_{C}\frac{P\left(z\right)Q\left(z^{-1}\right)}
    {A_{N}\left(z\right)A_{N}\left(z^{-1}\right)}\frac{dz}{z}
\end{align*}
where all the zeros of $A_{N}\left(z\right)$ lie within the contour $C$.
Two properties of $\Phi_{n}\left(z\right)$ under this inner product are:
\begin{enumerate}
\item The $\Phi_{n}\left(z\right)$ polynomials are
  orthogonal~\cite[Appendix D]{Parhi_VLSIDigitalSignalProcessingSystems}:
  \begin{align*}
    \langle \Phi_{m}\left(z\right),\Phi_{n}\left(z\right)\rangle = 0
    \;\text{if}\; m \ne n
  \end{align*}
\item For $1\le n \le N$, $\mathabs{k_{n}} < 1$ if-and-only-if the zeros of
  $A_N\left(z\right)$ lie within the unit
  circle~\cite[pages 72 and 74]{MarkelGray_AutocorrelationSpeechAnalysis}
\end{enumerate}
In addition, if $s_{n}=\sqrt{1-k_{n}^{2}}$\;, then the $\Phi_{n}$ polynomials are
orthonormal~\cite[Appendix D]{Parhi_VLSIDigitalSignalProcessingSystems}:
\begin{align*}
  \langle \Phi_{n}\left(z\right),\Phi_{n}\left(z\right)\rangle = 1
\end{align*}

\subsection{The Schur one-multiplier lattice filter}

In Equation~\ref{eq:SchurDegreeReduction}, choose $s_{n}=1-\epsilon_{n}k_{n}$,
where $\epsilon_{n}=\pm 1$, and initialise $\Lambda_{N}\left(z\right)=A_{N}\left(z\right)$. Then:
\begin{subequations}
\begin{align}
  \Lambda_{n-1}\left(z\right) &= \frac{z^{-1}\left\{\Lambda_{n}\left(z\right)-
                             k_{n}\hat{\Lambda}_{n}\left(z\right)\right\}}
                                {1-\epsilon_{n}k_{n}}
  \label{eqn:SchurOneMLambda} \\
  \hat{\Lambda}_{n-1}\left(z\right)
                              &= \frac{\left\{\hat{\Lambda}_{n}\left(z\right)-
                                       k_{n}\Lambda_{n}\left(z\right)\right\}}
                                      {1-\epsilon_{n}k_{n}}
  \label{eqn:SchurOneMLambdaHat}
\end{align}
\end{subequations}
where:
\begin{align*}
  k_{n}=\frac{\Lambda_{n}\left(0\right)}{\hat{\Lambda}_{n}\left(0\right)}
  =\frac{\Phi_{n}\left(0\right)}{\hat{\Phi}_{n}\left(0\right)}
\end{align*}
Rearranging Equation~\ref{eqn:SchurOneMLambdaHat} and substituting it into
Equation~\ref{eqn:SchurOneMLambda} gives:
\begin{align*}
  z\Lambda_{n-1}\left(z\right)
  &=\left(1+\epsilon_{n}k_{n}\right)\Lambda_{n}\left(z\right)-
    k_{n}\hat{\Lambda}_{n-1}\left(z\right)\\
  \hat{\Lambda}_{n}\left(z\right) & = k_{n}\Lambda_{n}\left(z\right)+
              \left(1-\epsilon_{n}k_{n}\right)\hat{\Lambda}_{n-1}\left(z\right)
\end{align*}
Figure~\ref{fig:Schur-one-multiplier-lattice-section} shows the corresponding
\emph{all pass Schur one multiplier lattice} filter section.

\begin{figure}
\centering
\includegraphics{schur_OneMultiplier_lattice_section}
\caption{All pass Schur one multiplier lattice filter section}
\label{fig:Schur-one-multiplier-lattice-section}
\end{figure}

The expansion of the numerator polynomial of the transfer function,
$F\left(z\right)$, in the orthogonal basis, $\Lambda_{n}\left(z\right)$,
is~\cite[Section 12.2.3]{Parhi_VLSIDigitalSignalProcessingSystems}:
\begin{align*}
B_{N}\left(z\right) & = \sum_{n=0}^{N}c_{n}\Lambda_{n}\left(z\right)
\end{align*}

Figure~\ref{fig:Second-order-tapped-Schur-one-multiplier-lattice-filter}
shows the tapped Schur one multiplier lattice filter implementation of a
second order transfer function.

\begin{figure}
\centering
\includegraphics{schur_OneMultiplier_2nd_Order}
\caption{Second-order tapped Schur one-multiplier lattice filter}
\label{fig:Second-order-tapped-Schur-one-multiplier-lattice-filter}
\end{figure}

The relation between $\Lambda_{n}\left(z\right)$ and $\Phi_{n}\left(z\right)$ is:
\begin{align*}
\Lambda_{N}\left(z\right) & = \Phi_{N}\left(z\right)\\
  \Lambda_{n}\left(z\right)
  &= \Phi_{n}\left(z\right)\sqrt{\frac{\left(1+\epsilon_{N}k_{N}\right)
    \left(1+\epsilon_{N-1}k_{N-1}\right)\cdots
    \left(1+\epsilon_{n+1}k_{n+1}\right)}
    {\left(1-\epsilon_{N}k_{N}\right)
    \left(1-\epsilon_{N-1}k_{N-1}\right)\cdots
    \left(1-\epsilon_{n+1}k_{n+1}\right)}}
\end{align*}
The $\Lambda_{n}\left(z\right)$ polynomials are orthogonal but not orthonormal
since:
\begin{align*}
\langle \Lambda_{n}\left(z\right),\Lambda_{n}\left(z\right)\rangle
&= \langle\hat{\Lambda}_{n}\left(z\right),\hat{\Lambda}_{n}\left(z\right)\rangle\\
&=\frac{\left(1+\epsilon_{N}k_{N}\right)\left(1+\epsilon_{N-1}k_{N-1}\right)
  \cdots\left(1+\epsilon_{n+1}k_{n+1}\right)}
  {\left(1-\epsilon_{N}k_{N}\right)\left(1-\epsilon_{N-1}k_{N-1}\right)
  \cdots\left(1-\epsilon_{n+1}k_{n+1}\right)}
\end{align*}

If the input signal is random and white with unit power then the average power
at an internal node, $x_{n}$, is
$\langle\Lambda_{n}\left(z\right),\Lambda_{n}\left(z\right)\rangle$. The
magnitude of $\langle\Lambda_{n}\left(z\right),\Lambda_{n}\left(z\right)\rangle$
can be adjusted by choosing the sign parameters,
$\epsilon_{N},\epsilon_{N-1}\hdots,\epsilon_{n+1}$.
\emph{Gray} and
\emph{Markel}~\cite[p. 496]{GrayMarkel_DigitalLatticeAndLadderFilterSynthesis}
suggest that one criterion for choosing the sign parameters is to require that
the node associated with the largest $k_{n}$ parameter in magnitude have the
largest amplitude. The sign parameters are found recursively by requiring that
the amplitudes at other nodes be as large as possible without exceeding the
maximum value. If the maximum occurs for $k_{l}$ then the recursion proceeds
for $n=l-1,l-2,\ldots,0$ and again for $n=l+1,l+2,\ldots,N$. The
recursion is simple because:
\begin{align*}
  \frac{\langle \Lambda_{n}\left(z\right),\Lambda_{n}\left(z\right)\rangle}
       {\langle \Lambda_{n+1}\left(z\right),\Lambda_{n+1}\left(z\right)\rangle}
  &= \frac{1+\epsilon_{n+1}k_{n+1}}{1-\epsilon_{n+1}k_{n+1}}
\end{align*}
By changing the sign parameter, this ratio can always be made smaller or larger 
than one. Algorithm~\ref{alg:One-multiplier-sign-assignment} shows the method
used to assign the sign parameters described by \emph{Gray} and
\emph{Markel}~\cite[p. 496]{GrayMarkel_DigitalLatticeAndLadderFilterSynthesis}.

\begin{algorithm}
Assume that $k_{l}$ has the largest magnitude of the $k_{n}$ for
$n=1,2,\ldots,N$. Define the quantities
\begin{align*}
  Q_{n}&=\frac{\langle\Lambda_{n}\left(z\right),\Lambda_{n}\left(z\right)\rangle}
             {\langle\Lambda_{l}\left(z\right),\Lambda_{l}\left(z\right)\rangle}\\
  q_{n}&=\frac{1+\left|k_{n}\right|}{1-\left|k_{n}\right|}
\end{align*}

so that $Q_{l}=1$. Each $Q_{n}$ should be as large as possible without
exceeding $Q_{l}$. Successive ratios are:
\begin{align}
\frac{Q_{n}}{Q_{n+1}} = \begin{cases}
q_{n} & if\,\epsilon_{n}=\phantom{-}\sign\left(k_{n}\right)\\
1/q_{n} & if\,\epsilon_{n}=-\sign\left(k_{n}\right)
\end{cases}\label{eq:OneMultRecursiveKCalc}
\end{align}

Now assign the $\epsilon_{n}$:
\begin{algorithmic}
\For {$n=l-1,l-2,\hdots,1$}
  \If {$Q_{n+1}<1/q_{n}$}
    \State $\epsilon_{n}=-\sign\left(k_{n}\right)$
  \Else
    \State $\epsilon_{n}=\sign\left(k_{n}\right)$
  \EndIf
\EndFor
\For {$n=l+1,l+2,\hdots,N$}
  \If {$Q_{n}<1/q_{n}$}
    \State $\epsilon_{n}=\sign\left(k_{n}\right)$
  \Else
    \State $\epsilon_{n}=-\sign\left(k_{n}\right)$
  \EndIf
\EndFor
\end{algorithmic}
\caption{One multiplier lattice sign
  assignment~\cite[p. 496]{GrayMarkel_DigitalLatticeAndLadderFilterSynthesis}.} 
\label{alg:One-multiplier-sign-assignment}
\end{algorithm}

\subsection{State variable descriptions of the Schur one multiplier lattice filter}
The state variable representation of a filter is:
\begin{align}
\left[\begin{array}{c}
\mathbf{x}^{\prime}\\
\mathbf{y}\end{array}\right] &= \left[\begin{array}{cc}
    \mathbf{A} & \mathbf{B} \\
    \mathbf{C} & \mathbf{D} \\
\end{array}\right]\left[\begin{array}{c}
\mathbf{x}\\
\mathbf{u}\end{array}\right]
\label{eqn:State-variable-representation}
\end{align}
where $\mathbf{u}$ is the input, $\mathbf{x}$ is the filter state, for
convenience $\mathbf{x}^{\prime}=z\,\mathbf{x}$, and $\mathbf{y}$ is the filter
output. $\mathbf{u}$ and $\mathbf{y}$ may be vectors. $\mathbf{A}$ is called
the state transition matrix.
Examination of Figure~\ref{fig:Schur-one-multiplier-lattice-section} and
Figure~\ref{fig:Second-order-tapped-Schur-one-multiplier-lattice-filter}
suggests that the calculation of the all pass output, $\hat{y}$, of the
tapped Schur lattice filter can be represented as a matrix product:

\begin{align*}
\left[\begin{array}{c}
\hat{y}_{0}\\
x_{1}\\
x_{2}\\
\vdots\\
x_{N-1}\\
u \end{array}\right] & = \left[\begin{array}{cccccc}
1 & 0 & \cdots &  & \cdots & 0\\
0 & 1 &  &  &  & \vdots\\
\vdots &  & \ddots\\
 &  &  & \ddots &  & \vdots\\
0 &  &  &  & 1 & 0\\
0 & \cdots &  & \cdots & 0 & 1
\end{array}\right]\left[\begin{array}{c}
x_{0}\\
x_{1}\\
x_{2}\\
\vdots\\
x_{N-1}\\
u
\end{array}\right]
\end{align*}
\begin{align*}
\left[\begin{array}{c}
x_{0}^{\prime}\\
\hat{y}_{1}\\
x_{2}\\
\vdots\\
x_{N-1}\\
u
\end{array}\right] & = \left[\begin{array}{cccccc}
-k_{1} & \left(1+k_{1}\epsilon_{1}\right) & 0 & \cdots & \cdots & 0\\
\left(1-k_{1}\epsilon_{1}\right) & k_{1} & 0 &  &  & \vdots\\
0 & 0 & 1\\
\vdots &  &  & \ddots &  & \vdots\\
0 &  &  &  & 1 & 0\\
0 & \cdots &  & \cdots & 0 & 1
\end{array}\right]\left[\begin{array}{c}
\hat{y}_{0}\\
x_{1}\\
x_{2}\\
\vdots\\
x_{N-1}\\
u
\end{array}\right]
\end{align*}
\begin{align*}
\left[\begin{array}{c}
x_{0}^{\prime}\\
x_{1}^{\prime}\\
\hat{y}_{2}\\
\vdots\\
x_{N-1}\\     
u
\end{array}\right] & = \left[\begin{array}{cccccc}
1 & 0 & 0 & 0 & \cdots & 0\\
0 & -k_{2} & \left(1+k_{2}\epsilon_{2}\right) & 0 &  & \vdots\\
0 & \left(1-k_{2}\epsilon_{2}\right) & k_{2} & 0\\
\vdots &  &  & \ddots &  & \vdots\\
0 &  &  &  & 1 & 0\\
0 & \cdots &  & \cdots & 0 & 1
\end{array}\right]\left[\begin{array}{c}
x_{0}^{\prime}\\
\hat{y}_{1}\\
x_{2}\\
\vdots\\
x_{N-1}\\
u
\end{array}\right]
\end{align*}
\begin{align*}
  \vdots\\
\end{align*}
\begin{align*}
\left[\begin{array}{c}
x_{0}^{\prime}\\
x_{1}^{\prime}\\
\vdots\\
x_{N-2}^{\prime}\\
\hat{y}_{N-1}\\
u
\end{array}\right] & = \left[\begin{array}{cccccc}
1 & 0 & \cdots &  & \cdots & 0\\
0 & 1 &  &  &  & \vdots\\
\vdots &  & \ddots &  &  & \vdots\\
0 &  &  & -k_{N-1} & \left(1+k_{N-1}\epsilon_{N-1}\right) & 0\\
0 &  &  & \left(1-k_{N-1}\epsilon_{N-1}\right) & k_{N-1} & 0\\
0 & \cdots & \cdots & 0 & 0 & 1
\end{array}\right]\left[\begin{array}{c}
x_{0}^{\prime}\\
\vdots\\
x_{N-3}^{\prime}\\
\hat{y}_{N-2}\\
x_{N-1}\\
u
\end{array}\right]
\end{align*}
\begin{align*}
\left[\begin{array}{c}
x_{0}^{\prime}\\
x_{1}^{\prime}\\
x_{2}^{\prime}\\
\vdots\\
x_{N-1}^{\prime}\\
\hat{y}
\end{array}\right] & = \left[\begin{array}{cccccc}
1 & 0 & \cdots &  & \cdots & 0\\
0 & 1 &  &  &  & \vdots\\
\vdots &  & \ddots & & & \vdots\\
0 &  &  & 1 & 0 & 0\\
0 &  &  &  0 & -k_{N} & \left(1+k_{N}\epsilon_{N}\right)\\
0 & \cdots &  & 0 & \left(1-k_{N}\epsilon_{N}\right) & k_{N}
\end{array}\right]\left[\begin{array}{c}
x_{0}^{\prime}\\
x_{1}^{\prime}\\
\vdots\\
x_{N-2}^{\prime}\\
\hat{y}_{N-1}\\
u
\end{array}\right]
\end{align*}
The construction of the state variable description of the Schur one multiplier
lattice filter is summarised in
Algorithm~\ref{alg:Construction-state-variable-One-Multiplier-lattice}.

\begin{algorithm}
Given $\left\{k_{1},k_{2},\hdots,k_{N}\right\}$,
$\left\{\epsilon_{1},\epsilon_{2},\hdots,\epsilon_{N}\right\}$ and
$\left\{c_{0},c_{1},\hdots,c_{N}\right\}$:
\begin{algorithmic}
\State $\hat{y}_{0} = x_{0}$
\For{$n=1,\hdots,N-1$}
  \State $x_{n-1}^{\prime}=-k_{n}\hat{y}_{n-1}+\left(1+k_{n}\epsilon_{n}\right)x_{n}$
  \State $\hat{y}_{n}= \left(1-k_{n}\epsilon_{n}\right)\hat{y}_{n-1}+k_{n}x_{n}$
\EndFor
\State $x_{N-1}^{\prime} = -k_{N}\hat{y}_{N-1}+\left(1+k_{N}\epsilon_{N}\right)u$
\State $\hat{y} = \left(1-k_{N}\epsilon_{N}\right)\hat{y}_{N-1}+k_{N}u$
\State $y  = c_{0}x_{0}+c_{1}x_{1}+\cdots+c_{N-1}x_{N-1}+c_{N}u$
\end{algorithmic}
\caption{Construction of a state variable description of the Schur
  one multiplier lattice filter.}
\label{alg:Construction-state-variable-One-Multiplier-lattice}
\end{algorithm}

The state variable matrix calculations for a second order tapped Schur one
muliplier lattice filter are:
\begin{align*}
  \left[\begin{array}{c}
    x_{0}^{\prime} \\
    x_{1}^{\prime} \\
    \hat{y} \\
    y\end{array}\right] &= 
  \left[\begin{array}{ccc}- k_{1} & 1+\epsilon_{1} k_{1} & 0\\
-k_{2}\left(1-\epsilon_{1}k_{1}\right) & -k_{1}k_{2}& 1+\epsilon_{2}k_{2} \\
 \left(1-\epsilon_{1}k_{1}\right)\left(1-\epsilon_{2}k_{2}\right) &
   k_{1}\left(1-\epsilon_{2}k_{2}\right) & k_{2} \\
 c_{0} & c_{1} & c_{2} \end{array}\right] \left[\begin{array}{c}
    x_{0} \\
    x_{1} \\
    u \end{array} \right]
\end{align*}
The latency in the arithmetic calculation of $y$ and $\hat{y}$
increases with the filter order. \emph{Parhi}~\cite[Chapter 4 and Section 12.8]
{Parhi_VLSIDigitalSignalProcessingSystems} describes the retiming and pipelining
of lattice filters for reduced latency. One method is to design the lattice
filter with coefficients of the denominator polynomial, $A_{N}\left(z\right)$,
only in $z^{-2n}$, where $N$ is even and $n=1,\hdots,\frac{N}{2}$.
\emph{Deczky}~\cite{Deczky_MinPSynthesisIIRDigitalFilters} and
\emph{Richards}~\cite{Richards_DeczkyRecursiveDecimator} describe such filters
as $R=2$.
Figure~\ref{fig:Fourth-order-tapped-Schur-one-multiplier-lattice-filter-R2}
shows such a fourth order tapped Schur one multiplier lattice filter with and
without pipelining.

\begin{figure}
\begin{subfigure}[b]{\textwidth}
\centering
\includegraphics[scale=\DesignOfSchurLatticeFiltersIncludeScale]{schur_OneMultiplier_4th_Order_R2}
\caption{Without pipelining}
\label{subfig:Fourth-order-tapped-Schur-one-multiplier-lattice-R2-without}
\end{subfigure}
\begin{subfigure}[b]{\textwidth}
\centering
\includegraphics[scale=\DesignOfSchurLatticeFiltersIncludeScale]{schur_OneMultiplier_4th_Order_R2_Pipelined}
\caption{With pipelining}
\label{subfig:Fourth-order-tapped-Schur-one-multiplier-lattice-R2-with}
\end{subfigure}
\caption{Fourth-order tapped Schur one-multiplier lattice filter with denominator
  coefficients only for $z^{-2n}$}
\label{fig:Fourth-order-tapped-Schur-one-multiplier-lattice-filter-R2}
\end{figure}

The state variable matrix calculations for a pipelined fourth order tapped Schur
one-muliplier lattice filter with denominator polynomial coefficients only for
$z^{-2n}$ are:

\begin{align*}
  \left[\begin{array}{c}
    x_{0}^{\prime} \\
    x_{1}^{\prime} \\
    x_{2}^{\prime} \\
    x_{3}^{\prime} \\
    x_{4}^{\prime} \\
    \hat{y} \\
    y\end{array}\right] &= 
  \left[\begin{array}{cccccc} 0 & 1 & 0 & 0 & 0 & 0\\
    -k_{2} & 0 & 0 & 0 & 1+\epsilon_{2}k_{2} & 0\\
    1-\epsilon_{2}k_{2} & 0 & 0 & 0 & k_{2} & 0\\
    c_{0} & c_{1} & 0 & 0 & c_{2} & 0 \\
    0 & 0 & -k_{4} & 0 & 0 & 1+\epsilon_{4}k_{4}\\
    0 & 0 & 1-\epsilon_{4}k_{4}  & 0 & 0 & k_{4} \\
    0 & 0 & 0 & 1 & c_{3} & c_{4}\end{array}\right] \left[\begin{array}{c}
    x_{0} \\
    x_{1} \\
    x_{2} \\
    x_{3} \\
    x_{4} \\
    u \end{array} \right]
\end{align*}

The construction of the state variable description of the pipelined tapped Schur
one multiplier lattice filter with denominator polynomial coefficients only for
$z^{-2n}$ is summarised in
Algorithm~\ref{alg:State-variable-pipelined-One-Multiplier-R2-lattice}.

\begin{algorithm}
Given $N$ even, $\left\{k_{2},k_{4}\hdots,k_{N}\right\}$,
$\left\{\epsilon_{2},\epsilon_{4},\hdots,\epsilon_{N}\right\}$ and
$\left\{c_{0},c_{1},\hdots,c_{N}\right\}$:
\begin{algorithmic}
\State $x^{\prime}_{0}=x_{1}$
\State $x^{\prime}_{1}=-k_{2}x_{0}+\left(1+k_{2}\epsilon_{2}\right)x_{4}$
\State $x^{\prime}_{2}=\left(1-k_{2}\epsilon_{2}\right)x_{0}+k_{2}x_{4}$
\State $x^{\prime}_{3}=c_{0}x_{0}+c_{1}x_{1}+c_{2}x_{4}$

\For{$n=2,\hdots,\frac{N}{2}-1$}
\State $x^{\prime}_{3n-2}=-k_{2n}x_{3n-4}+\left(1+k_{2n}\epsilon_{2n}\right)x_{3n+1}$
\State $x^{\prime}_{3n-1}=\left(1-k_{2n}\epsilon_{2n}\right)x_{3n-4}+ k_{2n}x_{3n+1}$
\State $x^{\prime}_{3n}=x_{3n-3}+c_{2n-1}x_{3n-2}+c_{2n}x_{3n+1}$
\EndFor

\State $x_{3\frac{N}{2}-2}^{\prime} = -k_{N}x_{3\frac{N}{2}-4}+
\left(1+k_{N}\epsilon_{N}\right)u$
\State $\hat{y} = \left(1-k_{N}\epsilon_{N}\right)x_{3\frac{N}{2}-4}+k_{N}u$
\State $y = x_{3\frac{N}{2}-3}+c_{N-1}x_{3\frac{N}{2}-2}+c_{N}u$
\end{algorithmic}
\caption{Construction of a state variable description of the pipelined tapped
  Schur one multiplier lattice filter with denominator polynomial coefficients
  only for $z^{-2n}$.}
\label{alg:State-variable-pipelined-One-Multiplier-R2-lattice}
\end{algorithm}

\subsection{Frequency response of the tapped Schur one multiplier lattice filter\label{sec:Frequency-response-of-tapped-Schur-one-multiplier}}
Eliminating the filter state, $\mathbf{x}$, from
Equation~\ref{eqn:State-variable-representation} gives:
\begin{align}
\mathbf{F}\left(z\right) &= 
    \mathbf{C}\left(z\mathbf{I}-\mathbf{A}\right)^{-1}\mathbf{B}+\mathbf{D}
\label{eqn:State-variable-response}
\end{align}
\emph{Thiele}~\cite{Theile_SensitivityLinearStateSpaceSystems} uses the identity:
\begin{align*}
  \mathbf{R}\mathbf{R}^{-1} &= \mathbf{I} \\
\frac{\partial\mathbf{R}}{\partial\chi}\mathbf{R}^{-1} 
  +\mathbf{R}\frac{\partial\mathbf{R}^{-1}}{\partial\chi} &= 0 \\
  \frac{\partial\mathbf{R}}{\partial\chi} 
    &=-\mathbf{R}\frac{\partial\mathbf{R}^{-1}}{\partial\chi}\mathbf{R}
\end{align*}
\emph{Theile} shows the sensitivity functions of $\mathbf{F}\left(z\right)$ with
respect to the components $\alpha$, $\beta$, $\gamma$ and $\delta$ of
$\mathbf{A}$, $\mathbf{B}$, $\mathbf{C}$ and $\mathbf{D}$ respectively:
\begin{align*}
\frac{\partial\mathbf{F}}{\partial{}z} &= 
-\mathbf{C}\left(z\mathbf{I}-\mathbf{A}\right)^{-2}\mathbf{B} \\
  \frac{\partial\mathbf{F}}{\partial\alpha} &= 
\mathbf{C}\left(z\mathbf{I}-\mathbf{A}\right)^{-1}\frac{\partial\mathbf{A}}{\partial\alpha}
\left(z\mathbf{I}-\mathbf{A}\right)^{-1}\mathbf{B} \\
\frac{\partial\mathbf{F}}{\partial\beta} &= 
\mathbf{C}\left(z\mathbf{I}-\mathbf{A}\right)^{-1}\\
\frac{\partial\mathbf{F}}{\partial\gamma} &= 
\left(z\mathbf{I}-\mathbf{A}\right)^{-1}\mathbf{B}\\
\frac{\partial\mathbf{F}}{\partial\delta} &= \mathbf{I}
\end{align*}
Substituting $z=e^{\imath\omega}$ into Equation~\ref{eqn:State-variable-response}
gives the complex frequency response of the state variable filter on the unit
circle. In the following I will use
$\mathbf{R}=\left(e^{\imath\omega}\mathbf{I}-\mathbf{A}\right)^{-1}$. The
components $\alpha$ of $\mathbf{A}$, etc. may themselves be functions of other
variables (for example, the $c$ and $k$ coefficients of a tapped Schur one
multiplier lattice filter) that I will represent by $\chi$. The sensitivity
functions of the complex frequency response
$\mathbf{F}\left(e^{\imath\omega}\right)$ with respect to $\omega$ and $\chi$ are:
\begin{align*}
  \frac{\partial\mathbf{R}}{\partial{}\omega}
  =& -\imath{}e^{\imath\omega}\mathbf{R}\mathbf{R}\\
\frac{\partial\mathbf{F}}{\partial\omega}=&
-\imath e^{\imath\omega}\mathbf{C}\mathbf{R}\mathbf{R}\mathbf{B} \\
  \frac{\partial\mathbf{F}}{\partial{}\chi}=
   &\frac{\partial\mathbf{C}}{\partial\chi}\mathbf{R}\mathbf{B}+
\mathbf{C}\mathbf{R}\frac{\partial\mathbf{A}}{\partial\chi}\mathbf{R}\mathbf{B}+
\mathbf{C}\mathbf{R}\frac{\partial\mathbf{B}}{\partial\chi}+
     \frac{\partial\mathbf{D}}{\partial\chi}\\
\frac{\partial{}^{2}\mathbf{F}}{\partial\chi\partial\omega}=&
-\imath{}e^{\imath\omega}\left[
\frac{\partial\mathbf{C}}{\partial\chi}\mathbf{R}\mathbf{R}\mathbf{B}+
\mathbf{C}\mathbf{R}\mathbf{R}\frac{\partial\mathbf{A}}{\partial\chi}\mathbf{R}\mathbf{B}+
\mathbf{C}\mathbf{R}\frac{\partial\mathbf{A}}{\partial\chi}\mathbf{R}\mathbf{R}\mathbf{B}+
\mathbf{C}\mathbf{R}\mathbf{R}\frac{\partial\mathbf{B}}{\partial\chi}\right]
\end{align*}

The complex frequency response of the filter is:
\begin{align*}
\mathbf{F} &=\Re\mathbf{F} + \imath\Im\mathbf{F}
\end{align*}

The squared-magnitude response of the filter is:
\begin{align*}
\left|\mathbf{F}\right|^{2} &=\Im \mathbf{F}^{2} + \Re \mathbf{F}^{2}
\end{align*}

The gradients of the squared-magnitude response of the filter with respect to 
the Schur lattice filter coefficients are:
\begin{align*}
\frac{\partial\left|\mathbf{F}\right|^{2}}{\partial\chi}
&= 2\left[\Im \mathbf{F}\Im\frac{\partial\mathbf{F}}{\partial\chi}+
          \Re \mathbf{F}\Re\frac{\partial\mathbf{F}}{\partial\chi}\right]
\end{align*}

The gradient of the squared-magnitude response of the filter with respect to 
the angular frequency is:
\begin{align*}
\frac{\partial\left|\mathbf{F}\right|^{2}}{\partial\omega}
&= 2\left[\Im \mathbf{F}\Im\frac{\partial{}\mathbf{F}}{\partial\omega}+
          \Re \mathbf{F}\Re\frac{\partial{}\mathbf{F}}{\partial\omega}\right]
\end{align*}
so that:
\begin{align*}
  \frac{\partial^{2}  \left|\mathbf{F}\right|^{2}}
  {\partial\omega\partial\chi}
  &= 2\left[ \Im\frac{\partial{}\mathbf{F}}{\partial\chi}
             \Im\frac{\partial{}\mathbf{F}}{\partial\omega}+
    \Im \mathbf{F}\Im\frac{\partial^{2}\mathbf{F}}{\partial\omega\partial\chi}+
    \Re\frac{\partial{}\mathbf{F}}{\partial\chi}
    \Re\frac{\partial{}\mathbf{F}}{\partial\omega}+
    \Re \mathbf{F}\Re\frac{\partial^{2}\mathbf{F}}{\partial\omega\partial\chi}
  \right]
\end{align*}

The phase response, $P$, of the filter is:
\begin{align*}%
P&=\arctan\frac{\Im \mathbf{F}}{\Re \mathbf{F}}
\end{align*}

The gradients of the phase response of the filter with respect to 
the Schur lattice filter coefficients are given by:
\begin{align*}
\left|\mathbf{F}\right|^{2}\frac{\partial{}P}{\partial\chi}&=
\Re \mathbf{F}\Im\frac{\partial{}\mathbf{F}}{\partial\chi}-
\Im \mathbf{F}\Re\frac{\partial{}\mathbf{F}}{\partial\chi}
\end{align*}

The group delay response, $T$, of the filter is:
\begin{align*}
T &= -\frac{\partial P}{\partial\omega}
\end{align*}
so that:
\begin{align*}
\left|\mathbf{F}\right|^{2}T &= -\left[\Re \mathbf{F}\Im\frac{\partial \mathbf{F}}{\partial\omega} -
                              \Im \mathbf{F}\Re\frac{\partial \mathbf{F}}{\partial\omega}\right]
\end{align*}

The gradients of the group delay response with respect to the Schur lattice
filter coefficients are given by:
\begin{align*}
  \frac{\partial\left|\mathbf{F}\right|^{2}}{\partial\chi}T+
  \left|\mathbf{F}\right|^{2}\frac{\partial{}T}{\partial\chi} = 
&-\left[\Re\frac{\partial{}\mathbf{F}}{\partial\chi}
\Im\frac{\partial{}\mathbf{F}}{\partial\omega} +
\Re \mathbf{F} \Im\frac{\partial^{2}\mathbf{F}}{\partial\chi\partial\omega} -
  \Im\frac{\partial{}\mathbf{F}}{\partial\chi}
  \Re\frac{\partial{}\mathbf{F}}{\partial\omega} -
  \Im \mathbf{F}\Re\frac{\partial^{2}\mathbf{F}}
                        {\partial\chi\partial\omega}\right]
\end{align*}

\section{Computer-Aided-Design of Schur lattice filters}
The filter design procedure commences with the specification of the required
filter amplitude, phase and group delay responses. The optimisation of an IIR
frequency response with respect to the coefficients is not a convex problem. I
hope to find a ``good enough'' design rather than a global optimum.

One formulation of the filter optimisation problem is to minimise the
\emph{weighted squared error} of the frequency response of $F\left(z\right)$:
\begin{align}
\begin{split}
  \textbf{minimise}\quad &\mathcal{E}_{F}\left(\chi\right) =
  \int W\left(\omega\right)\left|F\left(\omega\right)
    -F_{d}\left(\omega\right)\right|^{2}d\omega \\
\textbf{subject to}\quad &F\left(\omega\right)\text{ is stable} 
\end{split}
\label{eqn:Definition-of-squared-response-error}
\end{align}
where $\chi$ is the coefficient vector of the filter, $F\left(z\right)$,
$\mathcal{E}_{F}\left(\chi\right)$ is the weighted sum of the squared error,
$F\left(\omega\right)$ is the filter frequency response,
$W\left(\omega\right)$ is the frequency weighting and
$F_{d}\left(\omega\right)$ is the desired filter complex frequency response.
The integrand of Equation~\ref{eqn:Definition-of-squared-response-error} could
be the weighted sum of one or more of the squared-magnitude, phase or group
delay responses.

The filter response optimisation problem can also be expressed as a
\emph{mini-max} optimisation problem:
\begin{align}
\label{eqn:Definition-of-minimax-response-error}
\begin{split}
\textbf{minimise}\quad &\max\;\left|F\left(\omega\right)
  -F_{d}\left(\omega\right)\right|\\
\textbf{subject to}\quad &F\left(\omega\right)\text{ is stable} 
\end{split}
\end{align}

I begin by finding an initial filter design that approximates the desired
response, then optimising the minimum mean-squared-error of the
calculated response and, if desired, finding a mini-max approximation to
the desired response. Finally, I describe two methods of
searching for integer valued filter coefficients: a depth-first branch-and-bound
search and a successive relaxation method.

\subsection{Finding an initial filter\label{sec:Finding-initial-filter}}
The initial filter could be an FIR filter or a ``classical'' Butterworth,
Chebyshev, etc. IIR filter design.
\emph{Deczky}~\cite{Deczky_MinPSynthesisIIRDigitalFilters} and
\emph{Richards}~\cite{Richards_DeczkyRecursiveDecimator} begin with an
initial filter consisting of a ``by-eye'' arrangement of the filter poles and
zeros. \emph{Tarczynski et
  al.}~\cite{TarczynskiCainHermanowiczRojewski_WISEMethodDesignIIRFilters}
propose minimising the filter response mean-squared-error, $\mathcal{E}_{F}$,
weighted with a barrier function:
\begin{align}
\label{eqn:WISE-barrier-function}
\left(1-\lambda\right)\mathcal{E}_{F}+\lambda\sum^{T+M}_{t=T+1}h^{2}\left(t\right)
\end{align}
where $\lambda$, $T$ and $M$ are suitable constants and $h\left(t\right)$ is the
impulse response of the filter $H\left(z\right)=\frac{1}{A_{N}\left(z\right)}$.
The barrier function (the second part of
Equation ~\ref{eqn:WISE-barrier-function}) is intended to be small when the
filter, $F\left(z\right)$, is stable and increase rapidly when the poles
approach the unit circle. \emph{Tarczynski et al.} provide heuristics for
selecting $\lambda$, $T$ and $M$. Typically,
$\lambda \in \left[10^{-10},10^{-3}\right]$, $T\in \left[ 100, 500 \right]$ and
$M=NR$. \emph{Roberts and Mullis}~\cite[Section 8.3]
{RobertsMullis_DigitalSignalProcessing} show that the state space description of
the direct-form implementation of $H\left(z\right)$ is:
\begin{align*}
\left[\begin{array}{c}
x\left(t+1\right)\\
y\left(t\right)
\end{array}\right] &= \left[\begin{array}{cc}
A & B\\
C & D
\end{array}\right]\left[\begin{array}{c}
x\left(t\right)\\
u\left(t\right)
\end{array}\right]
\end{align*}
where:
\begin{align*}
A &= \left[\begin{array}{cccc}
0 & 1 & \cdots & 0\\
\vdots & \vdots & \ddots & \vdots\\
0 & 0 & \cdots & 1\\
-a_{N} & -a_{N-1} & \cdots & -a_{1}
\end{array}\right]\\
B &= \left[\begin{array}{ccccc}0 & 0 & \cdots & 0 & 1\end{array}\right]^{\top}\\
C &= \left[\begin{array}{ccc}-a_{N} & \cdots & -a_{1}\end{array}\right]\\
D &= 1
\end{align*}

The corresponding impulse response
is~\cite[Equation 8.3.21]{RobertsMullis_DigitalSignalProcessing}:
\begin{align*}
h\left(t\right) &= \begin{cases}
0 & t<0\\
D & t=0\\
CA^{t-1}B & t>0
\end{cases}
\end{align*}

\subsection{Minimum-Mean-Squared-Error optimisation of IIR filters}
Minimisation of the mean squared error of the filter response proceeds by
successive solution of the following linear approximation problem
for $\Delta_{\chi}$:
\begin{align}
\begin{split}
  \textbf{minimise}\quad & \epsilon+\beta \\
  \textbf{subject to}\quad & \mathnorm{\mathcal{E}_{F}\left(\chi\right) +
          \Delta_{\chi}\nabla\mathcal{E}_{F}\left(\chi\right)}\le\epsilon \\
  & \mathnorm{\Delta_{\chi}}\le\beta \\
  & F\left(\omega\right)\text{ is stable}
\end{split}
\label{eqn:Optimisation-of-minimum-mean-squared-response-error}
\end{align}
After each step the new coefficient vector is $\chi+\Delta_{\chi}$. 

In fact I replace $\mathcal{E}_{F}$ with an approximation to the sum at discrete
frequencies of the weighted squared error of the real valued squared magnitude,
phase, group delay and the gradient of the squared magnitude with respect to
angular frequency. For example, the first order approximation to the squared
error of the group delay, $T$, is:
\begin{align*}
  \bar{\mathcal{E}}_{T}\left(\chi+\Delta_{\chi}\right)
  &\approx \sum_{\omega} W_{T}\left(\omega\right)
     \left[T\left(\chi,\omega\right) + \right.
    \left. \Delta_{\chi}\nabla_{\chi}T\left(\chi,\omega\right) - \right.
    \left. T_{d}\left(\omega\right)\right]^{2}
\end{align*}
with gradient with respect to $\chi$:
\begin{align*}
  \nabla_{\chi} \bar{\mathcal{E}}_{T}\left(\chi\right)
  &\approx 2 \sum_{\omega} W_{T}\left(\omega\right)
    \left[T\left(\chi,\omega\right)-T_{d}\left(\omega\right)\right]
    \nabla_{\chi}T\left(\chi,\omega\right)
\end{align*}
The required gradients of the state variable filter frequency response were
derived in Section~\ref{sec:Frequency-response-of-tapped-Schur-one-multiplier}.

\subsection{Peak-Constrained-Least-Squares optimisation of IIR filters}
Linear phase FIR filter design
algorithms~\cite{Hofstetter_DesignNonRecursiveDigitalFilters,
  ParksMcClellan_ChebyshevApproxNonRecursiveDigitalFilters,
  SelesnickBurrus_ExchangeAlgorithmsLinearPhaseFIRFilters,
  SelesnickLangBurrus_ConstrainedLeastSquareMultiBandFIRFilters} commonly
employ an ``exchange algorithm'' that, given an initial approximation to the
desired amplitude response, finds the extremal frequencies of that response and
interpolates the coefficients to obtain a new set of filter coefficients that
achieves the desired ripple values at all of those frequencies. The process
repeats until a satisfactory mini-max amplitude response is found. I have
successfully solved the IIR mini-max problem of
Equation~\ref{eqn:Definition-of-minimax-response-error} with a modified version
of the ``Peak-Constrained-Least-Squares'' (PCLS) algorithm of \emph{Selesnick},
\emph{Lang} and \emph{Burrus}~\cite[p.498]
{SelesnickLangBurrus_ConstrainedLeastSquareMultiBandFIRFilters}. They describe
the exchange of amplitude response, $A\left(\omega\right)$, constraints as
follows:
\begin{quotation}
  After each iteration, the algorithm checks the values of
  $A\left(\omega\right)$ over the previous constraint set frequencies. $\hdots$
  However, if it is found that $A\left(\omega\right)$ violates the constraints
  at some frequency belonging to the previous constraint set, $R$, then i) that
  frequency where the violation is greatest is appended to the current
  constraint set, $S$, and ii) the same frequency is removed from the
  record of previous constraint set frequencies, $R$.
\end{quotation}

Figure~\ref{fig:Modified-Selesnick-Lang-Burrus-algorithm} shows a flow-chart of
the modified \emph{Selesnick}, \emph{Lang} and \emph{Burrus} exchange algorithm.

\begin{figure}
\centering
\includegraphics[scale=\DesignOfSchurLatticeFiltersIncludeScale]{slb_exchange_algorithm}
\caption{Modified Selesnick, Lang and Burrus exchange algorithm~\cite[p.498]
  {SelesnickLangBurrus_ConstrainedLeastSquareMultiBandFIRFilters}.}
\label{fig:Modified-Selesnick-Lang-Burrus-algorithm}
\end{figure}

Similarly to $\bar{\mathcal{E}}_{T}$, the PCLS constraints for the group delay
response, $T\left(\chi,\omega\right)$, are approximated to first order by:
\begin{align*}
  T_{l}\left(\omega\right) \le
  T\left(\chi,\omega\right) + \Delta_{\chi}\nabla_{\chi}T\left(\chi,\omega\right)
  \le T_{u}\left(\omega\right)
\end{align*}
where $T_{l}\left(\omega\right)$ and $T_{u}\left(\omega\right)$ are the
lower and upper constraints on the group delay.

The modified PCLS mini-max optimisation is:
\begin{align}
\begin{split}
  \textbf{minimise}\quad & \epsilon+\beta \\
  \textbf{subject to}\quad & \mathnorm{\bar{\mathcal{E}}_{F}\left(\chi\right) +
          \Delta_{\chi}\nabla\bar{\mathcal{E}}_{F}\left(\chi\right)}\le\epsilon \\
  & \mathnorm{\Delta_{\chi}}\le\beta \\
  & F\left(\omega\right)\text{ is stable} \\
  & \text{PCLS response constraints are satisfied}
\end{split}
\label{eqn:Optimisation-of-PCLS-response-error}
\end{align}

\subsection{Second Order Cone Programming}
 \emph{Alizadeh} and \emph{Goldfarb}~\cite
{AlizadehGoldfarb_SecondOrderCone} describe Second Order Cone
Programming (SOCP) as the solution of a class of convex optimisation problems in
which a linear function is minimised subject to a set of conic constraints.
In the following example I use the \emph{SeDuMi}
(\emph{Se}lf-\emph{Du}al-\emph{Mi}nimisation) SOCP solver originally written
by \emph{Jos Sturm}~\cite{Sturm_SeDuMi_GitHub}. 
\emph{Lu}~\cite[Section III]{Lu_SedumiRemarks} provides an example of expressing
an optimisation problem of
Equation~\ref{eqn:Optimisation-of-PCLS-response-error} in the form accepted by
SeDuMi. In \emph{Lu}'s notation the problem is:
\begin{subequations}
\begin{align}
\textbf{minimise}  \quad&\boldsymbol{b}^{\top}\boldsymbol{x}
\label{eqn:Lu-SeDuMi-Remarks-problem-format}\\
\textbf{subject to}\quad&\|\boldsymbol{A}_{i}^{\top}\boldsymbol{x}+
\boldsymbol{c}_{i}\| \le \boldsymbol{b}_{i}^{\top}\boldsymbol{x}+d_{i}
\quad \text{for}\;i=1,\hdots,q
\label{eqn:Lu-SeDuMi-Remarks-problem-format-conic-constraints}\\
&\boldsymbol{D}^{\top}\boldsymbol{x}+\boldsymbol{f}\ge\boldsymbol{0}
\label{eqn:Lu-SeDuMi-Remarks-problem-format-linear-constraints}
\end{align}
\end{subequations}
where $\boldsymbol{x}\in\mathfrak{\mathbb{R}}^{m\times 1}$, 
$\boldsymbol{b}\in\mathfrak{\mathbb{R}}^{m\times 1}$,
$\boldsymbol{A}_{i}\in\mathfrak{\mathbb{R}}^{m\times\left(n_{i}-1\right)}$
\footnote{$n_{i}-1$ to allow for the column $b_{i}$ in the matrix 
$\boldsymbol{A}_{t}^{\left(i\right)}$ and $d_{i}$ in the column vector 
$\boldsymbol{c}_{t}^{\left(i\right)}$}, 
$\boldsymbol{c}_{i}\in\mathfrak{\mathbb{R}}^{\left(n_{i}-1\right)\times 1}$,
$\boldsymbol{b}_{i}\in\mathfrak{\mathbb{R}}^{m\times 1}$, 
$d_{i}\in\mathfrak{\mathbb{R}}$ for $1 \le i \le q$,
$\boldsymbol{D}\in\mathfrak{\mathbb{R}}^{m\times p}$ and
$\boldsymbol{f}\in\mathfrak{\mathbb{R}}^{p\times 1}$. The problem is cast into
SeDuMi format by defining:
\begin{align*}
\boldsymbol{A}_{t}&=\left[-\boldsymbol{D} \quad \boldsymbol{A}_{t}^{\left(1\right)}
\hdots \quad \boldsymbol{A}_{t}^{\left(q\right)}\right]\\
\boldsymbol{A}_{t}^{\left(i\right)} &= -\left[\boldsymbol{b}_{i} \quad
\boldsymbol{A}_{i}\right]\nonumber\\
\boldsymbol{b}_{t}&=-\boldsymbol{b}\nonumber\\
\boldsymbol{c}_{t}&=\left[\boldsymbol{f};\quad \boldsymbol{c}_{t}^{\left(1\right)}
;\quad \hdots\quad \boldsymbol{c}_{t}^{\left(q\right)}\right]\nonumber\\
\boldsymbol{c}_{t}^{\left(i\right)}&=\left[d_{i} ;\quad \boldsymbol{c}_{i}\right]
\nonumber
\end{align*}

For Equation~\ref{eqn:Optimisation-of-PCLS-response-error},
$\boldsymbol{x}=\left[\epsilon,\beta,\chi\right]$,
Equation~\ref{eqn:Lu-SeDuMi-Remarks-problem-format-conic-constraints} applies
the conic constraints on the linear approximation to the minimum mean squared
error of the frequency response and 
Equation~\ref{eqn:Lu-SeDuMi-Remarks-problem-format-linear-constraints} applies
the linear constraints on both the reflection coefficients,
$\mathabs{k_{n}}<1$, and the PCLS frequency response constraints.

\section{\label{sec:Design-IIR-lattice-filters-with-signed-digit-coefficients}Design of digital filters with signed-digit coefficients}
\subsection{The canonical signed-digit representation of filter coefficients}

\emph{Parhi}~\cite[Section 13.6]{Parhi_VLSIDigitalSignalProcessingSystems} lists
the following properties of the \emph{canonical} binary signed-digit (CSD)
representation of a binary signed-digit number $A=a_{W-1}a_{W-2}\cdots{}a_{1}a_{0}$
where each $a_{i}\in \left\{-1,0,1\right\}$:
\begin{itemize}
\item no 2 consecutive bits in a CSD number are non-zero 
\item the CSD representation of a number contains the miniumum possible number
of non-zero bits, thus the name \emph{canonic}
\item the CSD representation of a number is unique
\end{itemize}
The first property breaks the carry chain when adding two CSD numbers.
\emph{Parhi}~\cite[Section 13.6.1]{Parhi_VLSIDigitalSignalProcessingSystems}
shows an algorithm that calculates the \emph{canonical} binary signed-digit 
representation from the two's complement representation, reproduced here as
Algorithm~\ref{alg:twos-complement-to-signed-digit-conversion}.

\begin{algorithm}[htbp]
Denote the two's complement representation of the number $A$ as 
$A=\hat{a}_{W-1}\hat{a}_{W-2}\cdots\hat{a}_{1}\hat{a}_{0}$. \\
Denote the CSD representation of $A$ as $A=a_{W-1}a_{W-2}\cdots{}a_{1}a_{0}$.

\begin{algorithmic}
\State $\hat{a}_{-1}=0$
\State $\gamma_{-1}=0$
\State $\hat{a}_{W}=\hat{a}_{W-1}$
\For{$k=0,\hdots,W-1$}
\State $\theta_{i}=\hat{a}_{i}\oplus\hat{a}_{i-1}$
\State $\gamma_{i}=\bar{\gamma}_{i-1}\theta_{i}$
\State $a_{i}=\left(1-2\hat{a}_{i+1}\right)\gamma_{i}$
\EndFor
\end{algorithmic}
\caption{Conversion of 2's complement numbers to the canonical signed-digit
  representation (\emph{Parhi}~\cite[Section
  13.6.1]{Parhi_VLSIDigitalSignalProcessingSystems}).}
\label{alg:twos-complement-to-signed-digit-conversion}
\end{algorithm}

\emph{Lim et al.}~\cite{Lim_SignedPowerOfTwoAllocationDigitalFilters} describe
a method of allocating  a limited number of signed power-of-two digit terms 
to the fixed point coefficients of a digital filter. The method is based 
on the belief that ``\emph{allocating the SPT terms in such a way that all the 
coefficient values have the same quantisation step-size to coefficient
sensitivity ratio will lead to a good design}''. 

\emph{Lim et al.} first prove properties of the canonical signed-digit
representation~\cite[Section II]{Lim_SignedPowerOfTwoAllocationDigitalFilters}.
In particular:

\emph{Property 1}: Define $S_{Q}$ as the set of contiguous integers that can be 
represented by up to $Q$ signed digits. The largest integer in $S_{Q}$ is
$J_{Q}=\sum_{l=0}^{Q-1}2^{2l+1}$.

\emph{Property 5}: On average, $0.72Q$ signed-digits are required to represent
the integers in $S_{Q}$.

\emph{Lim et al.} show that an estimate of the number of signed digits, $Q$, 
required to represent $J_{Q}$ is:
\begin{align*}
  Q \approx \frac{1}{2}\log_{2}J_{Q}+0.31
\end{align*}
Replacing $J_{Q}$ by an integer $n\in S_{Q}$, an estimate of the average number
of terms, $Q_{A}$, required to represent $n$ is:
\begin{align*}
  Q_{A}&\approx 0.72Q\\
      &\approx 0.36 \log_{2}n+0.22
\end{align*}

Now suppose that $D$ signed digits are available to represent two positive
integers $n_{1}$ and $n_{2}$. If $n_{1}\approx n_{2}$ then each integer is
allocated $\frac{D}{2}$ bits. If $n_{1}>n_{2}$ then \emph{Lim et al.} argue 
that the number of additional signed-digits, $Q_{E}$, required to represent 
$n_{1}$ is:
\begin{align*}
  Q_{E}&\approx 0.36\log_{2}\lfloor \frac{n_{1}}{n_{2}}\rfloor
\end{align*}
where $\lfloor\;\rfloor$ represents the integer part. In general, 
$Q_{E}$ is not an integer.

\emph{Lim et al.} go on to consider the allocation of signed digits to the
coefficients of a symmetric FIR filter. The change in the frequency response,
$F\left(\omega\right)$, of a filter due to a change $\Delta\chi_{n}$ in
coefficient $\chi_{n}$ is:
\begin{align*}
\Delta{} F\left(\omega,n\right)&  \approx
\frac{\partial{}F\left(\omega\right)}{\partial\chi_{n}}\Delta{}\chi_{n}
\end{align*}
\emph{Lim et al.} use the average of the coefficient sensitivity to define a
cost for the $n$'th coefficient:
\begin{align*}
cost_{n}&=0.36\log_{2}\left|\chi_{n}\right| + 0.36\log_{2}\int^{\pi}_{0}
\left|\frac{\partial{}F\left(\omega\right)}{\partial{}\chi_{}}\right|d\omega
\end{align*}

Given a total of $D$ signed-digits, \emph{Lim et al.} assign a single 
signed-digit at a time to the coefficient with the largest cost. After a 
coefficient is given a signed-digit, its cost is decreased by one. The
process is repeated until all $D$ digits have been allocated.

\subsection{Branch-and-bound search for the filter coefficients}
Given the $K$ floating point coefficients, $\chi$, of a filter, an exhaustive
search of the upper and lower bounds on the integer or signed-digit
approximations to these coefficients would require
$\mathcal{O}\left(2^{K}\right)$ comparisons of the corresponding filter
approximation error.
\emph{Branch-and-bound}~\cite{LandDoig_AutomaticMethodSolvingDiscreteProgrammingProblems},~\cite[p.627]{Sedgewick_AlgorithmsInCPlusPlus}
is a heuristic for reducing the 
number of branches searched in a binary decision tree. At each branch of the
binary tree the solution is compared to the estimated lower bounds on the cost
of the full path proceeding from that branch. If the cost of that full path is
greater than that of the best full path found so far then further search on
that path is abandoned.  Figure~\ref{fig:branch-bound-tree-search-algorithm}
shows a flow diagram of an implementation of the algorithm using a stack. The
floating point filter coefficients, $\chi$, are approximated by the
signed-digit coefficients, $\bar{\chi}$. Each coefficient, $\chi_{n}$
and $\bar{\chi}_{n}$, is bounded by the corresponding signed-digit numbers $u_{n}$
and $l_{n}$ so that $l_{n}\le\chi_{n}\le{}u_{n}$ and
$l_{n}\le\bar{\chi}_{n}\le u_{n}$. The search for the set of coefficients with
minimum cost is ``depth-first'', starting at the root of the search tree and
fixing successive coefficients.  \emph{Ito et
  al.}~\cite{Ito_PowersOfTwoAllocationFIR} recommend choosing, at each branch,
the $\chi_{n}$ with the greatest difference $u_{n}-l_{n}$. The two sub-problems
at that branch fix $\bar{\chi}_{n}$ to $l_{n}$ and $u_{n}$. One of the two
sub-problems is pushed onto a stack and the other is solved and the cost
calculated. I assume that this cost is the least possible for the remaining
coefficients on the current branch. If the cost of the current sub-problem is
greater than the current minimum cost then the current branch is abandoned and a
new sub-problem is popped off the problem stack. Otherwise, if the search has
reached the maximum depth of the tree then the current solution is a new
signed-digit minimum cost solution. If the current cost is less than the
minimum cost and the search has not reached the maximum depth then the search
continues with a new branch.

\begin{figure}
\centering
\includegraphics[scale=\DesignOfSchurLatticeFiltersIncludeScale]{branch_bound_algorithm}
\caption{Branch-and-bound depth-first search algorithm}
\label{fig:branch-bound-tree-search-algorithm}
\end{figure}

\subsection{Successive relaxation search for the filter coefficients}
A successive relaxation search for the filter coefficients fixes one
coefficient and optimises the filter frequency response with respect to the
remaining free coefficients. This is called a \emph{relaxation} of the
optimisation. The relaxation is repeated until all the coefficients are fixed.
At each coefficient relaxation step the script finds the upper and lower
signed-digit approximations to the current set of active coefficients and
selects the coefficient with the largest difference in those approximations.

\section{An example : design of a low pass differentiator filter}
This section describes the design of a low pass differentiator filter
implemented as $F_{0}\left(z\right)=1-z^{-1}$ in series with a tapped Schur
one-multiplier lattice correction filter, $C\left(z\right)$, having denominator
polynomial coefficients only for terms in $z^{-2n}$. In the pass band, the
desired complex frequency response of a differentiator filter is:
\begin{align*}
  F_{d}\left(\omega\right)=-\imath\frac{\omega}{2}e ^{-\imath\omega t_{p}}
\end{align*}
where $t_{p}$ is the nominal filter pass band group delay in samples.
The squared-magnitude response of the correction filter is:
\begin{align*}
  \mathabs{C\left(\omega\right)}^{2}
  &= \frac{\mathabs{F_{d}\left(\omega\right)}^{2}}
    {\mathabs{F_{0}\left(\omega\right)}^{2}}
\end{align*}
where:
\begin{align*}
  \mathabs{F_{d}\left(\omega\right)} &= \frac{\omega}{2} \\
  \mathabs{F_{d}\left(\omega\right)}^{2}&= \frac{\omega^{2}}{4} \\  
  \frac{d\mathabs{F_{d}\left(\omega\right)}^{2}}{d\omega}
     &= \frac{\omega}{2} = \mathabs{F_{d}\left(\omega\right)}
\end{align*}
and the response of the zero at $z=1$ is:
\begin{align*}
  F_{0}\left(\omega\right) &= 1-e^{\imath\omega} \\
        &=2\imath{} e^{\imath\frac{\omega}{2}}\sin\frac{\omega}{2} \\
  \mathabs{F_{0}\left(\omega\right)} &= 2\sin\frac{\omega}{2} \\
  \frac{d\mathabs{F_{0}\left(\omega\right)}^{2}}{d\omega}
        &=2\mathabs{F_{0}\left(\omega\right)}
          \frac{d\mathabs{F_{0}\left(\omega\right)}}{d\omega} \\
        &= 2\sin\omega
\end{align*}
By the chain rule for differentiation, the gradient of
$\mathabs{F\left(\omega\right)}^{2}$ is:
\begin{align*}
  \frac{d\mathabs{F_{d}\left(\omega\right)}^{2}}{d\omega}
  &= \mathabs{F_{0}\left(\omega\right)}^{2}
     \frac{d\mathabs{C\left(\omega\right)}^{2}}{d\omega} +
    \frac{\mathabs{F_{d}\left(\omega\right)}^{2}}
    {\mathabs{F_{0}\left(\omega\right)}^{2}}
    \frac{d\mathabs{F_{0}\left(\omega\right)}^{2}}{d\omega}
\end{align*}
Substituting and rearranging to obtain the desired gradient of
$\mathabs{C\left(\omega\right)}^{2}$ in terms of the desired overall squared
amplitude response, $\mathabs{F_{d}\left(\omega\right)}^{2}$, and the
squared magnitude response of the zero at $z=1$,
$\mathabs{F_{0}\left(\omega\right)}^{2}$:
\begin{align*}
  \frac{d\mathabs{C\left(\omega\right)}^{2}}{d\omega}
  &= \mathabs{F_{d}\left(\omega\right)}
\left[\frac{1-\mathabs{F_{d}\left(\omega\right)}\cot\frac{\omega}{2}}{\mathabs{F_{0}\left(\omega\right)}^{2} }\right]
\end{align*}

\subsection{Initial R=2 low pass differentiator filter}
Figure~\ref{fig:Schur-OneM-lattice-correction-lowpass-differentiator-R2-initial}
shows the frequency response of the initial R=2 low pass differentiator filter
found by the method of \emph{Tarczynski et al.}, described in
Section~\ref{sec:Finding-initial-filter}. The correction filter order is $N=10$,
filter pass band and stop band edge frequencies are $0.2$ and $0.4$\;(normalised
to the digital sample rate), the nominal pass band phase is $1.5\pi$
radians (adjusted for delay) and the nominal pass band group delay is $9$
samples.

\begin{figure}
\centering
\scalebox{\DesignOfSchurLatticeFiltersPdfScale}{\input{schurOneMlattice_socp_slb_lowpass_differentiator_R2_test_initial_response}}
\caption{Amplitude, phase and group delay responses of an initial 
  low pass differentiator filter composed of a Schur one-multiplier lattice
  correction filter with a denominator polynomial having terms only in $z^{-2}$
  in series with $1-z^{-1}$.}
\label{fig:Schur-OneM-lattice-correction-lowpass-differentiator-R2-initial}
\end{figure}

The coefficients of the numerator and denominator polynomials of the initial
correction filter transfer function are:
\begin{small}
\verbatiminput{schurOneMlattice_socp_slb_lowpass_differentiator_R2_test_N0_coef.m}
\verbatiminput{schurOneMlattice_socp_slb_lowpass_differentiator_R2_test_D0R_coef.m}
\end{small}

The corresponding Schur one-multiplier lattice coefficients of the initial
correction filter are:
\begin{small}
\verbatiminput{schurOneMlattice_socp_slb_lowpass_differentiator_R2_test_k0_coef.m}
\verbatiminput{schurOneMlattice_socp_slb_lowpass_differentiator_R2_test_epsilon0_coef.m}
\verbatiminput{schurOneMlattice_socp_slb_lowpass_differentiator_R2_test_c0_coef.m}
\end{small}

\subsection{PCLS design of the R=2 low pass differentiator filter}
The low pass differentiator correction filter was PCLS optimised with 
amplitude pass band error peak-to-peak ripple of $0.0009$,
amplitude stop band peak ripple of $0.007$,
phase pass band peak-to-peak ripple of $0.0002\pi$\; radians,
group delay pass band peak-to-peak ripple of $0.006$\; samples and
correction filter gradient of amplitude-squared pass band error peak-to-peak
ripple of $0.02$.

The Schur one-multiplier lattice coefficients of the PCLS SOCP optimised
correction filter are:
\begin{small}
\verbatiminput{schurOneMlattice_socp_slb_lowpass_differentiator_R2_test_k2_coef.m}
\verbatiminput{schurOneMlattice_socp_slb_lowpass_differentiator_R2_test_epsilon2_coef.m}
\verbatiminput{schurOneMlattice_socp_slb_lowpass_differentiator_R2_test_c2_coef.m}
\end{small}

The corresponding correction filter numerator and denominator polynomial
coefficients are:
\begin{small}
\verbatiminput{schurOneMlattice_socp_slb_lowpass_differentiator_R2_test_N2_coef.m}
\verbatiminput{schurOneMlattice_socp_slb_lowpass_differentiator_R2_test_D2_coef.m}
\end{small}
Figure~\ref{fig:Schur-OneM-lattice-lowpass-differentiator-R2}
shows the  amplitude error, phase and group delay response errors of the $R=2$
low pass differentiator filter after PCLS SOCP optimisation.
Figure~\ref{fig:Schur-OneM-lattice-lowpass-differentiator-R2-dCsqdw-error}
shows the error in the gradient of the squared-magnitude response of the $R=2$
correction filter after PCLS SOCP optimisation.
Figure~\ref{fig:Schur-OneM-lattice-lowpass-differentiator-R2-pz} shows
the pole-zero plot of the $R=2$ low pass differentiator filter.
Figure~\ref{fig:Schur-OneM-lattice-correction-lowpass-differentiator-direct-sens}
shows the gradients of the correction filter squared-magnitude, phase and delay
with respect to the coefficients of the direct form implementation of the filter.
Figure~\ref{fig:Schur-OneM-lattice-correction-lowpass-differentiator-Schur-sens}
shows the gradients of the correction filter squared-magnitude, phase and delay
with respect to the coefficients of the Schur lattice implementation of the
filter.

\begin{figure}
\centering
\scalebox{\DesignOfSchurLatticeFiltersPdfScale}{\input{schurOneMlattice_socp_slb_lowpass_differentiator_R2_test_pcls_error}}
\caption{PCLS optimised amplitude error, phase and group delay responses of a
  low pass differentiator filter composed of a Schur one-multiplier lattice
  correction filter with a denominator polynomial having terms only in $z^{-2}$
  in series with $1-z^{-1}$.}
\label{fig:Schur-OneM-lattice-lowpass-differentiator-R2}
\end{figure}

\begin{figure}
\centering
\scalebox{\DesignOfSchurLatticeFiltersPdfScale}{\input{schurOneMlattice_socp_slb_lowpass_differentiator_R2_test_pcls_dCsqdw_error}}
\caption{PCLS optimised pass band error of the gradient of the squared magnitude
  response of the correction filter after PCLS SOCP optimisation.}
\label{fig:Schur-OneM-lattice-lowpass-differentiator-R2-dCsqdw-error}
\end{figure}

\begin{figure}
\centering
\scalebox{\DesignOfSchurLatticeFiltersPdfScale}{\input{schurOneMlattice_socp_slb_lowpass_differentiator_R2_test_pcls_pz}}
\caption{Pole-zero plot of the PCLS optimised low pass differentiator
  filter composed of a Schur one-multiplier lattice correction filter with a
  denominator polynomial having terms only in $z^{-2}$ in series with $1-z^{-1}$.}
\label{fig:Schur-OneM-lattice-lowpass-differentiator-R2-pz}
\end{figure}

\begin{figure}
\centering
\scalebox{\DesignOfSchurLatticeFiltersPdfScale}{\input{schurOneMlattice_socp_slb_lowpass_differentiator_R2_test_direct_sensitivity}}
\caption{Gradients of the pass band response with respect to the coefficients of
  the direct form implementation of the PCLS optimised low pass differentiator
  correction filter.}
\label{fig:Schur-OneM-lattice-correction-lowpass-differentiator-direct-sens}
\end{figure}

\begin{figure}
\centering
\scalebox{\DesignOfSchurLatticeFiltersPdfScale}{\input{schurOneMlattice_socp_slb_lowpass_differentiator_R2_test_schur_sensitivity}}
\caption{Gradients of the pass band response with respect to the coefficients of
  the tapped Schur lattice implementation of the PCLS optimised low pass
  differentiator correction filter.}
\label{fig:Schur-OneM-lattice-correction-lowpass-differentiator-Schur-sens}
\end{figure}

\subsection{Search for the signed-digit coefficients of the R=2 low pass differentiator filter}

When truncated to $12$ bits and $3$ signed-digits the Schur one-multiplier
lattice coefficients of the PCLS optimised correction filter are:
\begin{small}
\verbatiminput{socp_relaxation_schurOneMlattice_lowpass_differentiator_R2_12_nbits_test_k0_sd_no_alloc_coef.m}
\verbatiminput{socp_relaxation_schurOneMlattice_lowpass_differentiator_R2_12_nbits_test_c0_sd_no_alloc_coef.m}
\end{small}

The numbers of signed-digits that the heuristic of \emph{Lim et al.} allocates to
each cofficient of the PCLS optimised correction filter are:
\begin{small}
\verbatiminput{socp_relaxation_schurOneMlattice_lowpass_differentiator_R2_12_nbits_test_k_allocsd_digits.m}
\verbatiminput{socp_relaxation_schurOneMlattice_lowpass_differentiator_R2_12_nbits_test_c_allocsd_digits.m}
\end{small}

The signed-digit coefficients of the PCLS optimised filter with the
numbers of signed-digits allocated by the heuristic of \emph{Lim et al.} are:
\begin{small}
\verbatiminput{socp_relaxation_schurOneMlattice_lowpass_differentiator_R2_12_nbits_test_k0_sd_coef.m}
\verbatiminput{socp_relaxation_schurOneMlattice_lowpass_differentiator_R2_12_nbits_test_c0_sd_coef.m}
\end{small}

The signed-digit coefficients of the PCLS optimised filter with the
numbers of signed-digits allocated by the heuristic of \emph{Lim et al.} and
found with branch-and-bound search are:
\begin{small}
\verbatiminput{branch_bound_schurOneMlattice_lowpass_differentiator_R2_12_nbits_test_k_min_coef.m}
\verbatiminput{branch_bound_schurOneMlattice_lowpass_differentiator_R2_12_nbits_test_c_min_coef.m}
\end{small}
The corresponding correction filter transfer function polynomials found with
branch-and-bound search are:
\begin{small}
\verbatiminput{branch_bound_schurOneMlattice_lowpass_differentiator_R2_12_nbits_test_N_min_coef.m}
\verbatiminput{branch_bound_schurOneMlattice_lowpass_differentiator_R2_12_nbits_test_D_min_coef.m}
\end{small}

The signed-digit coefficients of the PCLS optimised filter with the
numbers of signed-digits allocated by the heuristic of \emph{Lim et al.} and
found with SOCP relaxation search are:
\begin{small}
\verbatiminput{socp_relaxation_schurOneMlattice_lowpass_differentiator_R2_12_nbits_test_k_min_coef.m}
\verbatiminput{socp_relaxation_schurOneMlattice_lowpass_differentiator_R2_12_nbits_test_c_min_coef.m}
\end{small}
The corresponding correction filter transfer function polynomials found with
SOCP relaxation search are:
\begin{small}
\verbatiminput{socp_relaxation_schurOneMlattice_lowpass_differentiator_R2_12_nbits_test_N_min_coef.m}
\verbatiminput{socp_relaxation_schurOneMlattice_lowpass_differentiator_R2_12_nbits_test_D_min_coef.m}
\end{small}

Table~\ref{tab:Comp-schurOneMlattice-lowpass-diff-R2-12-nbits-cost}
compares, for each filter, a cost function, the maximum pass band and stop band
response errors, the total number of signed-digits required by the $12$-bit 
coefficients and the number of shift-and-add operations required to implement
the correction filter coefficient multiplications of the correction filter.

\begin{table}
\centering
\begin{threeparttable}
\begin{tabular}{lrrrrrrrr} \\ \toprule
 &Correction&Max. pass&Max. pass &Max. stop &Max. pass   &Max. pass  &$12$-bit&Shift-\\
 &filter    &amplitude&amplitude &amplitude &phase error &delay error&signed  &and- \\
 &cost      &error    &rel. error&error     &(rad./$\pi$)&(samples)  &digits  &adds \\
\midrule
\input{comparison_no_POP_schurOneMlattice_lowpass_differentiator_R2_12_nbits_test_cost.tab}
\\ \bottomrule
\end{tabular}
\end{threeparttable}
\caption[Comparison of low pass differentiator filter fixed point implementations]
{Comparison of the response errors and the total number of signed digits in the
  12-bit coefficients, each having an average of 3-signed digits, allocated by
  the method of \emph{Lim et al.}, required for a tapped Schur one-multiplier
  lattice, R=2, low pass differentiator correction filter found by
  branch-and-bound and SOCP relaxation search.}
\label{tab:Comp-schurOneMlattice-lowpass-diff-R2-12-nbits-cost}
\end{table}

Figure~\ref{fig:Comp-schurOneMlattice-lowpass-diff-R2-12-nbits-pass}
compares the pass band amplitude response of each filter for
$12$-bit $3$-signed-digit coefficients and $12$-bit coefficients with an average
of $3$-signed-digits allocated by the method of \emph{Lim et al.} found with
branch-and-bound search and SOCP relaxation search.
Figure~\ref{fig:Comp-schurOneMlattice-lowpass-diff-R2-12-nbits-relative-pass}
compares the pass band relative amplitude response of each filter.
Figure~\ref{fig:Comp-schurOneMlattice-lowpass-diff-R2-12-nbits-stop}
compares the stop band amplitude response of each filter.
Figure~\ref{fig:Comp-schurOneMlattice-lowpass-diff-R2-12-nbits-phase} shows
the pass band phase reslonse of each filter.
~\ref{fig:Comp-schurOneMlattice-lowpass-diff-R2-12-nbits-delay} shows the
pass band group-delay response of each filter.
Figure~\ref{fig:Comp-schurOneMlattice-lowpass-diff-R2-12-nbits-dCsqdw} shows
the gradient of the squared magnitude response of the correction filter.

\begin{figure}
\centering
\scalebox{\DesignOfSchurLatticeFiltersPdfScale}{\input{comparison_no_POP_schurOneMlattice_lowpass_differentiator_R2_12_nbits_test_pass}}
\caption{Comparison of the pass band response error of
  a tapped Schur one-multiplier, R=2, lattice lowpass differentiator correction
  filter with floating point coefficients, 12-bit 3-signed-digit coefficients,
  and 12-bit signed-digit coefficients each having an average of 3-signed-digits
  allocated by the method of \emph{Lim et al.}, found by branch-and-bound and
  SOCP relaxation search.}
\label{fig:Comp-schurOneMlattice-lowpass-diff-R2-12-nbits-pass}
\end{figure}

\begin{figure}
\centering
\scalebox{\DesignOfSchurLatticeFiltersPdfScale}{\input{comparison_no_POP_schurOneMlattice_lowpass_differentiator_R2_12_nbits_test_relative_pass}}
\caption{Comparison of the pass band relative response error of
  a tapped Schur one-multiplier, R=2, lattice lowpass differentiator correction
  filter with floating point coefficients, 12-bit 3-signed-digit coefficients,
  and 12-bit signed-digit coefficients each having an average of 3-signed-digits
  allocated by the method of \emph{Lim et al.}, found by branch-and-bound and
  SOCP relaxation search.}
\label{fig:Comp-schurOneMlattice-lowpass-diff-R2-12-nbits-relative-pass}
\end{figure}

\begin{figure}
\centering
\scalebox{\DesignOfSchurLatticeFiltersPdfScale}{\input{comparison_no_POP_schurOneMlattice_lowpass_differentiator_R2_12_nbits_test_stop}}
\caption{Comparison of the stop band response error of
  a tapped Schur one-multiplier, R=2, lattice lowpass differentiator correction
  filter with floating point coefficients, 12-bit 3-signed-digit coefficients,
  and 12-bit signed-digit coefficients each having an average of 3-signed-digits
  allocated by the method of \emph{Lim et al.}, found by branch-and-bound and
  SOCP relaxation search.}
\label{fig:Comp-schurOneMlattice-lowpass-diff-R2-12-nbits-stop}
\end{figure}

\begin{figure}
\centering
\scalebox{\DesignOfSchurLatticeFiltersPdfScale}{\input{comparison_no_POP_schurOneMlattice_lowpass_differentiator_R2_12_nbits_test_phase}}
\caption{Comparison of the pass band phase response errors of a tapped Schur
  one-multiplier, R=2, lattice low pass differentiator correction filter with
  floating point coefficients, 12-bit 3-signed-digit coefficients, and 12-bit
  signed-digit coefficients each having an average of 3-signed-digits allocated
  by the method of \emph{Lim et al.}, found by branch-and-bound and
  SOCP relaxation search.}
\label{fig:Comp-schurOneMlattice-lowpass-diff-R2-12-nbits-phase}
\end{figure}

\begin{figure}
\centering
\scalebox{\DesignOfSchurLatticeFiltersPdfScale}{\input{comparison_no_POP_schurOneMlattice_lowpass_differentiator_R2_12_nbits_test_delay}}
\caption{Comparison of the pass band group delay response errors of a tapped
  Schur one-multiplier, R=2, lattice low pass differentiator correction filter
  with floating point coefficients, 12-bit 3-signed-digit coefficients, and
  12-bit signed-digit coefficients each having an average of 3-signed-digits
  allocated by the method of \emph{Lim et al.}, found by branch-and-bound and
  SOCP relaxation search.}
\label{fig:Comp-schurOneMlattice-lowpass-diff-R2-12-nbits-delay}
\end{figure}

\begin{figure}
\centering
\scalebox{\DesignOfSchurLatticeFiltersPdfScale}{\input{comparison_no_POP_schurOneMlattice_lowpass_differentiator_R2_12_nbits_test_dCsqdw}}
\caption{Comparison of the error in the pass band gradient of the squared
  magnitude response of a tapped Schur one-multiplier, R=2, lattice low pass
  differentiator correction filter with floating point coefficients, 12-bit
  3-signed-digit coefficients, and 12-bit signed-digit coefficients each having
  an average of 3-signed-digits allocated by the method of \emph{Lim et al.},
  found by branch-and-bound and SOCP relaxation search.}
\label{fig:Comp-schurOneMlattice-lowpass-diff-R2-12-nbits-dCsqdw}
\end{figure}

\subsection{Comparison with a maximally-linear  FIR low pass differentiator filter}

\emph{Selesnick}~\cite{Selesnick_MaximallyFlatLowPassDigitalDifferentiators}
describes the design of maximally linear low pass FIR differentiator filters:
\begin{align*}
  F\left(z\right)&=\left(\frac{1-z^{-1}}{2}\right)
                   \left(\frac{1+z^{-1}}{2}\right)^{K}
                   z^{-L}\sum_{n=0}^{L}c_{n}\left[\frac{-z+2-z^{-1}}{4}\right]^{n}
\end{align*}
where the $c_{n}$ are defined recursively:
\begin{align*}
c_{0}&=2\\
c_{1}&=K+\frac{1}{3} \\
    & \hdots \\
c_{n}&=\frac{\left(8n^{2}+4Kn-10n-K+3\right)c_{n-1}-\left(2n+K-3\right)^{2}c_{n-2}}
       {2n\left(2n+1\right)}
\end{align*}
The length of the impulse response is $N=K+2L+2$. The coefficients of the maximally linear anti-symmetric FIR
low pass differentiator with $L=20$ and $K=21$ are\footnote{For odd length,
  $N=63$, $h0\left(32\right)=0$.}:
\begin{small}
\verbatiminput{socp_relaxation_directFIRantisymmetric_lowpass_differentiator_12_nbits_test_h0_coef.m}
\end{small}
By construction, the phase response of the anti-symmetric FIR low pass
differentiator filter is $\frac{\pi}{2}$ radians. This FIR filter was truncated
to $16$ distinct coefficients, the same number as the Schur lattice, and a group
delay of $16$ samples. After SOCP-relaxation optimisation to $12$-bits with an
average of $3$ signed-digits allocated by the heuristic of \emph{Lim et al.}
the distinct coefficients of the filter are:
\begin{small}
\verbatiminput{socp_relaxation_directFIRantisymmetric_lowpass_differentiator_12_nbits_test_hM_min_coef.m}
\end{small}

Table~\ref{tab:Comp-schurOneMlattice-lowpass-diff-R2-12-nbits-hM-min-cost}
compares a cost function, the maximum pass band and stop band response errors,
the total number of signed-digits required by the $12$-bit coefficients and the
number of shift-and-add operations required to implement the coefficient
multiplications of the floating point FIR, floating point truncated FIR and the
SOCP-relaxation optimised truncated FIR and the the SOCP-relaxation optimised
truncated FIR and the branch-and-bound optimised truncated FIR filter.

\begin{table}
\centering
\begin{threeparttable}
\begin{tabular}{lrrrrrr} \\ \toprule
 &Low pass        &Max. pass&Max. pass &Max. stop &$12$-bit&Shift-\\
 &differentiator  &amplitude&amplitude &amplitude &signed  &and- \\
 &cost            &error    &rel. error&error     &digits  &adds \\
\midrule
\input{comparison_no_POP_schurOneMlattice_lowpass_differentiator_R2_12_nbits_test_hM_min_cost.tab}
\\ \bottomrule
\end{tabular}
\end{threeparttable}
\caption[Comparison of low pass differentiator filter FIR implementations]
{Comparison of the response errors and the total number of signed digits in the
  12-bit coefficients, each having an average of 3-signed digits, allocated by
  the method of \emph{Lim et al.}, required for a low pass differentiator filter
  implemented as an anti-symmetric FIR filter found by SOCP relaxation search
  and branch-and-bound search.}
\label{tab:Comp-schurOneMlattice-lowpass-diff-R2-12-nbits-hM-min-cost}
\end{table}

Figure~\ref{fig:Comp-schurOneMlattice-lowpass-diff-R2-12-nbits-hM-min-pass}
compares the pass band error amplitude responses of a low pass
differentiator filter implemented as a maximally linear FIR filter with floating
point coefficients, that FIR filter truncated to $16$ distinct coefficients and
the truncated filter coefficients SOCP-relaxation optimised for $12$-bit
$3$-signed-digit coefficients and $12$-bit coefficients with an average of
$3$-signed-digits allocated by the method of \emph{Lim et al.}.
Similarly,
Figure~\ref{fig:Comp-schurOneMlattice-lowpass-diff-R2-12-nbits-hM-min-pass-relative}
compares the pass band relative error amplitude responses and
Figure~\ref{fig:Comp-schurOneMlattice-lowpass-diff-R2-12-nbits-hM-min-stop}
compares the stop band error amplitude responses.

\begin{figure}
\centering
\scalebox{\DesignOfSchurLatticeFiltersPdfScale}{\input{comparison_no_POP_schurOneMlattice_lowpass_differentiator_R2_12_nbits_test_hM_min_pass_error}}
\caption{Comparison of the pass band error amplitude responses of a low pass
  differentiator filter implemented as a maximally linear FIR filter with
  floating point coefficients, that FIR filter truncated to $16$ distinct
  coefficients and the truncated filter coefficients SOCP-relaxation optimised
  for $12$-bit $3$-signed-digit coefficients and $12$-bit coefficients with an
  average of $3$-signed-digits allocated by the method of \emph{Lim et al.}.}
\label{fig:Comp-schurOneMlattice-lowpass-diff-R2-12-nbits-hM-min-pass}
\end{figure}

\begin{figure}
\centering
\scalebox{\DesignOfSchurLatticeFiltersPdfScale}{\input{comparison_no_POP_schurOneMlattice_lowpass_differentiator_R2_12_nbits_test_hM_min_pass_relative_error}}
\caption{Comparison of the pass band relative error responses of a low pass
  differentiator filter implemented as a maximally linear FIR filter with
  floating point coefficients, that FIR filter truncated to $16$ distinct
  coefficients and the truncated filter coefficients SOCP-relaxation optimised
  for $12$-bit $3$-signed-digit coefficients and $12$-bit coefficients with an
  average of $3$-signed-digits allocated by the method of \emph{Lim et al.}.}
\label{fig:Comp-schurOneMlattice-lowpass-diff-R2-12-nbits-hM-min-pass-relative}
\end{figure}

\begin{figure}
\centering
\scalebox{\DesignOfSchurLatticeFiltersPdfScale}{\input{comparison_no_POP_schurOneMlattice_lowpass_differentiator_R2_12_nbits_test_hM_min_stop}}
\caption{Comparison of the stop band error amplitude responses of a low pass
  differentiator filter implemented as a maximally linear FIR filter with
  floating point coefficients, that FIR filter truncated to $16$ distinct
  coefficients and the truncated filter coefficients SOCP-relaxation optimised
  for $12$-bit $3$-signed-digit coefficients and $12$-bit coefficients with an
  average of $3$-signed-digits allocated by the method of \emph{Lim et al.}.}
\label{fig:Comp-schurOneMlattice-lowpass-diff-R2-12-nbits-hM-min-stop}
\end{figure}

Table~\ref{tab:Comp-schurOneMlattice-lowpass-diff-R2-12-nbits-Schur-FIR-cost}
compares a cost function, the maximum pass band and stop band response errors,
the total number of
signed-digits required by the $12$-bit coefficients and the number of
shift-and-add operations required to
implement the coefficient multiplications of the SOCP-relaxation optimised
truncated FIR and Schur lattice filters and the branch-and-bound optimised
truncated FIR and Schur lattice filters.

\begin{table}
\centering
\begin{threeparttable}
\begin{tabular}{lrrrrrr} \\ \toprule
 &Low pass        &Max. pass&Max. pass &Max. stop &$12$-bit&Shift-\\
 &differentiator  &amplitude&amplitude &amplitude &signed  &and- \\
 &cost            &error    &rel. error&error     &digits  &adds \\
\midrule
\input{comparison_no_POP_schurOneMlattice_lowpass_differentiator_R2_12_nbits_test_Schur_FIR_cost.tab}
\\ \bottomrule
\end{tabular}
\end{threeparttable}
\caption[Comparison of low pass differentiator filter FIR and Schur lattice implementations]
{Comparison of the response errors and the total number of signed digits in the
  12-bit coefficients, each having an average of 3-signed digits, allocated by
  the method of \emph{Lim et al.}, required for a low pass differentiator filter
  implemented as an anti-symmetric FIR filter and a tapped Schur one-multiplier
  lattice, R=2, correction filter found by SOCP relaxation search and
  branch-and-bound search.}
\label{tab:Comp-schurOneMlattice-lowpass-diff-R2-12-nbits-Schur-FIR-cost}
\end{table}

Figure~\ref{fig:Comp-schurOneMlattice-lowpass-diff-R2-12-nbits-Schur-FIR-pass}
compares the pass band amplitude error responses of a low pass
differentiator filter implemented as a maximally linear FIR filter truncated to
$16$ distinct coefficients and a tapped Schur one-multiplier lattice, R=2,
correction filter. The coefficients are SOCP-relaxation and branch-and-bound
optimised for $12$-bit coefficients with an average of
$3$-signed-digits each, allocated by the method of \emph{Lim et al.}\;.
Similarly,
Figure~\ref{fig:Comp-schurOneMlattice-lowpass-diff-R2-12-nbits-Schur-FIR-pass-relative}
compares the pass band relative error amplitude responses and
Figure~\ref{fig:Comp-schurOneMlattice-lowpass-diff-R2-12-nbits-Schur-FIR-stop}
compares the stop band error amplitude responses.

\begin{figure}
\centering
\scalebox{\DesignOfSchurLatticeFiltersPdfScale}{\input{comparison_no_POP_schurOneMlattice_lowpass_differentiator_R2_12_nbits_test_Schur_FIR_pass_error}}
\caption{Comparison of the pass band amplitude error responses of a low pass
  differentiator filter implemented as a maximally linear FIR filter truncated
  to $16$ distinct coefficients and a tapped Schur one-multiplier lattice, R=2,
  correction filter. The coefficients are SOCP-relaxation and branch-and-bound
  optimised for $12$-bit coefficients with an average of
  $3$-signed-digits each, allocated by the method of \emph{Lim et al.}\;.}
\label{fig:Comp-schurOneMlattice-lowpass-diff-R2-12-nbits-Schur-FIR-pass}
\end{figure}

\begin{figure}
\centering
\scalebox{\DesignOfSchurLatticeFiltersPdfScale}{\input{comparison_no_POP_schurOneMlattice_lowpass_differentiator_R2_12_nbits_test_Schur_FIR_pass_relative_error}}
\caption{Comparison of the pass band amplitude relative error responses of a low
  pass differentiator filter implemented as a maximally linear FIR filter
  truncated to $16$ distinct coefficients and a tapped Schur one-multiplier
  lattice, R=2, correction filter. The coefficients are SOCP-relaxation and
  branch-and-bound optimised for $12$-bit coefficients with an average of
  $3$-signed-digits each, allocated by the method of \emph{Lim et al.}\;.}
\label{fig:Comp-schurOneMlattice-lowpass-diff-R2-12-nbits-Schur-FIR-pass-relative}
\end{figure}

\begin{figure}
\centering
\scalebox{\DesignOfSchurLatticeFiltersPdfScale}{\input{comparison_no_POP_schurOneMlattice_lowpass_differentiator_R2_12_nbits_test_Schur_FIR_stop}}
\caption{Comparison of the stop band amplitude responses of a low pass
  differentiator filter implemented as a maximally linear FIR filter truncated
  to $16$ distinct coefficients and a tapped Schur one-multiplier lattice, R=2,
  correction filter. The coefficients are SOCP-relaxation and branch-and-bound
  optimised for $12$-bit coefficients with an average of
  $3$-signed-digits each, allocated by the method of \emph{Lim et al.}\;.}
\label{fig:Comp-schurOneMlattice-lowpass-diff-R2-12-nbits-Schur-FIR-stop}
\end{figure}

\clearpage
\section{Summary}
The Schur lattice digital filter has a simple stability constraint. This article
demonstrates the realisation of an IIR digital filter transfer function with
stability and frequency response constraints by optimising the floating point
and fixed point coefficients of a tapped one-multiplier Schur lattice digital
filter. The performance of the tapped, R=2, Schur lattice filter is compared
to that of a maximally-flat anti-symmetric FIR low pass differentiator filter.
The tapped, R=2, Schur lattice filter has lower group delay and better pass band
relative amplitude error. The truncated anti-symmetric FIR filter has better
stop band amplitude suppression and, by construction, constant phase and group
delay.

\bibliographystyle{plain}
\bibliography{DesignOfSchurLatticeFilters}

\end{document}
'
  \end{comment}
  \usepackage[hidelinks,colorlinks=false]{hyperref}
\else
  \usepackage[unicode=true,pdfusetitle,bookmarks=true,backref=page,
              bookmarksnumbered=false,bookmarksopen=false,breaklinks=false,
              pdfborder={1 0 0},colorlinks=true]{hyperref}
  \usepackage[all]{hypcap} % Needed to help hyperlinks direct correctly
\fi

\setlength{\parindent}{0em}
\setlength{\parskip}{\bigskipamount}
\floatstyle{ruled}
\newfloat{algorithm}{tbp}{loa}
\floatname{algorithm}{Algorithm}

\newcommand{\DesignOfSchurLatticeFiltersPdfScale}{1}
\newcommand{\DesignOfSchurLatticeFiltersIncludeScale}{0.85}

\newcommand{\coloneq}{\mathrel{\resizebox{\widthof{$\mathord{=}$}}
    {\height}{ $\!\!\resizebox{1.2\width}{0.8\height}
      {\raisebox{0.23ex}{$\mathop{:}$}}\!\!=\!\!$ }}}

\DeclareMathOperator{\sinc}{sinc}
\DeclareMathOperator{\jsn}{sn}
\DeclareMathOperator{\jns}{ns}
\DeclareMathOperator{\jcn}{cn}
\DeclareMathOperator{\jnc}{nc}
\DeclareMathOperator{\jdn}{dn}
\DeclareMathOperator{\jnd}{nd}
\DeclareMathOperator{\jsd}{sd}
\DeclareMathOperator{\jds}{ds}
\DeclareMathOperator{\jcd}{cd}
\DeclareMathOperator{\jdc}{dc}
\DeclareMathOperator{\jsc}{sc}
\DeclareMathOperator{\jcs}{cs}
\DeclareMathOperator{\jam}{am}
\DeclareMathOperator{\jarcsn}{arcsn}
\DeclareMathOperator{\jarcsc}{arcsc}
\DeclareMathOperator{\jarccn}{arccn}
\DeclareMathOperator{\jarcdn}{arcdn}
\DeclareMathOperator{\jarccd}{arccd}
\DeclareMathOperator{\jarccs}{arccs}
\DeclareMathOperator{\cosec}{cosec}
\DeclareMathOperator{\cotan}{cotan}
\DeclareMathOperator{\sech}{sech}
\DeclareMathOperator{\arctanh}{arctanh}
\DeclareMathOperator{\arccosh}{arccosh}
\DeclareMathOperator{\sign}{sign}
\DeclarePairedDelimiter{\ceil}{\lceil}{\rceil}
\DeclarePairedDelimiter{\floor}{\lfloor}{\rfloor}
\DeclarePairedDelimiter{\nint}{\lfloor}{\rceil}
\DeclarePairedDelimiter{\mathabs}{\lvert}{\rvert}
\DeclarePairedDelimiter{\mathnorm}{\lVert}{\rVert}
\DeclareMathOperator{\mathnull}{null}
\DeclareMathOperator{\mathspan}{span}
\DeclareMathOperator{\mathrank}{rank}
\DeclareMathOperator{\mathrange}{range}
\DeclareMathOperator{\mathtrace}{trace}
\DeclareMathOperator{\mathadj}{adj}
\DeclareMathOperator{\mathdiag}{diag}
\DeclareMathOperator{\mathtpltz}{Toeplitz}
\DeclareMathOperator{\mathcohull}{cohull}
\DeclareMathOperator{\mathcone}{cone}
\DeclareMathOperator{\mathri}{ri}
\DeclareMathOperator{\mathaff}{aff}
\DeclareMathOperator{\mathconv}{conv}
\DeclareMathOperator{\mathhe}{He}
\DeclareMathOperator{\mathsq}{Sq}
\DeclareMathOperator{\mathvec}{vec}
\DeclareMathOperator{\mathconj}{\ast} % {\textsf{\tiny{H}}}
\DeclareMathOperator{\mathreverse}{\textsf{\tiny{F}}}

\title{On the design of Schur lattice digital filters}
\author{Robert G. Jenssen}
\date{\today}

\begin{document}

\maketitle

\begin{abstract}
  This article describes a computer-aided procedure for approximating an
  arbitrary transfer function by a tapped one-multiplier all pass lattice
  filter with integer coefficients. The all pass lattice has a simple
  stability criterion, that the lattice coefficients have magnitude less
  than unity, and, in addition, has good coefficient sensitivity and round-off
  noise properties. The procedure consists of finding an initial filter
  transfer function, performing constrained minimum-mean-squared-error
  optimisation of the filter response and searching for the fixed point
  filter coefficients.
\end{abstract}

\section{Introduction}

The Infinite Impulse Response (IIR) digital filter transfer function is
a rational polynomial:
\begin{align}
  F\left(z\right) = \frac{B_{N}\left(z\right)}{A_{N}\left(z\right)}
  &= \frac{b_{0}+b_{1}z^{-1}+\cdots + b_{N}z^{-N}}
    {1+a_{1}z^{-1} + \cdots + a_{N}z^{-N}}
\label{eqn:Transfer-function-F}
\end{align}
Here $z$ is a complex number and the response of the filter with respect to
angular frequency, $\omega$, is $F\left(e^{\imath\omega}\right)$, where
$\omega=2\pi$ corresponds to the digital sampling frequency. 
This article describes a method for computer-aided design of the frequency
response of $F\left(e^{\imath\omega}\right)$. If the filter has
$A_{N}\left(z\right)=1$ then it is a Finite Impulse Response (FIR) filter and
the optimisation problem is convex, has no constraints on the coefficients and
is solved by well-known algorithms that minimise the peak error or the
least-squared-error~\cite{Medlin_LagrangeMultiplierFIRMultirate,SelesnickBurrus_ExchangeAlgorithmsLinearPhaseFIRFilters,SelesnickLangBurrus_ConstrainedLeastSquareFIRFilters,SelesnickLangBurrus_ConstrainedLeastSquareMultiBandFIRFilters,McClellanParks_ComputerProgramFIRLinearPhaseFilters,Iwasaki_GeneralizedKYPFrequencyDomainInequalities,Davidson_LMISpectralConstraints,ParksMcClellan_ChebyshevApproxNonRecursiveDigitalFilters}.
Otherwise, the filter is an Infinite Impulse (IIR) filter. An IIR filter is
stable if the roots, $P_{k}$, of the denominator polynomial,
$A_{N}\left(z\right)$, lie within the unit circle, $\left|P_{k}\right|<1$.
\emph{Lang}~\cite{Lang_LSDesignIIRDigitalFiltersPoleRadiusConstraint} applies
Rouch\'{e}'s theorem to constrain the zeros of $A_{N}\left(z\right)$ during
optimisation. \emph{Lu} and
\emph{Hinamoto}~\cite{LuHinamoto_IIRRobustStabilityQuadraticProgramming}
decompose $F\left(z\right)$ into a series connection of second-order sections
and apply the ``stabiliy triangle'' to each section to ensure filter stability
after optimisation. If the transfer function is represented in gain-pole-zero
form:
\begin{align*}
  F\left(z\right)&= G\frac{\prod_{k=1}^{N}\left(z-Z_{k}\right)}
                   {\prod_{k=1}^{N}\left(z-P_{k}\right)}
\end{align*}
where $G$ is the gain factor, the $Z_{k}$ are the roots of $B_{N}\left(z\right)$
and the $P_{k}$ are the roots of $A_{N}\left(z\right)$, then the constraints on
the pole locations of $F\left(z\right)$ are $\left|P_{k}\right|<1$. 
\emph{Deczky}~\cite{Deczky_MinPSynthesisIIRDigitalFilters} and
\emph{Richards}~\cite{Richards_DeczkyRecursiveDecimator} demonstrate the
optimisation of stable IIR filter transfer functions expressed in gain-pole-zero
form.
Lattice filters are an alternative implementation of the IIR filter
transfer function~\cite{VaidyanathanMitra_RobustFilterStructures}.
\emph{Vaidyanathan et
  al.}~\cite{VaidyanathanMitraNuevo_LowSensitivityIIRDigitalFilters,
  RegaliaMitraVaidyanathan_DigitalAllPassFilterVersatileSignalProcessing} show
that a subset of the $F\left(z\right)$ transfer functions can be factored into
the parallel sum of two all pass filters and that the resulting implementation
has good coefficient sensitivity properties. They show lattice filter
implementations of these all pass sections that are ``structurally bounded''
meaning that the section is all pass regardless of the values of the
coefficients. \emph{Gray} and
\emph{Markel}~\cite{GrayMarkel_DigitalLatticeAndLadderFilterSynthesis,
  MarkelGray_AutocorrelationSpeechAnalysis} describe the implementation of
$F\left(z\right)$ as a tapped all pass lattice filter based on the Schur
polynomial decomposition algorithm~\cite{SchurMethodsInOperatorTheory_Kailath,
  SchurMethodsInOperatorTheory_Schur_I}.
The Schur algorithm enables the decomposition of the denominator polynomial of
the transfer function, $F\left(z\right)$, into a set of orthogonal
polynomials. A result of this decompostion is a set of coefficients, often
called ``reflection coefficients'', that have magnitude less than unity
if-and-only-if the poles of the transfer function lie within the unit circle in
the complex plane. Further, the numerator polynomial of $F\left(z\right)$ can be
expressed as a weighted sum of the orthogonal polynomials. These coefficients and
orthogonal polynomials correspond to a tapped all pass lattice filter
implementation of the transfer function that has good round-off noise and
coefficient sensitivity when the coefficients are truncated to integer values.
This article reviews computer-aided techniques for designing tapped Schur
lattice filters with integer coefficients : finding an initial IIR filter
transfer function, constrained minimum-mean-squared-error optimisation of the
filter frequency response in terms of the lattice coefficients and searching for
integer filter coefficients. As an example, I demonstrate the design of a tapped
Schur lattice implementation of a low pass differentiator filter. The
coefficients and frequency response plots for this example are created with
Octave~\cite{Eaton_Octave,Octave_OptimPackage,Octave_SignalPackage}.
The source code for this article is available at
\url{https://github.com/robertgj/DesignOfIIRFilters}.

\section{Schur lattice filters}
\subsection{The Schur polynomial decomposition}
In 1918 \emph{Issai Schur}~\cite{SchurMethodsInOperatorTheory_Schur_I}
published a paper entitled ``\emph{On power series which are bounded in the
  interior of the unit circle}''. From the abstract:
\begin{quotation}
  The continued fraction algorithm introduced here very easily supplies an
  intrinsically important parametric representation for the coefficients of the
  power series to be considered.
\end{quotation}
\emph{Kailath}~\cite{SchurMethodsInOperatorTheory_Kailath} reviews the impact of
this algorithm on modern signal processing. \emph{Gray} and \emph{Markel}
describe the application of a similar algorithm to the autocorrelation analysis
of speech~\cite{MarkelGray_AutocorrelationSpeechAnalysis} and to the synthesis
of lattice digital
filters~\cite{GrayMarkel_DigitalLatticeAndLadderFilterSynthesis}. 
In the following I use the inner product method of \emph{Markel} and
\emph{Gray}~\cite{MarkelGray_AutocorrelationSpeechAnalysis} and
\emph{Parhi}~\cite[Chapter 12 and Appendix
D]{Parhi_VLSIDigitalSignalProcessingSystems}.

Initialise an $N$'th order polynomial as:
\begin{align*}
  \Phi_{N}\left(z\right) &= A_{N}\left(z\right) = \sum_{n=0}^{N}\phi_{n}z^{n}
\end{align*}
and define the reverse polynomial:
\begin{align*}
\hat{\Phi}_{N}\left(z\right)=z^{N}\Phi_{N}\left(z^{-1}\right)
\end{align*}
The transfer function $\hat{\Phi}_{N}\left(z\right)/\Phi_{N}\left(z\right)$ 
represents an all pass filter.
The Schur decomposition forms the polynomial $\Phi_{N-1}\left(z\right)$ as:
\begin{align*}
  \Phi_{N-1}\left(z\right) &= \frac{z^{-1}\left[\Phi_{N}\left(z\right)-
                             k_{N}\hat{\Phi}_{N}\left(z\right)\right]}
                             {s_{N}} \\
\end{align*}
where $s_{N}$ is a scaling constant and
$k_{N}=\phi_{0}/\phi_{N}=\Phi_{N}\left(0\right)/\hat{\Phi}_{N}\left(0\right)$.
The degree of $\Phi_{N-1}\left(z\right)$ is $1$ less than that of
$\Phi_{N}\left(z\right)$ since, by a change of variables, the numerator is:
\begin{align*}
 z^{-1}\left\{ \phi_{N}\Phi_{N}\left(z\right)-\phi_{0}\hat{\Phi}_{N}\left(z\right)\right\}
& = z^{-1}\sum^{N}_{n=0}\left\{\phi_{N}\phi_{n}z^{n}-\phi_{0}\phi_{N-i}z^{n}\right\}\\
& = \sum^{N}_{n=1}\left\{\phi_{N}\phi_{n}-\phi_{0}\phi_{N-i}\right\}z^{n-1}\\
\end{align*}
This degree reduction procedure is continued, resulting in the set of
polynomials $\left\{\Phi_{N}\left(z\right),\Phi_{N-1}\left(z\right),\ldots,
 \Phi_{0}\left(z\right)\right\}$:
\begin{align}
  \Phi_{n-1}\left(z\right) &= \frac{z^{-1}\left\{\Phi_{n}\left(z\right)-
                             k_{n}\hat{\Phi}_{n}\left(z\right)\right\}}
                             {s_{n}}
\label{eq:SchurDegreeReduction}
\end{align}
and
\begin{align*}
  \hat{\Phi}_{n-1}\left(z\right) &= \frac{\left\{\hat{\Phi}_{n}\left(z\right)-
                                         k_{n}\Phi_{n}\left(z\right)\right\}}
                                        {s_{n}}
\end{align*}

\emph{Markel} and \emph{Gray}~\cite[Equations 13a and 13b]
{MarkelGray_AutocorrelationSpeechAnalysis} define an inner product of two
polynomials, $P\left(z\right)$ and $Q\left(z\right)$:
\begin{align*}
  \langle P\left(z\right),Q\left(z\right)\rangle &= \frac{1}{2\pi \imath}
    \ointctrclockwise_{C}\frac{P\left(z\right)Q\left(z^{-1}\right)}
    {A_{N}\left(z\right)A_{N}\left(z^{-1}\right)}\frac{dz}{z}
\end{align*}
where all the zeros of $A_{N}\left(z\right)$ lie within the contour $C$.
Two properties of $\Phi_{n}\left(z\right)$ under this inner product are:
\begin{enumerate}
\item The $\Phi_{n}\left(z\right)$ polynomials are
  orthogonal~\cite[Appendix D]{Parhi_VLSIDigitalSignalProcessingSystems}:
  \begin{align*}
    \langle \Phi_{m}\left(z\right),\Phi_{n}\left(z\right)\rangle = 0
    \;\text{if}\; m \ne n
  \end{align*}
\item For $1\le n \le N$, $\mathabs{k_{n}} < 1$ if-and-only-if the zeros of
  $A_N\left(z\right)$ lie within the unit
  circle~\cite[pages 72 and 74]{MarkelGray_AutocorrelationSpeechAnalysis}
\end{enumerate}
In addition, if $s_{n}=\sqrt{1-k_{n}^{2}}$\;, then the $\Phi_{n}$ polynomials are
orthonormal~\cite[Appendix D]{Parhi_VLSIDigitalSignalProcessingSystems}:
\begin{align*}
  \langle \Phi_{n}\left(z\right),\Phi_{n}\left(z\right)\rangle = 1
\end{align*}

\subsection{The Schur one-multiplier lattice filter}

In Equation~\ref{eq:SchurDegreeReduction}, choose $s_{n}=1-\epsilon_{n}k_{n}$,
where $\epsilon_{n}=\pm 1$, and initialise $\Lambda_{N}\left(z\right)=A_{N}\left(z\right)$. Then:
\begin{subequations}
\begin{align}
  \Lambda_{n-1}\left(z\right) &= \frac{z^{-1}\left\{\Lambda_{n}\left(z\right)-
                             k_{n}\hat{\Lambda}_{n}\left(z\right)\right\}}
                                {1-\epsilon_{n}k_{n}}
  \label{eqn:SchurOneMLambda} \\
  \hat{\Lambda}_{n-1}\left(z\right)
                              &= \frac{\left\{\hat{\Lambda}_{n}\left(z\right)-
                                       k_{n}\Lambda_{n}\left(z\right)\right\}}
                                      {1-\epsilon_{n}k_{n}}
  \label{eqn:SchurOneMLambdaHat}
\end{align}
\end{subequations}
where:
\begin{align*}
  k_{n}=\frac{\Lambda_{n}\left(0\right)}{\hat{\Lambda}_{n}\left(0\right)}
  =\frac{\Phi_{n}\left(0\right)}{\hat{\Phi}_{n}\left(0\right)}
\end{align*}
Rearranging Equation~\ref{eqn:SchurOneMLambdaHat} and substituting it into
Equation~\ref{eqn:SchurOneMLambda} gives:
\begin{align*}
  z\Lambda_{n-1}\left(z\right)
  &=\left(1+\epsilon_{n}k_{n}\right)\Lambda_{n}\left(z\right)-
    k_{n}\hat{\Lambda}_{n-1}\left(z\right)\\
  \hat{\Lambda}_{n}\left(z\right) & = k_{n}\Lambda_{n}\left(z\right)+
              \left(1-\epsilon_{n}k_{n}\right)\hat{\Lambda}_{n-1}\left(z\right)
\end{align*}
Figure~\ref{fig:Schur-one-multiplier-lattice-section} shows the corresponding
\emph{all pass Schur one multiplier lattice} filter section.

\begin{figure}
\centering
\includegraphics{schur_OneMultiplier_lattice_section}
\caption{All pass Schur one multiplier lattice filter section}
\label{fig:Schur-one-multiplier-lattice-section}
\end{figure}

The expansion of the numerator polynomial of the transfer function,
$F\left(z\right)$, in the orthogonal basis, $\Lambda_{n}\left(z\right)$,
is~\cite[Section 12.2.3]{Parhi_VLSIDigitalSignalProcessingSystems}:
\begin{align*}
B_{N}\left(z\right) & = \sum_{n=0}^{N}c_{n}\Lambda_{n}\left(z\right)
\end{align*}

Figure~\ref{fig:Second-order-tapped-Schur-one-multiplier-lattice-filter}
shows the tapped Schur one multiplier lattice filter implementation of a
second order transfer function.

\begin{figure}
\centering
\includegraphics{schur_OneMultiplier_2nd_Order}
\caption{Second-order tapped Schur one-multiplier lattice filter}
\label{fig:Second-order-tapped-Schur-one-multiplier-lattice-filter}
\end{figure}

The relation between $\Lambda_{n}\left(z\right)$ and $\Phi_{n}\left(z\right)$ is:
\begin{align*}
\Lambda_{N}\left(z\right) & = \Phi_{N}\left(z\right)\\
  \Lambda_{n}\left(z\right)
  &= \Phi_{n}\left(z\right)\sqrt{\frac{\left(1+\epsilon_{N}k_{N}\right)
    \left(1+\epsilon_{N-1}k_{N-1}\right)\cdots
    \left(1+\epsilon_{n+1}k_{n+1}\right)}
    {\left(1-\epsilon_{N}k_{N}\right)
    \left(1-\epsilon_{N-1}k_{N-1}\right)\cdots
    \left(1-\epsilon_{n+1}k_{n+1}\right)}}
\end{align*}
The $\Lambda_{n}\left(z\right)$ polynomials are orthogonal but not orthonormal
since:
\begin{align*}
\langle \Lambda_{n}\left(z\right),\Lambda_{n}\left(z\right)\rangle
&= \langle\hat{\Lambda}_{n}\left(z\right),\hat{\Lambda}_{n}\left(z\right)\rangle\\
&=\frac{\left(1+\epsilon_{N}k_{N}\right)\left(1+\epsilon_{N-1}k_{N-1}\right)
  \cdots\left(1+\epsilon_{n+1}k_{n+1}\right)}
  {\left(1-\epsilon_{N}k_{N}\right)\left(1-\epsilon_{N-1}k_{N-1}\right)
  \cdots\left(1-\epsilon_{n+1}k_{n+1}\right)}
\end{align*}

If the input signal is random and white with unit power then the average power
at an internal node, $x_{n}$, is
$\langle\Lambda_{n}\left(z\right),\Lambda_{n}\left(z\right)\rangle$. The
magnitude of $\langle\Lambda_{n}\left(z\right),\Lambda_{n}\left(z\right)\rangle$
can be adjusted by choosing the sign parameters,
$\epsilon_{N},\epsilon_{N-1}\hdots,\epsilon_{n+1}$.
\emph{Gray} and
\emph{Markel}~\cite[p. 496]{GrayMarkel_DigitalLatticeAndLadderFilterSynthesis}
suggest that one criterion for choosing the sign parameters is to require that
the node associated with the largest $k_{n}$ parameter in magnitude have the
largest amplitude. The sign parameters are found recursively by requiring that
the amplitudes at other nodes be as large as possible without exceeding the
maximum value. If the maximum occurs for $k_{l}$ then the recursion proceeds
for $n=l-1,l-2,\ldots,0$ and again for $n=l+1,l+2,\ldots,N$. The
recursion is simple because:
\begin{align*}
  \frac{\langle \Lambda_{n}\left(z\right),\Lambda_{n}\left(z\right)\rangle}
       {\langle \Lambda_{n+1}\left(z\right),\Lambda_{n+1}\left(z\right)\rangle}
  &= \frac{1+\epsilon_{n+1}k_{n+1}}{1-\epsilon_{n+1}k_{n+1}}
\end{align*}
By changing the sign parameter, this ratio can always be made smaller or larger 
than one. Algorithm~\ref{alg:One-multiplier-sign-assignment} shows the method
used to assign the sign parameters described by \emph{Gray} and
\emph{Markel}~\cite[p. 496]{GrayMarkel_DigitalLatticeAndLadderFilterSynthesis}.

\begin{algorithm}
Assume that $k_{l}$ has the largest magnitude of the $k_{n}$ for
$n=1,2,\ldots,N$. Define the quantities
\begin{align*}
  Q_{n}&=\frac{\langle\Lambda_{n}\left(z\right),\Lambda_{n}\left(z\right)\rangle}
             {\langle\Lambda_{l}\left(z\right),\Lambda_{l}\left(z\right)\rangle}\\
  q_{n}&=\frac{1+\left|k_{n}\right|}{1-\left|k_{n}\right|}
\end{align*}

so that $Q_{l}=1$. Each $Q_{n}$ should be as large as possible without
exceeding $Q_{l}$. Successive ratios are:
\begin{align}
\frac{Q_{n}}{Q_{n+1}} = \begin{cases}
q_{n} & if\,\epsilon_{n}=\phantom{-}\sign\left(k_{n}\right)\\
1/q_{n} & if\,\epsilon_{n}=-\sign\left(k_{n}\right)
\end{cases}\label{eq:OneMultRecursiveKCalc}
\end{align}

Now assign the $\epsilon_{n}$:
\begin{algorithmic}
\For {$n=l-1,l-2,\hdots,1$}
  \If {$Q_{n+1}<1/q_{n}$}
    \State $\epsilon_{n}=-\sign\left(k_{n}\right)$
  \Else
    \State $\epsilon_{n}=\sign\left(k_{n}\right)$
  \EndIf
\EndFor
\For {$n=l+1,l+2,\hdots,N$}
  \If {$Q_{n}<1/q_{n}$}
    \State $\epsilon_{n}=\sign\left(k_{n}\right)$
  \Else
    \State $\epsilon_{n}=-\sign\left(k_{n}\right)$
  \EndIf
\EndFor
\end{algorithmic}
\caption{One multiplier lattice sign
  assignment~\cite[p. 496]{GrayMarkel_DigitalLatticeAndLadderFilterSynthesis}.} 
\label{alg:One-multiplier-sign-assignment}
\end{algorithm}

\subsection{State variable descriptions of the Schur one multiplier lattice filter}
The state variable representation of a filter is:
\begin{align}
\left[\begin{array}{c}
\mathbf{x}^{\prime}\\
\mathbf{y}\end{array}\right] &= \left[\begin{array}{cc}
    \mathbf{A} & \mathbf{B} \\
    \mathbf{C} & \mathbf{D} \\
\end{array}\right]\left[\begin{array}{c}
\mathbf{x}\\
\mathbf{u}\end{array}\right]
\label{eqn:State-variable-representation}
\end{align}
where $\mathbf{u}$ is the input, $\mathbf{x}$ is the filter state, for
convenience $\mathbf{x}^{\prime}=z\,\mathbf{x}$, and $\mathbf{y}$ is the filter
output. $\mathbf{u}$ and $\mathbf{y}$ may be vectors. $\mathbf{A}$ is called
the state transition matrix.
Examination of Figure~\ref{fig:Schur-one-multiplier-lattice-section} and
Figure~\ref{fig:Second-order-tapped-Schur-one-multiplier-lattice-filter}
suggests that the calculation of the all pass output, $\hat{y}$, of the
tapped Schur lattice filter can be represented as a matrix product:

\begin{align*}
\left[\begin{array}{c}
\hat{y}_{0}\\
x_{1}\\
x_{2}\\
\vdots\\
x_{N-1}\\
u \end{array}\right] & = \left[\begin{array}{cccccc}
1 & 0 & \cdots &  & \cdots & 0\\
0 & 1 &  &  &  & \vdots\\
\vdots &  & \ddots\\
 &  &  & \ddots &  & \vdots\\
0 &  &  &  & 1 & 0\\
0 & \cdots &  & \cdots & 0 & 1
\end{array}\right]\left[\begin{array}{c}
x_{0}\\
x_{1}\\
x_{2}\\
\vdots\\
x_{N-1}\\
u
\end{array}\right]
\end{align*}
\begin{align*}
\left[\begin{array}{c}
x_{0}^{\prime}\\
\hat{y}_{1}\\
x_{2}\\
\vdots\\
x_{N-1}\\
u
\end{array}\right] & = \left[\begin{array}{cccccc}
-k_{1} & \left(1+k_{1}\epsilon_{1}\right) & 0 & \cdots & \cdots & 0\\
\left(1-k_{1}\epsilon_{1}\right) & k_{1} & 0 &  &  & \vdots\\
0 & 0 & 1\\
\vdots &  &  & \ddots &  & \vdots\\
0 &  &  &  & 1 & 0\\
0 & \cdots &  & \cdots & 0 & 1
\end{array}\right]\left[\begin{array}{c}
\hat{y}_{0}\\
x_{1}\\
x_{2}\\
\vdots\\
x_{N-1}\\
u
\end{array}\right]
\end{align*}
\begin{align*}
\left[\begin{array}{c}
x_{0}^{\prime}\\
x_{1}^{\prime}\\
\hat{y}_{2}\\
\vdots\\
x_{N-1}\\     
u
\end{array}\right] & = \left[\begin{array}{cccccc}
1 & 0 & 0 & 0 & \cdots & 0\\
0 & -k_{2} & \left(1+k_{2}\epsilon_{2}\right) & 0 &  & \vdots\\
0 & \left(1-k_{2}\epsilon_{2}\right) & k_{2} & 0\\
\vdots &  &  & \ddots &  & \vdots\\
0 &  &  &  & 1 & 0\\
0 & \cdots &  & \cdots & 0 & 1
\end{array}\right]\left[\begin{array}{c}
x_{0}^{\prime}\\
\hat{y}_{1}\\
x_{2}\\
\vdots\\
x_{N-1}\\
u
\end{array}\right]
\end{align*}
\begin{align*}
  \vdots\\
\end{align*}
\begin{align*}
\left[\begin{array}{c}
x_{0}^{\prime}\\
x_{1}^{\prime}\\
\vdots\\
x_{N-2}^{\prime}\\
\hat{y}_{N-1}\\
u
\end{array}\right] & = \left[\begin{array}{cccccc}
1 & 0 & \cdots &  & \cdots & 0\\
0 & 1 &  &  &  & \vdots\\
\vdots &  & \ddots &  &  & \vdots\\
0 &  &  & -k_{N-1} & \left(1+k_{N-1}\epsilon_{N-1}\right) & 0\\
0 &  &  & \left(1-k_{N-1}\epsilon_{N-1}\right) & k_{N-1} & 0\\
0 & \cdots & \cdots & 0 & 0 & 1
\end{array}\right]\left[\begin{array}{c}
x_{0}^{\prime}\\
\vdots\\
x_{N-3}^{\prime}\\
\hat{y}_{N-2}\\
x_{N-1}\\
u
\end{array}\right]
\end{align*}
\begin{align*}
\left[\begin{array}{c}
x_{0}^{\prime}\\
x_{1}^{\prime}\\
x_{2}^{\prime}\\
\vdots\\
x_{N-1}^{\prime}\\
\hat{y}
\end{array}\right] & = \left[\begin{array}{cccccc}
1 & 0 & \cdots &  & \cdots & 0\\
0 & 1 &  &  &  & \vdots\\
\vdots &  & \ddots & & & \vdots\\
0 &  &  & 1 & 0 & 0\\
0 &  &  &  0 & -k_{N} & \left(1+k_{N}\epsilon_{N}\right)\\
0 & \cdots &  & 0 & \left(1-k_{N}\epsilon_{N}\right) & k_{N}
\end{array}\right]\left[\begin{array}{c}
x_{0}^{\prime}\\
x_{1}^{\prime}\\
\vdots\\
x_{N-2}^{\prime}\\
\hat{y}_{N-1}\\
u
\end{array}\right]
\end{align*}
The construction of the state variable description of the Schur one multiplier
lattice filter is summarised in
Algorithm~\ref{alg:Construction-state-variable-One-Multiplier-lattice}.

\begin{algorithm}
Given $\left\{k_{1},k_{2},\hdots,k_{N}\right\}$,
$\left\{\epsilon_{1},\epsilon_{2},\hdots,\epsilon_{N}\right\}$ and
$\left\{c_{0},c_{1},\hdots,c_{N}\right\}$:
\begin{algorithmic}
\State $\hat{y}_{0} = x_{0}$
\For{$n=1,\hdots,N-1$}
  \State $x_{n-1}^{\prime}=-k_{n}\hat{y}_{n-1}+\left(1+k_{n}\epsilon_{n}\right)x_{n}$
  \State $\hat{y}_{n}= \left(1-k_{n}\epsilon_{n}\right)\hat{y}_{n-1}+k_{n}x_{n}$
\EndFor
\State $x_{N-1}^{\prime} = -k_{N}\hat{y}_{N-1}+\left(1+k_{N}\epsilon_{N}\right)u$
\State $\hat{y} = \left(1-k_{N}\epsilon_{N}\right)\hat{y}_{N-1}+k_{N}u$
\State $y  = c_{0}x_{0}+c_{1}x_{1}+\cdots+c_{N-1}x_{N-1}+c_{N}u$
\end{algorithmic}
\caption{Construction of a state variable description of the Schur
  one multiplier lattice filter.}
\label{alg:Construction-state-variable-One-Multiplier-lattice}
\end{algorithm}

The state variable matrix calculations for a second order tapped Schur one
muliplier lattice filter are:
\begin{align*}
  \left[\begin{array}{c}
    x_{0}^{\prime} \\
    x_{1}^{\prime} \\
    \hat{y} \\
    y\end{array}\right] &= 
  \left[\begin{array}{ccc}- k_{1} & 1+\epsilon_{1} k_{1} & 0\\
-k_{2}\left(1-\epsilon_{1}k_{1}\right) & -k_{1}k_{2}& 1+\epsilon_{2}k_{2} \\
 \left(1-\epsilon_{1}k_{1}\right)\left(1-\epsilon_{2}k_{2}\right) &
   k_{1}\left(1-\epsilon_{2}k_{2}\right) & k_{2} \\
 c_{0} & c_{1} & c_{2} \end{array}\right] \left[\begin{array}{c}
    x_{0} \\
    x_{1} \\
    u \end{array} \right]
\end{align*}
The latency in the arithmetic calculation of $y$ and $\hat{y}$
increases with the filter order. \emph{Parhi}~\cite[Chapter 4 and Section 12.8]
{Parhi_VLSIDigitalSignalProcessingSystems} describes the retiming and pipelining
of lattice filters for reduced latency. One method is to design the lattice
filter with coefficients of the denominator polynomial, $A_{N}\left(z\right)$,
only in $z^{-2n}$, where $N$ is even and $n=1,\hdots,\frac{N}{2}$.
\emph{Deczky}~\cite{Deczky_MinPSynthesisIIRDigitalFilters} and
\emph{Richards}~\cite{Richards_DeczkyRecursiveDecimator} describe such filters
as $R=2$.
Figure~\ref{fig:Fourth-order-tapped-Schur-one-multiplier-lattice-filter-R2}
shows such a fourth order tapped Schur one multiplier lattice filter with and
without pipelining.

\begin{figure}
\begin{subfigure}[b]{\textwidth}
\centering
\includegraphics[scale=\DesignOfSchurLatticeFiltersIncludeScale]{schur_OneMultiplier_4th_Order_R2}
\caption{Without pipelining}
\label{subfig:Fourth-order-tapped-Schur-one-multiplier-lattice-R2-without}
\end{subfigure}
\begin{subfigure}[b]{\textwidth}
\centering
\includegraphics[scale=\DesignOfSchurLatticeFiltersIncludeScale]{schur_OneMultiplier_4th_Order_R2_Pipelined}
\caption{With pipelining}
\label{subfig:Fourth-order-tapped-Schur-one-multiplier-lattice-R2-with}
\end{subfigure}
\caption{Fourth-order tapped Schur one-multiplier lattice filter with denominator
  coefficients only for $z^{-2n}$}
\label{fig:Fourth-order-tapped-Schur-one-multiplier-lattice-filter-R2}
\end{figure}

The state variable matrix calculations for a pipelined fourth order tapped Schur
one-muliplier lattice filter with denominator polynomial coefficients only for
$z^{-2n}$ are:

\begin{align*}
  \left[\begin{array}{c}
    x_{0}^{\prime} \\
    x_{1}^{\prime} \\
    x_{2}^{\prime} \\
    x_{3}^{\prime} \\
    x_{4}^{\prime} \\
    \hat{y} \\
    y\end{array}\right] &= 
  \left[\begin{array}{cccccc} 0 & 1 & 0 & 0 & 0 & 0\\
    -k_{2} & 0 & 0 & 0 & 1+\epsilon_{2}k_{2} & 0\\
    1-\epsilon_{2}k_{2} & 0 & 0 & 0 & k_{2} & 0\\
    c_{0} & c_{1} & 0 & 0 & c_{2} & 0 \\
    0 & 0 & -k_{4} & 0 & 0 & 1+\epsilon_{4}k_{4}\\
    0 & 0 & 1-\epsilon_{4}k_{4}  & 0 & 0 & k_{4} \\
    0 & 0 & 0 & 1 & c_{3} & c_{4}\end{array}\right] \left[\begin{array}{c}
    x_{0} \\
    x_{1} \\
    x_{2} \\
    x_{3} \\
    x_{4} \\
    u \end{array} \right]
\end{align*}

The construction of the state variable description of the pipelined tapped Schur
one multiplier lattice filter with denominator polynomial coefficients only for
$z^{-2n}$ is summarised in
Algorithm~\ref{alg:State-variable-pipelined-One-Multiplier-R2-lattice}.

\begin{algorithm}
Given $N$ even, $\left\{k_{2},k_{4}\hdots,k_{N}\right\}$,
$\left\{\epsilon_{2},\epsilon_{4},\hdots,\epsilon_{N}\right\}$ and
$\left\{c_{0},c_{1},\hdots,c_{N}\right\}$:
\begin{algorithmic}
\State $x^{\prime}_{0}=x_{1}$
\State $x^{\prime}_{1}=-k_{2}x_{0}+\left(1+k_{2}\epsilon_{2}\right)x_{4}$
\State $x^{\prime}_{2}=\left(1-k_{2}\epsilon_{2}\right)x_{0}+k_{2}x_{4}$
\State $x^{\prime}_{3}=c_{0}x_{0}+c_{1}x_{1}+c_{2}x_{4}$

\For{$n=2,\hdots,\frac{N}{2}-1$}
\State $x^{\prime}_{3n-2}=-k_{2n}x_{3n-4}+\left(1+k_{2n}\epsilon_{2n}\right)x_{3n+1}$
\State $x^{\prime}_{3n-1}=\left(1-k_{2n}\epsilon_{2n}\right)x_{3n-4}+ k_{2n}x_{3n+1}$
\State $x^{\prime}_{3n}=x_{3n-3}+c_{2n-1}x_{3n-2}+c_{2n}x_{3n+1}$
\EndFor

\State $x_{3\frac{N}{2}-2}^{\prime} = -k_{N}x_{3\frac{N}{2}-4}+
\left(1+k_{N}\epsilon_{N}\right)u$
\State $\hat{y} = \left(1-k_{N}\epsilon_{N}\right)x_{3\frac{N}{2}-4}+k_{N}u$
\State $y = x_{3\frac{N}{2}-3}+c_{N-1}x_{3\frac{N}{2}-2}+c_{N}u$
\end{algorithmic}
\caption{Construction of a state variable description of the pipelined tapped
  Schur one multiplier lattice filter with denominator polynomial coefficients
  only for $z^{-2n}$.}
\label{alg:State-variable-pipelined-One-Multiplier-R2-lattice}
\end{algorithm}

\subsection{Frequency response of the tapped Schur one multiplier lattice filter\label{sec:Frequency-response-of-tapped-Schur-one-multiplier}}
Eliminating the filter state, $\mathbf{x}$, from
Equation~\ref{eqn:State-variable-representation} gives:
\begin{align}
\mathbf{F}\left(z\right) &= 
    \mathbf{C}\left(z\mathbf{I}-\mathbf{A}\right)^{-1}\mathbf{B}+\mathbf{D}
\label{eqn:State-variable-response}
\end{align}
\emph{Thiele}~\cite{Theile_SensitivityLinearStateSpaceSystems} uses the identity:
\begin{align*}
  \mathbf{R}\mathbf{R}^{-1} &= \mathbf{I} \\
\frac{\partial\mathbf{R}}{\partial\chi}\mathbf{R}^{-1} 
  +\mathbf{R}\frac{\partial\mathbf{R}^{-1}}{\partial\chi} &= 0 \\
  \frac{\partial\mathbf{R}}{\partial\chi} 
    &=-\mathbf{R}\frac{\partial\mathbf{R}^{-1}}{\partial\chi}\mathbf{R}
\end{align*}
\emph{Theile} shows the sensitivity functions of $\mathbf{F}\left(z\right)$ with
respect to the components $\alpha$, $\beta$, $\gamma$ and $\delta$ of
$\mathbf{A}$, $\mathbf{B}$, $\mathbf{C}$ and $\mathbf{D}$ respectively:
\begin{align*}
\frac{\partial\mathbf{F}}{\partial{}z} &= 
-\mathbf{C}\left(z\mathbf{I}-\mathbf{A}\right)^{-2}\mathbf{B} \\
  \frac{\partial\mathbf{F}}{\partial\alpha} &= 
\mathbf{C}\left(z\mathbf{I}-\mathbf{A}\right)^{-1}\frac{\partial\mathbf{A}}{\partial\alpha}
\left(z\mathbf{I}-\mathbf{A}\right)^{-1}\mathbf{B} \\
\frac{\partial\mathbf{F}}{\partial\beta} &= 
\mathbf{C}\left(z\mathbf{I}-\mathbf{A}\right)^{-1}\\
\frac{\partial\mathbf{F}}{\partial\gamma} &= 
\left(z\mathbf{I}-\mathbf{A}\right)^{-1}\mathbf{B}\\
\frac{\partial\mathbf{F}}{\partial\delta} &= \mathbf{I}
\end{align*}
Substituting $z=e^{\imath\omega}$ into Equation~\ref{eqn:State-variable-response}
gives the complex frequency response of the state variable filter on the unit
circle. In the following I will use
$\mathbf{R}=\left(e^{\imath\omega}\mathbf{I}-\mathbf{A}\right)^{-1}$. The
components $\alpha$ of $\mathbf{A}$, etc. may themselves be functions of other
variables (for example, the $c$ and $k$ coefficients of a tapped Schur one
multiplier lattice filter) that I will represent by $\chi$. The sensitivity
functions of the complex frequency response
$\mathbf{F}\left(e^{\imath\omega}\right)$ with respect to $\omega$ and $\chi$ are:
\begin{align*}
  \frac{\partial\mathbf{R}}{\partial{}\omega}
  =& -\imath{}e^{\imath\omega}\mathbf{R}\mathbf{R}\\
\frac{\partial\mathbf{F}}{\partial\omega}=&
-\imath e^{\imath\omega}\mathbf{C}\mathbf{R}\mathbf{R}\mathbf{B} \\
  \frac{\partial\mathbf{F}}{\partial{}\chi}=
   &\frac{\partial\mathbf{C}}{\partial\chi}\mathbf{R}\mathbf{B}+
\mathbf{C}\mathbf{R}\frac{\partial\mathbf{A}}{\partial\chi}\mathbf{R}\mathbf{B}+
\mathbf{C}\mathbf{R}\frac{\partial\mathbf{B}}{\partial\chi}+
     \frac{\partial\mathbf{D}}{\partial\chi}\\
\frac{\partial{}^{2}\mathbf{F}}{\partial\chi\partial\omega}=&
-\imath{}e^{\imath\omega}\left[
\frac{\partial\mathbf{C}}{\partial\chi}\mathbf{R}\mathbf{R}\mathbf{B}+
\mathbf{C}\mathbf{R}\mathbf{R}\frac{\partial\mathbf{A}}{\partial\chi}\mathbf{R}\mathbf{B}+
\mathbf{C}\mathbf{R}\frac{\partial\mathbf{A}}{\partial\chi}\mathbf{R}\mathbf{R}\mathbf{B}+
\mathbf{C}\mathbf{R}\mathbf{R}\frac{\partial\mathbf{B}}{\partial\chi}\right]
\end{align*}

The complex frequency response of the filter is:
\begin{align*}
\mathbf{F} &=\Re\mathbf{F} + \imath\Im\mathbf{F}
\end{align*}

The squared-magnitude response of the filter is:
\begin{align*}
\left|\mathbf{F}\right|^{2} &=\Im \mathbf{F}^{2} + \Re \mathbf{F}^{2}
\end{align*}

The gradients of the squared-magnitude response of the filter with respect to 
the Schur lattice filter coefficients are:
\begin{align*}
\frac{\partial\left|\mathbf{F}\right|^{2}}{\partial\chi}
&= 2\left[\Im \mathbf{F}\Im\frac{\partial\mathbf{F}}{\partial\chi}+
          \Re \mathbf{F}\Re\frac{\partial\mathbf{F}}{\partial\chi}\right]
\end{align*}

The gradient of the squared-magnitude response of the filter with respect to 
the angular frequency is:
\begin{align*}
\frac{\partial\left|\mathbf{F}\right|^{2}}{\partial\omega}
&= 2\left[\Im \mathbf{F}\Im\frac{\partial{}\mathbf{F}}{\partial\omega}+
          \Re \mathbf{F}\Re\frac{\partial{}\mathbf{F}}{\partial\omega}\right]
\end{align*}
so that:
\begin{align*}
  \frac{\partial^{2}  \left|\mathbf{F}\right|^{2}}
  {\partial\omega\partial\chi}
  &= 2\left[ \Im\frac{\partial{}\mathbf{F}}{\partial\chi}
             \Im\frac{\partial{}\mathbf{F}}{\partial\omega}+
    \Im \mathbf{F}\Im\frac{\partial^{2}\mathbf{F}}{\partial\omega\partial\chi}+
    \Re\frac{\partial{}\mathbf{F}}{\partial\chi}
    \Re\frac{\partial{}\mathbf{F}}{\partial\omega}+
    \Re \mathbf{F}\Re\frac{\partial^{2}\mathbf{F}}{\partial\omega\partial\chi}
  \right]
\end{align*}

The phase response, $P$, of the filter is:
\begin{align*}%
P&=\arctan\frac{\Im \mathbf{F}}{\Re \mathbf{F}}
\end{align*}

The gradients of the phase response of the filter with respect to 
the Schur lattice filter coefficients are given by:
\begin{align*}
\left|\mathbf{F}\right|^{2}\frac{\partial{}P}{\partial\chi}&=
\Re \mathbf{F}\Im\frac{\partial{}\mathbf{F}}{\partial\chi}-
\Im \mathbf{F}\Re\frac{\partial{}\mathbf{F}}{\partial\chi}
\end{align*}

The group delay response, $T$, of the filter is:
\begin{align*}
T &= -\frac{\partial P}{\partial\omega}
\end{align*}
so that:
\begin{align*}
\left|\mathbf{F}\right|^{2}T &= -\left[\Re \mathbf{F}\Im\frac{\partial \mathbf{F}}{\partial\omega} -
                              \Im \mathbf{F}\Re\frac{\partial \mathbf{F}}{\partial\omega}\right]
\end{align*}

The gradients of the group delay response with respect to the Schur lattice
filter coefficients are given by:
\begin{align*}
  \frac{\partial\left|\mathbf{F}\right|^{2}}{\partial\chi}T+
  \left|\mathbf{F}\right|^{2}\frac{\partial{}T}{\partial\chi} = 
&-\left[\Re\frac{\partial{}\mathbf{F}}{\partial\chi}
\Im\frac{\partial{}\mathbf{F}}{\partial\omega} +
\Re \mathbf{F} \Im\frac{\partial^{2}\mathbf{F}}{\partial\chi\partial\omega} -
  \Im\frac{\partial{}\mathbf{F}}{\partial\chi}
  \Re\frac{\partial{}\mathbf{F}}{\partial\omega} -
  \Im \mathbf{F}\Re\frac{\partial^{2}\mathbf{F}}
                        {\partial\chi\partial\omega}\right]
\end{align*}

\section{Computer-Aided-Design of Schur lattice filters}
The filter design procedure commences with the specification of the required
filter amplitude, phase and group delay responses. The optimisation of an IIR
frequency response with respect to the coefficients is not a convex problem. I
hope to find a ``good enough'' design rather than a global optimum.

One formulation of the filter optimisation problem is to minimise the
\emph{weighted squared error} of the frequency response of $F\left(z\right)$:
\begin{align}
\begin{split}
  \textbf{minimise}\quad &\mathcal{E}_{F}\left(\chi\right) =
  \int W\left(\omega\right)\left|F\left(\omega\right)
    -F_{d}\left(\omega\right)\right|^{2}d\omega \\
\textbf{subject to}\quad &F\left(\omega\right)\text{ is stable} 
\end{split}
\label{eqn:Definition-of-squared-response-error}
\end{align}
where $\chi$ is the coefficient vector of the filter, $F\left(z\right)$,
$\mathcal{E}_{F}\left(\chi\right)$ is the weighted sum of the squared error,
$F\left(\omega\right)$ is the filter frequency response,
$W\left(\omega\right)$ is the frequency weighting and
$F_{d}\left(\omega\right)$ is the desired filter complex frequency response.
The integrand of Equation~\ref{eqn:Definition-of-squared-response-error} could
be the weighted sum of one or more of the squared-magnitude, phase or group
delay responses.

The filter response optimisation problem can also be expressed as a
\emph{mini-max} optimisation problem:
\begin{align}
\label{eqn:Definition-of-minimax-response-error}
\begin{split}
\textbf{minimise}\quad &\max\;\left|F\left(\omega\right)
  -F_{d}\left(\omega\right)\right|\\
\textbf{subject to}\quad &F\left(\omega\right)\text{ is stable} 
\end{split}
\end{align}

I begin by finding an initial filter design that approximates the desired
response, then optimising the minimum mean-squared-error of the
calculated response and, if desired, finding a mini-max approximation to
the desired response. Finally, I describe two methods of
searching for integer valued filter coefficients: a depth-first branch-and-bound
search and a successive relaxation method.

\subsection{Finding an initial filter\label{sec:Finding-initial-filter}}
The initial filter could be an FIR filter or a ``classical'' Butterworth,
Chebyshev, etc. IIR filter design.
\emph{Deczky}~\cite{Deczky_MinPSynthesisIIRDigitalFilters} and
\emph{Richards}~\cite{Richards_DeczkyRecursiveDecimator} begin with an
initial filter consisting of a ``by-eye'' arrangement of the filter poles and
zeros. \emph{Tarczynski et
  al.}~\cite{TarczynskiCainHermanowiczRojewski_WISEMethodDesignIIRFilters}
propose minimising the filter response mean-squared-error, $\mathcal{E}_{F}$,
weighted with a barrier function:
\begin{align}
\label{eqn:WISE-barrier-function}
\left(1-\lambda\right)\mathcal{E}_{F}+\lambda\sum^{T+M}_{t=T+1}h^{2}\left(t\right)
\end{align}
where $\lambda$, $T$ and $M$ are suitable constants and $h\left(t\right)$ is the
impulse response of the filter $H\left(z\right)=\frac{1}{A_{N}\left(z\right)}$.
The barrier function (the second part of
Equation ~\ref{eqn:WISE-barrier-function}) is intended to be small when the
filter, $F\left(z\right)$, is stable and increase rapidly when the poles
approach the unit circle. \emph{Tarczynski et al.} provide heuristics for
selecting $\lambda$, $T$ and $M$. Typically,
$\lambda \in \left[10^{-10},10^{-3}\right]$, $T\in \left[ 100, 500 \right]$ and
$M=NR$. \emph{Roberts and Mullis}~\cite[Section 8.3]
{RobertsMullis_DigitalSignalProcessing} show that the state space description of
the direct-form implementation of $H\left(z\right)$ is:
\begin{align*}
\left[\begin{array}{c}
x\left(t+1\right)\\
y\left(t\right)
\end{array}\right] &= \left[\begin{array}{cc}
A & B\\
C & D
\end{array}\right]\left[\begin{array}{c}
x\left(t\right)\\
u\left(t\right)
\end{array}\right]
\end{align*}
where:
\begin{align*}
A &= \left[\begin{array}{cccc}
0 & 1 & \cdots & 0\\
\vdots & \vdots & \ddots & \vdots\\
0 & 0 & \cdots & 1\\
-a_{N} & -a_{N-1} & \cdots & -a_{1}
\end{array}\right]\\
B &= \left[\begin{array}{ccccc}0 & 0 & \cdots & 0 & 1\end{array}\right]^{\top}\\
C &= \left[\begin{array}{ccc}-a_{N} & \cdots & -a_{1}\end{array}\right]\\
D &= 1
\end{align*}

The corresponding impulse response
is~\cite[Equation 8.3.21]{RobertsMullis_DigitalSignalProcessing}:
\begin{align*}
h\left(t\right) &= \begin{cases}
0 & t<0\\
D & t=0\\
CA^{t-1}B & t>0
\end{cases}
\end{align*}

\subsection{Minimum-Mean-Squared-Error optimisation of IIR filters}
Minimisation of the mean squared error of the filter response proceeds by
successive solution of the following linear approximation problem
for $\Delta_{\chi}$:
\begin{align}
\begin{split}
  \textbf{minimise}\quad & \epsilon+\beta \\
  \textbf{subject to}\quad & \mathnorm{\mathcal{E}_{F}\left(\chi\right) +
          \Delta_{\chi}\nabla\mathcal{E}_{F}\left(\chi\right)}\le\epsilon \\
  & \mathnorm{\Delta_{\chi}}\le\beta \\
  & F\left(\omega\right)\text{ is stable}
\end{split}
\label{eqn:Optimisation-of-minimum-mean-squared-response-error}
\end{align}
After each step the new coefficient vector is $\chi+\Delta_{\chi}$. 

In fact I replace $\mathcal{E}_{F}$ with an approximation to the sum at discrete
frequencies of the weighted squared error of the real valued squared magnitude,
phase, group delay and the gradient of the squared magnitude with respect to
angular frequency. For example, the first order approximation to the squared
error of the group delay, $T$, is:
\begin{align*}
  \bar{\mathcal{E}}_{T}\left(\chi+\Delta_{\chi}\right)
  &\approx \sum_{\omega} W_{T}\left(\omega\right)
     \left[T\left(\chi,\omega\right) + \right.
    \left. \Delta_{\chi}\nabla_{\chi}T\left(\chi,\omega\right) - \right.
    \left. T_{d}\left(\omega\right)\right]^{2}
\end{align*}
with gradient with respect to $\chi$:
\begin{align*}
  \nabla_{\chi} \bar{\mathcal{E}}_{T}\left(\chi\right)
  &\approx 2 \sum_{\omega} W_{T}\left(\omega\right)
    \left[T\left(\chi,\omega\right)-T_{d}\left(\omega\right)\right]
    \nabla_{\chi}T\left(\chi,\omega\right)
\end{align*}
The required gradients of the state variable filter frequency response were
derived in Section~\ref{sec:Frequency-response-of-tapped-Schur-one-multiplier}.

\subsection{Peak-Constrained-Least-Squares optimisation of IIR filters}
Linear phase FIR filter design
algorithms~\cite{Hofstetter_DesignNonRecursiveDigitalFilters,
  ParksMcClellan_ChebyshevApproxNonRecursiveDigitalFilters,
  SelesnickBurrus_ExchangeAlgorithmsLinearPhaseFIRFilters,
  SelesnickLangBurrus_ConstrainedLeastSquareMultiBandFIRFilters} commonly
employ an ``exchange algorithm'' that, given an initial approximation to the
desired amplitude response, finds the extremal frequencies of that response and
interpolates the coefficients to obtain a new set of filter coefficients that
achieves the desired ripple values at all of those frequencies. The process
repeats until a satisfactory mini-max amplitude response is found. I have
successfully solved the IIR mini-max problem of
Equation~\ref{eqn:Definition-of-minimax-response-error} with a modified version
of the ``Peak-Constrained-Least-Squares'' (PCLS) algorithm of \emph{Selesnick},
\emph{Lang} and \emph{Burrus}~\cite[p.498]
{SelesnickLangBurrus_ConstrainedLeastSquareMultiBandFIRFilters}. They describe
the exchange of amplitude response, $A\left(\omega\right)$, constraints as
follows:
\begin{quotation}
  After each iteration, the algorithm checks the values of
  $A\left(\omega\right)$ over the previous constraint set frequencies. $\hdots$
  However, if it is found that $A\left(\omega\right)$ violates the constraints
  at some frequency belonging to the previous constraint set, $R$, then i) that
  frequency where the violation is greatest is appended to the current
  constraint set, $S$, and ii) the same frequency is removed from the
  record of previous constraint set frequencies, $R$.
\end{quotation}

Figure~\ref{fig:Modified-Selesnick-Lang-Burrus-algorithm} shows a flow-chart of
the modified \emph{Selesnick}, \emph{Lang} and \emph{Burrus} exchange algorithm.

\begin{figure}
\centering
\includegraphics[scale=\DesignOfSchurLatticeFiltersIncludeScale]{slb_exchange_algorithm}
\caption{Modified Selesnick, Lang and Burrus exchange algorithm~\cite[p.498]
  {SelesnickLangBurrus_ConstrainedLeastSquareMultiBandFIRFilters}.}
\label{fig:Modified-Selesnick-Lang-Burrus-algorithm}
\end{figure}

Similarly to $\bar{\mathcal{E}}_{T}$, the PCLS constraints for the group delay
response, $T\left(\chi,\omega\right)$, are approximated to first order by:
\begin{align*}
  T_{l}\left(\omega\right) \le
  T\left(\chi,\omega\right) + \Delta_{\chi}\nabla_{\chi}T\left(\chi,\omega\right)
  \le T_{u}\left(\omega\right)
\end{align*}
where $T_{l}\left(\omega\right)$ and $T_{u}\left(\omega\right)$ are the
lower and upper constraints on the group delay.

The modified PCLS mini-max optimisation is:
\begin{align}
\begin{split}
  \textbf{minimise}\quad & \epsilon+\beta \\
  \textbf{subject to}\quad & \mathnorm{\bar{\mathcal{E}}_{F}\left(\chi\right) +
          \Delta_{\chi}\nabla\bar{\mathcal{E}}_{F}\left(\chi\right)}\le\epsilon \\
  & \mathnorm{\Delta_{\chi}}\le\beta \\
  & F\left(\omega\right)\text{ is stable} \\
  & \text{PCLS response constraints are satisfied}
\end{split}
\label{eqn:Optimisation-of-PCLS-response-error}
\end{align}

\subsection{Second Order Cone Programming}
 \emph{Alizadeh} and \emph{Goldfarb}~\cite
{AlizadehGoldfarb_SecondOrderCone} describe Second Order Cone
Programming (SOCP) as the solution of a class of convex optimisation problems in
which a linear function is minimised subject to a set of conic constraints.
In the following example I use the \emph{SeDuMi}
(\emph{Se}lf-\emph{Du}al-\emph{Mi}nimisation) SOCP solver originally written
by \emph{Jos Sturm}~\cite{Sturm_SeDuMi_GitHub}. 
\emph{Lu}~\cite[Section III]{Lu_SedumiRemarks} provides an example of expressing
an optimisation problem of
Equation~\ref{eqn:Optimisation-of-PCLS-response-error} in the form accepted by
SeDuMi. In \emph{Lu}'s notation the problem is:
\begin{subequations}
\begin{align}
\textbf{minimise}  \quad&\boldsymbol{b}^{\top}\boldsymbol{x}
\label{eqn:Lu-SeDuMi-Remarks-problem-format}\\
\textbf{subject to}\quad&\|\boldsymbol{A}_{i}^{\top}\boldsymbol{x}+
\boldsymbol{c}_{i}\| \le \boldsymbol{b}_{i}^{\top}\boldsymbol{x}+d_{i}
\quad \text{for}\;i=1,\hdots,q
\label{eqn:Lu-SeDuMi-Remarks-problem-format-conic-constraints}\\
&\boldsymbol{D}^{\top}\boldsymbol{x}+\boldsymbol{f}\ge\boldsymbol{0}
\label{eqn:Lu-SeDuMi-Remarks-problem-format-linear-constraints}
\end{align}
\end{subequations}
where $\boldsymbol{x}\in\mathfrak{\mathbb{R}}^{m\times 1}$, 
$\boldsymbol{b}\in\mathfrak{\mathbb{R}}^{m\times 1}$,
$\boldsymbol{A}_{i}\in\mathfrak{\mathbb{R}}^{m\times\left(n_{i}-1\right)}$
\footnote{$n_{i}-1$ to allow for the column $b_{i}$ in the matrix 
$\boldsymbol{A}_{t}^{\left(i\right)}$ and $d_{i}$ in the column vector 
$\boldsymbol{c}_{t}^{\left(i\right)}$}, 
$\boldsymbol{c}_{i}\in\mathfrak{\mathbb{R}}^{\left(n_{i}-1\right)\times 1}$,
$\boldsymbol{b}_{i}\in\mathfrak{\mathbb{R}}^{m\times 1}$, 
$d_{i}\in\mathfrak{\mathbb{R}}$ for $1 \le i \le q$,
$\boldsymbol{D}\in\mathfrak{\mathbb{R}}^{m\times p}$ and
$\boldsymbol{f}\in\mathfrak{\mathbb{R}}^{p\times 1}$. The problem is cast into
SeDuMi format by defining:
\begin{align*}
\boldsymbol{A}_{t}&=\left[-\boldsymbol{D} \quad \boldsymbol{A}_{t}^{\left(1\right)}
\hdots \quad \boldsymbol{A}_{t}^{\left(q\right)}\right]\\
\boldsymbol{A}_{t}^{\left(i\right)} &= -\left[\boldsymbol{b}_{i} \quad
\boldsymbol{A}_{i}\right]\nonumber\\
\boldsymbol{b}_{t}&=-\boldsymbol{b}\nonumber\\
\boldsymbol{c}_{t}&=\left[\boldsymbol{f};\quad \boldsymbol{c}_{t}^{\left(1\right)}
;\quad \hdots\quad \boldsymbol{c}_{t}^{\left(q\right)}\right]\nonumber\\
\boldsymbol{c}_{t}^{\left(i\right)}&=\left[d_{i} ;\quad \boldsymbol{c}_{i}\right]
\nonumber
\end{align*}

For Equation~\ref{eqn:Optimisation-of-PCLS-response-error},
$\boldsymbol{x}=\left[\epsilon,\beta,\chi\right]$,
Equation~\ref{eqn:Lu-SeDuMi-Remarks-problem-format-conic-constraints} applies
the conic constraints on the linear approximation to the minimum mean squared
error of the frequency response and 
Equation~\ref{eqn:Lu-SeDuMi-Remarks-problem-format-linear-constraints} applies
the linear constraints on both the reflection coefficients,
$\mathabs{k_{n}}<1$, and the PCLS frequency response constraints.

\section{\label{sec:Design-IIR-lattice-filters-with-signed-digit-coefficients}Design of digital filters with signed-digit coefficients}
\subsection{The canonical signed-digit representation of filter coefficients}

\emph{Parhi}~\cite[Section 13.6]{Parhi_VLSIDigitalSignalProcessingSystems} lists
the following properties of the \emph{canonical} binary signed-digit (CSD)
representation of a binary signed-digit number $A=a_{W-1}a_{W-2}\cdots{}a_{1}a_{0}$
where each $a_{i}\in \left\{-1,0,1\right\}$:
\begin{itemize}
\item no 2 consecutive bits in a CSD number are non-zero 
\item the CSD representation of a number contains the miniumum possible number
of non-zero bits, thus the name \emph{canonic}
\item the CSD representation of a number is unique
\end{itemize}
The first property breaks the carry chain when adding two CSD numbers.
\emph{Parhi}~\cite[Section 13.6.1]{Parhi_VLSIDigitalSignalProcessingSystems}
shows an algorithm that calculates the \emph{canonical} binary signed-digit 
representation from the two's complement representation, reproduced here as
Algorithm~\ref{alg:twos-complement-to-signed-digit-conversion}.

\begin{algorithm}[htbp]
Denote the two's complement representation of the number $A$ as 
$A=\hat{a}_{W-1}\hat{a}_{W-2}\cdots\hat{a}_{1}\hat{a}_{0}$. \\
Denote the CSD representation of $A$ as $A=a_{W-1}a_{W-2}\cdots{}a_{1}a_{0}$.

\begin{algorithmic}
\State $\hat{a}_{-1}=0$
\State $\gamma_{-1}=0$
\State $\hat{a}_{W}=\hat{a}_{W-1}$
\For{$k=0,\hdots,W-1$}
\State $\theta_{i}=\hat{a}_{i}\oplus\hat{a}_{i-1}$
\State $\gamma_{i}=\bar{\gamma}_{i-1}\theta_{i}$
\State $a_{i}=\left(1-2\hat{a}_{i+1}\right)\gamma_{i}$
\EndFor
\end{algorithmic}
\caption{Conversion of 2's complement numbers to the canonical signed-digit
  representation (\emph{Parhi}~\cite[Section
  13.6.1]{Parhi_VLSIDigitalSignalProcessingSystems}).}
\label{alg:twos-complement-to-signed-digit-conversion}
\end{algorithm}

\emph{Lim et al.}~\cite{Lim_SignedPowerOfTwoAllocationDigitalFilters} describe
a method of allocating  a limited number of signed power-of-two digit terms 
to the fixed point coefficients of a digital filter. The method is based 
on the belief that ``\emph{allocating the SPT terms in such a way that all the 
coefficient values have the same quantisation step-size to coefficient
sensitivity ratio will lead to a good design}''. 

\emph{Lim et al.} first prove properties of the canonical signed-digit
representation~\cite[Section II]{Lim_SignedPowerOfTwoAllocationDigitalFilters}.
In particular:

\emph{Property 1}: Define $S_{Q}$ as the set of contiguous integers that can be 
represented by up to $Q$ signed digits. The largest integer in $S_{Q}$ is
$J_{Q}=\sum_{l=0}^{Q-1}2^{2l+1}$.

\emph{Property 5}: On average, $0.72Q$ signed-digits are required to represent
the integers in $S_{Q}$.

\emph{Lim et al.} show that an estimate of the number of signed digits, $Q$, 
required to represent $J_{Q}$ is:
\begin{align*}
  Q \approx \frac{1}{2}\log_{2}J_{Q}+0.31
\end{align*}
Replacing $J_{Q}$ by an integer $n\in S_{Q}$, an estimate of the average number
of terms, $Q_{A}$, required to represent $n$ is:
\begin{align*}
  Q_{A}&\approx 0.72Q\\
      &\approx 0.36 \log_{2}n+0.22
\end{align*}

Now suppose that $D$ signed digits are available to represent two positive
integers $n_{1}$ and $n_{2}$. If $n_{1}\approx n_{2}$ then each integer is
allocated $\frac{D}{2}$ bits. If $n_{1}>n_{2}$ then \emph{Lim et al.} argue 
that the number of additional signed-digits, $Q_{E}$, required to represent 
$n_{1}$ is:
\begin{align*}
  Q_{E}&\approx 0.36\log_{2}\lfloor \frac{n_{1}}{n_{2}}\rfloor
\end{align*}
where $\lfloor\;\rfloor$ represents the integer part. In general, 
$Q_{E}$ is not an integer.

\emph{Lim et al.} go on to consider the allocation of signed digits to the
coefficients of a symmetric FIR filter. The change in the frequency response,
$F\left(\omega\right)$, of a filter due to a change $\Delta\chi_{n}$ in
coefficient $\chi_{n}$ is:
\begin{align*}
\Delta{} F\left(\omega,n\right)&  \approx
\frac{\partial{}F\left(\omega\right)}{\partial\chi_{n}}\Delta{}\chi_{n}
\end{align*}
\emph{Lim et al.} use the average of the coefficient sensitivity to define a
cost for the $n$'th coefficient:
\begin{align*}
cost_{n}&=0.36\log_{2}\left|\chi_{n}\right| + 0.36\log_{2}\int^{\pi}_{0}
\left|\frac{\partial{}F\left(\omega\right)}{\partial{}\chi_{}}\right|d\omega
\end{align*}

Given a total of $D$ signed-digits, \emph{Lim et al.} assign a single 
signed-digit at a time to the coefficient with the largest cost. After a 
coefficient is given a signed-digit, its cost is decreased by one. The
process is repeated until all $D$ digits have been allocated.

\subsection{Branch-and-bound search for the filter coefficients}
Given the $K$ floating point coefficients, $\chi$, of a filter, an exhaustive
search of the upper and lower bounds on the integer or signed-digit
approximations to these coefficients would require
$\mathcal{O}\left(2^{K}\right)$ comparisons of the corresponding filter
approximation error.
\emph{Branch-and-bound}~\cite{LandDoig_AutomaticMethodSolvingDiscreteProgrammingProblems},~\cite[p.627]{Sedgewick_AlgorithmsInCPlusPlus}
is a heuristic for reducing the 
number of branches searched in a binary decision tree. At each branch of the
binary tree the solution is compared to the estimated lower bounds on the cost
of the full path proceeding from that branch. If the cost of that full path is
greater than that of the best full path found so far then further search on
that path is abandoned.  Figure~\ref{fig:branch-bound-tree-search-algorithm}
shows a flow diagram of an implementation of the algorithm using a stack. The
floating point filter coefficients, $\chi$, are approximated by the
signed-digit coefficients, $\bar{\chi}$. Each coefficient, $\chi_{n}$
and $\bar{\chi}_{n}$, is bounded by the corresponding signed-digit numbers $u_{n}$
and $l_{n}$ so that $l_{n}\le\chi_{n}\le{}u_{n}$ and
$l_{n}\le\bar{\chi}_{n}\le u_{n}$. The search for the set of coefficients with
minimum cost is ``depth-first'', starting at the root of the search tree and
fixing successive coefficients.  \emph{Ito et
  al.}~\cite{Ito_PowersOfTwoAllocationFIR} recommend choosing, at each branch,
the $\chi_{n}$ with the greatest difference $u_{n}-l_{n}$. The two sub-problems
at that branch fix $\bar{\chi}_{n}$ to $l_{n}$ and $u_{n}$. One of the two
sub-problems is pushed onto a stack and the other is solved and the cost
calculated. I assume that this cost is the least possible for the remaining
coefficients on the current branch. If the cost of the current sub-problem is
greater than the current minimum cost then the current branch is abandoned and a
new sub-problem is popped off the problem stack. Otherwise, if the search has
reached the maximum depth of the tree then the current solution is a new
signed-digit minimum cost solution. If the current cost is less than the
minimum cost and the search has not reached the maximum depth then the search
continues with a new branch.

\begin{figure}
\centering
\includegraphics[scale=\DesignOfSchurLatticeFiltersIncludeScale]{branch_bound_algorithm}
\caption{Branch-and-bound depth-first search algorithm}
\label{fig:branch-bound-tree-search-algorithm}
\end{figure}

\subsection{Successive relaxation search for the filter coefficients}
A successive relaxation search for the filter coefficients fixes one
coefficient and optimises the filter frequency response with respect to the
remaining free coefficients. This is called a \emph{relaxation} of the
optimisation. The relaxation is repeated until all the coefficients are fixed.
At each coefficient relaxation step the script finds the upper and lower
signed-digit approximations to the current set of active coefficients and
selects the coefficient with the largest difference in those approximations.

\section{An example : design of a low pass differentiator filter}
This section describes the design of a low pass differentiator filter
implemented as $F_{0}\left(z\right)=1-z^{-1}$ in series with a tapped Schur
one-multiplier lattice correction filter, $C\left(z\right)$, having denominator
polynomial coefficients only for terms in $z^{-2n}$. In the pass band, the
desired complex frequency response of a differentiator filter is:
\begin{align*}
  F_{d}\left(\omega\right)=-\imath\frac{\omega}{2}e ^{-\imath\omega t_{p}}
\end{align*}
where $t_{p}$ is the nominal filter pass band group delay in samples.
The squared-magnitude response of the correction filter is:
\begin{align*}
  \mathabs{C\left(\omega\right)}^{2}
  &= \frac{\mathabs{F_{d}\left(\omega\right)}^{2}}
    {\mathabs{F_{0}\left(\omega\right)}^{2}}
\end{align*}
where:
\begin{align*}
  \mathabs{F_{d}\left(\omega\right)} &= \frac{\omega}{2} \\
  \mathabs{F_{d}\left(\omega\right)}^{2}&= \frac{\omega^{2}}{4} \\  
  \frac{d\mathabs{F_{d}\left(\omega\right)}^{2}}{d\omega}
     &= \frac{\omega}{2} = \mathabs{F_{d}\left(\omega\right)}
\end{align*}
and the response of the zero at $z=1$ is:
\begin{align*}
  F_{0}\left(\omega\right) &= 1-e^{\imath\omega} \\
        &=2\imath{} e^{\imath\frac{\omega}{2}}\sin\frac{\omega}{2} \\
  \mathabs{F_{0}\left(\omega\right)} &= 2\sin\frac{\omega}{2} \\
  \frac{d\mathabs{F_{0}\left(\omega\right)}^{2}}{d\omega}
        &=2\mathabs{F_{0}\left(\omega\right)}
          \frac{d\mathabs{F_{0}\left(\omega\right)}}{d\omega} \\
        &= 2\sin\omega
\end{align*}
By the chain rule for differentiation, the gradient of
$\mathabs{F\left(\omega\right)}^{2}$ is:
\begin{align*}
  \frac{d\mathabs{F_{d}\left(\omega\right)}^{2}}{d\omega}
  &= \mathabs{F_{0}\left(\omega\right)}^{2}
     \frac{d\mathabs{C\left(\omega\right)}^{2}}{d\omega} +
    \frac{\mathabs{F_{d}\left(\omega\right)}^{2}}
    {\mathabs{F_{0}\left(\omega\right)}^{2}}
    \frac{d\mathabs{F_{0}\left(\omega\right)}^{2}}{d\omega}
\end{align*}
Substituting and rearranging to obtain the desired gradient of
$\mathabs{C\left(\omega\right)}^{2}$ in terms of the desired overall squared
amplitude response, $\mathabs{F_{d}\left(\omega\right)}^{2}$, and the
squared magnitude response of the zero at $z=1$,
$\mathabs{F_{0}\left(\omega\right)}^{2}$:
\begin{align*}
  \frac{d\mathabs{C\left(\omega\right)}^{2}}{d\omega}
  &= \mathabs{F_{d}\left(\omega\right)}
\left[\frac{1-\mathabs{F_{d}\left(\omega\right)}\cot\frac{\omega}{2}}{\mathabs{F_{0}\left(\omega\right)}^{2} }\right]
\end{align*}

\subsection{Initial R=2 low pass differentiator filter}
Figure~\ref{fig:Schur-OneM-lattice-correction-lowpass-differentiator-R2-initial}
shows the frequency response of the initial R=2 low pass differentiator filter
found by the method of \emph{Tarczynski et al.}, described in
Section~\ref{sec:Finding-initial-filter}. The correction filter order is $N=10$,
filter pass band and stop band edge frequencies are $0.2$ and $0.4$\;(normalised
to the digital sample rate), the nominal pass band phase is $1.5\pi$
radians (adjusted for delay) and the nominal pass band group delay is $9$
samples.

\begin{figure}
\centering
\scalebox{\DesignOfSchurLatticeFiltersPdfScale}{\input{schurOneMlattice_socp_slb_lowpass_differentiator_R2_test_initial_response}}
\caption{Amplitude, phase and group delay responses of an initial 
  low pass differentiator filter composed of a Schur one-multiplier lattice
  correction filter with a denominator polynomial having terms only in $z^{-2}$
  in series with $1-z^{-1}$.}
\label{fig:Schur-OneM-lattice-correction-lowpass-differentiator-R2-initial}
\end{figure}

The coefficients of the numerator and denominator polynomials of the initial
correction filter transfer function are:
\begin{small}
\verbatiminput{schurOneMlattice_socp_slb_lowpass_differentiator_R2_test_N0_coef.m}
\verbatiminput{schurOneMlattice_socp_slb_lowpass_differentiator_R2_test_D0R_coef.m}
\end{small}

The corresponding Schur one-multiplier lattice coefficients of the initial
correction filter are:
\begin{small}
\verbatiminput{schurOneMlattice_socp_slb_lowpass_differentiator_R2_test_k0_coef.m}
\verbatiminput{schurOneMlattice_socp_slb_lowpass_differentiator_R2_test_epsilon0_coef.m}
\verbatiminput{schurOneMlattice_socp_slb_lowpass_differentiator_R2_test_c0_coef.m}
\end{small}

\subsection{PCLS design of the R=2 low pass differentiator filter}
The low pass differentiator correction filter was PCLS optimised with 
amplitude pass band error peak-to-peak ripple of $0.0009$,
amplitude stop band peak ripple of $0.007$,
phase pass band peak-to-peak ripple of $0.0002\pi$\; radians,
group delay pass band peak-to-peak ripple of $0.006$\; samples and
correction filter gradient of amplitude-squared pass band error peak-to-peak
ripple of $0.02$.

The Schur one-multiplier lattice coefficients of the PCLS SOCP optimised
correction filter are:
\begin{small}
\verbatiminput{schurOneMlattice_socp_slb_lowpass_differentiator_R2_test_k2_coef.m}
\verbatiminput{schurOneMlattice_socp_slb_lowpass_differentiator_R2_test_epsilon2_coef.m}
\verbatiminput{schurOneMlattice_socp_slb_lowpass_differentiator_R2_test_c2_coef.m}
\end{small}

The corresponding correction filter numerator and denominator polynomial
coefficients are:
\begin{small}
\verbatiminput{schurOneMlattice_socp_slb_lowpass_differentiator_R2_test_N2_coef.m}
\verbatiminput{schurOneMlattice_socp_slb_lowpass_differentiator_R2_test_D2_coef.m}
\end{small}
Figure~\ref{fig:Schur-OneM-lattice-lowpass-differentiator-R2}
shows the  amplitude error, phase and group delay response errors of the $R=2$
low pass differentiator filter after PCLS SOCP optimisation.
Figure~\ref{fig:Schur-OneM-lattice-lowpass-differentiator-R2-dCsqdw-error}
shows the error in the gradient of the squared-magnitude response of the $R=2$
correction filter after PCLS SOCP optimisation.
Figure~\ref{fig:Schur-OneM-lattice-lowpass-differentiator-R2-pz} shows
the pole-zero plot of the $R=2$ low pass differentiator filter.
Figure~\ref{fig:Schur-OneM-lattice-correction-lowpass-differentiator-direct-sens}
shows the gradients of the correction filter squared-magnitude, phase and delay
with respect to the coefficients of the direct form implementation of the filter.
Figure~\ref{fig:Schur-OneM-lattice-correction-lowpass-differentiator-Schur-sens}
shows the gradients of the correction filter squared-magnitude, phase and delay
with respect to the coefficients of the Schur lattice implementation of the
filter.

\begin{figure}
\centering
\scalebox{\DesignOfSchurLatticeFiltersPdfScale}{\input{schurOneMlattice_socp_slb_lowpass_differentiator_R2_test_pcls_error}}
\caption{PCLS optimised amplitude error, phase and group delay responses of a
  low pass differentiator filter composed of a Schur one-multiplier lattice
  correction filter with a denominator polynomial having terms only in $z^{-2}$
  in series with $1-z^{-1}$.}
\label{fig:Schur-OneM-lattice-lowpass-differentiator-R2}
\end{figure}

\begin{figure}
\centering
\scalebox{\DesignOfSchurLatticeFiltersPdfScale}{\input{schurOneMlattice_socp_slb_lowpass_differentiator_R2_test_pcls_dCsqdw_error}}
\caption{PCLS optimised pass band error of the gradient of the squared magnitude
  response of the correction filter after PCLS SOCP optimisation.}
\label{fig:Schur-OneM-lattice-lowpass-differentiator-R2-dCsqdw-error}
\end{figure}

\begin{figure}
\centering
\scalebox{\DesignOfSchurLatticeFiltersPdfScale}{\input{schurOneMlattice_socp_slb_lowpass_differentiator_R2_test_pcls_pz}}
\caption{Pole-zero plot of the PCLS optimised low pass differentiator
  filter composed of a Schur one-multiplier lattice correction filter with a
  denominator polynomial having terms only in $z^{-2}$ in series with $1-z^{-1}$.}
\label{fig:Schur-OneM-lattice-lowpass-differentiator-R2-pz}
\end{figure}

\begin{figure}
\centering
\scalebox{\DesignOfSchurLatticeFiltersPdfScale}{\input{schurOneMlattice_socp_slb_lowpass_differentiator_R2_test_direct_sensitivity}}
\caption{Gradients of the pass band response with respect to the coefficients of
  the direct form implementation of the PCLS optimised low pass differentiator
  correction filter.}
\label{fig:Schur-OneM-lattice-correction-lowpass-differentiator-direct-sens}
\end{figure}

\begin{figure}
\centering
\scalebox{\DesignOfSchurLatticeFiltersPdfScale}{\input{schurOneMlattice_socp_slb_lowpass_differentiator_R2_test_schur_sensitivity}}
\caption{Gradients of the pass band response with respect to the coefficients of
  the tapped Schur lattice implementation of the PCLS optimised low pass
  differentiator correction filter.}
\label{fig:Schur-OneM-lattice-correction-lowpass-differentiator-Schur-sens}
\end{figure}

\subsection{Search for the signed-digit coefficients of the R=2 low pass differentiator filter}

When truncated to $12$ bits and $3$ signed-digits the Schur one-multiplier
lattice coefficients of the PCLS optimised correction filter are:
\begin{small}
\verbatiminput{socp_relaxation_schurOneMlattice_lowpass_differentiator_R2_12_nbits_test_k0_sd_no_alloc_coef.m}
\verbatiminput{socp_relaxation_schurOneMlattice_lowpass_differentiator_R2_12_nbits_test_c0_sd_no_alloc_coef.m}
\end{small}

The numbers of signed-digits that the heuristic of \emph{Lim et al.} allocates to
each cofficient of the PCLS optimised correction filter are:
\begin{small}
\verbatiminput{socp_relaxation_schurOneMlattice_lowpass_differentiator_R2_12_nbits_test_k_allocsd_digits.m}
\verbatiminput{socp_relaxation_schurOneMlattice_lowpass_differentiator_R2_12_nbits_test_c_allocsd_digits.m}
\end{small}

The signed-digit coefficients of the PCLS optimised filter with the
numbers of signed-digits allocated by the heuristic of \emph{Lim et al.} are:
\begin{small}
\verbatiminput{socp_relaxation_schurOneMlattice_lowpass_differentiator_R2_12_nbits_test_k0_sd_coef.m}
\verbatiminput{socp_relaxation_schurOneMlattice_lowpass_differentiator_R2_12_nbits_test_c0_sd_coef.m}
\end{small}

The signed-digit coefficients of the PCLS optimised filter with the
numbers of signed-digits allocated by the heuristic of \emph{Lim et al.} and
found with branch-and-bound search are:
\begin{small}
\verbatiminput{branch_bound_schurOneMlattice_lowpass_differentiator_R2_12_nbits_test_k_min_coef.m}
\verbatiminput{branch_bound_schurOneMlattice_lowpass_differentiator_R2_12_nbits_test_c_min_coef.m}
\end{small}
The corresponding correction filter transfer function polynomials found with
branch-and-bound search are:
\begin{small}
\verbatiminput{branch_bound_schurOneMlattice_lowpass_differentiator_R2_12_nbits_test_N_min_coef.m}
\verbatiminput{branch_bound_schurOneMlattice_lowpass_differentiator_R2_12_nbits_test_D_min_coef.m}
\end{small}

The signed-digit coefficients of the PCLS optimised filter with the
numbers of signed-digits allocated by the heuristic of \emph{Lim et al.} and
found with SOCP relaxation search are:
\begin{small}
\verbatiminput{socp_relaxation_schurOneMlattice_lowpass_differentiator_R2_12_nbits_test_k_min_coef.m}
\verbatiminput{socp_relaxation_schurOneMlattice_lowpass_differentiator_R2_12_nbits_test_c_min_coef.m}
\end{small}
The corresponding correction filter transfer function polynomials found with
SOCP relaxation search are:
\begin{small}
\verbatiminput{socp_relaxation_schurOneMlattice_lowpass_differentiator_R2_12_nbits_test_N_min_coef.m}
\verbatiminput{socp_relaxation_schurOneMlattice_lowpass_differentiator_R2_12_nbits_test_D_min_coef.m}
\end{small}

Table~\ref{tab:Comp-schurOneMlattice-lowpass-diff-R2-12-nbits-cost}
compares, for each filter, a cost function, the maximum pass band and stop band
response errors, the total number of signed-digits required by the $12$-bit 
coefficients and the number of shift-and-add operations required to implement
the correction filter coefficient multiplications of the correction filter.

\begin{table}
\centering
\begin{threeparttable}
\begin{tabular}{lrrrrrrrr} \\ \toprule
 &Correction&Max. pass&Max. pass &Max. stop &Max. pass   &Max. pass  &$12$-bit&Shift-\\
 &filter    &amplitude&amplitude &amplitude &phase error &delay error&signed  &and- \\
 &cost      &error    &rel. error&error     &(rad./$\pi$)&(samples)  &digits  &adds \\
\midrule
\input{comparison_no_POP_schurOneMlattice_lowpass_differentiator_R2_12_nbits_test_cost.tab}
\\ \bottomrule
\end{tabular}
\end{threeparttable}
\caption[Comparison of low pass differentiator filter fixed point implementations]
{Comparison of the response errors and the total number of signed digits in the
  12-bit coefficients, each having an average of 3-signed digits, allocated by
  the method of \emph{Lim et al.}, required for a tapped Schur one-multiplier
  lattice, R=2, low pass differentiator correction filter found by
  branch-and-bound and SOCP relaxation search.}
\label{tab:Comp-schurOneMlattice-lowpass-diff-R2-12-nbits-cost}
\end{table}

Figure~\ref{fig:Comp-schurOneMlattice-lowpass-diff-R2-12-nbits-pass}
compares the pass band amplitude response of each filter for
$12$-bit $3$-signed-digit coefficients and $12$-bit coefficients with an average
of $3$-signed-digits allocated by the method of \emph{Lim et al.} found with
branch-and-bound search and SOCP relaxation search.
Figure~\ref{fig:Comp-schurOneMlattice-lowpass-diff-R2-12-nbits-relative-pass}
compares the pass band relative amplitude response of each filter.
Figure~\ref{fig:Comp-schurOneMlattice-lowpass-diff-R2-12-nbits-stop}
compares the stop band amplitude response of each filter.
Figure~\ref{fig:Comp-schurOneMlattice-lowpass-diff-R2-12-nbits-phase} shows
the pass band phase reslonse of each filter.
~\ref{fig:Comp-schurOneMlattice-lowpass-diff-R2-12-nbits-delay} shows the
pass band group-delay response of each filter.
Figure~\ref{fig:Comp-schurOneMlattice-lowpass-diff-R2-12-nbits-dCsqdw} shows
the gradient of the squared magnitude response of the correction filter.

\begin{figure}
\centering
\scalebox{\DesignOfSchurLatticeFiltersPdfScale}{\input{comparison_no_POP_schurOneMlattice_lowpass_differentiator_R2_12_nbits_test_pass}}
\caption{Comparison of the pass band response error of
  a tapped Schur one-multiplier, R=2, lattice lowpass differentiator correction
  filter with floating point coefficients, 12-bit 3-signed-digit coefficients,
  and 12-bit signed-digit coefficients each having an average of 3-signed-digits
  allocated by the method of \emph{Lim et al.}, found by branch-and-bound and
  SOCP relaxation search.}
\label{fig:Comp-schurOneMlattice-lowpass-diff-R2-12-nbits-pass}
\end{figure}

\begin{figure}
\centering
\scalebox{\DesignOfSchurLatticeFiltersPdfScale}{\input{comparison_no_POP_schurOneMlattice_lowpass_differentiator_R2_12_nbits_test_relative_pass}}
\caption{Comparison of the pass band relative response error of
  a tapped Schur one-multiplier, R=2, lattice lowpass differentiator correction
  filter with floating point coefficients, 12-bit 3-signed-digit coefficients,
  and 12-bit signed-digit coefficients each having an average of 3-signed-digits
  allocated by the method of \emph{Lim et al.}, found by branch-and-bound and
  SOCP relaxation search.}
\label{fig:Comp-schurOneMlattice-lowpass-diff-R2-12-nbits-relative-pass}
\end{figure}

\begin{figure}
\centering
\scalebox{\DesignOfSchurLatticeFiltersPdfScale}{\input{comparison_no_POP_schurOneMlattice_lowpass_differentiator_R2_12_nbits_test_stop}}
\caption{Comparison of the stop band response error of
  a tapped Schur one-multiplier, R=2, lattice lowpass differentiator correction
  filter with floating point coefficients, 12-bit 3-signed-digit coefficients,
  and 12-bit signed-digit coefficients each having an average of 3-signed-digits
  allocated by the method of \emph{Lim et al.}, found by branch-and-bound and
  SOCP relaxation search.}
\label{fig:Comp-schurOneMlattice-lowpass-diff-R2-12-nbits-stop}
\end{figure}

\begin{figure}
\centering
\scalebox{\DesignOfSchurLatticeFiltersPdfScale}{\input{comparison_no_POP_schurOneMlattice_lowpass_differentiator_R2_12_nbits_test_phase}}
\caption{Comparison of the pass band phase response errors of a tapped Schur
  one-multiplier, R=2, lattice low pass differentiator correction filter with
  floating point coefficients, 12-bit 3-signed-digit coefficients, and 12-bit
  signed-digit coefficients each having an average of 3-signed-digits allocated
  by the method of \emph{Lim et al.}, found by branch-and-bound and
  SOCP relaxation search.}
\label{fig:Comp-schurOneMlattice-lowpass-diff-R2-12-nbits-phase}
\end{figure}

\begin{figure}
\centering
\scalebox{\DesignOfSchurLatticeFiltersPdfScale}{\input{comparison_no_POP_schurOneMlattice_lowpass_differentiator_R2_12_nbits_test_delay}}
\caption{Comparison of the pass band group delay response errors of a tapped
  Schur one-multiplier, R=2, lattice low pass differentiator correction filter
  with floating point coefficients, 12-bit 3-signed-digit coefficients, and
  12-bit signed-digit coefficients each having an average of 3-signed-digits
  allocated by the method of \emph{Lim et al.}, found by branch-and-bound and
  SOCP relaxation search.}
\label{fig:Comp-schurOneMlattice-lowpass-diff-R2-12-nbits-delay}
\end{figure}

\begin{figure}
\centering
\scalebox{\DesignOfSchurLatticeFiltersPdfScale}{\input{comparison_no_POP_schurOneMlattice_lowpass_differentiator_R2_12_nbits_test_dCsqdw}}
\caption{Comparison of the error in the pass band gradient of the squared
  magnitude response of a tapped Schur one-multiplier, R=2, lattice low pass
  differentiator correction filter with floating point coefficients, 12-bit
  3-signed-digit coefficients, and 12-bit signed-digit coefficients each having
  an average of 3-signed-digits allocated by the method of \emph{Lim et al.},
  found by branch-and-bound and SOCP relaxation search.}
\label{fig:Comp-schurOneMlattice-lowpass-diff-R2-12-nbits-dCsqdw}
\end{figure}

\subsection{Comparison with a maximally-linear  FIR low pass differentiator filter}

\emph{Selesnick}~\cite{Selesnick_MaximallyFlatLowPassDigitalDifferentiators}
describes the design of maximally linear low pass FIR differentiator filters:
\begin{align*}
  F\left(z\right)&=\left(\frac{1-z^{-1}}{2}\right)
                   \left(\frac{1+z^{-1}}{2}\right)^{K}
                   z^{-L}\sum_{n=0}^{L}c_{n}\left[\frac{-z+2-z^{-1}}{4}\right]^{n}
\end{align*}
where the $c_{n}$ are defined recursively:
\begin{align*}
c_{0}&=2\\
c_{1}&=K+\frac{1}{3} \\
    & \hdots \\
c_{n}&=\frac{\left(8n^{2}+4Kn-10n-K+3\right)c_{n-1}-\left(2n+K-3\right)^{2}c_{n-2}}
       {2n\left(2n+1\right)}
\end{align*}
The length of the impulse response is $N=K+2L+2$. The coefficients of the maximally linear anti-symmetric FIR
low pass differentiator with $L=20$ and $K=21$ are\footnote{For odd length,
  $N=63$, $h0\left(32\right)=0$.}:
\begin{small}
\verbatiminput{socp_relaxation_directFIRantisymmetric_lowpass_differentiator_12_nbits_test_h0_coef.m}
\end{small}
By construction, the phase response of the anti-symmetric FIR low pass
differentiator filter is $\frac{\pi}{2}$ radians. This FIR filter was truncated
to $16$ distinct coefficients, the same number as the Schur lattice, and a group
delay of $16$ samples. After SOCP-relaxation optimisation to $12$-bits with an
average of $3$ signed-digits allocated by the heuristic of \emph{Lim et al.}
the distinct coefficients of the filter are:
\begin{small}
\verbatiminput{socp_relaxation_directFIRantisymmetric_lowpass_differentiator_12_nbits_test_hM_min_coef.m}
\end{small}

Table~\ref{tab:Comp-schurOneMlattice-lowpass-diff-R2-12-nbits-hM-min-cost}
compares a cost function, the maximum pass band and stop band response errors,
the total number of signed-digits required by the $12$-bit coefficients and the
number of shift-and-add operations required to implement the coefficient
multiplications of the floating point FIR, floating point truncated FIR and the
SOCP-relaxation optimised truncated FIR and the the SOCP-relaxation optimised
truncated FIR and the branch-and-bound optimised truncated FIR filter.

\begin{table}
\centering
\begin{threeparttable}
\begin{tabular}{lrrrrrr} \\ \toprule
 &Low pass        &Max. pass&Max. pass &Max. stop &$12$-bit&Shift-\\
 &differentiator  &amplitude&amplitude &amplitude &signed  &and- \\
 &cost            &error    &rel. error&error     &digits  &adds \\
\midrule
\input{comparison_no_POP_schurOneMlattice_lowpass_differentiator_R2_12_nbits_test_hM_min_cost.tab}
\\ \bottomrule
\end{tabular}
\end{threeparttable}
\caption[Comparison of low pass differentiator filter FIR implementations]
{Comparison of the response errors and the total number of signed digits in the
  12-bit coefficients, each having an average of 3-signed digits, allocated by
  the method of \emph{Lim et al.}, required for a low pass differentiator filter
  implemented as an anti-symmetric FIR filter found by SOCP relaxation search
  and branch-and-bound search.}
\label{tab:Comp-schurOneMlattice-lowpass-diff-R2-12-nbits-hM-min-cost}
\end{table}

Figure~\ref{fig:Comp-schurOneMlattice-lowpass-diff-R2-12-nbits-hM-min-pass}
compares the pass band error amplitude responses of a low pass
differentiator filter implemented as a maximally linear FIR filter with floating
point coefficients, that FIR filter truncated to $16$ distinct coefficients and
the truncated filter coefficients SOCP-relaxation optimised for $12$-bit
$3$-signed-digit coefficients and $12$-bit coefficients with an average of
$3$-signed-digits allocated by the method of \emph{Lim et al.}.
Similarly,
Figure~\ref{fig:Comp-schurOneMlattice-lowpass-diff-R2-12-nbits-hM-min-pass-relative}
compares the pass band relative error amplitude responses and
Figure~\ref{fig:Comp-schurOneMlattice-lowpass-diff-R2-12-nbits-hM-min-stop}
compares the stop band error amplitude responses.

\begin{figure}
\centering
\scalebox{\DesignOfSchurLatticeFiltersPdfScale}{\input{comparison_no_POP_schurOneMlattice_lowpass_differentiator_R2_12_nbits_test_hM_min_pass_error}}
\caption{Comparison of the pass band error amplitude responses of a low pass
  differentiator filter implemented as a maximally linear FIR filter with
  floating point coefficients, that FIR filter truncated to $16$ distinct
  coefficients and the truncated filter coefficients SOCP-relaxation optimised
  for $12$-bit $3$-signed-digit coefficients and $12$-bit coefficients with an
  average of $3$-signed-digits allocated by the method of \emph{Lim et al.}.}
\label{fig:Comp-schurOneMlattice-lowpass-diff-R2-12-nbits-hM-min-pass}
\end{figure}

\begin{figure}
\centering
\scalebox{\DesignOfSchurLatticeFiltersPdfScale}{\input{comparison_no_POP_schurOneMlattice_lowpass_differentiator_R2_12_nbits_test_hM_min_pass_relative_error}}
\caption{Comparison of the pass band relative error responses of a low pass
  differentiator filter implemented as a maximally linear FIR filter with
  floating point coefficients, that FIR filter truncated to $16$ distinct
  coefficients and the truncated filter coefficients SOCP-relaxation optimised
  for $12$-bit $3$-signed-digit coefficients and $12$-bit coefficients with an
  average of $3$-signed-digits allocated by the method of \emph{Lim et al.}.}
\label{fig:Comp-schurOneMlattice-lowpass-diff-R2-12-nbits-hM-min-pass-relative}
\end{figure}

\begin{figure}
\centering
\scalebox{\DesignOfSchurLatticeFiltersPdfScale}{\input{comparison_no_POP_schurOneMlattice_lowpass_differentiator_R2_12_nbits_test_hM_min_stop}}
\caption{Comparison of the stop band error amplitude responses of a low pass
  differentiator filter implemented as a maximally linear FIR filter with
  floating point coefficients, that FIR filter truncated to $16$ distinct
  coefficients and the truncated filter coefficients SOCP-relaxation optimised
  for $12$-bit $3$-signed-digit coefficients and $12$-bit coefficients with an
  average of $3$-signed-digits allocated by the method of \emph{Lim et al.}.}
\label{fig:Comp-schurOneMlattice-lowpass-diff-R2-12-nbits-hM-min-stop}
\end{figure}

Table~\ref{tab:Comp-schurOneMlattice-lowpass-diff-R2-12-nbits-Schur-FIR-cost}
compares a cost function, the maximum pass band and stop band response errors,
the total number of
signed-digits required by the $12$-bit coefficients and the number of
shift-and-add operations required to
implement the coefficient multiplications of the SOCP-relaxation optimised
truncated FIR and Schur lattice filters and the branch-and-bound optimised
truncated FIR and Schur lattice filters.

\begin{table}
\centering
\begin{threeparttable}
\begin{tabular}{lrrrrrr} \\ \toprule
 &Low pass        &Max. pass&Max. pass &Max. stop &$12$-bit&Shift-\\
 &differentiator  &amplitude&amplitude &amplitude &signed  &and- \\
 &cost            &error    &rel. error&error     &digits  &adds \\
\midrule
\input{comparison_no_POP_schurOneMlattice_lowpass_differentiator_R2_12_nbits_test_Schur_FIR_cost.tab}
\\ \bottomrule
\end{tabular}
\end{threeparttable}
\caption[Comparison of low pass differentiator filter FIR and Schur lattice implementations]
{Comparison of the response errors and the total number of signed digits in the
  12-bit coefficients, each having an average of 3-signed digits, allocated by
  the method of \emph{Lim et al.}, required for a low pass differentiator filter
  implemented as an anti-symmetric FIR filter and a tapped Schur one-multiplier
  lattice, R=2, correction filter found by SOCP relaxation search and
  branch-and-bound search.}
\label{tab:Comp-schurOneMlattice-lowpass-diff-R2-12-nbits-Schur-FIR-cost}
\end{table}

Figure~\ref{fig:Comp-schurOneMlattice-lowpass-diff-R2-12-nbits-Schur-FIR-pass}
compares the pass band amplitude error responses of a low pass
differentiator filter implemented as a maximally linear FIR filter truncated to
$16$ distinct coefficients and a tapped Schur one-multiplier lattice, R=2,
correction filter. The coefficients are SOCP-relaxation and branch-and-bound
optimised for $12$-bit coefficients with an average of
$3$-signed-digits each, allocated by the method of \emph{Lim et al.}\;.
Similarly,
Figure~\ref{fig:Comp-schurOneMlattice-lowpass-diff-R2-12-nbits-Schur-FIR-pass-relative}
compares the pass band relative error amplitude responses and
Figure~\ref{fig:Comp-schurOneMlattice-lowpass-diff-R2-12-nbits-Schur-FIR-stop}
compares the stop band error amplitude responses.

\begin{figure}
\centering
\scalebox{\DesignOfSchurLatticeFiltersPdfScale}{\input{comparison_no_POP_schurOneMlattice_lowpass_differentiator_R2_12_nbits_test_Schur_FIR_pass_error}}
\caption{Comparison of the pass band amplitude error responses of a low pass
  differentiator filter implemented as a maximally linear FIR filter truncated
  to $16$ distinct coefficients and a tapped Schur one-multiplier lattice, R=2,
  correction filter. The coefficients are SOCP-relaxation and branch-and-bound
  optimised for $12$-bit coefficients with an average of
  $3$-signed-digits each, allocated by the method of \emph{Lim et al.}\;.}
\label{fig:Comp-schurOneMlattice-lowpass-diff-R2-12-nbits-Schur-FIR-pass}
\end{figure}

\begin{figure}
\centering
\scalebox{\DesignOfSchurLatticeFiltersPdfScale}{\input{comparison_no_POP_schurOneMlattice_lowpass_differentiator_R2_12_nbits_test_Schur_FIR_pass_relative_error}}
\caption{Comparison of the pass band amplitude relative error responses of a low
  pass differentiator filter implemented as a maximally linear FIR filter
  truncated to $16$ distinct coefficients and a tapped Schur one-multiplier
  lattice, R=2, correction filter. The coefficients are SOCP-relaxation and
  branch-and-bound optimised for $12$-bit coefficients with an average of
  $3$-signed-digits each, allocated by the method of \emph{Lim et al.}\;.}
\label{fig:Comp-schurOneMlattice-lowpass-diff-R2-12-nbits-Schur-FIR-pass-relative}
\end{figure}

\begin{figure}
\centering
\scalebox{\DesignOfSchurLatticeFiltersPdfScale}{\input{comparison_no_POP_schurOneMlattice_lowpass_differentiator_R2_12_nbits_test_Schur_FIR_stop}}
\caption{Comparison of the stop band amplitude responses of a low pass
  differentiator filter implemented as a maximally linear FIR filter truncated
  to $16$ distinct coefficients and a tapped Schur one-multiplier lattice, R=2,
  correction filter. The coefficients are SOCP-relaxation and branch-and-bound
  optimised for $12$-bit coefficients with an average of
  $3$-signed-digits each, allocated by the method of \emph{Lim et al.}\;.}
\label{fig:Comp-schurOneMlattice-lowpass-diff-R2-12-nbits-Schur-FIR-stop}
\end{figure}

\clearpage
\section{Summary}
The Schur lattice digital filter has a simple stability constraint. This article
demonstrates the realisation of an IIR digital filter transfer function with
stability and frequency response constraints by optimising the floating point
and fixed point coefficients of a tapped one-multiplier Schur lattice digital
filter. The performance of the tapped, R=2, Schur lattice filter is compared
to that of a maximally-flat anti-symmetric FIR low pass differentiator filter.
The tapped, R=2, Schur lattice filter has lower group delay and better pass band
relative amplitude error. The truncated anti-symmetric FIR filter has better
stop band amplitude suppression and, by construction, constant phase and group
delay.

\bibliographystyle{plain}
\bibliography{DesignOfSchurLatticeFilters}

\end{document}
